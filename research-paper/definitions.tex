\section{Definitions}
\label{sec:definitions}



\renewcommand{\varSecKeyAlice}{\textit{sk}_0}
\renewcommand{\varPubKeyAlice}{\textit{pk}_0}
\renewcommand{\varSecKeyBob}{\textit{sk}_1}
\renewcommand{\varPubKeyBob}{\textit{pk}_1}
\renewcommand{\varSigAlice}{\ensuremath{\tilde{\sigma_0}}}
\renewcommand{\varSigBob}{\ensuremath{\tilde{\sigma_1}}}
\newcommand{\varSig}{\ensuremath{\tilde{\sigma_b}}}
\newcommand{\varSigOther}{\ensuremath{\tilde{\sigma}_{1-b}}}
\newcommand{\procVerFinSig}{\ensuremath{\mathsf{vrfFinSig}}}
\renewcommand{\varSigApt	}{\ensuremath{\hat{\sigma}_b}}
\newcommand{\varSigAptOther}{\ensuremath{\hat{\sigma}_{1-b}}}
\renewcommand{\varSigAptBob}{\ensuremath{\hat{\sigma}_1}}
\newcommand{\procSignOracleOne}[1]{\ensuremath{\mathcal{O}_s(#1)}}
\newcommand{\varPubKeyOneSide}{\ensuremath{\varPubKey_b}}

\begin{definition}[Two Party Staggered Signature Scheme]
\label{def:two-party-scheme}	
A \emph{two party staggered signature scheme $\varSigSchemeMP$} is a tuple of processes $(\procKeyGenPtId,$  $\procSignPtId, \procVerfPtSigId,$ $\procFinSigId)$ defined as follows: 
    
    \begin{itemize}
    	\item $((\varSecKeyAlice, \varPubKeyAlice, \varPubKey), (\varSecKeyBob, \varPubKeyBob, \varPubKey )) \opFunResult \procKeyGenPt{\varSecParam}{\varSecParam}$: The distributed key generation protocol takes as input the security parameter from both Alice and Bob and returns the tuple $(\varSecKeyAlice, \varPubKeyAlice)$ to Alice (similar to Bob) where $(\varSecKeyAlice, \varPubKeyAlice)$ is a pair of private and corresponding public keys. Moreover, both participants learn $\varPubKey$, that intuitively represents the combination of $\varPubKeyAlice$ and $\varPubKeyBob$. 
    	
    	\item $(\varSigAlice \opSeperate \varSigBob) \opFunResult \procSignPt{\varMsg}{\varSecKeyAlice, \varPubKey}{\varSecKeyBob, \varPubKey}$: In the distributed first stage signing protocol, each participant takes as input the message $\varMsg$, its secret key ($\varSecKey$) and the composite public key $\varPubKey$. This protocol outputs $(\varSigAlice, \varSigBob)$, that is, a half signature to each participant.  
    	
    	\item $\cnstTrueorFalse \opFunResult \procVerfPtSig{\varSig}{\varMsg}{\varPubKeyOneSide}{}$:  The verification algorithm for a half signature is a DPT function that takes as input the half signature $\varSig$, a message $m$, and the corresponding participant public key $\varPubKeyOneSide$. The algorithm returns 1 if the verification was successful or 0 otherwise.
        \item $\varSigFin \opFunResult \procFinSig{\varSigAlice}{\varSigBob}$: The finalize signature algorithm is a DPT function that takes as input two half  signatures and combines them into a final signature.
        
        \item $\cnstTrueorFalse \opFunResult \procVerFinSig(\varSigFin, \varMsg, \varPubKey) $: The verification algorithm for the final signature is a DPT function that takes as input the final signature $\varSigFin$, the message $\varMsg$ and public key $\varPubKey$ and returns $1$ if the verification is successful or $0$ otherwise.  
    \end{itemize}
\end{definition}


\begin{definition}[Two Party Staggered Signature Correctness]
\label{def:two-party-sig-correctness}
	For all $((\varSecKeyAlice, \varPubKeyAlice, \varPubKey),$ $(\varSecKeyBob, \varPubKeyBob, \varPubKey)) \opFunResult \procKeyGenPt{\varSecParam}{\varSecParam}$ and for all messages $\varMsg$, the following conditions should hold:
	\begin{itemize}
		\item Let $(\varSigAlice \opSeperate \varSigBob) \opFunResult \procSignPt{\varMsg}{\varSecKeyAlice, \varPubKey}{\varSecKeyBob, \varPubKey}$. Then, $1 \opFunResult \procVerfPtSig{\varSig}{\varMsg}{\varPubKeyOneSide}{}$ for $b \in \set{0,1}$. 
		\item  Let $\varSigFin \opFunResult \procFinSig{\varSigAlice}{\varSigBob}$. Then, $1 \opFunResult \procVerFinSig(\varSigFin, \varMsg, \varPubKey)$. 
	\end{itemize}
\end{definition}


\begin{definition}[Two Party Staggered Signature Security]
\label{def:two-party-sig-security}
	A two party staggered signature scheme is \emph{existentially unforgeable under chosen-message attacks} if for every probabilistic polynomial-time adversary $\cnstAdversary$ there exists a negligible function $\funNegl{\varSecParam}$ such that 

\[ \prob{\procExpForgeSimple{\varSecParam} \opEqNoQ 1} \opSmEq \funNegl{\varN} \]

\begin{center}
    \fbox{
    \begin{varwidth}{\textwidth}
        \procedure[linenumbering]{$\procExpForgeSimple{\varSecParam}$} {
         \varSet \opAssign \cnstEmptySet \\
        ((\varSecKeyAlice, \varPubKeyAlice), (\varSecKeyBob, \varPubKeyBob)) \opFunResult \procKeyGenPt{\varSecParam}{\varSecParam}\\
        (\varMsg^* \opSeperate (\varSigAlice^*, \varSigBob^*)) \opFunResult \cnstAdversary^{\procSignOracleOne{\cdot}}(\varPubKeyAlice, \varPubKeyBob, \varSecKeyBob) \\
        \varSignature^* \opFunResult \procFinSig{\varSigAlice^*}{\varSigBob^*} \\
        \pcreturn ((\varMsg) \opNotIn \varSet  \opAnd \procVerFinSig(\varSignature^*, \varMsg^*, \varPubKey))
        }\\[2\baselineskip]
        \procedure[linenumbering]{$\procSignOracleOne{\varMsg}$} {
        (\varSigAlice \opSeperate \varSigBob) \opFunResult \procSignPt{\varMsg}{\varSecKeyAlice, \varPubKey}{\varSecKeyBob, \varPubKey}\\
        \varSet \opAssign \varSet \opUnion \{\varMsg\} \\
        \pcreturn (\varSigAlice \opSeperate \varSigBob)
        }
    \end{varwidth}
    }
\end{center}

\end{definition}



\begin{definition}[Flexible Adaptor Signatures]
A \emph{flexible adaptor signature} scheme $\varSigSchemeApt$ wrt. a hard relation $R$ and two party staggered signature scheme $\varSigSchemeMP$ consists of three algorithms $(\procAptSigId, \procVerifyAptSigId, \procExtWitId)$ defined as follows:


\begin{itemize}
        \item $\varSigApt \opFunResult \procAptSig{\varSig}{\varWit}$ takes as input a half signature $\varSig$ and a witness value $\varWit$ and it outputs an adapted signature $\varSigApt$. 
        \item $\cnstTrueorFalse \opFunResult \procVerifyAptSig{\varSigApt}{\varMsg}{\varPubKeyOneSide}{\varStatement}{}$ takes as input an adapted half signature $\varSigApt$, a message $\varMsg$, a couple of public keys $\varPubKeyAlice$, $\varPubKeyBob$ and a statement $\varStatement$ it outputs if $1$ if $\varSigApt$ is a valid adapted signature under statement $\varStatement$ and public keys $\varPubKeyAlice$, $\varPubKeyBob$. 
        \item $\varWit \opFunResult \procExtWit{\varSignature}{\varSigApt}{\varSigOther}$ takes as input a full signature $\varSignature$, a half signature in the adapted form $\varSigApt$ and the other half signature $\varSigOther$ and returns the witness $\varWit$.
    \end{itemize}

\end{definition}




\begin{definition}[Flexible Adaptor Signature Correctness]
We say that a flexible adaptor signature scheme is \emph{correct} if for every message $\varMsg \in \cnstBinary{*}$, keypairs $((\varSecKeyAlice, \varPubKeyAlice), (\varSecKeyBob, \varPubKeyBob)) \opFunResult \procKeyGenPt{\varSecParam}{\varSecParam}$, every statement/witness pair $(\varStatement \opSeperate \varWit)$ in a relation $\cnstRelation$ and pair of half signatures $(\varSigAlice \opSeperate \varSigBob) \opFunResult \procSignPt{\varMsg}{\varSecKeyAlice, \varPubKey}{\varSecKeyBob, \varPubKey}$ and the corresponding final signature $\varSigFin \opFunResult \procFinSig{\varSigAlice}{\varSigBob}$ it holds that 
	
	\begin{itemize}
		\item $1 \opFunResult \procVerifyAptSig{\procAptSig{\varSig}{\varWit}}{\varMsg}{\varPubKeyOneSide}{\varStatement}{}$
		\item $\varWit \opFunResult \procExtWit{\varSignature}{\procAptSig{\varSig}{\varWit}}{\varSigOther}$
	\end{itemize}

\end{definition}


\begin{definition}[Flexible Adaptor Signature Unforgeability] Let $\varSigSchemeMP$ be a secure two party staggered signature scheme as defined in~\cref{def:two-party-scheme}. Then, we say that a flexible adaptor signature scheme $\varSigSchemeApt$ is \emph{unforgeable under chosen message attack}  if for every probabilistic polynomial-time adversary $\cnstAdversary$ there exists a negligible function $\funNegl{\varSecParam}$ such that 
	
	\[ \prob{\procExpForg{\varSecParam} \opEqNoQ 1} \opSmEq \funNegl{\varN} \]

\begin{center}
    \fbox{
    \begin{varwidth}{\textwidth}
        \procedure[linenumbering]{$\procExpForg{\varSecParam}$} {
        ((\varSecKeyAlice, \varPubKeyAlice), (\varSecKeyBob, \varPubKeyBob)) \opFunResult \procKeyGenPt{\varSecParam}{\varSecParam}\\
        (\varMsg^*) \opFunResult \cnstAdversary^{\procSignOracle{\cdot}{\cdot}}(\varPubKeyAlice, \varPubKeyBob, \varSecKeyBob) \\
        (\varSigAlice \opSeperate \varSigBob) \opFunResult \procSignPt{\varMsg}{\varSecKeyAlice, \varPubKey}{\varSecKeyBob, \varPubKey}\\
        (\varWit, \varStatement) \opFunResult \procGenR{\varSecParam}\\
        \varSigAptBob^* \opFunResult \cnstAdversary^{\procSignOracle{\cdot}{\cdot}}(\varSigBob, \varStatement) \\
        \pcreturn \procVerifyAptSig{\varSigAptBob^*}{\varMsg^*}{\varPubKeyBob}{\varStatement}{}}\\[2\baselineskip]
        \procedure[linenumbering]{$\procSignOracle{\varMsg}{\varSecKeyBob}$} {
        (\varSigAlice \opSeperate \varSigBob) \opFunResult \procSignPt{\varMsg}{\varSecKeyAlice, \varPubKey}{\varSecKeyBob, \varPubKey}\\
        \pcreturn (\varSigAlice \opSeperate \varSigBob)
        }
    \end{varwidth}
    }
\end{center}

	
\end{definition}


\begin{definition}[Flexible Adaptor Signature Witness Extractability] 
	
Let $\varSigSchemeMP$ be a secure two party staggered signature scheme as defined in~\cref{def:two-party-scheme}. Then, we say that a flexible adaptor signature scheme $\varSigSchemeApt$ achieves \emph{witness extractability} if for every probabilistic polynomial-time adversary $\cnstAdversary$ there exists a negligible function $\funNegl{\varSecParam}$ such that 

	 \[ \prob{\procExpExt{\varN} \opEqNoQ 1} \opSmEq \funNegl{\varN} \]
	
    \begin{center}
        \fbox{
        \begin{varwidth}{\textwidth}
            \procedure[linenumbering]{$\procExpExt{\varN}$} {
            ((\varSecKeyAlice, \varPubKeyAlice), (\varSecKeyBob, \varPubKeyBob)) \opFunResult \procKeyGenPt{\varSecParam}{\varSecParam} \\
            (\varMsg \opSeperate \varStatement) \opFunResult \cnstAdversary^{\procSignOracle{\cdot}{\cdot}}(\varPubKeyAlice, \varPubKeyBob, \varSecKeyBob) \\
            (\varSigAlice \opSeperate \varSigBob)\opFunResult \procSignPt{\varMsg}{\varSecKeyAlice, \varPubKey}{\varSecKeyBob, \varPubKey} \\
            (\varSigAptBob \opSeperate \varSigFin) \opFunResult \cnstAdversary^{\procSignOracle{\cdot}{\cdot}}(\varSigAlice) \\
            \funStar{\varWit} \opFunResult \procExtWit{\varSigFin}{\varSigAlice}{\varSigAptBob} \\
            \pcreturn ((\varStatement \opSeperate \funStar{\varWit}) \opNotIn \cnstRelation \opAnd \procVerFinSig(\varSigFin, \varMsg, \varPubKey))
            }\\

            \procedure[linenumbering]{$\procSignOracle{\varMsg}{\varPubKey}$} {
            \varSet \opAssign \varSet \opUnion \varMsg \\
            (\varSigAlice \opSeperate \varSigBob) \opFunResult \procSignPt{\varMsg}{\varSecKeyAlice, \varPubKey}{\varSecKeyBob, \varPubKey} \\
            \pcreturn  (\varSigAlice \opSeperate \varSigBob)
            }
        \end{varwidth}
        }
    \end{center}
\end{definition}





































