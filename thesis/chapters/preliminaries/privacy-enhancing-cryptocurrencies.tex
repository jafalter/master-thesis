\urldef\urlblockexp\url{https://blockstream.info/}
\urldef\urlchainanal\url{https://www.chainalysis.com/}

As seen in \cref{sec:pre:bitcoin} in Bitcoin funds are stored in UTXOs which can be identified by an address.
The value being transferred in a transaction is given in plain text therefore by the nature of Bitcoin's public blockchain it is possible for anyone to simply look up the amount stored in an address.
So called Block Explorers\urlblockexp make such a lookup straight forward.
It is further possible to link multiple addresses together, further weakening the anonymity of the system~\cite{barber2012bitter}.
Attempts such as CoinJoin~\cite{maxwell2013coinjoin} or its successor CoinShuffle~\cite{ruffing2014coinshuffle} introduced protocols that can mitigate this linkability issue in Bitcoin.

\subsection{Zero Knowledge Proofs} \label{sec:pre:privacy:zeroknowlegde}

\subsection{Range Proofs} \label{sec:pre:rangeproof}

\begin{definition}[Range proof System]\label{def:pre:rangeproof}
    A range proof system $\varRProofSystemParam{\varCommitScheme}$ with regards to a homomorphic commitment scheme $\varCommitScheme$ consists of a tupel of functions $(\procRProofSetupId, \procProofId, \procVerfProofId)$.
    \begin{itemize}
        \item $\varRProofParams \opFunResult \procRProofSetup{\varSecParam}{\varI}{\varJ}$: The rangeproof setup algorithm takes as input a security paramter $\varSecParam$ as well as two numbers
        $\varI$ and $\varJ$ which are treated as exponents of 2 to define the lower and upper bound of the rangeproof protocol.
        \item $\varProof \opFunResult \procProof{\varCommitment}{\varValue}{\varBlindingFactor}$: The proof algorithm is a DPT function which takes as input a commitment $\varCommitment$ a value $\varValue$ and
        a blinding factor $\varBlindingFactor$. It will output a proof $\varProof$ attesting to the statement that the value $\varValue$ of commitment $\varCommitment$ is in between the range $\langle \varLowerBound, \varUpperBound \rangle$ as
        defined during the $\procRProofSetupId$ function.
        \item $\{1,0\} \opFunResult \procVerfProof{\varProof}{\varCommitment}$: The proof verification algorithm is a DPT function which verifies the validity of the proof $\varProof$ with regards to the commitment
        $\varCommitment$. It will output 1 upon a successfull verification or 0 otherwise.
    \end{itemize}
\end{definition}

\begin{definition}[Multiparty Rangeproof System]\label{def:pre:mp-rangeproof}
    A Multiparty Rangeproof Sytem $\varMPRProofSystemParam{\varCommitScheme}$ with regards to a homomorphic commitment scheme $\varCommitScheme$ is an extension of the regular Rangeproof Sytem with the following
    distributed protocol $\procDRProofId$.
    \begin{itemize}
        \item $\varProof \opFunResult \procDRProof{\varCommitment}{\varValue}{\varBlindingFactorAlice}{\varBlindingFactorBob}$: The distributed proof protocol allows two parties Alice and Bob, each owning a share of the
        commitment $\varCommitment$ to cooperate in order to produce a valid range proof $\varProof$ without a party learning the blinding factor share from the other party.
    \end{itemize}
\end{definition}

For MP proofs~\cite{klinec2020privacy}