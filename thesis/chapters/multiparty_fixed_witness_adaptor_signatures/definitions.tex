A Two-Party Signature Scheme is an extension of a Signature Scheme shown in~\cref{def:pre:signature-scheme}, which allows us to distribute signature generation for a composite public key shared between two parties Alice and Bob.
Alice and Bob want to collaborate to generate a signature valid under the composite public key $\varPubKey \opAssign \varPubKeyAlice \opAddPoint \varPubKeyBob$ without revealing their secret keys to each other.
The definition below was constructed with the goal in mind of formalizing exactly what is currently implemented and used in Mimblewimble based Cryptocurrencies.

\begin{definition}[Two-Party Signature Scheme]
    \label{def:sig:two-party-sig}

    A \emph{Two-Party Signature Scheme $\varSigSchemeMP$} extends a Signature Scheme $\varSigScheme$ with a tuple of protocols and algorithms\\
    ($\procKeyGenPtId, \procSignPrtId , \procVerfPtSigId, \procFinSigId)$ defined as follows:

    \begin{itemize}
        \item $((\varSecKeyAlice, \varPubKeyAlice, \varNonceAlice, \varSigContext), (\varSecKeyBob, \varPubKeyBob, \varNonceBob, \varSigContext)) \opFunResult \procKeyGenPt{\varSecParam}{\varSecParam}$: The distributed key generation protocol takes as input the security parameter from both Alice and Bob.
        It returns the tuple $(\varSecKeyAlice, \varPubKeyAlice, \varNonceAlice, \varSigContext)$ to Alice (similar to Bob) where $(\varSecKeyAlice, \varPubKeyAlice)$ is a pair of private and corresponding public keys, $\varNonceAlice$ a secret nonce and $\varSigContext$ is the signature context containing parameters shared between Alice and Bob.
        We introduce $\varSigContext$ for the participants to share and update parameters with each other during the protocol execution.
        Note that this context always has to be consistent between the two parties.
        If Alice were to update $\varSigContext$, she has to send the updated version to Bob to continue the protocol.

        \item $(\varSigAlice) \opFunResult \procSignPrt{\varMsg}{\varSecKeyAlice}{\varNonceAlice}{\varSigContext}$: The partial signing algorithm is a DPT function that takes as input the message $\varMsg$, the share of the secret key $\varSecKeyAlice$ and nonce $\varNonceAlice$ (similar for Bob), and the shared signature context $\varSigContext$. The procedure outputs $(\varSigAlice)$, that is, a share of the signature to a participant.

        \item $\cnstTrueorFalse \opFunResult \procVerfPtSig{\varSigAlice}{\varMsg}{\varPubKeyAlice}$: The share verification algorithm is a DPT function that takes as input a signature share $\varSigAlice$, a message $m$, and the other participant's public key $\varPubKeyAlice$ (similar $\varPubKeyBob$ for Bob's partial signature).
        The algorithm returns 1 if the verification was successful or 0 otherwise.

        \item $\varSigFin \opFunResult \procFinSig{\varSigAlice}{\varSigBob}$: The finalize signature algorithm is a DPT function that takes as input two shares of the signatures and combines them into a final signature valid under the composite public key $\varPubKey \opEqNoQ \varPubKeyAlice \opAddPoint \varPubKeyBob$.

    \end{itemize}

\end{definition}

We require the Two-Party Signature Scheme to be correct as well as secure as of~\cref{subsec:pre:security}.
For the security of the distributed key-generation protocol $\procKeyGenPtId$, special care needs to be taken to protect the scheme against rogue-key attacks.
In such an attack one of the public keys is computed as a function of the other parties public key, allowing the corrupted signer to produce forged signatures under the honest users public key without knowing its secret key~\cite{maxwell2019simple}.


From~\cref{def:sig:two-party-sig}, we now derive a Two-Party Adaptor Signature Scheme $\varSigSchemeApt$, allowing one of the participants to hide a secret witness value inside his partial signature.
\begin{definition}[Two-Party Fixed Witness Adaptor Signature Scheme]
    \label{def:sig:two-party-fixed-wit-apt-sig}
    Given a pair $(\varWit, \varStatement) \opIn \cnstRelation$, (where $\cnstRelation$ is a hard relation as of \cref{def:pre:hard-relation}) a Two-Party Fixed Witness Adaptor Signature Scheme $\varSigSchemeApt$ is an extension to $\varSigSchemeMP$ with the following algorithms.

    \[ \varSigSchemeApt \opAssign (\varSigSchemeMP \opConc \procAptSigId \opConc \procVerifyAptSigId \opConc \procExtWitId) \]

    \begin{itemize}
        \item $\varSigAptAlice \opFunResult \procAptSig{\varSigAlice}{\varWit}$: The mask signature algorithm is a DPT function that takes as input a partial signature $\varSigAlice$ and a secret witness value $\varWit$.
        The procedure will output a masked partial signature $\varSigAptAlice$ that can be verified to contain $\varWit$ using the $\procVerifyAptSigId$ function without revealing $\varWit$.

        \item $\cnstTrueorFalse \opFunResult \procVerifyAptSig{\varSigAptAlice}{\varMsg}{\varPubKeyAlice}{\varStatement}$: The masked signature verification algorithm is a DPT function that takes as input a masked partial signature $\varSigAptAlice$, the other participant's public key $\varPubKeyAlice$ and a statement $\varStatement$.
        The function will verify the partial signature's validity and that it was masked with the secret witness $\varWit$.

        \item $\varWit \opFunResult \procExtWit{\varSigFin}{\varSigAlice}{\varSigAptBob}$: The witness extraction algorithm is a DPT function that lets Alice extract the secret witness $\varWit$ after having learned the final composite signature $\varSigFin$.
        As input, it expects the partial signatures $\varSigAlice$ and $\varSigAptBob$ shared between the participants during protocol execution and the final composite signature $\varSigFin$.
        Consequently, only protocol participants knowing the partial signatures exchanged during the protocol can run this algorithm.
    \end{itemize}
\end{definition}

Similar to how it is defined in~\cite{aumayr2020bitcoinchannels}, additionally to regular Correctness, as described  in~\cref{def:pre:signature-scheme}, we require our Signature Scheme to satisfy Adaptor Signature Correctness.
This property is given when one can complete every masked partial signature generated by $\procAptSigId$ into a final signature for all pairs $(\varWit \opSeperate \varStatement) \opIn \cnstRelation$.
And it will then be possible to extract the witness computing $\procExtWitId$ with the required parameters.

\begin{definition}[Adaptor Signature Correctness]
    \label{def:sig:apt-sig-correctness}
    More formally, \emph{Adaptor Signature Correctness} is given if for every security parameter $\varN \in \cnstNatural$, message $\varMsg \in \cnstBinary{*}$, \\ keypairs $\langle (\varSecKeyAlice, \varPubKeyAlice, \varNonceAlice, \varSigContext), (\varSecKeyBob, \varPubKeyBob, \varNonceBob, \varSigContext) \rangle \opFunResult \procKeyGenPt{\varSecParam}{\varSecParam}$ with their composite public key $\varSigContext.\varPubKey \opEqNoQ \varPubKeyAlice \opAddPoint \varPubKeyBob$ and every statement/witness pair $(\varStatement \opSeperate \varWit) \opFunResult \procGenR{\varSecParam}$ it must hold that:
    \[
        \Pr\left[
        \begin{array}{c}
            \:\procVerf{\varMsg}{\varSigFin}{\varSigContext.\varPubKey} \opEqNoQ 1                                         \\
            \opAnd                                                                                              \\
            \: \procVerifyAptSig{\varSigAptBob}{\varMsg}{\varPubKeyBob}{\varStatement} \opEqNoQ 1             \\
            \opAnd                                                                                              \\
            \:(\funStar{\varWit}, \varStatement) \opIn \cnstRelation
        \end{array}
        \middle\vert
        \begin{array}{l}
            (\varWit, \varStatement) \opFunResult \procGenR{\varSecParam} \\
            \varSigAlice \opFunResult \procSignPrt{\varMsg}{\varSecKeyAlice}{\varNonceAlice}{\varSigContext}        \\
            \varSigBob \opFunResult \procSignPrt{\varMsg}{\varSecKeyBob}{\varNonceBob}{\varSigContext}              \\
            \varSigAptBob \opFunResult \procAptSig{\varSigBob}{\varWit}                                             \\
            \varSigFin \opFunResult \procFinSig{\varSigAlice}{\varSigBob}                                           \\
            \funStar{\varWit} \opFunResult \procExtWit{\varSigFin}{\varSigAlice}{\varSigAptBob}
        \end{array}
        \right]=1.
    \]
\end{definition}