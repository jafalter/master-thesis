We now prove that the outlined schnorr-based instantiation is correct, i.e. Adaptor Signature Correctness holds, as well as secure with regards to the definition~\ref{subsec:pre:security}.

\subsection{Adaptor Signature Correctness}\label{subsec:sig:aptsig-correctness}

To prove that Adaptor Signature Correctness holds we have 3 statements to prove, first we prove that $\procVerf{\varMsg}{\varSigFin}{\varSigContext.\varPubKey} \opEq 1$ holds in our
schnorr-based instantiation of the signature scheme, whereas $\varSigContext$ is setup such that $\varPubKey \opEqNoQ \varPubKeyAlice \opAddPoint \varPubKeyBob$.

\begin{proof}
    \label{prf:apt-schnorr-pre-sig-corr}
    For this prove we assume the setup already specified in definition~\ref{def:sig:apt-sig-correctness}.
    The proof is by showing equality of the equation checked by the verifier of the final signature by continuous substitutions in the left side of equation:
    \begin{align}
        \funGen{\varS} &\opEqNoQ \varRand \opAddPoint \opPointScalar{\varPubKey}{\varSchnorrChallenge} \\
        \funGen{\varSAlice} \opAddPoint \funGen{\varSBob} & \\
        \funGen{\varNonceAlice \opAddScalar \varSchnorrChallenge \opTimesScalar \varSecKeyAlice} \opAddPoint \funGen{\varNonceBob \opAddScalar \varSchnorrChallenge \opTimesScalar \varSecKeyBob} & \\
        \funGen{\varNonceAlice} \opAddPoint \opPointScalar{\varPubKeyAlice}{\varSchnorrChallenge} \opAddPoint \funGen{\varNonceBob} \opAddPoint \opPointScalar{\varPubKeyBob}{\varSchnorrChallenge} & \\
        \varRandAlice \opAddPoint \opPointScalar{\varPubKeyAlice}{\varSchnorrChallenge} \opAddPoint \varRandBob \opAddPoint \opPointScalar{\varPubKeyBob}{\varSchnorrChallenge} & \\
        \varRand \opAddPoint \opPointScalar{\varPubKey}{\varSchnorrChallenge} & \opEqNoQ \varRand \opAddPoint \opPointScalar{\varPubKey}{\varSchnorrChallenge} \\
        1 & \opEqNoQ 1
    \end{align}

    It remains to prove that with the same setup $\procVerifyAptSig{\varSigAptBob}{\varMsg}{\varPubKeyBob}{\varStatement} \opEq 1$ and
    $(\varStatement \opSeperate \funStar{\varWit}) \opIn \cnstRelation$ hold.

    \[
        \procVerifyAptSig{\varSigAptBob}{\varMsg}{\varPubKeyBob}{\varStatement} \opEq 1
    \]
    The proof is by continuous substitutions in the equation checked by the verifier:
    \begin{align}
        \funGen{\varSigAptBob} &\opEqNoQ \varRandBob \opAddPoint \opPointScalar{\varPubKeyBob}{\varSchnorrChallenge} \opAddPoint \varStatement \\
        \funGen{\varSigBob \opAddScalar \varWit} & \\
        \funGen{\varNonceBob \opAddScalar \varSecKeyBob \opTimesScalar \varSchnorrChallenge \opAddScalar \varWit} & \\
        \funGen{\varNonceBob} \opAddPoint \funGen{\varSecKeyBob \opTimesScalar \varSchnorrChallenge} \opAddScalar \funGen{\varWit} & \\
        \varRandBob \opAddPoint \opPointScalar{\varPubKeyBob}{\varSchnorrChallenge} \opAddPoint \varStatement &\opEqNoQ \varRandBob \opAddPoint \opPointScalar{\varPubKeyBob}{\varSchnorrChallenge} \opAddPoint \varStatement \\
        1 &\opEqNoQ 1
    \end{align}
    We now continue to prove the last equation required:
    \[
        ((\varStatement \opSeperate \funStar{\varWit}) \opIn \cnstRelation)
    \]
    We do this by showing that $\varWit$ is calculated correctly in $\procExtWitId$:
    \begin{align}
        \varWit \opAssign & \varSApt \opSub (\varS \opSub \varSAlice) \\
        & \varSApt \opSub ((\varSAlice \opAddScalar \varSBob ) \opSub \varSAlice ) \\
        & \varSBob \opAddScalar \varWit \opSub (\varSBob) \\
        \varWit \opAssign & \varWit \\
    \end{align}
\end{proof}

\subsection{Security}\label{subsec:sig:secureaptscheme}

We have shown that the outline signature scheme is correct, next we have to prove its security.
We will prove security for both the $\procDSignId$ and $\procDAptSignId$ protocols in the hybrid model which was layed out by Yehuda Lindell in~\cite{lindell2017simulate}.
In particular we will use the $\procZKfId$-model in which we assume that we have access to a constant-round protocol $\procZKfId$ that computes the zero-knowledge proof of knowledge functionality for any $\cnstNP$ relation $\cnstRelation$.

