This section will first define procedures and protocols to construct Mimblewimble transactions and prove their security.
The formalizations will be similar to those found by Fuchsbauer et al. in their cryptographic investigation of the Mimblewimble protocol~\cite{fuchsbauer2019aggregate}.
In particular, the final transaction output from our protocols should be a valid transaction as by the definitions of Fuchsbauer et al.
As we will only focus on the transaction building protocol, the notions of cut through, transaction merging, coin minting (coinbase transactions), and publishing transactions to the ledger, all discussed in~\cref{sec:pre:mimblewimble} and formalized by Fuchsbauer et al. in~\cite{fuchsbauer2019aggregate}, will not be the topic of this formalization.

As an extension to the regular transaction protocol \emph{Mimblewimble Transaction Scheme}, which we will define first, we will additionally define two further schemes:
The first of them titled \emph{Extended Mimblewimble Transaction Scheme}, will provide additional functions to create and spend coins owned by two keys instead of just one, thereby enabling coins owned by multiple parties, which is similar to a mutlisig address in Bitcoin~\cite{antonopoulos2014mastering}.
The second extended definition is called \emph{Contract Mimblewimble Transaction Scheme}, in which we further add algorithms that allow embedding primitive smart contracts to the transaction building protocol.
Both the \emph{Extended Mimblewimble Transaction Scheme} and \emph{Contract Mimblewimble Transaction Scheme} are constructed to provide the functionality that is later needed to build the final Atomic Swap protocol, which we will introduce in~\cref{sec:atom:atomic-swap}.

We will proceed by providing an instantiation of the three transaction schemes in~\cref{sec:atom:inst}, which can be implemented and deployed on a Mimblewimble based cryptocurrency such as Beam or Grin.
In~\cref{sec:atom:protocols}, we define two-party protocols from the outlined schemes to construct Mimblewimble transactions.
~\Cref{sec:atom:security} shows the proofs that the formalizations are correct and the protocols secure in the malicious setting as shown in~\cref{subsec:pre:security}.
Finally, in~\cref{sec:atom:atomic-swap}, we describe an Atomic Swap protocol from these building blocks, allowing two parties to securely and trustlessly swap funds from a Mimblewimble based blockchain with those on another blockchain, such as Bitcoin.

\section{Definitions}\label{sec:atom:definitions}
A Two-Party Signature Scheme is an extension of a Signature Scheme shown in~\cref{def:pre:signature-scheme}, which allows us to distribute signature generation for a composite public key shared between two parties Alice and Bob.
Alice and Bob want to collaborate to generate a signature valid under the composite public key $\varPubKey \opAssign \varPubKeyAlice \opAddPoint \varPubKeyBob$ without revealing their secret keys to each other.
The definition below was constructed with the goal in mind of formalizing exactly what is currently implemented and used in Mimblewimble-based cryptocurrencies.

\begin{definition}[Two-Party Signature Scheme]
    \label{def:sig:two-party-sig}

    A \emph{Two-Party Signature Scheme $\varSigSchemeMP$} extends a Signature Scheme $\varSigScheme$ with a tuple of protocols and algorithms\\
    ($\procKeyGenPtId, \procSignPrtId , \procVerfPtSigId, \procFinSigId)$ defined as follows:

    \begin{asparaitem}
        \item $((\varSecKeyAlice, \varPubKeyAlice, \varNonceAlice, \varSigContext), (\varSecKeyBob, \varPubKeyBob, \varNonceBob, \varSigContext)) \opFunResult \procKeyGenPt{\varSecParam}{\varSecParam}$: The distributed key generation protocol takes as input the security parameter from both Alice and Bob.
        It returns the tuple $(\varSecKeyAlice, \varPubKeyAlice, \varNonceAlice, \varSigContext)$ to Alice (similar to Bob) where $(\varSecKeyAlice, \varPubKeyAlice)$ is a pair of private and corresponding public keys, $\varNonceAlice$ a secret nonce and $\varSigContext$ is the signature context containing parameters shared between Alice and Bob.
        We introduce $\varSigContext$ for the participants to share and update parameters with each other during the protocol execution.
        Note that this context always has to be consistent between the two parties.
        If Alice were to update $\varSigContext$, she has to send the updated version to Bob to continue the protocol.

        \item $(\varSigAlice) \opFunResult \procSignPrt{\varMsg}{\varSecKeyAlice}{\varNonceAlice}{\varSigContext}$: The partial signing algorithm is a DPT function that takes as input the message $\varMsg$, the share of the secret key $\varSecKeyAlice$ and nonce $\varNonceAlice$ (similar for Bob), and the shared signature context $\varSigContext$. The procedure outputs $(\varSigAlice)$, that is, a share of the signature to a participant.

        \item $\cnstTrueorFalse \opFunResult \procVerfPtSig{\varSigAlice}{\varMsg}{\varPubKeyAlice}$: The share verification algorithm is a DPT function that takes as input a signature share $\varSigAlice$, a message $m$, and the other participant's public key $\varPubKeyAlice$ ($\varPubKeyBob$ for Bob's partial signature).
        The algorithm returns 1 if the verification was successful or 0 otherwise.

        \item $\varSigFin \opFunResult \procFinSig{\varSigAlice}{\varSigBob}$: The finalize signature algorithm is a DPT function that takes as input two shares of the signatures and combines them into a final signature valid under the composite public key $\varPubKey \opEqNoQ \varPubKeyAlice \opAddPoint \varPubKeyBob$.

    \end{asparaitem}

\end{definition}

We require the Two-Party Signature Scheme to be correct as well as secure as of~\cref{subsec:pre:security}.
For the security of the distributed key-generation protocol $\procKeyGenPtId$, special care needs to be taken to protect the scheme against rogue-key attacks.
In such an attack one of the public keys is computed as a function of the other parties public key, allowing the corrupted signer to produce forged signatures under the honest users public key without knowing its secret key~\cite{maxwell2019simple}.


From~\cref{def:sig:two-party-sig}, we now derive a Two-Party Adaptor Signature Scheme $\varSigSchemeApt$, allowing one of the participants to hide a secret witness value inside his partial signature.
\begin{definition}[Two-Party Fixed Witness Adaptor Signature Scheme]
    \label{def:sig:two-party-fixed-wit-apt-sig}
    Given a pair $(\varWit, \varStatement) \opIn \cnstRelation$, (where $\cnstRelation$ is a hard relation as of \cref{def:pre:hard-relation}) a Two-Party Fixed Witness Adaptor Signature Scheme $\varSigSchemeApt$ is an extension to $\varSigSchemeMP$ with the following algorithms.

    \[ \varSigSchemeApt \opAssign (\varSigSchemeMP \opConc \procAptSigId \opConc \procVerifyAptSigId \opConc \procExtWitId) \]

    \begin{asparaitem}
        \item $\varSigAptAlice \opFunResult \procAptSig{\varSigAlice}{\varWit}$: The mask signature algorithm is a DPT function that takes as input a partial signature $\varSigAlice$ and a secret witness value $\varWit$.
        The procedure will output a masked partial signature $\varSigAptAlice$ that can be verified to contain $\varWit$ using the $\procVerifyAptSigId$ function without revealing $\varWit$.

        \item $\cnstTrueorFalse \opFunResult \procVerifyAptSig{\varSigAptAlice}{\varMsg}{\varPubKeyAlice}{\varStatement}$: The masked signature verification algorithm is a DPT function that takes as input a masked partial signature $\varSigAptAlice$, the other participant's public key $\varPubKeyAlice$ and a statement $\varStatement$.
        The function will verify the partial signature's validity and that it was masked with the secret witness $\varWit$.

        \item $\varWit \opFunResult \procExtWit{\varSigFin}{\varSigAlice}{\varSigAptBob}$: The witness extraction algorithm is a DPT function that lets Alice extract the secret witness $\varWit$ after having learned the final composite signature $\varSigFin$.
        As input, it expects the partial signatures $\varSigAlice$ and $\varSigAptBob$ shared between the participants during protocol execution and the final composite signature $\varSigFin$.
        Consequently, only protocol participants knowing the partial signatures exchanged during the protocol can run this algorithm.
    \end{asparaitem}
\end{definition}

Similar to how it is defined in~\cite{aumayr2020bitcoinchannels}, additionally to regular Correctness, as described  in~\cref{def:pre:signature-scheme}, we require our Signature Scheme to satisfy Adaptor Signature Correctness.
This property is given when one can complete every masked partial signature generated by $\procAptSigId$ into a final signature for all pairs $(\varWit \opSeperate \varStatement) \opIn \cnstRelation$.
And it will then be possible to extract the witness computing $\procExtWitId$ with the required parameters.

\begin{definition}[Adaptor Signature Correctness]
    \label{def:sig:apt-sig-correctness}
    More formally, \emph{Adaptor Signature Correctness} is given if for every security parameter $\varN \in \cnstNatural$, message $\varMsg \in \cnstBinary{*}$, \\ keypairs $\langle (\varSecKeyAlice, \varPubKeyAlice, \varNonceAlice, \varSigContext), (\varSecKeyBob, \varPubKeyBob, \varNonceBob, \varSigContext) \rangle \opFunResult \procKeyGenPt{\varSecParam}{\varSecParam}$ with their composite public key $\varSigContext.\varPubKey \opEqNoQ \varPubKeyAlice \opAddPoint \varPubKeyBob$ and every statement/witness pair $(\varStatement \opSeperate \varWit) \opFunResult \procGenR{\varSecParam}$ it must hold that:
    \[
        \Pr\left[
        \begin{array}{c}
            \:\procVerf{\varMsg}{\varSigFin}{\varSigContext.\varPubKey} \opEqNoQ 1                                         \\
            \opAnd                                                                                              \\
            \: \procVerifyAptSig{\varSigAptBob}{\varMsg}{\varPubKeyBob}{\varStatement} \opEqNoQ 1             \\
            \opAnd                                                                                              \\
            \:(\funStar{\varWit}, \varStatement) \opIn \cnstRelation
        \end{array}
        \middle\vert
        \begin{array}{l}
            (\varWit, \varStatement) \opFunResult \procGenR{\varSecParam} \\
            \varSigAlice \opFunResult \procSignPrt{\varMsg}{\varSecKeyAlice}{\varNonceAlice}{\varSigContext}        \\
            \varSigBob \opFunResult \procSignPrt{\varMsg}{\varSecKeyBob}{\varNonceBob}{\varSigContext}              \\
            \varSigAptBob \opFunResult \procAptSig{\varSigBob}{\varWit}                                             \\
            \varSigFin \opFunResult \procFinSig{\varSigAlice}{\varSigBob}                                           \\
            \funStar{\varWit} \opFunResult \procExtWit{\varSigFin}{\varSigAlice}{\varSigAptBob}
        \end{array}
        \right]=1.
    \]
\end{definition}

\section{Instantiation}\label{sec:atom:inst}
\urldef\urlgrinexplained\url{https://medium.com/@brandonarvanaghi/grin-transactions-explained-step-by-step-fdceb905a853}

This section will provide an instantiation of the Transaction Scheme definitions found in~\cref{def:atom:mw-tx-scheme},~\cref{def:atom:ext-mw-tx-scheme}, and~\cref{def:atom:apt-ext-mw-tx-scheme}.
One can implement the instantiations in a cryptocurrency based on the Mimblewimble protocol such as Beam and Grin.

\subsection{Mimblewimble Transaction Scheme}\label{subsec:atom:mw-tx-scheme}

First, we provide an instantiation of the simplest form of a transaction in which a sender wants to transfer some value $\varFundValue$ to a receiver.
For the protocol's execution we assume to have access to a homomorphic Commitment Scheme such as Pedersen Commitment $\varCommitScheme$, shown in~\cref{def:pre:pedersen}.
Furthermore, we require a Range Proof System $\varRProofSystem$ as described in~\cref{def:pre:rangeproof} and a Two-Party Signature Scheme $\varSigSchemeMP$ as of~\cref{def:sig:two-party-sig}.

To make the pseudocode for the transaction protocol easier to read, we first introduce two auxiliary functions $\procCreateCoinId$
and $\procCreatePreTxId$.
The coin creation function will take as input a value $\varValue$ and a blinding factor $\varBlindingFactor$. It will create and output a new spendable coin $\varSpendableCoin$ already containing a range proof $\varProof$ attesting to the statement that the coins value $\varValue$ is within the valid range as defined for the blockchain.
The transaction creation algorithm $\procCreatePreTxId$ takes as input a message $\varMsg$, a list of input coins $\funArray{\varCoinInp}$, a list of output coins $\funArray{\varCoinOut}$, a list of range proofs $\funArray{\varProof}$, a signature context $\varSigContext$, a list of Commitments $\funArray{\varCommitment}$, a signature $\varSignature$, and a lock time $\varTime$ and will collect the input data into a transaction object.

\begin{center}
    \fbox{
    \begin{varwidth}{\textwidth}
        \procedure[linenumbering]{$\procCreateCoin{\varValue}{\varBlindingFactor}$} {
        \varCommitment \opFunResult \procCommit{\varValue}{\varBlindingFactor} \\
        \varProof \opFunResult \procProof{\varCoin}{\varValue}{\varBlindingFactor} \\
        \pcreturn (\varCommitment, \varBlindingFactor, \varValue, \varProof)
        }
        \procedure[linenumbering]{$\procCreatePreTx{\varMsg}{\funArray{\varCoinInp}}{\funArray{\varCoinOut}}{\funArray{\varProof}}{\varSigContext}{\funArray{\varCommitment}}{\varSignature}{\varTime}$}{
        \pcreturn ( \pcskipln \\
        \varMsg \opAssign \varMsg, \pcskipln \\
        \varInputs \opAssign \funArray{\varCoinInp}, \pcskipln \\
        \varOutputs \opAssign \funArray{\varCoinOut}, \pcskipln \\
        \varProofs \opAssign \funArray{\varProof}, \pcskipln \\
        \varSigContext \opAssign \varSigContext, \pcskipln \\
        \varCommits \opAssign \funArray{\varCommitment}, \pcskipln \\
        \varSignature \opAssign \varSignature, \pcskipln \\
        \varTime \opAssign \varTime
        )
        }
    \end{varwidth}
    }
\end{center}

In~\cref{fig:inst-mw-tx-1} and~\cref{fig:inst-mw-tx-2}, we provide an instantiation of the Mimblewimble Transaction Scheme using the auxiliary functions provided before.

In the $\procSendCoinsId$ function the sender creates his change output coin, which is the difference between the value stored in his input coins and the value transferred to a receiver.
He sets up the signature context with his parameters and gets a pre-transaction $\varPreTx$, newly created spendable output coin $\varSpendableCoinAlice$, and a signing key $\varSecKeyAlice$ and secret nonce $\varNonceAlice$ as output.
The pre-transaction can then be sent to a receiver.
Note that, as we have already explained earlier, our instantiation differs from the one described by Fuchsbauer et al.~\cite{fuchsbauer2019aggregate} in that the sender does not yet sign the transaction during $\procSendCoinsId$, because we are using a Two-Party Signature Scheme~\cref{def:sig:two-party-sig} instead of an aggregatable signature scheme~\cref{def:atom:aggsig}.

In $\procRecvCoinsId$, the receiver of a pre-transaction will verify the senders proof $\varProofBob$, create his output coin $\varCoinOutBob$, add his parameters to the signature context and then create his partial signature $\varSigBob$.
The function returns an updated version of the pre-transaction $\varPreTx$ that the receiver can send back to the sender, and the newly created spendable output $\varSpendableCoinBob$.

Now in $\procFinTxId$, the original sender will validate the updated pre-transaction $\varPreTx$ sent to him by the receiver.
If he finds it as valid, he will only now create his partial signature and finally finalize the two partial signatures into the final composite one, with which he can then build the final transaction.

\begin{figure}
    \begin{center}
        \fbox{
        \begin{varwidth}{\textwidth}
            \procedure[linenumbering]{$\procSendCoins{\funArray{\varSpendableCoin}}{\varFundValue}{\varTime}$} {
            \varValue \opFunResult \sum_{\varI \opAssign 0}^{\varI \opSm \varN}(\varSpendableCoin_{i}.\varValue) \\
            \pcif \varFundValue \opGreaterThen \varValue
            \t \pcreturn \cnstFalsum \\
            \pcif \opExists \varI \opNotEq \varJ : \varSpendableCoin[\varI] \opEqNoQ \varSpendableCoin[\varJ]
            \t \pcreturn \cnstFalsum \\
            \varMsg \opAssign \cnstBinary{*} \\
            (\funStar{\varBlindingFactorAlice}, \varNonceAlice) \sample \cnstIntegersPrimeWithoutZero{\varPrime} \< \< \\
            \funStar{\varSpendableCoinAlice} \opFunResult \procCreateCoin{\varValue \opSub \varFundValue}{\funStar{\varBlindingFactorAlice}} \\
            \{ \varCoinOutAlice, \funStar{\varBlindingFactorAlice}, \varValueAlice, \varProofAlice \} \opFunResult \funStar{\varSpendableCoinAlice} \\
            \varSecKeyAlice \opAssign \funStar{\varBlindingFactorAlice} \opSub \sum_{\varI \opAssign 0}^{\varI \opSm \varN}(\varSpendableCoin_{i}.\varBlindingFactor) \\
            \varSigContext \opAssign \{ \varPubKey \opAssign \cnstIdentityElement, \varRand \opAssign \cnstIdentityElement \} \\
            \varSigContext \opFunResult \procSetupCtx{\varSigContext}{\funGen{\varSecKeyAlice}}{\funGen{\varNonceAlice}} \\
            \varPreTx \opFunResult \procCreatePreTx{\varMsg}{\varSpendableCoin.\varCommitment}{\funArray{\varCoinOutAlice}}{\funArray{\varProofAlice}}{\varSigContext}{\funArray{\funGen{\varSecKeyAlice}}}{\cnstEmptySet}{\varTime} \\
            \pcreturn (\varPreTx, \funStar{\varSpendableCoinAlice}, (\varSecKeyAlice, \varNonceAlice))
            } \par
            \procedure[linenumbering]{$\procRecvCoins{\varPreTx}{\varFundValue}$} {
            (\varMsg,\varInputs,\varOutputs,\varProofs,\varSigContext,\varCommits,\cnstEmptySet,\varTime) \opFunResult \varPreTx \\
            \pcif \procVerfProof{\varProofs[0]}{\varOutputs[0]} \opEqNoQ 0 \\
            \t \pcreturn \cnstFalsum \\
            (\funStar{\varBlindingFactorBob},\varNonceBob) \sample \cnstIntegersPrimeWithoutZero{\varPrime} \\
            \funStar{\varSpendableCoinBob} \opFunResult \procCreateCoin{\varFundValue}{\funStar{\varBlindingFactorBob}} \\
            \{ \varCoinOutBob, \funStar{\varBlindingFactorBob}, \varValueBob, \varProofBob \} \opFunResult \funStar{\varSpendableCoinBob} \\
            \varSecKeyBob \opAssign \funStar{\varBlindingFactorBob} \\
            \varSigContext \opFunResult \procSetupCtx{\varSigContext}{\funGen{\varSecKeyBob}}{\funGen{\varNonceBob}} \\
            \varSigBob \opFunResult \procSignPrt{\varMsg}{\varSecKeyBob}{\varNonceBob}{\varSigContext} \\
            \varPreTx \opFunResult \procCreatePreTx{\varMsg}{\varInputs}{\varOutputs \opConc \varCoinOutBob}{\varProofs \opConc \varProofBob}{\varSigContext}{\varCommits \opConc \funGen{\varSecKeyBob}}{\varSigBob}{\varTime} \\
            \pcreturn (\varPreTx, \funStar{\varSpendableCoinBob})
            }
        \end{varwidth}
        }
    \end{center}
    \caption{Instantiation of Mimblewimble Transaction Scheme part 1. \label{fig:inst-mw-tx-1}}
\end{figure}
\begin{figure}
    \begin{center}
        \fbox{
        \begin{varwidth}{\textwidth}
            \procedure[linenumbering]{$\procFinTx{\varPreTx}{\varSecKeyAlice}{\varNonceAlice}$} {
            (\varMsg,\varInputs,\varOutputs,\varProofs,\varSigContext,\varCommits,\varSigBob,\varTime) \opFunResult \varPreTx \\
            \pcif \procVerfProof{\varProofs[1]}{\varOutputs[1]} \opEqNoQ 0 \\
            \t \pcreturn \cnstFalsum \\
            \pcif \procVerfPtSig{\varSigBob}{\varMsg}{\varCommits[1]} \opEqNoQ 0 \\
            \t \pcreturn \cnstFalsum \\
            \varSigAlice \opFunResult \procSignPrt{\varMsg}{\varSecKeyAlice}{\varNonceAlice}{\varSigContext} \\
            \varSigFin \opFunResult \procFinSig{\varSigAlice}{\varSigBob} \\
            \varTx \opFunResult \procCreatePreTx{\varMsg}{\varInputs}{\varOutputs}{\varProofs}{\varSigContext}{\varCommits}{\varSigFin}{\varTime} \\
            \pcreturn \varTx
            } \par
            \procedure[linenumbering]{$\procVerfTx{\varTx}$} {
            (\varMsg,\varInputs,\varOutputs,\varProofs,\varSigContext,\varCommits,\varSignature,\varTime) \opFunResult \varTx \\
            \varExcess \opEqNoQ \sum(\varOutputs) \opSub \sum(\varInputs) \\
            \pcreturn (\opForAll \varI \opNotEq \varJ : \varInputs[\varI] \opNotEq \varInputs[\varJ] \opAnd \varOutputs[\varI] \opNotEq \varOutputs[\varJ]) \text{ and } \pcskipln \\
            \t \varInputs \opUnion \varOutputs \opEqNoQ \cnstEmptySet \text{ and } (\opForAll \varI : \procVerfProof{\varProofs[\varI]}{\varOutputs[\varI]}) \text{ and } \procVerf{\varMsg}{\varSignature}{\varExcess}
            }
        \end{varwidth}
        }
    \end{center}
    \caption{Instantiation of Mimblewimble Transaction Scheme part 2. \label{fig:inst-mw-tx-2}}
\end{figure}

\subsection{Extended Mimblewimble Transaction Scheme}\label{subsec:atom:ext-tx-scheme}

\Cref{fig:atom:dsendcoins} shows an instantiation of the $\procDSendCoinsId$ function of the Extended Mimblewimble Transaction Scheme.
We have an array of spendable input coins, which keys are shared between two parties Alice and Carol.
We use Carol here to not confuse this party with the receiver, which we previously called Bob.
Although Carol and Bob could be the same person, they do not necessarily have to be.

The protocol starts with both Alice and Carol creating her change outputs with values $\varValueAlice$ and $\varValueCarol$.
Alice then creates the initial pre-transaction $\varPreTx$ and sends it to Carol, who verifies Alice's output, adds her outputs and parameters, and sends back $\varPreTx$, which Alice verifies.
The protocol returns $\varPreTx$ to both parties, which can then be transmitted to the receiver by any of the two parties, as well as the secret signing information $(\varSecKeyAlice, \varNonceAlice)$, $(\varSecKeyCarol, \varNonceCarol)$.

\newgeometry{margin=2cm}
\begin{landscape}
    \thispagestyle{plain}
    \begin{figure}
        \begin{center}
            \fbox{
            \procedure[linenumbering,skipfirstln]{$\procDSendCoins{\funArray{\varPtSpendableCoinAlice}}{\funArray{\varPtSpendableCoinCarol}}{\varFundValue}{\varTime}$}{
            Alice \< \< Carol \\
            \varValue \opFunResult \sum_{\varI \opAssign 0}^{\varI \opSm \varN}(\varSpendableCoin_{i}.\varValue) \< \< \varValue \opFunResult \sum_{\varI \opAssign 0}^{\varI \opSm \varN}(\varSpendableCoin_{i}.\varValue) \\
            \pcif \varFundValue \opGreaterThen \varValue \opOr \opExists \varI \opNotEq \varJ : \varPtSpendableCoinAlice[\varI] \opEqNoQ \varPtSpendableCoinAlice[\varJ] \< \< \pcif \varFundValue \opGreaterThen \varValue \opOr \opExists \varI \opNotEq \varJ : \varPtSpendableCoinAlice[\varI] \opEqNoQ \varPtSpendableCoinAlice[\varJ] \\
            \t \pcreturn \cnstFalsum \< \< \t \pcreturn \cnstFalsum \\
            \varValue_{rem} \opEqNoQ \varValue \opSub \varFundValue \\
            \varValueAlice, \varValueCarol \opFunResult \{ 0, \varValue_{rem} \} \text{ s.t. } \varValueAlice \opAddScalar \varValueCarol \opEqNoQ \varValue_{rem}\\
            \< \sendmessageright*{\varValueCarol} \\
            \varMsg \opAssign \cnstBinary{*} \\
            (\funStar{\varBlindingFactorAlice}, \varNonceAlice) \sample \cnstIntegersPrimeWithoutZero{\varPrime} \< \< (\funStar{\varBlindingFactorCarol}, \varNonceCarol) \sample \cnstIntegersPrimeWithoutZero{\varPrime} \\
            \funStar{\varSpendableCoinAlice} \opFunResult \procCreateCoin{\varValueAlice}{\funStar{\varBlindingFactorAlice}} \< \< \funStar{\varSpendableCoinCarol} \opFunResult \procCreateCoin{\varValueCarol}{\funStar{\varBlindingFactorCarol}} \\
            \{ \varCoinOutAlice, \funStar{\varBlindingFactorAlice}, \varValueAlice, \varProofAlice \} \opFunResult \funStar{\varSpendableCoinAlice} \< \< \{ \varCoinOutCarol, \funStar{\varBlindingFactorCarol}, \varValueCarol, \varProofCarol \} \opFunResult \funStar{\varSpendableCoinCarol} \\
            \varSecKeyAlice \opAssign \funStar{\varBlindingFactorAlice} \opSub \sum \funArray{\varBlindingFactorAlice} \< \< \varSecKeyCarol \opAssign \funStar{\varBlindingFactorCarol} \opSub \sum \funArray{\varBlindingFactorCarol} \\
            \varSigContext \opAssign \{ \varPubKey \opAssign \cnstIdentityElement, \varRand \opAssign \cnstIdentityElement \} \< \< \\
            \varSigContext \opFunResult \procSetupCtx{\varSigContext}{\funGen{\varSecKeyAlice}}{\funGen{\varNonceAlice}} \< \< \\
            \varPreTx \opFunResult \pcskipln \\
            \procCreatePreTx{\varMsg}{\funArray{\varCoinInp}}{\funArray{\varCoinOutAlice}}{\funArray{\varProofAlice}}{\varSigContext}{\funArray{\funGen{\varNonceAlice}}}{\cnstEmptySet}{\varTime} \< \< \\
            \< \sendmessageright*{\varPreTx} \< \\
            \< \< (\varMsg,\varInputs,\varOutputs,\varProofs,\varSigContext,\varCommits,\varTime) \opFunResult \varPreTx \\
            \< \< \pcif \procVerfProof{\varProofs[0]}{\varOutputs[0]} \opEqNoQ 0 \\
            \< \< \t \pcreturn \cnstFalsum \\
            \< \< \varSigContext \opFunResult \procSetupCtx{\varSigContext}{\funGen{\varSecKeyCarol}}{\funGen{\varNonceCarol}} \\
            \< \< \funStarAlt{\varPreTx} \opFunResult \procCreatePreTx{\varMsg}{\varInputs}{\varOutputs \opConc \varCoinOutCarol}{\varProof \opConc \varProofCarol}{\varSigContext}{\varCommits \opConc \funGen{\varNonceCarol}}{\cnstEmptySet}{\varTime} \\
            \< \sendmessageleft*{\funStarAlt{\varPreTx}} \< \\
            \pcif \procVerfProof{\funStarAlt{\varPreTx}.\varProofs[1]}{\funStarAlt{\varPreTx}.\varOutputs[1]} \opEqNoQ 0 \< \< \\
            \t \pcreturn \cnstFalsum \< \< \\
            \pcreturn (\funStarAlt{\varPreTx}, \funStar{\varSpendableCoinAlice}, (\varSecKeyAlice, \varNonceAlice)) \< \< \pcreturn (\funStarAlt{\varPreTx}, \funStar{\varSpendableCoinCarol}, (\varSecKeyCarol, \varNonceCarol))
            }
            }
        \end{center}
        \caption{Extended Mimblewimble Transaction Scheme - $\procDSendCoinsId$ \label{fig:atom:dsendcoins}}
    \end{figure}
\end{landscape}
\restoregeometry

\Cref{fig:atom:drecv} shows an instantiation of the $\procDRecvCoinsId$ function of the Extended Mimblewimble Transaction Scheme.
Calling this protocol, two receivers, Bob and Carol, want to create a receiving shared coin $\varCoinShared$ with value $\varFundValue$ and key shares $(\varBlindingFactorAlice, \varBlindingFactorCarol)$.
The protocol starts by both receivers verifying the sender's output(s).
Bob begins by creating a coin with fund value $\varFundValue$ and his share of the newly created blinding factor and sends it over to Carol.
Carol finalizes the shared coin by adding a Commitment to her blinding factor to the coin and sends it back, together with the commitment.
Bob verifies the updated shared coin's validity, after which the two parties engage in two two-party protocols to create their partial signature and coin range proof.
Finally, they make the updated pre-transaction $\varPreTx$ which can be sent back to the transaction sender.

\newgeometry{margin=2cm}
\begin{landscape}
    \thispagestyle{plain}
    \begin{figure}
        \begin{center}
            \fbox{
            \procedure[linenumbering,skipfirstln]{$\procDRecvCoins{\varPreTx}{\varFundValue}$} {
            Bob \< \< \< \< Carol \\
            (\varMsg,\varInputs,\varOutputs,\varProofs,\varSigContext,\varCommits,\cnstEmptySet,\varTime) \opFunResult \varPreTx \\
            \pcforeach \varOutputs \textit{ as } (\varIterator => \varCoinOut) \< \< \< \< \pcforeach \varOutputs \textit{ as } (\varIterator => \varCoinOut) \\
            \t \pcif \procVerfProof{\varProofs[\varIterator]}{\varCoinOut[\varIterator]} \opEqNoQ 0 \< \< \< \< \t \pcif \procVerfProof{\varProofs[\varIterator]}{\varCoinOut[\varIterator]} \opEqNoQ 0 \\
            \t \pcreturn \cnstFalsum \< \< \< \< \t \pcreturn \cnstFalsum \\
            (\funStar{\varBlindingFactorBob}, \varNonceBob) \sample \cnstIntegersPrimeWithoutZero{\varPrime} \< \< \< \< \\
            (\varCoinShared, \cdot, \cdot, \cdot) \opFunResult \procCreateCoin{\varFundValue}{\funStar{\varBlindingFactorBob}} \< \< \< \< \\
            \varSecKeyBob \opAssign \funStar{\varBlindingFactorBob} \< \< \< \< \\
            \< \sendmessagerightx{4}{\varPreTx, \varCoinShared} \< \\
            \< \< \< \< (\funStar{\varBlindingFactorCarol}, \varNonceCarol) \sample \cnstIntegersPrimeWithoutZero{\varPrime} \\
            \< \< \< \< \varSecKeyCarol \opAssign \funStar{\varBlindingFactorCarol} \\
            \< \< \< \< \funStarAlt{\varCoinShared} \opAssign \varCoinShared \opAddPoint \funGen{\varBlindingFactorCarol} \\
            \< \< \< \< \funStarAlt{\varPreTx} \opAssign \varPreTx \\
            \< \< \< \< \funStarAlt{\varPreTx}.\varOutputs[] \opAssign \funStarAlt{\varCoinShared} \\
            \< \sendmessageleftx{4}{\funStarAlt{\varPreTx}, \funGen{\varSecKeyCarol}} \< \\
            \{\cdots \funStarAlt{\varCoinShared}\} \opFunResult \funStarAlt{\varPreTx}.\varOutputs \< \< \< \< \\
            \pcif \funStarAlt{\varCoinShared} \opNotEq \varCoinShared \opAddPoint \funGen{\varSecKeyCarol} \< \< \< \< \\
            \t \pcreturn \cnstFalsum \< \< \< \< \\
            \varProofBobCarol \opFunResult \procDRProofL{\funStarAlt{\varCoinShared}}{\varFundValue}{\varSecKeyBob} \< \< \< \< \varProofBobCarol \opFunResult \procDRProofL{\funStarAlt{\varCoinShared}}{\varFundValue}{\varSecKeyCarol} \\
            \funStar{\varPtSpendableCoinBob} \opAssign \{ \varCoinShared, \varFundValue, \funStar{\varBlindingFactorBob}, \varProofBobCarol \} \< \< \< \< \funStar{\varPtSpendableCoinCarol} \opAssign \{ \varCoinShared, \varFundValue, \funStar{\varBlindingFactorCarol}, \varProofBobCarol \} \\
            (\varSigBobCarol, \varPubKeyBobCarol) \opFunResult \procDSignL{\varMsg}{\varSecKeyBob}{\varNonceBob} \< \< \< \< (\varSigBobCarol, \varPubKeyBobCarol) \opFunResult \procDSignR{\varMsg}{\varSecKeyCarol}{\varNonceCarol} \\
            (\cdot, \cdot, \funStar{\varSigContext}) \opFunResult \varSigBobCarol \< \< \< \< (\cdot, \cdot, \funStar{\varSigContext}) \opFunResult \varSigBobCarol \\
            \funStarAlt{\varSigContext} \opFunResult \procSetupCtx{\varSigContext}{\funStar{\varSigContext}.\varPubKey}{\funStar{\varSigContext}.\varRand} \< \< \< \< \funStarAlt{\varSigContext} \opFunResult \procSetupCtx{\varSigContext}{\funStar{\varSigContext}.\varPubKey}{\funStar{\varSigContext}.\varRand} \\
            \funStar{\varPreTx} \opFunResult \< \< \< \< \funStar{\varPreTx} \opFunResult \\
            \procCreatePreTx{\varMsg}{\varInputs}{\varOutputs \opConc \funStarAlt{\varCoinShared}}{\varProofs \opConc \varProofBobCarol}{\funStarAlt{\varSigContext}}{\varCommits \opConc \varPubKeyBobCarol}{\varSigBobCarol}{\varTime} \< \< \< \<  \procCreatePreTx{\varMsg}{\varInputs}{\varOutputs \opConc \funStarAlt{\varCoinShared}}{\varProofs \opConc \varProofBobCarol}{\funStarAlt{\varSigContext}}{\varCommits \opConc \varPubKeyBobCarol}{\varSigBobCarol}{\varTime} \\
            \pcreturn (\funStar{\varPreTx}, \funStar{\varPtSpendableCoinBob}) \< \< \< \< \pcreturn (\funStar{\varPreTx}, \funStar{\varPtSpendableCoinCarol})
            }
            }
        \end{center}
        \caption{Extended Mimblewimble Transaction Scheme - $\procDRecvCoinsId$ \label{fig:atom:drecv}}
    \end{figure}
\end{landscape}
\restoregeometry

Finally, \cref{fig:ext-mim-tx-fin} shows the implementation of the $\procDFinTxId$ protocol.
Running this protocol, the two transaction senders, each owning a share of the input coins keys, will cooperate to produce a signature share valid under their input coins and change outputs.
They can combine the partial signatures into the final one and finalize the transaction.

\begin{figure}
    \begin{center}
        \fbox{
        \procedure[linenumbering,skipfirstln]{$\procDFinTx{\varPreTx}{\varSecKeyAlice}{\varNonceAlice}{\varSecKeyCarol}{\varNonceCarol}$} {
        Alice \< \< Carol \\
        (\varMsg,\varInputs,\varOutputs,\varProofs,\varSigContext,\varCommits,\varSigBob,\varTime) \opFunResult \varPreTx \< \< (\varMsg,\varInputs,\varOutputs,\varProofs,\varSigContext,\varCommits,\varSigBob,\varTime) \opFunResult \varPreTx \\
        \pcif \procVerfProof{\varProofs[1]}{\varOutputs[1]} \opEqNoQ 0 \< \< \pcif \procVerfProof{\varProofs[1]}{\varOutputs[1]} \opEqNoQ 0 \\
        \t \pcreturn \cnstFalsum \< \< \t \pcreturn \cnstFalsum \\
        \pcif \procVerfPtSig{\varSigBob}{\varMsg}{\varCommits[1]} \opEqNoQ 0 \< \< \pcif \procVerfPtSig{\varSigBob}{\varMsg}{\varCommits[1]} \opEqNoQ 0 \\
        \t \pcreturn \cnstFalsum \< \< \t \pcreturn \cnstFalsum \\
        \varSigAliceCarol \opFunResult \procDSignL{\varMsg}{\varSecKeyAlice}{\varNonceAlice} \< \< \varSigAliceCarol \opFunResult \procDSignR{\varMsg}{\varSecKeyCarol}{\varNonceCarol} \\
        \varSigFin \opFunResult \procFinSig{\varSigBob}{\varSigAliceCarol} \< \< \varSigFin \opFunResult \procFinSig{\varSigBob}{\varSigAliceCarol} \\
        \varTx \opFunResult \procCreatePreTx{\varMsg}{\varInputs}{\varOutputs}{\varProofs}{\varSigContext}{\varCommits}{\varSigFin}{\varTime} \< \< \varTx \opFunResult \procCreatePreTx{\varMsg}{\varInputs}{\varOutputs}{\varProofs}{\varSigContext}{\varCommits}{\varSigFin}{\varTime} \\
        \pcreturn \varTx \< \< \pcreturn \varTx
        }
        }
    \end{center}
    \caption{Extended Mimblewimble Transaction Scheme - $\procDFinTxId$ \label{fig:ext-mim-tx-fin}}
\end{figure}

\subsection{Contract Mimblewimble Transaction Scheme}

\Cref{fig:inst-apt-mw-tx-recv} shows an instantiation of the $\procAptRecvCoinsId$ algorithm.
Before updating the pre-transaction $\varPreTx$, Bob masks his partial signature with the witness value $\varWit$.
The procedure then returns the pre-transaction $\varPreTx$ containing Bob's masked partial signature and the statement $\varStatement$, a Commitment to the witness value $\varWit$.

\begin{figure}
    \begin{center}
        \fbox{
        \begin{varwidth}{\textwidth}
            \procedure[linenumbering]{$\procAptRecvCoins{\varPreTx}{\varFundValue}{\varWit}$} {
            (\varMsg,\varInputs,\varOutputs,\varProofs,\varSigContext,\varCommits,\cnstEmptySet, \varTime) \opFunResult \varPreTx \\
            \pcif \procVerfProof{\varProofs[0]}{\varOutputs[0]} \opEqNoQ 0 \\
            \t \pcreturn \cnstFalsum \\
            (\funStar{\varBlindingFactorBob},\varNonceBob) \sample \cnstIntegersPrimeWithoutZero{\varPrime} \\
            (\varCoinOutBob,\varProofBob) \opFunResult \procCreateCoin{\varFundValue}{\funStar{\varBlindingFactorBob}} \\
            \varSecKeyBob \opAssign \funStar{\varBlindingFactorBob} \\
            \varSigContext \opFunResult \procSetupCtx{\varSigContext}{\funGen{\varSecKeyBob}}{\funGen{\varNonceBob}} \\
            \varSigBob \opFunResult \procSignPrt{\varMsg}{\varSecKeyBob}{\varSigContext.\varPubKey}{\varSigContext.\varRand} \\
            \varSigAptBob \opFunResult \procAptSig{\varSigBob}{\varWit} \\
            \varPreTx \opFunResult \procCreatePreTx{\varMsg}{\varInputs}{\varOutputs \opConc \varCoinOutBob}{\varProofs \opConc \varProofBob}{\varSigContext}{\varCommits \opConc \funGen{\varNonceBob}}{\varSigAptBob}{\varTime} \\
            \pcreturn (\varPreTx, (\varCoinOutBob, \funStar{\varBlindingFactorBob}),\varSigBob)
            }
        \end{varwidth}
        }
    \end{center}
    \caption{Contract Mimblewimble Transaction Scheme - $\procAptRecvCoinsId$. \label{fig:inst-apt-mw-tx-recv}}
\end{figure}

In~\cref{fig:inst-apt-mw-tx-fin}, we show the updated distributed version of the transaction finalization protocol.
Again Alice verifies the pre-transaction $\varPreTx$ received by Bob and then cooperates with Bob in the $\procDSignId$ protocol to build the partial signature for their shared coin.
Note that Alice cannot finalize the signature (and consequently the transaction) as she only knows Bob's masked signature share ($\varSigAptBob$), but not the original one ($\varSigBob$), which is needed for the $\procFinSigId$ function.
Therefore, Bob completes the transaction and outputs it, while Alice outputs $\varSigAliceBob$ to retrieve $\varX$.

\begin{figure}
    \begin{center}
        \fbox{
        \procedure[linenumbering,skipfirstln]{$\procDAptFinTx{\varPreTx}{\varSecKeyAlice}{\varNonceAlice}{\varStatement}{\varSecKeyBob}{\varNonceBob}{\varSigBob}$} {
        Alice \< \< Bob \\
        (\varMsg,\varInputs,\varOutputs,\varProofs,\varSigContext,\varCommits,\varSigAptBob,\varTime) \opFunResult \varPreTx \< \< (\varMsg,\varInputs,\varOutputs,\varProofs,\varSigContext,\varCommits,\varSigAptBob,\varTime) \opFunResult \varPreTx \\
        \pcif \procVerfProof{\varProofs[1]}{\varOutputs[1]} \opEqNoQ 0 \< \< \\
        \t \pcreturn \cnstFalsum \< \< \\
        \pcif \procVerifyAptSig{\varSigBob}{\varMsg}{\varCommits[1]}{\varStatement} \opEqNoQ 0 \< \< \\
        \t \pcreturn \cnstFalsum \< \< \\
        \varSigAliceBob \opFunResult \procDSignL{\varMsg}{\varSecKeyAlice}{\varNonceAlice} \< \< \varSigAliceBob \opFunResult \procDSignL{\varMsg}{\varSecKeyBob}{\varNonceBob} \\
        \< \< \varSigFin \opFunResult \procFinSig{\varSigAliceCarol}{\varSigBob} \\
        \< \< \varTx \opFunResult \procCreatePreTx{\varMsg}{\varInputs}{\varOutputs}{\varProofs}{\varSigContext}{\varCommits}{\varSigFin}{\varTime} \\
        \pcreturn \varSigAliceBob \< \< \pcreturn \varTx
        }
        }
    \end{center}
    \caption{Adapted Extended Mimblewimble Transaction Scheme - $\procDAptFinTxId$. \label{fig:inst-apt-mw-tx-fin}}
\end{figure}

\section{Protocols}\label{sec:atom:protocols}
In this section we specify three protocols to build Mimblewimble transactions from the definitions found in~\ref{sec:atom:definitions}.
Later in section~\ref{sec:atom:security} we will prove the security of those protocols and finally in section~\ref{sec:atom:atomic-swap} we will use those protocols to build our Atomic Swap.

\subsection{Simple Mimblewimble Transaction - $\procDBuildMwTxId$} \label{subsec:atom:simple-mw-tx}

$\procDBuildMwTxId$ is a protocol between a sender and receiver which builds a mimblewimble transaction transferring a value $\varFundValue$ from the sender to a receiver for a Mimblewimble Transaction scheme as defined in~\ref{def:atom:mw-tx-scheme}
It takes as input a list of spendable coins $\funArray{\varSpendableCoin}$, a transaction value $\varFundValue$, and an optional timelock $\varTime$ from the sender, the same transaction value $\varFundValue$ from the receiver and uses the functions defined earlier to output a valid transaction $\varTx$ as well as the newly spendable coins to both parties.
\[ \langle (\varTx, \funStar{\varSpendableCoinAlice}), (\varTx, \funStar{\varSpendableCoinBob}) \rangle \opFunResult \procDBuildMwTx{\funStar{\varSpendableCoin}}{\varFundValue}{\varTime} \]
Figure~\ref{fig:d-build-mw-tx} show the implementation of the $\procDBuildMwTxId$.

\begin{figure}
    \fbox{
    \begin{varwidth}{\textwidth}
        \procedure[linenumbering,skipfirstln]{$\procDBuildMwTx{\funArray{\varSpendableCoin}}{\varFundValue}{\varTime}$}{
        Alice \< \< Bob \\
        (\varPreTx, \funStar{\varSpendableCoinAlice}, (\varSecKeyAlice, \varNonceAlice)) \pcskipln \\
        \opFunResult \procSendCoins{\funArray{\varSpendableCoin}}{\varFundValue}{\varTime} \\
        \< \sendmessageright*{\varPreTx} \< \\
        \< \< (\funStarAlt{\varPreTx}, \funStar{\varSpendableCoinBob}) \opFunResult \procRecvCoins{\varPreTx}{\varFundValue} \\
        \< \sendmessageleft*{\funStarAlt{\varPreTx}} \\
        \varTx \opFunResult \procFinTx{\funStarAlt{\varPreTx}}{\varSecKeyAlice}{\varNonceAlice} \\
        \< \sendmessageright*{\varTx} \\
        \pcreturn (\varTx, \funStar{\varSpendableCoinAlice}) \< \< \pcreturn (\varTx, \funStar{\varSpendableCoinBob})
        }
    \end{varwidth}
    }
    \caption{$\procDBuildMwTxId$ two-party protocol to build a new transaction} \label{fig:d-build-mw-tx}
\end{figure}

\subsection{Shared Output Mimblewimble Transaction - $\procDSharedOutputMwTxId$} \label{subsec:atom:shared-out-mw-tx}

$\procDSharedOutputMwTxId$ is a protocol between a sender and a receiver.
It builds a mimblewimble transaction transferring value from a sender for the Extendend Mimblewimble Transaction Scheme in ~\ref{def:atom:ext-mw-tx-scheme}.
However, instead of simply sending value to a receiver it sends it to a shared coin, for which both the sender and receiver know one part of the opening.
As input it again takes a list of spendable coins $\funArray{\varSpendableCoin}$, a transaction value $\varFundValue$ and an optional timelock $\varTime$ from the sender and the same transaction value $\varFundValue$ from the receiver.
It outputs the final transaction $\varTx$ to both parties, Alice will receiver her spendable change output $\funStar{\varSpendableCoinAlice}$ and both parties will receive their part of the shared spendable coin $\funStar{\varPtSpendableCoinAlice}$, $\funStar{\varPtSpendableCoinBob}$.

\[ \langle (\varTx, \funStar{\varSpendableCoinAlice}, \funStar{\varPtSpendableCoinAlice}), (\varTx, \funStar{\varPtSpendableCoinBob}) \rangle \opFunResult \procDSharedOutputMwTx{\funArray{\varSpendableCoin}}{\varFundValue}{\varTime} \]

One use case of this transaction protocol is to lock funds between two users, which can then be redeemed by both parties cooperating.

Figure~\ref{fig:d-shared-out-mw-tx} shows the implementation of the protocol.

\begin{figure}
    \fbox{
    \begin{varwidth}{\textwidth}
        \procedure[linenumbering,skipfirstln]{$\procDSharedOutputMwTx{\funArray{\varSpendableCoin}}{\varFundValue}{\varTime}$}{
        Alice \< \< Bob \\
        (\varPreTx, \funStar{\varSpendableCoinAlice}, (\varSecKeyAlice, \varNonceAlice)) \pcskipln \\
        \opFunResult \procSendCoins{\funArray{\varSpendableCoin}}{\varFundValue}{\varTime} \\
        \< \sendmessageright*{\varPreTx} \< \\
        (\funStarAlt{\varPreTx}, \funStar{\varPtSpendableCoinAlice}) \< \< (\funStarAlt{\varPreTx}, \funStar{\varPtSpendableCoinBob}) \pcskipln \\
        \opFunResult \procDRecvCoinsL{\varPreTx}{\varFundValue}  \< \< \opFunResult \procDRecvCoinsR{\varFundValue} \\
        \varTx \opFunResult \procFinTx{\funStarAlt{\varPreTx}}{\varSecKeyAlice}{\varNonceAlice} \\
        \< \sendmessageright*{\varTx} \\
        \pcreturn (\varTx, \funStar{\varSpendableCoinAlice}, \funStar{\varPtSpendableCoinAlice}) \< \< \pcreturn (\funStar{\varPtSpendableCoinBob})
        }
    \end{varwidth}
    }
    \caption{$\procDSharedOutputMwTxId$ two-party protocol to build a new transaction with a shared output} \label{fig:d-shared-out-mw-tx}
\end{figure}

\subsection{Shared Input Mimblewimble Transaction $\procDSharedInpMwTxId$} \label{subsec:atom:shared-inp-mw-tx}

$\procDSharedInpMwTxId$ is a protocol between a sender and a receiver.
It builds a mimblewimble transaction transferring value from a coin shared between the sender and receiver to a receiver again for the Extended Mimblewimble Transaction Scheme outlined in~\ref{def:atom:ext-mw-tx-scheme}
As input it takes a list of partial spendable coins $\funArray{\varPtSpendableCoinAlice}$, a transaction value $\varFundValue$ and an optional timelock $\varTime$ from the sender the other part of the shared spendable coins $\varPtSpendableCoinBob$ as well as the same transaction value $\varFundValue$ from the receiver.
It outputs a final transaction $\varTx$ to both parties, as well as the new outputs $\funStar{\varSpendableCoinAlice}, \funStar{\varSpendableCoinBob}$ to the respective owner.

\[ \langle (\varTx, \funStar{\varSpendableCoinAlice}), (\varTx, \funStar{\varSpendableCoinBob}) \rangle \opFunResult \procDSharedInpMwTx{\funArray{\varPtSpendableCoinAlice}}{\varFundValue}{\varTime}{\funArray{\varPtSpendableCoinBob}} \]

The protocol can be used to redeem funds which are locked created with the $\procDSharedInpMwTxId$ protocol.

Figure~\ref{fig:d-shared-inp-mw-tx} shows the implementation of the protocol.

\begin{figure}
    \fbox{
    \begin{varwidth}{\textwidth}
        \procedure[linenumbering,skipfirstln]{$\procDSharedInpMwTx{\funArray{\varPtSpendableCoinAlice}}{\varFundValue}{\varTime}{\funArray{\varPtSpendableCoinBob}}$}{
        Alice \< \< Bob \\
        (\varPreTx, \funStar{\varSpendableCoinAlice}, (\varSecKeyAlice, \varNonceAlice)) \< \< (\varPreTx, (\varSecKeyBob, \varNonceBob)) \pcskipln \\
        \opFunResult \procDSendCoinsL{\funArray{\varPtSpendableCoinAlice}}{\varFundValue}{\varTime} \< \< \opFunResult \procDSendCoinsL{\funArray{\varPtSpendableCoinBob}}{\varFundValue}{\varTime} \\
        \< \< (\funStarAlt{\varPreTx}, \funStar{\varSpendableCoinBob}) \opFunResult \procRecvCoins{\varPreTx}{\varFundValue} \\
        \< \sendmessageleft*{\funStarAlt{\varPreTx}} \\
        \varTx \opFunResult \procDFinTxL{\funStarAlt{\varPreTx}}{\varSecKeyAlice}{\varNonceAlice} \< \< \varTx \opFunResult \procDFinTxL{\funStarAlt{\varPreTx}}{\varSecKeyBob}{\varNonceBob} \\
        \pcreturn (\varTx, \funStar{\varSpendableCoinAlice}) \< \< \pcreturn (\varTx, \funStar{\varSpendableCoinBob})
        }
    \end{varwidth}
    }
    \caption{$\procDSharedOutputMwTxId$ two-party protocol to build a new transaction from a shared output} \label{fig:d-shared-inp-mw-tx}
\end{figure}

\subsection{Contract Mimblewimble Transaction - $\procDScriptMwTxId$} \label{subsec:atom:script-mw-tx}

$\procDScriptMwTxId$ is a protocol between a sender and a receiver for the Script Mimblewimble Transaction Scheme defined in~\ref{def:atom:apt-ext-mw-tx-scheme}.
Similar to the $\procDSharedInpMwTxId$ it spends an input coin which is shared between the sender and receiver.
Additionally, we utilize the adapted signature protocol from~\ref{def:sig:two-party-fixed-wit-apt-sig} to let the receiver hide a secret witness value $\varWit$ in the transaction signature which the sender can extract from the final transaction, thereby allowing the construction of primitive contracts.

\[ \langle (\varTx, \funStar{\varSpendableCoinAlice}, \varWit), (\varTx, \funStar{\varSpendableCoinBob}) \rangle \opFunResult \procDScriptMwTx{\funArray{\varPtSpendableCoinAlice}}{\varFundValue}{\varTime}{\varStatement}{\funArray{\varPtSpendableCoinBob}}{\varWit} \]

Figure~\ref{fig:d-script-tx} shows the implementation of the protocol.

\begin{figure}
    \fbox{
    \begin{varwidth}{\textwidth}
        \procedure[linenumbering,skipfirstln]{$\procDScriptMwTx{\funArray{\varPtSpendableCoinAlice}}{\varFundValue}{\varTime}{\varStatement}{\funArray{\varPtSpendableCoinBob}}{\varWit}$}{
        Alice \< \< Bob \\
        (\varPreTx, \funStar{\varSpendableCoinAlice}, (\varSecKeyAlice, \varNonceAlice)) \< \< (\varPreTx, (\varSecKeyBob, \varNonceBob)) \pcskipln \\
        \opFunResult \procDSendCoinsL{\funArray{\varPtSpendableCoinAlice}}{\varFundValue}{\varTime} \< \< \opFunResult \procDSendCoinsL{\funArray{\varPtSpendableCoinBob}}{\varFundValue}{\varTime} \\
        \< \< (\funStarAlt{\varPreTx}, \funStar{\varSpendableCoinBob}, \varSigBob)  \pcskipln \\
        \< \< \opFunResult \procAptRecvCoins{\varPreTx}{\varFundValue}{\varWit}  \\
        \< \sendmessageleft*{\funStarAlt{\varPreTx}, \funStarAlt{\varStatement}} \\
        \pcif \varStatement \opNotEq \cnstFalsum \opAnd \varStatement \opNotEq \funStarAlt{\varStatement} \\
        \t \pcreturn \cnstFalsum \\
        \varSigAptBob \opFunResult \funStarAlt{\varPreTx}.\varSignature \\
        \varSigAliceBob \< \< \varTx \pcskipln \\
        \opFunResult \procDAptFinTxL{\funStarAlt{\varPreTx}}{\varSecKeyAlice}{\varNonceAlice}{\varStatement} \< \< \opFunResult \procDAptFinTxR{\funStarAlt{\varPreTx}}{\varSecKeyBob}{\varNonceBob}{\varSigBob} \\
        \< \sendmessageleft*{\varTx} \\
        \varWit \opFunResult \procExtWit{\varTx.\varSignature}{\varSigAliceBob}{\varSigAptBob} \\
        \pcreturn (\varTx, \funStar{\varSpendableCoinAlice}, \varWit) \< \< \pcreturn (\varTx, \funStar{\varSpendableCoinBob})
        }
    \end{varwidth}
    }
    \caption{$\procDScriptMwTxId$ two-party protocol to build a primitive contract transaction} \label{fig:d-script-tx}
\end{figure}









\section{Correctness \& Security} \label{sec:atom:security}
In this section we will prove the correctness and security of the instantiation described in~\ref{sec:atom:atomic-inst}.
We start by proving \emph{Transaction Scheme Correctness}, \emph{Extended Transaction Scheme Correctness} and \emph{Adapted Transaction Scheme Correctness} for the three outlined transaction schemes $\varMWScheme, \varextMWScheme$ and $\varaptMWScheme$.
We then continue by showing that all three security definitions (\emph{Inflation-Resistence}, \emph{Theft-Resistence}, \emph{Transaction indistinguishability}) hold again for all three transaction types.
The general process will be to find a proof for the basic transaction protocol defined in~\ref{def:atom:mw-tx-scheme} and then reduce the extended and adapted case to the basic one.

\subsection{Correctness} \label{subsec:atom:correctness}

We will argue \emph{Transaction Scheme Correctness} follows from the correctness of the commitment scheme $\varCommitScheme$, two-party signature scheme $\varSigScheme$ as well as the correctness of the range proof system $\varProofSystem$ used in the transaction protocol.
If the transaction was constructed correctly (that is by calling the procedures $\procSendCoinsId, \procRecvCoinsId, \procFinTxId$, the distributed variants $\procDSendCoinsId, \procDRecvCoinsId, \procDFinTxId$ or the adapted ones $\procAptRecvCoinsId, \procDAptFinTxId$ with valid inputs) it must follow that the final transaction has correct commitments, rangeproofs and a valid signature and $\procVerfTxId$ will therefore return 1.
We construct the following theorem:

\begin{theorem}\label{lem:atom:correctness}
    \emph{Transaction Scheme Correctness}, \emph{Extended Transaction Scheme Correctness} or \emph{Adapted Transaction Scheme Correctness} for a transaction system $\varMWSchemeParams{\varCommitScheme}{\varSigScheme}{\varProofSystem}$, $\varextMWSchemeParams{\varCommitScheme}{\varSigScheme}{\varProofSystem}$ or $\varaptMWSchemeParams{\varCommitScheme}{\varSigScheme}{\varProofSystem}$ holds if and only if the underlying Commitment Scheme $\varCommitScheme,$ Two-Party Signature Scheme $\varSigSchemeMP$ and Rangeproof system $\varProofSystem$ are correct.
\end{theorem}

\begin{proof}
    We assume there are two honest participants Alice and Bob, there exists a list of input coins $\funArray{\varCoinInp}$ with blinding factors $\funArray{\varBlindingFactorAlice}$ and total value $\varValue$ known to Alice, and some amount $\varFundValue$ which Alice wants to transfer to Bob.
    For \emph{Transaction Scheme Correctness} to hold $\procVerfTx{\varTx}$ must return 1 with overwhealming probability for the two parties creating a transaction $\varTx$ in the following three steps:
    \begin{enumerate}
        \item $(\varPreTx, (\varSecKeyAlice, \varNonceAlice)) \opFunResult \procSendCoins{\funArray{\varCoinInp}}{\funArray{\varBlindingFactorAlice}}{\varFundValue}{\varValue}{\cnstFalsum}$
        \item $\funStar{\varPreTx} \opFunResult \procRecvCoins{\varPreTx}{\varFundValue}$
        \item $\varTx \opFunResult \procFinTx{\funStar{\varPreTx}}{\varSecKeyAlice}{\varNonceAlice}$
    \end{enumerate}
    Here are the following three requirements for $\procVerfTx{\varTx}$ to return 1:
    \begin{enumerate}
        \item The list of input coins $\varInputs$, as well as output coins $\varOutputs$ must each inlcude at least one element.
        \item For each output coin $\varCoinOut$ the rangeproof $\varProof$ must verify.
        \item The transaction signature under the public key $\varPubKey \opAssign \cnstSumZeroToN \varOutputs \opSub \cnstSumZeroToN \varInputs$ must verify.
    \end{enumerate}

    Condition 1 must hold trivially, as already in $\procSendCoinsId$ Alice will create the output $\varCoinOutAlice$ and initialize the pre-transaction with $\funArray{\varCoinInp}$ and $\funArray{\varCoinOutAlice}$, thereby already after $\procSendCoinsId$ both input and output lists are non empty and this condition is fulfilled.

    Condition 2 follows from the implementation of the $\procCreateCoinId$ function called in $\procSendCoinsId$ as well as $\procRecvCoinsId$.
    In the function a rangeproof is computed for the new coin $\varCoin$ with value $\varValue$ and blinding factor $\varBlindingFactor$ as $\varProof \opFunResult \procProof{\varCoin}{\varValue}{\varBlindingFactor}$.
    Given that our Rangeproof system $\varProof$ system has to be correct $\procVerfProof{\varProof}{\varCoin} \opEqNoQ 1$ must hold for all coins created with the $\procCreateCoinId$ routine.
    Therefore Condition 2 must hold.

    For condition 3 we must look at how the secret keys $\varSecKeyAlice$ and $\varSecKeyBob$ are constructed.
    From the instantiation of $\procSendCoinsId$ we can see that Alice's share will be $\varSecKeyAlice \opAssign \funStar{\varBlindingFactorAlice} \opSub \cnstSumZeroToN \funArray{\varBlindingFactorAlice}$, where $\funStar{\varBlindingFactorAlice}$ is the blinding factor to her output and $\funArray{\varBlindingFactorAlice}$ are the blinding factors to her input coins.
    Bobs secret key is constructed like $\varSecKeyBob \opAssign \funStar{\varBlindingFactorBob}$, so it corresponds to the blinding factor of his output.
    From the construction of the two-party signature scheme in~\ref{def:sig:two-party-sig} we know that therefore the final signature will be valid under the following public key:
    \[ \funStar{\varPubKey} \opAssign \funGen{\varSecKeyAlice} \opAddPoint \funGen{\varSecKeyBob} \]
    Given how the secret keys are constructed we arrive at:
    \[ \funStar{\varPubKey} \opAssign \funGen{\funStar{\varBlindingFactorAlice}} \opAddPoint \cnstSumZeroToN \funArray{\funGen{- \varBlindingFactorAlice}} \opAddPoint \funGen{\varBlindingFactorBob} \]
    If we can show that the public key $\varPubKey$ computed in $\procVerfTxId$ is the same as above, $\procVerf{\varMsg}{\varSignature}{\varPubKey} \opEqNoQ 1$ must hold and therefore condition 3 would be proven.
    We show this by a stepwise conversion of the initial equation computing $\varPubKey$ until we arrive at the equation for $\funStar{\varPubKey}$:
    \begin{gather}
        \varPubKey \opEqNoQ \funStar{\varPubKey} \\
        \cnstSumZeroToN \varOutputs \opSub \cnstSumZeroToN \varInputs \opEqNoQ \funGen{\funStar{\varBlindingFactorAlice}} \opAddPoint \cnstSumZeroToN \funArray{\funGen{- \varBlindingFactorAlice}} \opAddPoint \funGen{\varBlindingFactorBob} \\
        \varCoinOutAlice \opAddPoint \varCoinOutBob \opAddPoint \cnstSumZeroToN \funArray{(\varCoinInp)^{-1}}  \opEqNoQ\\
        (\funGen{\funStar{\varBlindingFactorAlice}} \opAddPoint \funGenH{\varValue \opSub \varFundValue}) \opAddPoint
        (\funGen{\funStar{\varBlindingFactorBob}} \opAddPoint \funGenH{\varFundValue}) \opAddPoint
        \cnstSumZeroToN \funArray{(\funGen{- \varBlindingFactorAlice}, \funGenH{- \varValue_i})} \opEqNoQ \\
        \funGen{\funStar{\varBlindingFactorAlice}} \opAddPoint \funGen{\funStar{\varBlindingFactorBob}} \opAddPoint \cnstSumZeroToN \funGen{- \varBlindingFactorAlice} \opEqNoQ \funGen{\funStar{\varBlindingFactorAlice}} \opAddPoint \funGen{\funStar{\varBlindingFactorBob}} \opAddPoint \cnstSumZeroToN \funGen{- \varBlindingFactorAlice} \\
        1 \opEqNoQ 1
    \end{gather}
    From step 5.3 to 5.4 we replace every coin $\varCoin$ by its instantiation for a pedersen commitment $\varCoin \opEqNoQ \funGen{\varValue} \opAddPoint \funGenH{\varValue}$.

    From step 5.4 to 5.4 we rely on the fact that if Alice is honest $\varValue \opEqNoQ \cnstSumZeroToN \varValue_i$, therefore also $(\varValue \opSub \varFundValue) \opAddScalar\varFundValue \opEqNoQ \cnstSumZeroToN \varValue_i$ must hold.
    From that we can infer that $\funGenH{\varValue \opSub \varFundValue} \opAddPoint \funGenH{\varFundValue} \opAddPoint \cnstSumZeroToN \funGenH{- \varValue_i}$ must cancel out.
    We have managed to show that condition 3 must hold and can conclude that \emph{Transaction Scheme Correctness} holds.

    We will now argue that the same deriviation holds for \emph{Extended Transaction Scheme Correctness} \emph{Adapted Transaction Scheme Correctness}.

    Condition 1 again follows trivially from the construction of $\procDSendCoinsId$ for the same reasons we have already layed out.

    $\procDSendCoinsId$, $\procDRecvCoinsId$, $\procAptRecvCoinsId$ all rely on the same $\procCreateCoinId$ routine to create output coins, thereby condition 2 will hold for the same reasons as layed out before.

    In the case of \emph{Extended Transaction Scheme Correctness} the blinding factors for the input coins $\funArray{\varCoinInp}$ are shared.
    However, we can easily reduce this case to the proof for the regular case:
    In $\procDSendCoinsId$ Alice and Carol construct their secret keys as follows:
    \begin{gather}
        \varSecKeyAlice \opAssign \funStar{\varBlindingFactorAlice} \opSub \cnstSumZeroToN \varBlindingFactorAlice \\
        \varSecKeyCarol \opAssign \funStar{\varBlindingFactorCarol} \opSub \cnstSumZeroToN \varBlindingFactorCarol
    \end{gather}
    $\varSecKeyAlice$ and $\varSecKeyCarol$ are then inputs to $\procDFinTxId$ in which a partial signature $\varSigAliceCarol$ is calculated, by both Alice and Carol signing with their secret key.
    Assume the general key from before in which we have a single secret key $\varSecKeyAlice$.
    We can split $\varSecKeyAlice$ into arbitrarily chosen shares $(\varSecKeyAlice)_1, (\varSecKeyAlice)_2$ with $\varSecKeyAlice \opEqNoQ (\varSecKeyAlice)_1 + (\varSecKeyAlice)_2$.
    By the definition of Two-Party Signatures~\ref{def:sig:two-party-sig} the combined signature from $(\varSecKeyAlice)_1, (\varSecKeyAlice)_2$ will be valid under $\funGen{\varSecKeyAlice}$.
    Thereby we can treat $\varSecKeyAlice$ and $\varSecKeyCarol$ from $\procSendCoinsId$ as arbitrary shares of a combined $\varSecKeyAliceCarol$.
    It follows from the addtive homomorphic property of the elliptic curve that a signature valid under $\funGen{\varSecKeyAliceCarol}$ must also be valid under $\funGen{\varSecKeyAlice} \opAddPoint \funGen{\varSecKeyCarol}$.
    The case of two receivers calling $\procDRecvCoinsId$ is symmetric.
    From this we can conclude that condition 3 must also hold for the \emph{Extended Transaction Scheme}. \\
    Now for the \emph{Adapted Extended Transaction Scheme} the same argument holds.
    The only difference in this scheme is that in $\procDAptFinTxId$ Bob (instead of Alice) will call $\procFinSigId$, as only he knows his unadapted partial signature $\varSigBob$.
    However, the construction of the signature remains unchanged, therefore the reduction we provided before must hold for the same reasons.

    As we have managed to show that all $\procVerfTx{\varTx} \opEqNoQ 1$ must hold for all three transaction schemes, assuming the participants are honest and the underlaying commitment scheme $\varCommitScheme$, two signature scheme $\varSigSchemeMP$ and rangeproof system $\varProofSystem$ are correct theorem~\ref{lem:atom:correctness} must also hold.
\end{proof}

In the remains to show that the commitment scheme, signature scheme and rangeproof system used in the given instantiations of the three transaction schemes are correct.

All three transaction schemes use Pedersen Commitments as the commitment system $\varCommitScheme$, for which a security and correctness evaluation can be found in~\cite{pedersen1991non}.
For the correctness of the rangeproof system $\varProofSystem$ we refer the reader again to the bulletproof paper~\cite{bunz2018bulletproofs}.
The instantiation of the Mimblewimble Transaction Scheme $\varMWScheme$ (defined in~\ref{def:atom:mw-tx-scheme} instantiated in figure~\ref{fig:inst-mw-tx}) uses a standard Two-Party Signature Scheme $\varSigSchemeMP$ defined in~\ref{def:sig:two-party-sig} for which a correctness proof is available by Maxwell et al. in~\cite{maxwell2019simple}.

For the Extended Mimblewimble Transaction Scheme $\varextMWScheme$ defined in~\ref{def:atom:ext-mw-tx-scheme} and instantiated in figure~\ref{fig:ext-mim-tx-spend},~\ref{fig:ext-mim-tx-recv} and~\ref{fig:ext-mim-tx-fin} the same signature scheme $\varSigSchemeMP$ and rangeproof system $\varProofSystem$ is used, therefore we can make the same argument as for the previous case to show \emph{Extended Transaction Scheme Correctness} holds.

For the Adapted Extended Mimblewimble Transaction Scheme $\varaptMWScheme$, which we have defined in~\ref{def:atom:apt-ext-mw-tx-scheme}, and have shown an instantiation in figure~\ref{fig:inst-apt-mw-tx-recv}, and~\ref{fig:inst-apt-mw-tx-fin}, the Two Party Fixed Witness Adaptor Schnorr Signature Scheme $\varSigSchemeApt$, which we have defined in~\ref{def:sig:two-party-fixed-wit-apt-sig}, is used.
We have already shown correctness of this signature scheme in section~\ref{sec:sig:two-party-apt-security} by proofing that \emph{Adaptor Signature Correctness} holds.
Furthermore, the scheme used a multiparty version of the bulletproof system $\varMPRProofSystem$ for which we refer the reader to~\cite{klinec2020privacy} for a proof of its correctness.

\subsection{Inflation Resistance} \label{subsec:atom:prf-inflation-resistance}

To prove that Inflation Resistance as defined in~\ref{def:atom:inflation-resistance} holds we must show that the probability of a PPT adversary $\cnstAdversary$ winning the $\procInflateId$ game is neglible.
We will prove that the Inflation Resistance property holds for the Mimblewimble Transaction Protocol as seen in figure~\ref{fig:inst-mw-tx} by providing a game-based reduction from $\procInflateId$ to the discrete logarithm assumption from definition~\ref{def:pre:discretelog}.

For the reduction to work we must assume that the signature scheme $\varSigScheme$ used in the transaction scheme is correct and EUF-CMA secure.
Apart from that we also require the employed rangeproof system $\varProofSystem$ and additive homomorphic commitment scheme $\varCommitScheme$ to be correct, as well as the commitment scheme to be additive homomorphic (as defined in definition~\ref{def:pre:homo-com}) which is secure under hiding and binding.
Lastly the rangeproof system has to be setup such that proofs are for a strictly positive range, such that commitments to negative numbers are disallowed.
From this we arrive at the following Theorem:

\begin{theorem}\label{theo:atom:infl-resistence}
\emph{Inflation Resistance} holds for a transaction system $\varMWSchemeParams{\varCommitScheme}{\varSigScheme}{\varProofSystem}$ if and only if the underlying additive homomorphic commitment Scheme $\varCommitScheme$, signature scheme $\varSigScheme$ and rangeproof system $\varProofSystem$ with range $[x,y]$ of $x >= 0, y > 0$ are correct, the signature scheme is EUF-CMA secure, and the commitment fulfills hiding and binding.
\end{theorem}

\begin{proof}\label{prf:atom:infl-resistance}
    We will provide a stepwise reduction from the $\procInflateId$ game to the discrete logarithm assumption.
    In each step we will provide a modified game which the PPT adversary $\cnstAdversary$ will be able to win with overwhelming probability given the assumption that he can win the previous game until we arrive at a contraction.
    The first game $\procGameInfId{0}$ is very similar to the original $\procInflateId$ game, just that we replace $\procCommitId$ with its implementation for pedersen commitments.

    \begin{center}
        \fbox{
        \begin{varwidth}{\textwidth}
            \procedure[linenumbering]{$\procGameInf{0}{\varN}{\varValue}$} {
            \varRProofParams \opFunResult \procRProofSetup{\varSecParam}{0}{2^{64}} \\
            \varPrime \sample \cnstIntegersPrimeWithoutZero{p} \\
            \varG, \varH \sample \cnstGroup_\varPrime \\
            \varCommitParams \opFunResult \procSetupComPed{\varG}{\varH} \\
            \varBlindingFactorAlice \sample \cnstIntegersPrimeWithoutZero{\varN} \\
            (\varCoinInp, \varProof) \opFunResult \procCreateCoin{\varValue}{\varBlindingFactorAlice} \\
            (\varPreTx, (\varSecKeyAlice, \varNonceAlice)) \opFunResult \procSendCoins{\funArray{\varCoinInp}}{\varBlindingFactorAlice}{\varValue}{\varValue}{\cnstFalsum} \\
            (\funStar{\varPreTx}, \funArray{(\varCoin_i, \varValue_i, \funStar{\varBlindingFactor_i})}) \opFunResult \cnstAdversary (\varPreTx) \\
            \varTx \opFunResult \procFinTx{\funStar{\varPreTx}}{\varSecKeyAlice}{\varNonceAlice} \\
            \pcreturn (\cnstSumZeroToN \varValue_i) \opGreaterThen \varValue \opAnd \opForAll_{i=0}^{n} \varCoin_i \opIn \varOutputs \opEqNoQ \funGenH{\varValue_i} \opAddPoint \funGen{\funStar{\varBlindingFactor_i}} \opAnd \procVerfTx{\varTx} \opEqNoQ 1 \\
            \opAnd \varTx.inp \opEqNoQ \funArray{\varCoinInp}
            }
        \end{varwidth}
        }
    \end{center}

    In $\procGameInfId{1}$ we reduce the winning condition, we require the range proof of the output coins to be valid, and the final signature to be valid which both follow from the instantiation of $\procVerfTxId$.
    We also move the computation of the final public key $\varPubKey$ from $\procVerfTxId$ inside the game.

    \begin{center}
        \fbox{
        \begin{varwidth}{\textwidth}
            \procedure[linenumbering]{$\procGameInf{1}{\varN}{\varValue}$} {
            \varRProofParams \opFunResult \procRProofSetup{\varSecParam}{0}{2^{64}} \\
            \varPrime \sample \cnstIntegersPrimeWithoutZero{p} \\
            \varG, \varH \sample \cnstGroup_\varPrime \\
            \varCommitParams \opFunResult \procSetupComPed{\varG}{\varH} \\
            \varBlindingFactorAlice \sample \cnstIntegersPrimeWithoutZero{\varN} \\
            (\varCoinInp, \varProof) \opFunResult \procCreateCoin{\varValue}{\varBlindingFactorAlice} \\
            (\varPreTx, (\varSecKeyAlice, \varNonceAlice)) \opFunResult \procSendCoins{\funArray{\varCoinInp}}{\varBlindingFactorAlice}{\varValue}{\varValue}{\cnstFalsum} \\
            (\funStar{\varPreTx}, \funArray{(\varCoin_i, \varValue_i, \funStar{\varBlindingFactor_i})}) \opFunResult \cnstAdversary (\varPreTx) \\
            \varTx \opFunResult \procFinTx{\varPreTx}{\varSecKeyAlice}{\varNonceAlice} \\
            \pcforeach \varTx.\varOutputs \textit{ as } (\varIterator => \varCoin_i) \\
            \t \pcif \procVerfProof{\varTx.\varProofs[\varIterator]}{\varCoin_i} \opEqNoQ 0 \\
            \t \t \pcreturn 0 \\
            \varPubKey \opAssign \cnstSumOneToN \varTx.\varOutputs \opSub \cnstSumOneToN \varTx.\varInputs \\
            \pcreturn (\cnstSumZeroToN \varValue_i) \opGreaterThen \varValue \opAnd \opForAll_{i=0}^{n} \varCoin_i \opIn \varOutputs \opEqNoQ \funGenH{\varValue_i} \opAddPoint \funGen{\funStar{\varBlindingFactor_i}} \opAnd \procVerf{\varTx.\varSignature}{\varTx.\varMsg}{\varPubKey} \opEqNoQ 1 \\
            \opAnd \varTx.inp \opEqNoQ \funArray{\varCoinInp}
            }
        \end{varwidth}
        }
    \end{center}

    Knowing how $\varOutputs$ and $\varInputs$ has to look like we can also write $\varPubKey$ as:
    \[ \varPubKey \opAssign \funSum{\varIterator \opAssign 0}{\varN} \varTx.\varOutputs \opAddPoint \varCoinInp^{-1} \]
    Given the construction of pedersen commitments from definition~\ref{def:pre:pedersen} and the definition of a Mimblewimble coin~\ref{def:pre:coin} we can also write the equation like this:
    \[ \varPubKey \opAssign \funSum{\varIterator \opAssign 0}{\varN} (\funGenH{\varValue_i} \opAddPoint \funGen{\funStar{\varBlindingFactor_i}}) \opAddPoint \funGenH{- \varValue} \opAddPoint \funGen{- \varBlindingFactorAlice} \]
    We also have the winning condition of $(\cnstSumZeroToN \varValue_i) \opGreaterThen \varValue$, that means the sum of all output values has to be greater then the input value $\varValue$.
    Therefore there has to exist some remainder $\varRemainder \opAssign \funSum{\varIterator \opAssign 0}{\varN}(v_i) \opSub v$.
    Now we can write the equation like this:
    \[  \varPubKey \opAssign \funSum{\varIterator \opAssign 0}{\varN} \funGen{\funStar{\varBlindingFactor_i}} \opAddPoint \funGen{- \varBlindingFactorAlice} \opAddPoint \funGenH{\varRemainder} \]
    Again from the definition of pedersen commitments~\ref{def:pre:pedersen} we know that $\varH$ and $\varG$ are adjacent generators so there exists an unknown exponent $\varX$ such that $\varH \opEqNoQ \funGen{x}$.
    Therefore we arrive at the following equation:
    \[  \varPubKey \opAssign \funSum{\varIterator \opAssign 0}{\varN} \funGen{\funStar{\varBlindingFactor_i}} \opAddPoint \funGen{- \varBlindingFactorAlice} \opAddPoint (\funGen{\varX})^{\varRemainder} \]
    We modify our game with this equation, additionally we require the adversary to reveal an opening $\varOpening \opEqNoQ \varX \opTimesScalar \varRemainder$.
    We add the following win condition $\funGen{\varOpening} \opEqNoQ \funGenH{\varRemainder}$ to check the validity of the opening.

    \begin{center}
        \fbox{
        \begin{varwidth}{\textwidth}
            \procedure[linenumbering]{$\procGameInf{2}{\varN}{\varValue}$} {
            \varRProofParams \opFunResult \procRProofSetup{\varSecParam}{0}{2^{64}} \\
            \varPrime \sample \cnstIntegersPrimeWithoutZero{p} \\
            \varG, \varH \sample \cnstGroup_\varPrime \\
            \varCommitParams \opFunResult \procSetupComPed{\varG}{\varH} \\
            \varBlindingFactorAlice \sample \cnstIntegersPrimeWithoutZero{\varN} \\
            (\varCoinInp, \varProof) \opFunResult \procCreateCoin{\varValue}{\varBlindingFactorAlice} \\
            (\varPreTx, (\varSecKeyAlice, \varNonceAlice)) \opFunResult \procSendCoins{\funArray{\varCoinInp}}{\varBlindingFactorAlice}{\varValue}{\varValue}{\cnstFalsum} \\
            (\funStar{\varPreTx}, \funArray{(\varCoin_i, \varValue_i, \funStar{\varBlindingFactor_i})}, \varOpening) \opFunResult \cnstAdversary (\varPreTx) \\
            \varTx \opFunResult \procFinTx{\varPreTx}{\varSecKeyAlice}{\varNonceAlice} \\
            \pcforeach \varTx.\varOutputs \textit{ as } (\varIterator => \varCoin_i) \\
            \t \pcif \procVerfProof{\varTx.\varProofs[\varIterator]}{\varCoin_i} \opEqNoQ 0 \\
            \t \t \pcreturn 0 \\
            \varRemainder \opAssign \cnstSumZeroToN (\varValue_i) \opSub \varValue \\
            \t \pcreturn 0 \\
            \varPubKey \opAssign \funSum{\varIterator \opAssign 0}{\varN} \funGen{\funStar{\varBlindingFactor_i}} \opAddPoint \funGen{- \varBlindingFactorAlice} \opAddPoint (\funGen{\varX})^{\varRemainder} \\
            \pcreturn \varRemainder \opGreaterThen 0 \opAnd \opForAll_{i=0}^{n} \varCoin_i \opIn \varOutputs \opEqNoQ \funGenH{\varValue_i} \opAddPoint \funGen{\funStar{\varBlindingFactor_i}} \opAnd \procVerf{\varTx.\varSignature}{\varTx.\varMsg}{\varPubKey} \\
            \opAnd \varTx.inp \opEqNoQ \funArray{\varCoinInp} \opAnd \funGen{\varOpening} \opEqNoQ \funGenH{\varRemainder}
            }
        \end{varwidth}
        }
    \end{center}

    We argue that if the adversary can win $\procGameInfId{1}$ he also can win $\procGameInfId{2}$.
    As we have layed out before, to produce a valid signature under $\varPubKey$ the following values have to be known $\varBlindingFactorAlice^{-1}$, $\funArray{\varBlindingFactor_i}$, $\varRemainder$ and $\varX$.
    Since Alice will provide the signature under $\funGen{\varBlindingFactorAlice^{-1}}$, Bob has to provide his signature under $\cnstSumZeroToN (\funGen{\varBlindingFactor_i}) \funGen{\varRemainder \opTimesScalar \varX}$, which means he must know the opening $\varRemainder \opTimesScalar \varX$.
    However, here is where we arrive at a contradiction, as by the definition of pedersen commitments~\ref{def:pre:pedersen} a $\varX$ such that $\varH \opEqNoQ \funGen{\varX}$ must not be known.
    We show that given this contradiction, an adversary $\cnstAdversary$ is able to consistently break the discrete logarithm game by giving a simulation via the $\procGameInfId{2}$ game.

    In order to break the discrete logarithm of a cyclic group $\cnstGroup_\varPrime$ a PPT adversary $\cnstAdversary$ would have to win the following game:
    \begin{center}
        \fbox{
        \begin{varwidth}{\textwidth}
            \procedure[linenumbering]{$\procDiscreteLogGame{\varPrime}$} {
            \varG \sample \cnstGroup_\varPrime \\
            \varX \sample \cnstIntegersPrimeWithoutZero{\varPrime} \\
            \varPubKey \opFunResult \funGen{\varX} \\
            \funStar{\varX} \opFunResult \cnstAdversary (\varPubKey) \\
            \pcreturn \varPubKey \opEqNoQ \funGen{\funStar{\varX}}
            }
        \end{varwidth}
        }
    \end{center}

    We construct the following extractor function $\procExtractXId$, which let's us extract $\varX$, given the opening $\varOpening$ and the remainder $\varRemainder$ of the $\procGameInfId{2}$:

    \begin{center}
        \fbox{
        \begin{varwidth}{\textwidth}
            \procedure[linenumbering]{$\procExtractX{\varOpening}{\varRemainder}$} {
            \pcreturn \varOpening \opSub \varRemainder
            }
        \end{varwidth}
        }
    \end{center}

    Given $\procExtractXId$ the adversary now is able to win $\procDiscreteLogGameId$ as follows:
    Given $\varG$ and $\varPubKey$ by the $\procDiscreteLogGameId$ we can construct a simulation by setting up the pedersen commitment scheme during $\procGameInfId{2}$ as $\procSetupComPed{\varG}{\varPubKey}$.
    After the game has completed the adversary can compute $\varRemainder \opAssign \cnstSumZeroToN (\varValue_i) \opSub \varValue$ and afterward $\varX \opFunResult \procExtractX{\varOpening}{\varRemainder}$
    As the commitment scheme is setup with $\varG$ and $\varPubKey$, $\varPubKey \opEqNoQ \funGen{\varX}$ must hold, and $\varX$ will therefore be accepted as a solution to the $\\procDiscreteLogGameId$ game.
    We have therefore arrived at a contradiction and can conclude that theorem~\ref{theo:atom:infl-resistence} must hold.
\end{proof}

We now argue that the same reduction must hold for the instantiation of the extended and adapted mimblewimble transaction scheme from figures~\ref{fig:ext-mim-tx-fin},~\ref{fig:ext-mim-tx-recv},~\ref{fig:ext-mim-tx-spend} and~\ref{fig:inst-apt-mw-tx-fin},~\ref{fig:inst-apt-mw-tx-recv}.

\begin{theorem} \label{theo:atom:inf-resistence-ext}
    \emph{Inflation Resistance} holds for an \emph{Extended Mimblewimble Transaction Scheme} and \emph{Adapted Mimblewimble Transaction Scheme} iff it holds for \emph{Mimblewimble Transaction Scheme}
\end{theorem}

\begin{proof}
    For the proof we look at the additional function included in the extended and adapted variant and proof that if we replace one of the original ones with the new ones the reduction found in~\ref{prf:atom:infl-resistance} still holds.

    \begin{itemize}
        \item Case 1: $\procDSendCoinsId$ is used instead of $\procSendCoinsId$: This means the key to the input coin would be distributed between two parties, therefore we would modify our game to include a second challenger, holding part of the input coin blinding factor.
        We would also require to call $\procDFinTxId$ instead of the regular $\procFinTxId$ to complete the transaction.
        The transaction structure, implementation of $\procVerfTxId$ and behaviour of the adversary is left unchanged, the reduction therefore still holds.
        \item Case 2: $\procDRecvCoinsId$ is used on the receiver side.
        Again we have to introduce a second challenger to the game which will run the $\procDRecvCoinsId$ protocol together with the adversary.
        However, again this does not change our reduction, the transaction structure, as well as $\procVerfTxId$ stays the same and the adversary has access to all parameters needed to run $\procExtractXId$.
        \item Case 3: $\procAptRecvCoinsId$ is expected to be called by the adversary $\cnstAdversary$.
        This means that $\cnstAdversary$ has to provide an adapted signature $\varSigAptBob$ and statement $\varStatement$ which he can simply do by computing $(\varWit, \varStatement) \opFunResult \procGenRId$ and $\varSigAptBob \opFunResult \procAptSig{\varSigBob}{\varWit}$.
        We then have to modify the game to call the $\procDAptFinTxId$ to finalize the transaction.
        Again transaction structure, $\procVerfTxId$ is unchanged and Bob has access to all parameters needed for calling $\procExtractXId$.
    \end{itemize}

    We have shown that by using the functions provided in the extended and adapted Mimblewimble transaction schemes, the properties used in the reduction of~\ref{prf:atom:infl-resistance} remain unchanged and therfore theorem~\ref{theo:atom:inf-resistence-ext} must hold.
\end{proof}

\subsection{Theft Resistance} \label{subsec:atom:prf-theft-resistance}

TODO

\subsection{Transaction indistinguishability} \label{subsec:atom:prf-tx-indistinguishability}

TODO

\section{Atomic Swap Protocol}\label{sec:atom:atomic-swap}
With the outlined Adapted Mimblewimble Transaction Scheme from definition~\ref{def:atom:apt-ext-mw-tx-scheme} and protocols from \ref{sec:atom:protocols} we can now construct an Atomic Swap protocol with another cryptocurrency.
In this thesis we will explain a swap with Bitcoin, as at present Bitcoin and Bitcoin-like cryptocurrencies are the most widely adopted.
We will generally refer to the ``Bitcoin side`` and the ``Mimblewimble side`` of the swap to be most generic.
Upon implementation one has to decide for a specific implementation, for example BTC on the Bitcoin side and Grin on
the Mimblewimble side.
On the Bitcoin side we define three DPT functions $(\procLockAddrId, \procVerifyLockId, \procSpendBtcId)$.
\begin{itemize}
    \item $(\varScriptPubKey) \opFunResult \procLockAddr{\varPubKeyAlice}{\varPubKeyBob}{\varStatement}{\varTime}$:
    The locking script function lets Bob construct a Bitcoin script only spendable by Alice if she receives the discrete logarithm $\varWit$ of $\varStatement$ with $\varStatement \opEqNoQ \funGen{\varWit}$.
    Additionally, the function requires Bobs public key $\varPubKeyBob$ and a timelock $\varTime$ (given as a block number) as input which allows Bob to reclaim his funds after some time if the atomic swap was not completed successfully.
    The function will create and return a Bitcon script $\varScriptPubKey$ to which Bob can send funds using a P2SH transaction.
    To spend this output Alice will have to provide a signature under her public key $\varPubKeyAlice$ and $\varStatement$, which she is able to provide, once acquired $\varWit$.
    This construction is similar although simpler to the locking mechanism described by Malavolta et al.
    For a in-depth security analysis of this concept we refer the interested reader to their paper ~\cite{malavolta2019anonymous}.
    For a concrete Bitcoin Script realizing this functionality see section~\ref{ch:implementation}.
    \item $\{ 1,0 \} \opFunResult \procVerifyLock{\varPubKeyAlice}{\varPubKeyBob}{\varStatement}{\varValue}{\varTime}{\varUTXO_{lock}}$:
    The lock verification algorithm takes as input Alices, Bobs public keys and the statement $\varStatement$ and the UTXO $\varUTXO_{lock}$.
    The function will compute the Bitcoin lock script $\varScriptPubKey$ as created by $\procLockAddrId$ check equality with $\varUTXO_{lock}$ and if the value locked under the UTXO equals $\varValue$.
    Upon successful verification the function returns 1, otherwise 0.
    \item $\varTx \opFunResult \procSpendBtc{\varInputs}{\varOutputs}{\varSecKey}$:
    The spend Bitcoin functionality is a wrapper around the $\procBuildTransactionId, \procSignTransactionId$ defined in~\ref{subsec:pre:bitcointx}.
    It constructs and signs a transaction spending the UTXOs given in $\varInputs$ and creates the fresh UTXOs in $\varOutputs$.
    It returns a signed transaction which then can be broadcast.
\end{itemize}

\subsection{Setup phase}\label{subsec:atom:setup}

We assume Alice owns Mimblewimble coins $\funArray{\varSpendableCoin}$ with the total value $\varValueMw$ and Bob
Bitcoin locked in some UTXO $\varUTXO$ with a value of $\varValueBtc$ and secret spending key $\varSecKey_{btc}$.
Before the protocol can start the two parties must agree on the value they want to swap, the exchange rate of the currencies and a time after which the swap should be canceled.
After coming to an agreement the following variables are defined and known by both Alice and Bob:
\begin{itemize}
    \item $\varSecParam$ A security parameter.
    \item $\varAmountBtc$ The amount of Bitcoin Bob will swap to Alice.
    \item $\varAmountMW$ The amount of the Mimblewimble coin Alice will swap to Bob.
    \item $\varTimeBTC$ The locktime as a blockheight for the Bitcoin side.
    \item $\varTimeMW$ The locktime as a blockheight for the Mimblewimble side.
\end{itemize}
We collect this shared variables in an initial swap state $\varSwpState$:
\[ \varSwpState \opAssign \{ \varSecParam, \varAmountBtc, \varAmountMW, \varTimeBTC, \varTimeMW \} \]

In practice, we need to consider that exchange rates might fluctuate, furthermore timeouts have to be calculated separately for each chain.
The problems with cross chain payments are discussed by Tairi et al. in~\cite{tairi2019a2l}, they propose to use a fixed exchange rate for each day and to use a real world timeout like one day and then calculate the specific block numbers by taking the average block time of the blockchain into account.
In our setup we can also fix the exchange rate at the beginning of the protocol, which stays unchanged during protocol execution.
If the exchange rate fluctuates and one party is negatively impacted he or she could still decide to stop being cooperative which means the coins would be returned to the original owners after the timeout.

There is furthermore the problem of transaction fees, which we do not consider for this formalization.
Depending on the current network load the participants need to agree on a fee that they are willing to pay for each network.
It needs to be considered that if fees are picked to low, it might take time for transactions to be confirmed, and the swap will take longer, if they are picked high the participants will lose funds.

We formalize the protocol $\procSetupSwapId$ in figure~\ref{fig:setup-swap}.
The protocol takes as input the shared swap state $\varSwpState$ from both parties.
From Alice her Mimblewimble input coins $\funArray{\varSpendableCoin}$ with the summed up value $\varValueMw$ is furthermore required as an input.
From Bob we require a list of UTXO's $\funArray{\varUTXO}$ he wants to spend, he also needs to provide their spending keys $\funArray{\varSecKey_{btc}}$ and their summed of total value $\varValueBtc$, although this could also be read from the blockchain.

The protocol starts by both parties creating and exchanging keys.
Bob now creates two new Bitcoin outputs $\varUTXO_{lock}$ and $\varUTXO_{B}$, of which one is the locked Bitcoins which Alice might retrieve later (or Bob after time $\varTimeBTC$ has passed), and the other Bobs change output. (Difference between what is stored in the input UTXO and what should be sent to Alice).
After Bob has published the transaction sending value to the new outputs, he will provide Alice with the statement $\varStatement$ under which the Bitcoins' are locked together with Alice's public key.
Alice can now verify that the funds on Bitcoin side are indeed correctly locked.
After that she will collaborate with Bob to spend her Mimblewimble coins into an output shared by both parties.
Immediately after, both parties collaborate again to spend this shared coin back to Alice with a timelock of $\varTimeMW$.
It is immanent that Alice does not publish the first transaction (A -> AB) before the timelocked refund transaction (AB -> A) was signed, otherwise her funds are locked in the shared output without the possibility of refund if Bob refuses to cooperate.
The setup protocol concludes with the funds locked up in both chains and ready to be swapped and outputs the updated swap state $\varSwpState$ to both parties.
Additionally, it outputs Alice's part $\funStar{\varPtSpendableCoinAlice}$ of the locked mimblewimble coin, her change output on the mimbewimble side $\funStar{\varSpendableCoinAlice}$, her secret key $\varSecKeyAlice$ for the Bitcoin side and $\funStarAlt{\varSpendableCoinAlice}$, which is refund coin, only valid after $\varTimeMW$.
For Bob it furthermore outputs his part $\funStar{\varPtSpendableCoinBob}$ of the locked mimblewimble coin, his change output on the bitcoin side $\varUTXO_{B}$ and the secret witness value $\varWit$, which shall be revealed to Alice in the execution phase.

\newgeometry{margin=2cm}
\begin{landscape}
    \thispagestyle{plain}
    \begin{figure}
        \fbox{
        \procedure[linenumbering,skipfirstln]{$\procSetupSwap{\varSwpState}{\funArray{\varSpendableCoin}}{\varValueMw}{\funArray{\varUTXO}}{\funArray{\varSecKey_{btc}}}{\varValueBtc}$} {
        Alice \< \< \< \< Bob \\
        \varKeyPairAlice \opFunResult \procSetup{\varSecParam} \< \< \< \< \varKeyPairBob \opFunResult \procSetup{\varSecParam} \\
        \< \< \< \< (\varWit, \varStatement) \opFunResult \procSetup{\varSecParam} \\
        \< \sendmessagerightx{4}{\varPubKeyAlice} \< \\
        \< \sendmessageleftx{4}{\varPubKeyBob} \< \\
        \< \< \< \< \varScriptPubKey \opFunResult \procLockAddr{\varPubKeyAlice}{\varStatement}{\varPubKeyBob}{\varTimeBTC} \\
        \< \< \< \< \varUTXO_{lock} \opFunResult \procCreateUTXO{\varAmountBtc}{\varScriptPubKey} \\
        \< \< \< \< \varUTXO_{B} \opFunResult \procCreateUTXO{\varValueBtc - \varAmountBtc}{\varPubKeyBob} \\
        \< \< \< \< \varBtcTx \opFunResult \procSpendBtc{\funArray{\varUTXO}}{\funArray{\varUTXO_{lock}, \varUTXO_{B}}}{\funArray{\varSecKey_{btc}}} \\
        \< \< \< \< \procPublishBtc{\funArray{\varBtcTx}} \\
        \< \< \< \< \varSwpState \opAssign \varSwpState \opUnion (\varStatement, \varUTXO_{lock}) \\
        \< \sendmessageleftx{4}{\varStatement,\varUTXO_{lock}} \< \\
        \pcif \procVerifyLock{\varPubKeyAlice}{\varPubKeyBob}{\varStatement}{\varAmountBtc}{\varTimeBTC}{\varUTXO_{lock}} \opEqNoQ 0 \\
        \t \pcreturn \cnstFalsum \< \< \< \< \\
        \varSwpState \opAssign \varSwpState \opUnion (\varStatement, \varUTXO_{lock}) \\
        (\varMwFundTx, \funStar{\varSpendableCoinAlice},\funStar{\varPtSpendableCoinAlice}) \< \< \< \< (\funStar{\varPtSpendableCoinBob}) \pcskipln \\
        \opFunResult \procDSharedOutputMwTxL{\funArray{\varSpendableCoin}}{\varAmountMW}{\cnstFalsum} \< \< \< \< \opFunResult \procDSharedOutputMwTxR{\varAmountMW} \\
        (\varMwRefundTx, \funStarAlt{\varSpendableCoinAlice}) \< \< \< \< \varMwRefundTx \pcskipln \\
        \opFunResult \procDSharedInpMwTxL{\funStar{\varPtSpendableCoinAlice}}{\varAmountMW}{\varTimeMW} \< \< \< \< \opFunResult \procDSharedInpMwTxR{\funStar{\varPtSpendableCoinBob}}{\varAmountMW} \\
        \procPublishMW{\funArray{\varMwFundTx,\varMwRefundTx}}  \\
        \pcreturn (\varSwpState, \funStar{\varPtSpendableCoinAlice}, \funStar{\varSpendableCoinAlice}, \varSecKeyAlice, \funStarAlt{\varSpendableCoinAlice}) \< \< \< \< \pcreturn (\varSwpState, \funStar{\varPtSpendableCoinBob}, \varUTXO_{B}, \varWit)
        }
        }
        \caption{Atomic Swap - $\procSetupSwapId$.}\label{fig:setup-swap}
    \end{figure}
\end{landscape}
\restoregeometry

\subsection{Execution Phase}\label{subsec:atom:exec}

First we need to define an additional auxiliary function $\procVerifyTimeId$ with the following interface:
\[ \{0,1\} \opFunResult \procVerifyTime{\varChain}{\varTime} \]
This function will verify that there is sufficient time to execute the atomic swap protcol.
As input it takes a chain paramter $\varChain$ (in our case this could be either BTC or MW) and a block height $\varTime$.
The routine will verify that the current height of the blockchain is marginally below $\varTime$.
If this is the case it will return 1, or 0 otherwise.
How much time exactly should be left for the function to return 1 is implementation specific, and could be set to for instance one day.
We now define a protocol $\procExcSwapId$ to execute the Atomic Swap between some amount $\varAmountBtc$ on the Bitcoin side and some amount on the Mimblewimble side $\varAmountMW$.
We assume the participants have successfully run the $\procSetupSwapId$ protocol and both know the updated swap state $\varSwpState$ as returned by the setup protocol.
Both parties need to provide their part of the locked mimblewimble coins as input to the protocol.
Additionally, Alice needs to provide her secret key for the bitcoin side $\varSecKeyAlice$ and Bob the secret witness value $\varWit$.
The protocol starts with both parties checking that there is enough time left to complete the protocol.
After the check they will run the $\procDScriptMwTxId$ protocol in which they spend the locked Mimblewimble output to Bob, while at the same time revealing $\varWit$ to Alice.
Either one of the parties can now publish the transaction to the mimblewimble network, which concludes the swap on the mimbewimble side, as Bob is now in full control of the funds.
Alice, knowing $\varWit$, creates now a new UTXO where she then sends the funds from the Bitcoin lock.
After publishing this transaction to the Bitcoin network, Alice is in full possession of the swapped funds on the Bitcoin side and the Atomic Swap is completed.
The protocol outputs their newly created output/coin to each party.

\newgeometry{margin=2cm}
\begin{landscape}
    \thispagestyle{plain}
    \begin{figure}
        \fbox{
        \procedure[linenumbering,skipfirstln]{$\procExcSwap{\varSwpState}{\funStar{\varPtSpendableCoinAlice}}{\varSecKeyAlice}{\funStar{\varPtSpendableCoinBob}}{\varWit}$} {
        Alice \< \< \< \< Bob \\
        (\varAmountMW, \varAmountBtc, \varTimeMW, \varTimeBTC, \varUTXO_{lock}, \varStatement) \opFunResult \varSwpState \< \< \< \< (\varAmountMW, \varAmountBtc, \varTimeMW, \varTimeBTC) \opFunResult \varSwpState \\
        \pcif \procVerifyTime{BTC}{\varTimeBTC} \opEqNoQ 0 \opOr \procVerifyTime{MW}{\varTimeMW} \opEqNoQ 0 \< \< \< \< \pcif \procVerifyTime{BTC}{\varTimeBTC} \opEqNoQ 0 \opOr \procVerifyTime{MW}{\varTimeMW} \opEqNoQ 0 \\
        \t \pcreturn \cnstFalsum \< \< \< \< \t \pcreturn \cnstFalsum \\
        (\varMwTx, \cnstEmptySet, \varWit) \< \< \< \< (\varMwTx, \funStar{\varSpendableCoinBob}) \pcskipln \\
        \opFunResult \procDScriptMwTxL{\funStar{\varPtSpendableCoinAlice}}{\varAmountMW}{\cnstFalsum}{\varStatement} \< \< \< \< \opFunResult \procDScriptMwTxR{\funStar{\varPtSpendableCoinBob}}{\varAmountMW}{\varWit} \\
        \procPublishMW{\varMwTx} \< \< \< \< \procPublishMW{\varMwTx} \\
        (\funStarAlt{\varSecKeyAlice}, \funStarAlt{\varPubKeyAlice}) \opFunResult \procSetup{\varSecParam} \< \< \< \< \\
        \varUTXO_{A} \opFunResult \procCreateUTXO{\varAmountBtc}{\funStarAlt{\varPubKeyAlice}} \< \< \< \< \\
        \varBtcTx \opFunResult \procSpendBtc{\funArray{\varUTXO_{lock}}}{\funArray{\varUTXO_{A}}}{\funArray{\varSecKeyAlice, \varWit}} \< \< \< \< \\
        \procPublishBtc{\funStar{\varBtcTx}} \< \< \< \< \\
        \pcreturn (\varUTXO_{A}) \< \< \< \< \pcreturn (\funStar{\varSpendableCoinBob})
        }
        }
        \caption{Atomic Swap - $\procSetupSwapId$. \label{fig:exec-swap}}
    \end{figure}
\end{landscape}
\restoregeometry

\subsection{Refunding}\label{subsec:atom:refund}

If one party refused to cooperate or goes offline the coins can be returned to the original owner.
On the Bitcoin side this is the case as Bob can simply spend the locked output with his private key $\varSecKeyBob$ after the timeout $\varTimeBTC$ has passed.
He then can simply construct and sign a transaction spending the output to a new UTXO which is in his full possession.
He even could prepare this transaction upfront and broadcast it, once the the blocknumber hits $\varTimeBTC$ the transaction will become valid and get mined.
Again we stress the importance of using appropriate timeouts, if a timeout is too short the swap might get cancelled if there are some delays, if the timeout is too long the funds might be locked for an unnessary amount of time.

On the Mimblewimble side the second transaction spending the shared output back to Alice guarantees that her funds are returned to her after the timeout $\varTimeMW$ hits.
For this reason it is so important that Alice publishes both the fund and refund transaction at the same time.
If she would publish the funding transaction first, Bob could refuse to cooperate for the refund transaction, in which case the funds would stay in the locking output only retrievable if both parties cooperate.
If the swap executes successful the refund transaction would get discarded by miners, as it then is no longer valid even after the timeout $\varTimeMW$.