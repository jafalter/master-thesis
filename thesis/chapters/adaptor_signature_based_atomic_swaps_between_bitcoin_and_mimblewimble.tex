This section will first define procedures and protocols to construct Mimblewimble transactions and prove their security.
The formalizations will be similar to those found by Fuchsbauer et al. in their cryptographic investigation of the Mimblewimble protocol~\cite{fuchsbauer2019aggregate}.
In particular, the final transaction output from our protocols should be a valid transaction as by the definitions of Fuchsbauer et al.
As we will only focus on the transaction building protocol~(\cref{subsec:pre:mimblewimble:txbuild}), the notions of cut through~(\cref{par:pre:mimblewimble:cut}), transaction merging~(\cref{sec:pre:mimblewimble:merge}), coin minting (see coinbase transactions \cref{subsec:pre:mimblwimble-tx}), and publishing transactions to the ledger (\cref{subsec:pre:mimblewimble:ledger}), all formalized by Fuchsbauer et al. in~\cite{fuchsbauer2019aggregate}, will not be the topic of this formalization.

As an extension to the regular transaction protocol \emph{Mimblewimble Transaction Scheme}, which we will define first, we will additionally define two further schemes:
The first of them titled \emph{Extended Mimblewimble Transaction Scheme}, will provide additional functions to create and spend coins owned by two parties instead of just one, thereby enabling coins owned by multiple parties, which is similar to a mutlisig address in Bitcoin~\cite{antonopoulos2014mastering}.
The second extended definition is called \emph{Contract Mimblewimble Transaction Scheme}, in which we further add algorithms that allow embedding primitive smart contracts to the transaction building protocol.
Both the \emph{Extended Mimblewimble Transaction Scheme} and \emph{Contract Mimblewimble Transaction Scheme} are constructed to provide the functionality that is later needed to build the final Atomic Swap protocol, which we will introduce in~\cref{sec:atom:atomic-swap}.

We will proceed by providing an instantiation of the three transaction schemes in~\cref{sec:atom:inst}, which can be implemented and deployed on a Mimblewimble-based Cryptocurrency such as Beam or Grin.
In~\cref{sec:atom:protocols}, we define two-party protocols from the outlined schemes to construct Mimblewimble transactions.
~\Cref{sec:atom:security} shows the proofs that the formalizations are correct and the protocols secure in the malicious setting as defined in~\cref{subsec:pre:security}.
Finally, in~\cref{sec:atom:atomic-swap}, we describe an Atomic Swap protocol from these building blocks, allowing two parties to securely and trustlessly swap funds from a Mimblewimble based blockchain with those on another blockchain, such as Bitcoin.

\section{Definitions}\label{sec:atom:definitions}
\urldef\urlgrinexplained\url{https://tinyurl.com/y63hc4ua}

As we have already discussed in~\cref{sec:pre:mimblewimble} for creating a transaction in Mimblewimble, it is immanent that both the sender and receiver collaborate and exchange messages via a secure channel.
To construct the transaction protocol, we assume that we have access to a Two-Party Signature Scheme $\varSigSchemeMP$ as described in~\cref{def:sig:two-party-sig}, a Range Proof System as shown in~\cref{def:pre:rangeproof} such as Bulletproofs, as described in~\cref{sec:pre:rangeproof} and a homomorphic Commitment Scheme $\varCommitScheme$ as defined in~\cref{def:pre:homo-com} such as Pedersen Commitments seen in~\cref{def:pre:pedersen}.

Fuchsbauer et al. have defined three procedures, $\procFuchsSend$, $\procFuchsRcv$, and $\procFuchsLdgr$, regarding creating a transaction.
$\procFuchsSend$ called by the sender will create a pre-transaction, $\procFuchsRcv$ takes the pre-transaction and adds the receiver's output and $\procFuchsLdgr$ (again called by the sender) verifies and publishes the final transaction to the blockchain ledger.
As we have already pointed out in this thesis, we won't discuss the transaction publishing phase.
Therefore we will not cover the publishing functionality of the $\procFuchsLdgr$ procedure.
However, we will use the verification capabilities of the algorithm.
That means the transactions created by our protocol must be compatible with the $\procFuchsVer$ functionality formalized by Fuchsbauer et at. and internally used by $\procFuchsLdgr$.
We can, however, assume that a transaction $\varTx$ for which $\procFuchsVer \opEqNoQ 1$ holds, could be published to the ledger using the $\procFuchsLdgr$ algorithm. (Given the inputs used in the transaction are present and unspent on the ledger)

Originally Fuchsbauer et al. have defined the creation of a Mimblewimble transaction as a two-step, two-party protocol.
A sender owning a set of input coins calls $\procFuchsSend$ to create an initial pre-transaction signed already by the sender and then forwarded to the fund receiver.
The receiver then calls $\procFuchsRcv$ to add his output coins with the correct value.
His signature is then aggregated with the sender's signature and thereby finalizing the transaction $\varTx$.
Any party (knowing the final $\varTx$) can now call $\procFuchsLdgr$ to verify and publish the transaction to the ledger.

We now want to motivate why in the following, we found it necessary to redefine some of the algorithms already laid out by Fuchsbauer et al.
The main reason is that we are using the notion of two-party signatures as of~\cref{def:sig:two-party-sig} instead of aggregatable signatures, which are employed in their paper.
While aggregatable signatures are a similar concept to the two-party signatures, we can find some essential differences.
Ultimately, the two-party signatures is, as we shall see, the more appropriate and secure choice for the formalization.
First of all, we need to define the notion of an Aggregatable Signature Scheme:
\begin{definition}[Aggregatable Signature Scheme] \label{def:atom:aggsig}
    A Signature Scheme $\varSigScheme$ can be called aggregatable if for two signatures $\varSignature_1$ and $\varSignature_2$, valid for a message $\varMsg$ under the public keys $\varPubKey_1$ and $\varPubKey_2$ we can construct an aggregated signature $\varSignature_a$ valid for the same message $\varMsg$ under the composite public key $\varPubKey_a \opEqNoQ \varPubKey_1 \opAddPoint \varPubKey_2$
\end{definition}
In the Schnorr Signature Scheme, we can only aggregate signatures by primitively concatenating the individual signatures like $\varSignature_1 \opConc \varSignature_2$.
The verifier would then check the validity of $\varSignature_1$ and $\varSignature_2$ independently under the public keys $\varPubKey_1, \varPubKey_2$ and finally check if $\varPubKey_a \opEqNoQ \varPubKey_1 \opAddPoint \varPubKey_2$~\cite{fuchsbauer2019aggregate}. \\
The reason why we can not simply add up the signatures is the following:
Recall the structure of a Schnorr signature $(\varS, \varRand)$, imagine we would try to create an aggregated signature like $\varSignature_a \opEqNoQ (\varS_1 \opAddScalar \varS_2, \varRand_1 \opAddPoint \varRand_2)$, then this would not be a valid signature anymore.
$\varS$ is calculated as $\varS \opEqNoQ \varNonce \opAddScalar \varSchnorrChallenge \opTimesScalar \varSecKey$ where $\varSchnorrChallenge \opEqNoQ \funHash{\varMsg \opConc \varRand \opConc \varPubKey}$.
As we have changed the nonce Commitment $\varRand$ and the public key $\varPubKey$ in our aggregated signature the Schnorr challenge $\varSchnorrChallenge$ will be different from the one used by the individual signers, thereby making the verification algorithm return 0.
We can fix this issue by having the individual signers use the final composite $\varRand$ and $\varPubKey$ for their Schnorr challenge calculation, which is exactly what we are doing in the Schnorr-based instantiation of the Two-Party Signature Scheme in~\cref{fig:twoparty-schnorr}.
This detail, however, introduces the necessity for an initial setup phase in which the parties exchange messages to compute $\varRand$ and $\varPubKey$ from their shares.
Using the two-party Schnorr model instead of the aggregated Schnorr, we save space, as we only need to store one single signature instead of multiple.
Further, we also only need to store the final public key $\varPubKey$ and disregard the public key shares.
We also note that the two-party version is currently implemented in Grin and Beam in practice.\footnote{\urlgrinexplained}
Finally, there is another critical advantage that comes with the two-party Schnorr approach.
For the peers to start the signing process, the final composite $\varPubKey$ and nonce Commitment $\varRand$ need to be known.
That also entails that the flow pointed out in~\cite{fuchsbauer2019aggregate}, in which the transaction sender starts the signing process, and the receiver completes it is no longer possible.
Instead, the signing process can only start with the receiver's turn.
We need to introduce a third round.
The sender receives the partially signed pre-transaction from the receiver, adds his partial signature and only now can finalize the signature and thereby the transaction.
While having to add an additional round would seem like an inconvenience at first, we discover that we avoid being vulnerable to a \emph{Transaction Sniff Attack} by doing so.

For the following attack to be possible, we need to assume that the channel between the sender (Alice) and receiver (Bob) has been compromised therefore can no longer be considered secure.
We show that under this assumption, the formalization laid out by Fuchsbauer et al. would be vulnerable to the \emph{Transaction Sniff Attack}.
In contrast our formalization would still be secure.
\paragraph{Transaction Sniff Attack}
Imagine a sender Alice and receiver Bob.
Alice owns three Mimblewimble coins and wants to send one of them to Bob to pay for Bob's service.
They start the transaction-building process and communicate via a channel that they assume to be secure.
However, in reality, the channel they are using is insecure, and attacker $\cnstAdversary$ has managed to compromise it and is secretly listening to every message exchanged between the two.
With the notions defined by Fuchsbauer et al., Alice starts the protocol by running $\varPreTx \opFunResult \procFuchsSend(\cdot)$ and sending $\varPreTx$ to Bob via the channel.
Bob has received $\varPreTx$ from Alice but decides to wait with the protocol continuation because of some urgent task.
In the meantime, the malicious attacker managed to sniff $\varPreTx$ sent by Alice.
Already containing Alice's signature, all the attacker has to do is guess the value Alice might want to send, create an output coin with that value, add his signature, aggregate it with Alice and broadcast the final transaction to the network.
Since the range of possible amounts that Alice might want to transfer is limited, it is trivial for the attacker to guess it in polynomial time.
When now Bob comes back to finalize the transaction, he will discover that he is unable to continue with the protocol, as the transactions input coins are already spent and are now in possession of the attacker.

Starting the signing process only at the receiver's turn and introducing a third-round solves this issue because Alice adds the signature for her input coins only at the last step.
Using the Two-Party Signature Scheme instead of an Aggregatable Signature Scheme forces us to make this change because of the additional setup phase required.
Even if the attacker were able to sniff one of the pre-transactions exchanged between the parties, because Alice will only ever add the signature for her input coins at the end of the protocol, the attacker would not be able to compute a valid transaction.

We now define the standard \emph{Mimblewimble Transaction Scheme} that intuitively allows a sender to transfer value stored in a Mimblewimble coin to a receiver.
To improve the readability of our following formalizations, we introduce a wrapper $\varSpendableCoin$ that represents a spendable coin and contains a reference to the coin Commitment $\varCommitment$, range proof $\varProof$, and its (secret) spending information of the coins value $\varValue$ and blinding factor $\varBlindingFactor$.
\[ \varSpendableCoin \opAssign \{ \varCommitment, \varValue, \varBlindingFactor, \varProof \} \]
If we want to indicate that a spendable coin is used as an output coin in a transaction, we write $\funStar{\varSpendableCoin}$.

\begin{definition}[Mimblewimble Transaction Scheme]
    \label{def:atom:mw-tx-scheme}
    A Mimblewimble Transaction Scheme $\varMWSchemeParams{\varCommitScheme}{\varSigSchemeMP}{\varRProofSystem}$ with Commitment Scheme $\varCommitScheme$, Two-Party Signature Scheme $\varSigSchemeMP$, and Range Proof System $\varRProofSystem$ consists of the following tuple of procedures:
    \[ \varMWSchemeParams{\varCommitScheme}{\varSigSchemeMP}{\varRProofSystem} \opAssign ( \procSendCoinsId,\procRecvCoinsId,\procFinTxId,\procVerfTxId ) \]
    \begin{itemize}
        \item $(\varPreTx, \funStar{\varSpendableCoinAlice}, (\varSecKeyAlice, \varNonceAlice)) \opFunResult \procSendCoins{\funArray{\varSpendableCoin}}{\varFundValue}{\varTime}$: The spendCoins algorithm is a DPT function called by the sending party to initiate the spending of some input coins.
        As input, it takes a list of spendable coins $\funArray{\varSpendableCoin}$ and a value $\varFundValue$ which should be transferred to the receiver.
        Optionally a sender can pass a block height $\varTime$ to make this transaction only valid after a specific time.
        It outputs a pre-transaction $\varPreTx$ which can be sent to a receiver, Alice's spendable change output coin $\funStar{\varSpendableCoinAlice}$ as well as the senders signing key and secret nonce $(\varSecKeyAlice, \varNonceAlice)$ later used in the transaction signing process.
        \item $(\funStar{\varPreTx}, \funStar{\varSpendableCoinBob}) \opFunResult \procRecvCoins{\varPreTx}{\varFundValue}$: The receiveCoins algorithm is a DPT routine called by the receiver and takes as input a pre-transaction $\varPreTx$ and a fund value
        $\varFundValue$.
        It will output a modified pre-transaction $\funStar{\varPreTx}$ and Bob's new spendable output coin $\funStar{\varSpendableCoinBob}$, added to the transaction.
        At this stage, the transaction already has to be partially signed by the receiver.
        \item $\varTx \opFunResult \procFinTx{\varPreTx}{\varSecKeyAlice}{\varNonceAlice}$: The finalize algorithm is a DPT routine called by the transaction sender that takes as input a pre-transaction $\varPreTx$ and the senders signing key $\varSecKeyAlice$ and nonce $\varNonceAlice$.
        The function will output a finalized signed transaction $\varTx$.
        \item $\{1,0\} \opFunResult \procVerfTx{\varTx}$: The verification algorithm is the same as defined in the paper by Fuchsbauer et al.~\cite{fuchsbauer2019aggregate}, we still add it here for completeness.
        Note that in their work, one can find it under the name $\styleFunction{MW.Ver}$.
        We rename it here to $\procVerfId$ to fit with our naming scheme.
        If an invalid transaction is passed to the routine, it will output 0, 1 otherwise.
        Informally the algorithm verifies four conditions:
        \begin{enumerate}
            \item Condition 1: Every input and output coin only appears once in the transaction.
            \item Condition 2: The union of input and output coins is the empty set.
            \item Condition 3: For every output coin, the range proof verifies.
            \item Condition 4: The transaction signature verifies with the excess value of the transaction as the public key, which is calculated by summing up the output coins and subtracting the input coins. (See~\cref{sec:pre:mimblewimble})
        \end{enumerate}
    \end{itemize}
\end{definition}

We say a Mimblewimble Transaction Scheme is correct if the verification algorithm $\procVerfTxId$ returns 1 upon providing a transaction that is well balanced and contains a valid signature.
More formally:
\begin{definition}[Transaction Scheme Correctness]
    \label{def:atom:tx-scheme-correctness}
    For any transaction fund value $\varFundValue$ and list of spendable input coins $\funArray{\varSpendableCoin}$ with combined value $\varValue \opGreaterEq \varFundValue$ the following must hold:
    \[
        \Pr\left[
        \begin{array}{c}
            \: \procVerfTx{\varTx} \opEqNoQ 1
        \end{array}
        \middle\vert
        \begin{array}{l}
            \varFundValue \opSmEq \sum_{\varI \opAssign 0}^{\varI \opSm \varN}(\varSpendableCoin_{i}.\varValue) \\
            (\varPreTx, \cdot, (\varSecKeyAlice, \varNonceAlice)) \opFunResult \procSendCoins{\funArray{\varSpendableCoin}}{\varValue}{\cnstFalsum} \\
            (\funStar{\varPreTx}, \cdot) \opFunResult \procRecvCoins{\varPreTx}{\varFundValue} \\
            \varTx \opFunResult \procFinTx{\funStar{\varPreTx}}{\varSecKeyAlice}{\varNonceAlice}
        \end{array}
        \right]=1.
    \]
\end{definition}

In the following, we define the \emph{Extended Mimblewimble Transaction Scheme}, which intuitively extends the previous scheme with shared coin ownership functionalities, similar to multisignature addresses available in Bitcoin.

\begin{definition}[Extended Mimblewimble Transaction Scheme]
    \label{def:atom:ext-mw-tx-scheme}
    An extended Mimblewimble Transaction Scheme $\varextMWSchemeParams{\varCommitScheme}{\varSigSchemeMP}{\varMPRProofSystem}$ is an extension to $\varMWScheme$ with the following three distributed protocols:
    \begin{gather*}
        \varextMWSchemeParams{\varCommitScheme}{\varSigSchemeMP}{\varMPRProofSystem} \opAssign \\ \varMWSchemeParams{\varCommitScheme}{\varSigSchemeMP}{\varMPRProofSystem} \opConc (\procDSendCoinsId,\procDRecvCoinsId, \procDFinTxId)
    \end{gather*}
    Note that for this scheme, we require a Two-Party Range Proof System $\varMPRProofSystem$ as shown in~\cref{def:pre:mp-rangeproof}.
    Specifically, we need the system to provide a distributed proof computation protocol $\procDRProofId$.
    \begin{itemize}
        \item $\langle (\varPreTx, \funStar{\varSpendableCoinAlice}, (\varSecKeyAlice,\varNonceAlice)), (\varPreTx, \funStar{\varSpendableCoinCarol}, (\varSecKeyCarol,\varNonceCarol)) \rangle$ \\
        $\opFunResult \procDSendCoins{\funArray{\varSpendableCoinAlice}}{\funArray{\varSpendableCoinCarol}}{\varFundValue}{\varTime}$:
        The distributed coin spending algorithm takes as input a list of spendable input coins for which ownership is shared between Alice and Carol.
        Assume that both Alice and Carol own a coin $\varCoin$, then we have two blinding factors $\varBlindingFactorAlice, \varBlindingFactorCarol$, where $\varBlindingFactorAlice$ is known only to Alice and $\varBlindingFactorCarol$ only to Carol.
        Both blinding factors are required to spend the coin.
        Again optionally a block height $\varTime$ can be given to time lock the transaction.
        Similar to the single party version of the function its outputs are a pre-transaction $\varPreTx$ and change coin for each party $\funStar{\varSpendableCoinAlice}$ (resp. $\funStar{\varSpendableCoinCarol}$), and their signing information.
        \item $\langle (\funStar{\varPreTx}, \funStar{\varPtSpendableCoinBob}), (\funStar{\varPreTx}, \funStar{\varPtSpendableCoinCarol}) \rangle \opFunResult \procDRecvCoins{\varPreTx}{\varFundValue}$: The distributed coin receive procedure takes as input a pre-transaction $\varPreTx$ and a value $\varFundValue$ which should be transferred with the transaction.
        The distributed algorithm will generate an output coin with value $\varValue$, owned by both Bob and Carol (each knowing only a share of the coin Commitment's blinding factor).
        The output will be an updated pre-transaction $\funStar{\varPreTx}$, and the spendable shared output coins for each party $\funStar{\varPtSpendableCoinBob}$ (resp. $\funStar{\varPtSpendableCoinCarol}$).
        Note that the newly generated output coin can only be spent by both parties cooperating, as each share of the blinding factor is strictly required.
        We note here that creating more complex schemes in which a coin is spendable by knowing N out M keys would be possible by implementing Shamir's Secret Sharing algorithm, which can be found in~\cite{shamir1979share}.
        \item $\langle \varTx, \varTx \rangle \opFunResult \procDFinTx{\varPreTx}{\varSecKeyAlice}{\varNonceAlice}{\varSecKeyCarol}{\varNonceCarol}$: The distributed finalized transaction protocol has to be used to create a transaction spending a shared coin (i.e., the transaction was created with the $\procDSendCoinsId$ algorithm).
        In this case, we require signing information from both Alice and Carol.
    \end{itemize}
\end{definition}

Correctness is given very similarly to the standard scheme:

\begin{definition}[Extended Transaction Scheme Correctness]
    \label{def:atom:ext-tx-scheme-correctness}
    For any list of spendable coins $\funArray{\varSpendableCoin}$ with total value $\varValue$ greater than the transaction fund value $\varFundValue$ and split blinding factors $(\funArray{\varBlindingFactorAlice}, \funArray{\varBlindingFactorCarol})$, the following must hold:
    \[
        \Pr\left[
        \begin{array}{c}
            \: \procVerfTx{\varTx} \opEqNoQ 1
        \end{array}
        \middle\vert
        \begin{array}{l}
            \varFundValue \opSmEq \sum_{\varI \opAssign 0}^{\varI \opSm \varN}(\varSpendableCoin_{i}.\varValue) \\
            \langle (\varPreTx, \cdot, (\varSecKeyAlice, \varNonceAlice)), (\varPreTx, (\varSecKeyCarol, \varNonceCarol)) \rangle \opFunResult \\
            \procDSendCoins{\funArray{\varSpendableCoinAlice}}{\funArray{\varSpendableCoinCarol}}{\varFundValue}{\cnstFalsum} \\
            \langle (\funStar{\varPreTx}, \cdot)(\funStar{\varPreTx}, \cdot) \rangle \opFunResult \procDRecvCoins{\varPreTx}{\varFundValue} \\
            \varTx \opFunResult \procDFinTx{\funStar{\varPreTx}}{\varSecKeyAlice}{\varNonceAlice}{\varSecKeyCarol}{\varNonceCarol}
        \end{array}
        \right]=1.
    \]
\end{definition}

We define the \emph{Contract Mimblewimble Transaction Scheme}, which will extend the scheme with additional algorithms to create primitive contracts between the sending and receiving party.

\begin{definition}[Contract Mimblewimble Transaction Scheme]
    \label{def:atom:apt-ext-mw-tx-scheme}
    The contract version of the Mimblewimble Transaction Scheme updates the Extended Mimblewimble Transaction Scheme by providing a modified version of the single party receive routine and the distributed finalize transaction protocol.
    \begin{gather*}
        \varaptMWSchemeParams{\varCommitScheme}{\varSigSchemeMP}{\varMPRProofSystem} \opAssign \\ \varextMWSchemeParams{\varCommitScheme}{\varSigSchemeMP}{\varMPRProofSystem} \opConc (\procAptRecvCoinsId, \procDAptFinTxId)
    \end{gather*}
    \begin{itemize}
        \item $(\funStar{\varPreTx}, \funStar{\varSpendableCoinBob}, \varSigBob) \opFunResult \procAptRecvCoins{\varPreTx}{\varFundValue}{\varWit}$: The contract variant of the receive function takes an additional input, a secret witness value
        $\varWit$, hidden in the transaction signature and extracted by the other party after the completion of the protocol.
        Note that the routine also returns Bob's unmasked partial signature.
        The reason for this is that we later need the unmasked version to complete the signature and finalze the transaction.
        By not sharing this unmasked signature with Alice, Bob is the one who gets to finalize the transaction, which is different from the simpler protocol and is a crucial feature necessary for our Atomic Swap protocol.
        We want to stress here that $\procAptRecvCoinsId$ is only a single-party algorithm.
        We can only use it if we're going to create an output coin owned by a single receiver.
        It would, of course, be conceivable also to define a distributed version similar to $\procDRecvCoinsId$ of this functionality, allowing two receivers (or one of the two) to hide secret witness values, extractable later by the sender(s).
        However, as for the following protocols, such functionality is not needed, we omit it here.
        \item $\langle \varSigAliceBob, \varTx \rangle \opFunResult \procDAptFinTx{\funStar{\varPreTx}}{\varSecKeyAlice}{\varNonceAlice}{\varStatement}{\varSecKeyBob}{\varNonceBob}{\varSigBob}$: The finalize transaction algorithm's contract variant is a distributed protocol between the sender(s) and receiver.
        Additionally to the pre-transaction $\funStar{\varPreTx}$, the senders need to input their signing information.
        Bob needs to input the unmasked version of his partial signature as it is required for transaction completion.
        This protocol could also be implemented as a three-party protocol with two senders controlling a shared coin and a third receiver.
        However, in our case, which we will describe later in~\cref{sec:atom:protocols}, one of the two senders is also the receiver.
        We allowed ourselves to model this protocol as being between only two parties to simplify the formalization.
        In this version of the protocol, only Bob can finalize the transaction, which is different from $\procFinTxId$ and $\procDFinTxId$.
        The reason for that is that for the Atomic Swap execution, Bob needs to be the one in control of building the final transaction.
        If Alice were to build the final transaction before Bob, she will extract the witness value before the transaction has been published, which in the Atomic Swap scenario would mean she could steal the funds stored on the other chain.
        That is why the protocol does not return the final transaction $\varTx$ to Alice.
        Instead, the protocol will output the sender's partial signature, which Alice can later use to extract the final transaction's witness value.
    \end{itemize}
\end{definition}

Similar to before, we define Correctness for the adapted scheme:

\begin{definition}[Contract Transaction Scheme Correctness]
    \label{def:atom:apt-tx-scheme-correctness}
    For any transaction fund value $\varFundValue$ and list of input coins $\funArray{\varSpendableCoin}$ with combined value $\varValue \opGreaterEq \varFundValue$ and any witness value $\varX \opIn \cnstIntegersPrimeWithoutZero{\varPrime}$, the following must hold:
    \[
        \Pr\left[
        \begin{array}{c}
            \: \procVerfTx{\varTx} \opEqNoQ 1
        \end{array}
        \middle\vert
        \begin{array}{l}
            \varFundValue \opSmEq \sum_{\varI \opAssign 0}^{\varI \opSm \varN}(\varSpendableCoin_{i}.\varValue) \\
            (\varPreTx, \funStar{\varSpendableCoinAlice}, (\varSecKeyAlice, \varNonceAlice)) \opFunResult \procSendCoins{\funArray{\varSpendableCoin}}{\varFundValue}{\cnstFalsum} \\
            (\funStar{\varPreTx}, \funStar{\varSpendableCoinBob}, \varSigBob) \opFunResult \procAptRecvCoins{\varPreTx}{\varFundValue}{\varWit} \\
            \langle \varSigAliceCarol, \varTx \rangle \opFunResult \procDAptFinTx{\funStar{\varPreTx}}{\varSecKeyAlice}{\varNonceAlice}{\varStatement}{\varSecKeyCarol}{\varNonceCarol}{\varSigBob}
        \end{array}
        \right]=1.
    \]
\end{definition}


\section{Instantiation}\label{sec:atom:inst}
\urldef\urlgrinexplained\url{https://medium.com/@brandonarvanaghi/grin-transactions-explained-step-by-step-fdceb905a853}

In this section we will provide an instantiation of the transaction scheme definitions found in~\ref{def:atom:mw-tx-scheme},~\ref{def:atom:ext-mw-tx-scheme} and~\ref{def:atom:apt-ext-mw-tx-scheme}.
The instantiations can be implemented in a Cryptocurrency based on the Mimblewimble protocol such as Beam and Grin.

\subsection{Mimblewimble Transaction Scheme}

First we provide an instantiation of the simplest form of a transaction in which a sender wants to transfer some value $\varFundValue$ to a receiver.
For the execution of the protocol we assume to have access to a homomorphic commitment scheme such as Pedersen Commitment $\varCommitScheme$ as defined in definition~\ref{def:pre:pedersen}.
Furthermore we require a Rangeproof system $\varProofSystem$ as defined in~\ref{sec:pre:rangeproof} and a two-party signature scheme $\varSigSchemeMP$ as defined in~\ref{def:sig:two-party-sig}.

The make the pseudocode for the transaction protocol easier to read we first introduce two auxiliary functions $\procCreateCoinId$
and $\procCreatePreTxId$.
The coin creation function will take as input a value $\varValue$ and a blinding factor $\varBlindingFactor$. It will create and output a new spendable coin $\varSpendableCoin$ already containing a range proof $\varProof$ attesting to the statement that the coins value $\varValue$ is within the valid range as defined for the blockchain.
The transaction creation algorithm $\procCreatePreTxId$ takes as input a message $\varMsg$, a list of input coins $\funArray{\varCoinInp}$, a list of output coins $\funArray{\varCoinOut}$, a list of rangeproofs $\funArray{\varProof}$, a signature context $\varSigContext$, a list of commitments $\varCommitment$, a signature $\varSignature$, and a lock time $\varTime$ and will collect the input data into a transaction object.

\begin{center}
    \fbox{
    \begin{varwidth}{\textwidth}
        \procedure[linenumbering]{$\procCreateCoin{\varValue}{\varBlindingFactor}$} {
        \varCommitment \opFunResult \procCommit{\varValue}{\varBlindingFactor} \\
        \varProof \opFunResult \procProof{\varCoin}{\varValue}{\varBlindingFactor} \\
        \pcreturn (\varCommitment, \varBlindingFactor, \varValue, \varProof)
        }
        \procedure[linenumbering]{$\procCreatePreTx{\varMsg}{\funArray{\varCoinInp}}{\funArray{\varCoinOut}}{\funArray{\varProof}}{\varSigContext}{\funArray{\varCommitment}}{\varSignature}{\varTime}$}{
        \pcreturn ( \\
        \varMsg \opAssign \varMsg, \\
        \varInputs \opAssign \funArray{\varCoinInp}, \\
        \varOutputs \opAssign \funArray{\varCoinOut}, \\
        \varProofs \opAssign \funArray{\varProof}, \\
        \varSigContext \opAssign \varSigContext, \\
        \varCommits \opAssign \funArray{\varCommitment}, \\
        \varSignature \opAssign \varSignature, \\
        \varTime \opAssign \varTime
        )
        }
    \end{varwidth}
    }
\end{center}

In figure~\ref{fig:inst-mw-tx} we provide an instantiation of the Mimblewimble Transction Scheme using the auxiliary functions provided before.

In the $\procSendCoinsId$ function the sender creates his change output coin, which is the difference between the value stored in his input coins and the value which should be transferred to a receiver.
He sets up the signature context with his parameters and gets a pre-transaction $\varPreTx$, newly created spendable output coin $\varSpendableCoinAlice$, as well as a signing key $\varSecKeyAlice$ and secret nonce $\varNonceAlice$ as output.
The pre-transaction can then be sent to a receiver.
Note that this instantiation differs from the one described by Fuchsbauer et al.~\cite{fuchsbauer2019aggregate} in that the sender does not yet sign the transaction during $\procSendCoinsId$.
This has the reason that in our definition of the Two-Party Signature Scheme~\ref{def:sig:two-party-sig} the signature context $\varSigContext$ requires to be fully setup before a partial signature can be created, therefore signing can only start at the receivers turn, after the signature context has been completed.
In the referenced paper it is possible to start the signing earlier, because instead of using the notion of a two-party signing protocol, they instead rely on an aggregateable signature scheme.
The sender and receiver both will create their signatures which will then be aggregated into the final one.
However, we find that by using a two-party signature scheme for our formalization we are closer to what is implemented in practice~\footnote{\urlgrinexplained} .
Furthermore by starting the signing process at the receivers turn we avoid a potential problem:
If an adversary learns the already signed pre-transaction and transaction value $\varFundValue$ before the intended receiver, the adversary would be able to steal the coins by creating his malicious output coin together with his signature, which he could then aggregate to the senders pre-transaction.

In $\procRecvCoinsId$ the receiver of a pre-transaction will verify the senders proof $\varProofBob$, create his output coin $\varCoinOutBob$, add his parameters to the signature context and then create his partial signature $\varSigBob$.
The function returns an updated version of the pre-transaction $\varPreTx$ which can be sent back to the sender, as well as the newly created spendable output $\varSpendableCoinBob$.

Now in $\procFinTxId$ the original sender will validate the updated pre-transcation $\varPreTx$ sent to him by the receiver.
If he finds it as valid, he will only now create his partial signature and finally finalize the two partial signatures into the final composite one, with which he can then build the final transaction.

\begin{figure}
    \begin{center}
        \fbox{
        \begin{varwidth}{\textwidth}
            \procedure[linenumbering]{$\procSendCoins{\funArray{\varSpendableCoin}}{\varFundValue}{\varTime}$} {
            \varValue \opFunResult \sum_{\varI \opAssign 0}^{\varI \opSm \varN}(\varSpendableCoin_{i}.\varValue) \\
            \pcif \varFundValue \opGreaterThen \varValue
            \t \pcreturn \cnstFalsum \\
            \pcif \opExists \varI \opNotEq \varJ : \varSpendableCoin[\varI] \opEqNoQ \varSpendableCoin[\varJ]
            \t \pcreturn \cnstFalsum \\
            \varMsg \opAssign \cnstBinary{*} \\
            (\funStar{\varBlindingFactorAlice}, \varNonceAlice) \sample \cnstIntegersPrimeWithoutZero{\varPrime} \< \< \\
            \funStar{\varSpendableCoinAlice} \opFunResult \procCreateCoin{\varValue \opSub \varFundValue}{\funStar{\varBlindingFactorAlice}} \\
            \{ \varCoinOutAlice, \funStar{\varBlindingFactorAlice}, \varValueAlice, \varProofAlice \} \opFunResult \funStar{\varSpendableCoinAlice} \\
            \varSecKeyAlice \opAssign \funStar{\varBlindingFactorAlice} \opSub \sum_{\varI \opAssign 0}^{\varI \opSm \varN}(\varSpendableCoin_{i}.\varBlindingFactor) \\
            \varSigContext \opAssign \{ \varPubKey \opAssign \cnstIdentityElement, \varRand \opAssign \cnstIdentityElement \} \\
            \varSigContext \opFunResult \procSetupCtx{\varSigContext}{\funGen{\varSecKeyAlice}}{\funGen{\varNonceAlice}} \\
            \varPreTx \opFunResult \procCreatePreTx{\varMsg}{\varSpendableCoin.\varCommitment}{\funArray{\varCoinOutAlice}}{\funArray{\varProofAlice}}{\varSigContext}{\funArray{\funGen{\varSecKeyAlice}}}{\cnstEmptySet}{\varTime} \\
            \pcreturn (\varPreTx, \funStar{\varSpendableCoinAlice}, (\varSecKeyAlice, \varNonceAlice))
            } \\
            \procedure[linenumbering]{$\procRecvCoins{\varPreTx}{\varFundValue}$} {
            (\varMsg,\varInputs,\varOutputs,\varProofs,\varSigContext,\varCommits,\cnstEmptySet,\varTime) \opFunResult \varPreTx \\
            \pcif \procVerfProof{\varProofs[0]}{\varOutputs[0]} \opEqNoQ 0 \\
            \t \pcreturn \cnstFalsum \\
            (\funStar{\varBlindingFactorBob},\varNonceBob) \sample \cnstIntegersPrimeWithoutZero{\varPrime} \\
            \funStar{\varSpendableCoinBob} \opFunResult \procCreateCoin{\varFundValue}{\funStar{\varBlindingFactorBob}} \\
            \{ \varCoinOutBob, \funStar{\varBlindingFactorBob}, \varValueBob, \varProofBob \} \opFunResult \funStar{\varSpendableCoinBob} \\
            \varSecKeyBob \opAssign \funStar{\varBlindingFactorBob} \\
            \varSigContext \opFunResult \procSetupCtx{\varSigContext}{\funGen{\varSecKeyBob}}{\funGen{\varNonceBob}} \\
            \varSigBob \opFunResult \procSignPrt{\varMsg}{\varSecKeyBob}{\varNonceBob}{\varSigContext} \\
            \varPreTx \opFunResult \procCreatePreTx{\varMsg}{\varInputs}{\varOutputs \opConc \varCoinOutBob}{\varProofs \opConc \varProofBob}{\varSigContext}{\varCommits \opConc \funGen{\varSecKeyBob}}{\varSigBob}{\varTime} \\
            \pcreturn (\varPreTx, \funStar{\varSpendableCoinBob})
            } \\
            \procedure[linenumbering]{$\procFinTx{\varPreTx}{\varSecKeyAlice}{\varNonceAlice}$} {
            (\varMsg,\varInputs,\varOutputs,\varProofs,\varSigContext,\varCommits,\varSigBob,\varTime) \opFunResult \varPreTx \\
            \pcif \procVerfProof{\varProofs[1]}{\varOutputs[1]} \opEqNoQ 0 \\
            \t \pcreturn \cnstFalsum \\
            \pcif \procVerfPtSig{\varSigBob}{\varMsg}{\varCommits[1]} \opEqNoQ 0 \\
            \t \pcreturn \cnstFalsum \\
            \varSigAlice \opFunResult \procSignPrt{\varMsg}{\varSecKeyAlice}{\varNonceAlice}{\varSigContext} \\
            \varSigFin \opFunResult \procFinSig{\varSigAlice}{\varSigBob} \\
            \varTx \opFunResult \procCreatePreTx{\varMsg}{\varInputs}{\varOutputs}{\varProofs}{\varSigContext}{\varCommits}{\varSigFin}{\varTime} \\
            \pcreturn \varTx
            } \\
            \procedure[linenumbering]{$\procVerfTx{\varTx}$} {
            (\varMsg,\varInputs,\varOutputs,\varProofs,\varSigContext,\varCommits,\varSignature,\varTime) \opFunResult \varTx \\
            \varExcess \opEqNoQ \sum(\varOutputs) \opSub \sum(\varInputs) \\
            \pcreturn (\opForAll \varI \opNotEq \varJ : \varInputs[\varI] \opNotEq \varInputs[\varJ] \opAnd \varOutputs[\varI] \opNotEq \varOutputs[\varJ]) \text{ and } \pcskipln \\
                \t \varInputs \opUnion \varOutputs \opEqNoQ \cnstEmptySet \text{ and } (\opForAll \varI : \procVerfProof{\varProofs[\varI]}{\varOutputs[\varI]}) \text{ and } \procVerf{\varMsg}{\varSignature}{\varExcess}
            }
        \end{varwidth}
        }
    \end{center}
    \caption{Instantiation of Mimblewimble Transaction Scheme. \label{fig:inst-mw-tx}}
\end{figure}

\subsection{Extended Mimblewimble Transaction Scheme}\label{subsec:atom:ext-tx-scheme}

Figure~\ref{fig:ext-mim-tx-spend} shows an instantiation of the $\procDSendCoinsId$ function of the Extended Mimblewimble Transaction Scheme.
We have an array of spendable input coins which keys are shared between two parties Alice and Carol.
We use Carol here to not confuse this party with the receiver, which we previously called Bob.
Although Carol and Bob could be the same person, they not necessarily have to be.

The protocol starts with both Alice and Carol creating her change outputs with values $\varValueAlice$ and $\varValueCarol$.
Alice then creates the initial pre-transaction $\varPreTx$ and sends it to Carol who verifies Alice's output, adds her outputs and parameters and sends back $\varPreTx$, which Alice verifies.
The protocol returns $\varPreTx$ to both parties, which can then be transmitted to the receiver by any of the two parties, as well as the secret signing information $(\varSecKeyAlice, \varNonceAlice)$, $(\varSecKeyCarol, \varNonceCarol)$.

\newgeometry{margin=2cm}
\begin{landscape}
    \thispagestyle{plain}
    \begin{figure}
        \fbox{
        \procedure[linenumbering,skipfirstln]{$\procDSendCoins{\funArray{\varPtSpendableCoinAlice}}{\funArray{\varPtSpendableCoinCarol}}{\varFundValue}{\varTime}$}{
        Alice \< \< Carol \\
        \varValue \opFunResult \sum_{\varI \opAssign 0}^{\varI \opSm \varN}(\varSpendableCoin_{i}.\varValue) \< \< \varValue \opFunResult \sum_{\varI \opAssign 0}^{\varI \opSm \varN}(\varSpendableCoin_{i}.\varValue) \\
        \pcif \varFundValue \opGreaterThen \varValue \< \< \pcif \varFundValue \opGreaterThen \varValue \\
        \t \pcreturn \cnstFalsum \< \< \t \pcreturn \cnstFalsum \\
        \pcif \opExists \varI \opNotEq \varJ : \varPtSpendableCoinAlice[\varI] \opEqNoQ \varPtSpendableCoinAlice[\varJ] \< \< \pcif \opExists \varI \opNotEq \varJ : \varPtSpendableCoinCarol[\varI] \opEqNoQ \varPtSpendableCoinCarol[\varJ] \\
        \t \pcreturn \cnstFalsum \< \< \pcreturn \cnstFalsum \\
        \varMsg \opAssign \cnstBinary{*} \\
        (\funStar{\varBlindingFactorAlice}, \varNonceAlice) \sample \cnstIntegersPrimeWithoutZero{\varPrime} \< \< (\funStar{\varBlindingFactorCarol}, \varNonceCarol) \sample \cnstIntegersPrimeWithoutZero{\varPrime} \\
        \funStar{\varSpendableCoinAlice} \opFunResult \procCreateCoin{\varValueAlice}{\funStar{\varBlindingFactorAlice}} \< \< \funStar{\varSpendableCoinCarol} \opFunResult \procCreateCoin{\varValueCarol}{\funStar{\varBlindingFactorCarol}} \\
        \{ \varCoinOutAlice, \funStar{\varBlindingFactorAlice}, \varValueAlice, \varProofAlice \} \opFunResult \funStar{\varSpendableCoinAlice} \< \< \{ \varCoinOutCarol, \funStar{\varBlindingFactorCarol}, \varValueCarol, \varProofCarol \} \opFunResult \funStar{\varSpendableCoinCarol} \\
        \varSecKeyAlice \opAssign \funStar{\varBlindingFactorAlice} \opSub \sum \funArray{\varBlindingFactorAlice} \< \< \varSecKeyCarol \opAssign \funStar{\varBlindingFactorCarol} \opSub \sum \funArray{\varBlindingFactorCarol} \\
        \varSigContext \opAssign \{ \varPubKey \opAssign \cnstIdentityElement, \varRand \opAssign \cnstIdentityElement \} \< \< \\
        \varSigContext \opFunResult \procSetupCtx{\varSigContext}{\funGen{\varSecKeyAlice}}{\funGen{\varNonceAlice}} \< \< \\
        \varPreTx \opFunResult \pcskipln \\
        \procCreatePreTx{\varMsg}{\funArray{\varCoinInp}}{\funArray{\varCoinOutAlice}}{\funArray{\varProofAlice}}{\varSigContext}{\funArray{\funGen{\varNonceAlice}}}{\cnstEmptySet}{\varTime} \< \< \\
        \< \sendmessageright*{\varPreTx} \< \\
        \< \< (\varMsg,\varInputs,\varOutputs,\varProofs,\varSigContext,\varCommits,\funStar{\varTime}) \opFunResult \varPreTx \\
        \< \< \pcif \procVerfProof{\varProofs[0]}{\varOutputs[0]} \opEqNoQ 0 \opOr \varTime \opNotEq \funStar{\varTime} \\
        \< \< \t \pcreturn \cnstFalsum \\
        \< \< \varSigContext \opFunResult \procSetupCtx{\varSigContext}{\funGen{\varSecKeyCarol}}{\funGen{\varNonceCarol}} \\
        \< \< \funStarAlt{\varPreTx} \opFunResult \procCreatePreTx{\varMsg}{\varInputs}{\varOutputs \opConc \varCoinOutCarol}{\varProof \opConc \varProofCarol}{\varSigContext}{\varCommits \opConc \funGen{\varNonceCarol}}{\cnstEmptySet}{\varTime} \\
        \< \sendmessageleft*{\funStarAlt{\varPreTx}} \< \\
        \pcif \procVerfProof{\funStarAlt{\varPreTx}.\varProofs[1]}{\funStarAlt{\varPreTx}.\varOutputs[1]} \opEqNoQ 0 \< \< \\
        \t \pcreturn \cnstFalsum \< \< \\
        \pcreturn (\funStarAlt{\varPreTx}, \funStar{\varSpendableCoinAlice}, (\varSecKeyAlice, \varNonceAlice)) \< \< \pcreturn (\funStarAlt{\varPreTx}, \funStar{\varSpendableCoinCarol}, (\varSecKeyCarol, \varNonceCarol))
        }
        }
        \caption{Extended Mimblewimble Transaction Scheme - $\procDSendCoinsId$ \label{fig:ext-mim-tx-spend}}
    \end{figure}
\end{landscape}
\restoregeometry

Figure~\ref{fig:ext-mim-tx-recv} shows an instantiation of the $\procRecvCoinsId$ function of the Extended Mimblewimble Transaction Scheme.
Calling this protocol two receivers Bob and Carol want to create a receiving shared coin $\varCoinShared$ with value $\varFundValue$ and key shares $(\varBlindingFactorAlice, \varBlindingFactorCarol)$.
The protocol starts by both receivers verifing the senders output(s).
Bob starts by creating a coin with fund value $\varFundValue$ and his share of the newly create blinding factor and sends it over to Carol.
Carol finalizes the shared coin by adding a commitment to her blinding factor to the coin and sends it back, together with the commitment.
Bob verifies validity of the updated shared coin after which the two parties engage in two two-party protocols to create their partial signature and coin rangeproof.
Finally they create the updated pre-transaction $\varPreTx$ which can be sent back to the transaction sender.

\newgeometry{margin=2cm}
\begin{landscape}
    \thispagestyle{plain}
    \begin{figure}
        \fbox{
        \procedure[linenumbering,skipfirstln]{$\procDRecvCoins{\varPreTx}{\varFundValue}$} {
        Bob \< \< \< \< Carol \\
        (\varMsg,\varInputs,\varOutputs,\varProofs,\varSigContext,\varCommits,\cnstEmptySet,\varTime) \opFunResult \varPreTx \\
        \pcforeach \varOutputs \textit{ as } (\varIterator => \varCoinOut) \< \< \< \< \pcforeach \varOutputs \textit{ as } (\varIterator => \varCoinOut) \\
        \t \pcif \procVerfProof{\varProofs[\varIterator]}{\varCoinOut[\varIterator]} \opEqNoQ 0 \< \< \< \< \t \pcif \procVerfProof{\varProofs[\varIterator]}{\varCoinOut[\varIterator]} \opEqNoQ 0 \\
        \t \pcreturn \cnstFalsum \< \< \< \< \t \pcreturn \cnstFalsum \\
        (\funStar{\varBlindingFactorBob}, \varNonceBob) \sample \cnstIntegersPrimeWithoutZero{\varPrime} \< \< \< \< \\
        (\varCoinShared, \cdot, \cdot, \cdot) \opFunResult \procCreateCoin{\varFundValue}{\funStar{\varBlindingFactorBob}} \< \< \< \< \\
        \varSecKeyBob \opAssign \funStar{\varBlindingFactorBob} \< \< \< \< \\
        \< \sendmessagerightx{4}{\varPreTx, \varCoinShared} \< \\
        \< \< \< \< (\funStar{\varBlindingFactorCarol}, \varNonceCarol) \sample \cnstIntegersPrimeWithoutZero{\varPrime} \\
        \< \< \< \< \varSecKeyCarol \opAssign \funStar{\varBlindingFactorCarol} \\
        \< \< \< \< \funStarAlt{\varCoinShared} \opAssign \varCoinShared \opAddPoint \funGen{\varBlindingFactorCarol} \\
        \< \< \< \< \varPreTx.\varOutputs[] \opAssign \funStarAlt{\varCoinShared} \\
        \< \sendmessageleftx{4}{\varPreTx, \funGen{\varSecKeyCarol}} \< \\
        \{\cdots \funStarAlt{\varCoinShared}\} \opFunResult \varPreTx.\varOutputs \< \< \< \< \\
        \pcif \funStarAlt{\varCoinShared} \opNotEq \varCoinShared \opAddPoint \funGen{\varSecKeyCarol} \< \< \< \< \\
        \t \pcreturn \cnstFalsum \< \< \< \< \\
        \varProofBobCarol \opFunResult \procDRProofL{\funStarAlt{\varCoinShared}}{\varFundValue}{\varSecKeyBob} \< \< \< \< \varProofBobCarol \opFunResult \procDRProofL{\funStarAlt{\varCoinShared}}{\varFundValue}{\varSecKeyCarol} \\
        \funStar{\varPtSpendableCoinBob} \opAssign \{ \varCoinShared, \varFundValue, \funStar{\varBlindingFactorBob}, \varProofBobCarol \} \< \< \< \< \funStar{\varPtSpendableCoinCarol} \opAssign \{ \varCoinShared, \varFundValue, \funStar{\varBlindingFactorCarol}, \varProofBobCarol \} \\
        (\varSigBobCarol, \varPubKeyBobCarol) \opFunResult \procDSignL{\varMsg}{\varSecKeyBob}{\varNonceBob} \< \< \< \< (\varSigBobCarol, \varPubKeyBobCarol) \opFunResult \procDSignR{\varMsg}{\varSecKeyCarol}{\varNonceCarol} \\
        (\cdot, \cdot, \funStar{\varSigContext}) \opFunResult \varSigBobCarol \< \< \< \< (\cdot, \cdot, \funStar{\varSigContext}) \opFunResult \varSigBobCarol \\
        \funStarAlt{\varSigContext} \opFunResult \procSetupCtx{\varSigContext}{\funStar{\varSigContext}.\varPubKey}{\funStar{\varSigContext}.\varRand} \< \< \< \< \funStarAlt{\varSigContext} \opFunResult \procSetupCtx{\varSigContext}{\funStar{\varSigContext}.\varPubKey}{\funStar{\varSigContext}.\varRand} \\
        \< \varPreTx \opFunResult \procCreatePreTx{\varMsg}{\varInputs}{\varOutputs \opConc \funStarAlt{\varCoinShared}}{\varProofs \opConc \varProofBobCarol}{\funStarAlt{\varSigContext}}{\varCommits \opConc \varPubKeyBobCarol}{\varSigBobCarol}{\varTime} \< \\
        \pcreturn (\varPreTx, \funStar{\varPtSpendableCoinBob}) \< \< \< \< \pcreturn (\varPreTx, \funStar{\varPtSpendableCoinCarol})
        }
        }
        \caption{Extended Mimblewimble Transaction Scheme - $\procDRecvCoinsId$ \label{fig:ext-mim-tx-recv}}
    \end{figure}
\end{landscape}
\restoregeometry

\begin{figure}
    \fbox{
    \procedure[linenumbering,skipfirstln]{$\procDFinTx{\varPreTx}{\varSecKeyAlice}{\varNonceAlice}{\varSecKeyCarol}{\varNonceCarol}$} {
    Alice \< \< Carol \\
    (\varMsg,\varInputs,\varOutputs,\varProofs,\varSigContext,\varCommits,\varSigBob,\varTime) \opFunResult \varPreTx \< \< (\varMsg,\varInputs,\varOutputs,\varProofs,\varSigContext,\varCommits,\varSigBob,\varTime) \opFunResult \varPreTx \\
    \pcif \procVerfProof{\varProofs[1]}{\varOutputs[1]} \opEqNoQ 0 \< \< \pcif \procVerfProof{\varProofs[1]}{\varOutputs[1]} \opEqNoQ 0 \\
    \t \pcreturn \cnstFalsum \< \< \t \pcreturn \cnstFalsum \\
    \pcif \procVerfPtSig{\varSigBob}{\varMsg}{\varCommits[1]} \opEqNoQ 0 \< \< \pcif \procVerfPtSig{\varSigBob}{\varMsg}{\varCommits[1]} \opEqNoQ 0 \\
    \t \pcreturn \cnstFalsum \< \< \t \pcreturn \cnstFalsum \\
    \varSigAliceCarol \opFunResult \procDSignL{\varMsg}{\varSecKeyAlice}{\varNonceAlice} \< \< \varSigAliceCarol \opFunResult \procDSignR{\varMsg}{\varSecKeyCarol}{\varNonceCarol} \\
    \varSigFin \opFunResult \procFinSig{\varSigBob}{\varSigAliceCarol} \< \< \varSigFin \opFunResult \procFinSig{\varSigBob}{\varSigAliceCarol} \\
    \varTx \opFunResult \procCreatePreTx{\varMsg}{\varInputs}{\varOutputs}{\varProofs}{\varSigContext}{\varCommits}{\varSigFin}{\varTime} \< \< \varTx \opFunResult \procCreatePreTx{\varMsg}{\varInputs}{\varOutputs}{\varProofs}{\varSigContext}{\varCommits}{\varSigFin}{\varTime} \\
    \pcreturn \varTx \< \< \pcreturn \varTx
    }
    }
    \caption{Extended Mimblewimble Transaction Scheme - $\procDFinTxId$ \label{fig:ext-mim-tx-fin}}
\end{figure}

\subsection{Adapted Extended Mimblewimble Transaction Scheme}

Figure~\ref{fig:inst-apt-mw-tx-recv} shows an instantiation of the $\procAptRecvCoinsId$ algorithm.
Before updating the pre-transaction $\varPreTx$ Bob adapts his partial signature with the witness value $\varWit$.
The procedure then returns the pre-transaction $\varPreTx$ containing Bobs adapted partial signature, and the statement $\varStatement$ which is a commitment to the witness value $\varWit$.

\begin{figure}
    \begin{center}
        \fbox{
        \begin{varwidth}{\textwidth}
            \procedure[linenumbering]{$\procAptRecvCoins{\varPreTx}{\varFundValue}{\varWit}$} {
            (\varMsg,\varInputs,\varOutputs,\varProofs,\varSigContext,\varCommits,\cnstEmptySet, \varTime) \opFunResult \varPreTx \\
            \pcif \procVerfProof{\varProofs[0]}{\varOutputs[0]} \opEqNoQ 0 \\
            \t \pcreturn \cnstFalsum \\
            (\funStar{\varBlindingFactorBob},\varNonceBob) \sample \cnstIntegersPrimeWithoutZero{\varPrime} \\
            (\varCoinOutBob,\varProofBob) \opFunResult \procCreateCoin{\varFundValue}{\funStar{\varBlindingFactorBob}} \\
            \varSecKeyBob \opAssign \funStar{\varBlindingFactorBob} \\
            \varSigContext \opFunResult \procSetupCtx{\varSigContext}{\funGen{\varSecKeyBob}}{\funGen{\varNonceBob}} \\
            \varSigBob \opFunResult \procSignPrt{\varMsg}{\varSecKeyBob}{\varSigContext.\varPubKey}{\varSigContext.\varRand} \\
            \varSigAptBob \opFunResult \procAptSig{\varSigBob}{\varWit} \\
            \varPreTx \opFunResult \procCreatePreTx{\varMsg}{\varInputs}{\varOutputs \opConc \varCoinOutBob}{\varProofs \opConc \varProofBob}{\varSigContext}{\varCommits \opConc \funGen{\varNonceBob}}{\varSigAptBob}{\varTime} \\
            \pcreturn (\varPreTx, (\varCoinOutBob, \funStar{\varBlindingFactorBob}),\varSigBob)
            }
        \end{varwidth}
        }
    \end{center}
    \caption{Adapted Extended Mimblewimble Transaction Scheme - $\procAptRecvCoinsId$. \label{fig:inst-apt-mw-tx-recv}}
\end{figure}

In figure~\ref{fig:inst-apt-mw-tx-fin} we show the updated distributed version of the transaction finalization protocol.
Again Alice verifies the pre-transaction $\varPreTx$ received by Bob and then cooperates with Bob in the $\procDSignId$ protocol to build the partial signature for their shared coin.
Note that at this point Alice is not able to finalize the signature (and consequently the transaction) as she only knows Bobs adapted partial signature ($\varSigAptBob$), but not the original one ($\varSigBob$), which is needed for the $\procFinSigId$ function.
Therefore, Bob completes the transaction and outputs it, while Alice outputs $\varSigAliceBob$ with which see can then retreive $\varX$.

\begin{figure}
    \fbox{
    \procedure[linenumbering,skipfirstln]{$\procDAptFinTx{\varPreTx}{\varSecKeyAlice}{\varNonceAlice}{\varStatement}{\varSecKeyBob}{\varNonceBob}{\varSigBob}$} {
    Alice \< \< Bob \\
    (\varMsg,\varInputs,\varOutputs,\varProofs,\varSigContext,\varCommits,\varSigAptBob,\varTime) \opFunResult \varPreTx \< \< (\varMsg,\varInputs,\varOutputs,\varProofs,\varSigContext,\varCommits,\varSigAptBob,\varTime) \opFunResult \varPreTx \\
    \pcif \procVerfProof{\varProofs[1]}{\varOutputs[1]} \opEqNoQ 0 \< \< \\
    \t \pcreturn \cnstFalsum \< \< \\
    \pcif \procVerifyAptSig{\varSigBob}{\varMsg}{\varCommits[1]}{\varStatement} \opEqNoQ 0 \< \< \\
    \t \pcreturn \cnstFalsum \< \< \\
    \varSigAliceBob \opFunResult \procDSignL{\varMsg}{\varSecKeyAlice}{\varNonceAlice} \< \< \varSigAliceBob \opFunResult \procDSignL{\varMsg}{\varSecKeyBob}{\varNonceBob} \\
    \< \< \varSigFin \opFunResult \procFinSig{\varSigAliceCarol}{\varSigBob} \\
    \< \< \varTx \opFunResult \procCreatePreTx{\varMsg}{\varInputs}{\varOutputs}{\varProofs}{\varSigContext}{\varCommits}{\varSigFin}{\varTime} \\
    \pcreturn \varSigAliceBob \< \< \pcreturn \varTx
    }
    }
    \caption{Adapted Extended Mimblewimble Transaction Scheme - $\procDAptFinTxId$. \label{fig:inst-apt-mw-tx-fin}}
\end{figure}

\section{Protocols}\label{sec:atom:protocols}
This section specifies three protocols to build Mimblewimble transactions from the definitions found in~\cref{sec:atom:definitions}.
Later in~\cref{sec:atom:security}, we will prove the security of these protocols, and finally, in ~\cref{sec:atom:atomic-swap}, we will utilize them to build our Atomic Swap.

\subsection{Simple Mimblewimble Transaction - $\procDBuildMwTxId$} \label{subsec:atom:simple-mw-tx}

$\procDBuildMwTxId$ is a protocol between a sender and receiver which builds a Mimblewimble transaction transferring a value $\varFundValue$ from the sender to a receiver for a Mimblewimble Transaction Scheme as defined in~\cref{def:atom:mw-tx-scheme}.
It takes as input a list of spendable coins $\funArray{\varSpendableCoin}$, a transaction value $\varFundValue$, and an optional timelock $\varTime$ from the sender, the same transaction value $\varFundValue$ from the receiver, and uses the functions defined earlier to output a valid transaction $\varTx$ as well as the newly spendable coins to both parties.
\[ \langle (\varTx, \funStar{\varSpendableCoinAlice}), (\varTx, \funStar{\varSpendableCoinBob}) \rangle \opFunResult \procDBuildMwTx{\funStar{\varSpendableCoin}}{\varFundValue}{\varTime} \]
\Cref{fig:d-build-mw-tx} shows the implementation of the $\procDBuildMwTxId$.

\begin{figure}
    \begin{center}
    \fbox{
    \begin{varwidth}{\textwidth}
        \procedure[linenumbering,skipfirstln]{$\procDBuildMwTx{\funArray{\varSpendableCoin}}{\varFundValue}{\varTime}$}{
        Alice \< \< Bob \\
        (\varPreTx, \funStar{\varSpendableCoinAlice}, (\varSecKeyAlice, \varNonceAlice)) \pcskipln \\
        \opFunResult \procSendCoins{\funArray{\varSpendableCoin}}{\varFundValue}{\varTime} \\
        \< \sendmessageright*{\varPreTx} \< \\
        \< \< (\funStarAlt{\varPreTx}, \funStar{\varSpendableCoinBob}) \opFunResult \procRecvCoins{\varPreTx}{\varFundValue} \\
        \< \sendmessageleft*{\funStarAlt{\varPreTx}} \\
        \varTx \opFunResult \procFinTx{\funStarAlt{\varPreTx}}{\varSecKeyAlice}{\varNonceAlice} \\
        \< \sendmessageright*{\varTx} \\
        \pcreturn (\varTx, \funStar{\varSpendableCoinAlice}) \< \< \pcreturn (\varTx, \funStar{\varSpendableCoinBob})
        }
    \end{varwidth}
    }
    \end{center}
    \caption{$\procDBuildMwTxId$ two-party protocol to build a new transaction} \label{fig:d-build-mw-tx}
\end{figure}

\subsection{Shared Output Mimblewimble Transaction - $\procDSharedOutputMwTxId$} \label{subsec:atom:shared-out-mw-tx}

$\procDSharedOutputMwTxId$ is a protocol between a sender and a receiver.
It builds a Mimblewimble transaction transferring value from a sender for the Extended Mimblewimble Transaction Scheme in ~\cref{def:atom:ext-mw-tx-scheme}.
However, instead of simply sending value to a receiver, it sends it to a shared coin, for which both the sender and receiver know one part of the opening.
As input, it again takes a list of spendable coins $\funArray{\varSpendableCoin}$, a transaction value $\varFundValue$ and an optional timelock $\varTime$ from the sender, and the same transaction value $\varFundValue$ from the receiver.
It outputs the final transaction $\varTx$ to both parties, Alice will receive her spendable change output $\funStar{\varSpendableCoinAlice}$ and both parties will receive their part of the shared spendable coin $\funStar{\varPtSpendableCoinAlice}$, $\funStar{\varPtSpendableCoinBob}$.

\[ \langle (\varTx, \funStar{\varSpendableCoinAlice}, \funStar{\varPtSpendableCoinAlice}), (\varTx, \funStar{\varPtSpendableCoinBob}) \rangle \opFunResult \procDSharedOutputMwTx{\funArray{\varSpendableCoin}}{\varFundValue}{\varTime} \]

One use case of this transaction protocol is to lock funds between two users, which can then be redeemed by both parties cooperating.

\Cref{fig:d-shared-out-mw-tx} shows the implementation of the protocol.

\begin{figure}
    \begin{center}
    \fbox{
    \begin{varwidth}{\textwidth}
        \procedure[linenumbering,skipfirstln]{$\procDSharedOutputMwTx{\funArray{\varSpendableCoin}}{\varFundValue}{\varTime}$}{
        Alice \< \< Bob \\
        (\varPreTx, \funStar{\varSpendableCoinAlice}, (\varSecKeyAlice, \varNonceAlice)) \pcskipln \\
        \opFunResult \procSendCoins{\funArray{\varSpendableCoin}}{\varFundValue}{\varTime} \\
        \< \sendmessageright*{\varPreTx} \< \\
        (\funStarAlt{\varPreTx}, \funStar{\varPtSpendableCoinAlice}) \< \< (\funStarAlt{\varPreTx}, \funStar{\varPtSpendableCoinBob}) \pcskipln \\
        \opFunResult \procDRecvCoinsL{\varPreTx}{\varFundValue}  \< \< \opFunResult \procDRecvCoinsR \\
        \varTx \opFunResult \procFinTx{\funStarAlt{\varPreTx}}{\varSecKeyAlice}{\varNonceAlice} \\
        \< \sendmessageright*{\varTx} \\
        \pcreturn (\varTx, \funStar{\varSpendableCoinAlice}, \funStar{\varPtSpendableCoinAlice}) \< \< \pcreturn (\varTx, \funStar{\varPtSpendableCoinBob})
        }
    \end{varwidth}
    }
    \end{center}
    \caption{$\procDSharedOutputMwTxId$ two-party protocol to build a new transaction with a shared output} \label{fig:d-shared-out-mw-tx}
\end{figure}

\subsection{Shared Input Mimblewimble Transaction $\procDSharedInpMwTxId$} \label{subsec:atom:shared-inp-mw-tx}

$\procDSharedInpMwTxId$ is a protocol between a sender and a receiver.
It builds a Mimblewimble transaction transferring value from a coin shared between the sender and receiver to a receiver again for the Extended Mimblewimble Transaction Scheme outlined in~\cref{def:atom:ext-mw-tx-scheme}
As input, it takes a list of partial spendable coins $\funArray{\varPtSpendableCoinAlice}$, a transaction value $\varFundValue$, an optional timelock $\varTime$ from the sender, and the other part of the shared spendable coins $\varPtSpendableCoinBob$ and the same transaction value $\varFundValue$ from the receiver.
It outputs a final transaction $\varTx$ to both parties and the new outputs $\funStar{\varSpendableCoinAlice}, \funStar{\varSpendableCoinBob}$ to the respective owner.

\[ \langle (\varTx, \funStar{\varSpendableCoinAlice}), (\varTx, \funStar{\varSpendableCoinBob}) \rangle \opFunResult \procDSharedInpMwTx{\funArray{\varPtSpendableCoinAlice}}{\varFundValue}{\varTime}{\funArray{\varPtSpendableCoinBob}} \]

The protocol can be used to redeem funds that are locked created with the \\ $\procDSharedInpMwTxId$ protocol.

\Cref{fig:d-shared-inp-mw-tx} shows the implementation of the protocol.

\begin{figure}
    \begin{center}
    \fbox{
    \begin{varwidth}{\textwidth}
        \procedure[linenumbering,skipfirstln]{$\procDSharedInpMwTx{\funArray{\varPtSpendableCoinAlice}}{\varFundValue}{\varTime}{\funArray{\varPtSpendableCoinBob}}$}{
        Alice \< \< Bob \\
        (\varPreTx, \funStar{\varSpendableCoinAlice}, (\varSecKeyAlice, \varNonceAlice)) \< \< (\varPreTx, (\varSecKeyBob, \varNonceBob)) \pcskipln \\
        \opFunResult \procDSendCoinsL{\funArray{\varPtSpendableCoinAlice}}{\varFundValue}{\varTime} \< \< \opFunResult \procDSendCoinsL{\funArray{\varPtSpendableCoinBob}}{\varFundValue}{\varTime} \\
        \< \< (\funStarAlt{\varPreTx}, \funStar{\varSpendableCoinBob}) \opFunResult \procRecvCoins{\varPreTx}{\varFundValue} \\
        \< \sendmessageleft*[2cm]{\funStarAlt{\varPreTx}} \\
        \varTx \opFunResult \procDFinTxL{\funStarAlt{\varPreTx}}{\varSecKeyAlice}{\varNonceAlice} \< \< \varTx \opFunResult \procDFinTxL{\funStarAlt{\varPreTx}}{\varSecKeyBob}{\varNonceBob} \\
        \pcreturn (\varTx, \funStar{\varSpendableCoinAlice}) \< \< \pcreturn (\varTx, \funStar{\varSpendableCoinBob})
        }
    \end{varwidth}
    }
    \end{center}
    \caption{$\procDSharedOutputMwTxId$ two-party protocol to build a new transaction from a shared output} \label{fig:d-shared-inp-mw-tx}
\end{figure}

\subsection{Contract Mimblewimble Transaction - $\procDScriptMwTxId$} \label{subsec:atom:script-mw-tx}

$\procDScriptMwTxId$ is a protocol between a sender and a receiver for the Contract Mimblewimble Transaction Scheme defined in~\cref{def:atom:apt-ext-mw-tx-scheme}.
Similar to the $\procDSharedInpMwTxId$ it spends an input coin which is shared between the sender and receiver.
Additionally, we utilize the adapted signature protocol from~\cref{def:sig:two-party-fixed-wit-apt-sig} to let the receiver hide a secret witness value $\varWit$ in the transaction signature, which the sender can extract from the final transaction, thereby allowing primitive contracts.

\[ \langle (\varTx, \funStar{\varSpendableCoinAlice}, \varWit), (\varTx, \funStar{\varSpendableCoinBob}) \rangle \opFunResult \procDScriptMwTx{\funArray{\varPtSpendableCoinAlice}}{\varFundValue}{\varTime}{\varStatement}{\funArray{\varPtSpendableCoinBob}}{\varWit} \]

\Cref{fig:d-script-tx} shows the implementation of the protocol.

\begin{figure}
    \begin{center}
    \fbox{
    \begin{varwidth}{\textwidth}
        \procedure[linenumbering,skipfirstln]{$\procDScriptMwTx{\funArray{\varPtSpendableCoinAlice}}{\varFundValue}{\varTime}{\varStatement}{\funArray{\varPtSpendableCoinBob}}{\varWit}$}{
        Alice \< \< Bob \\
        (\varPreTx, \funStar{\varSpendableCoinAlice}, (\varSecKeyAlice, \varNonceAlice)) \< \< (\varPreTx, (\varSecKeyBob, \varNonceBob)) \pcskipln \\
        \opFunResult \procDSendCoinsL{\funArray{\varPtSpendableCoinAlice}}{\varFundValue}{\varTime} \< \< \opFunResult \procDSendCoinsL{\funArray{\varPtSpendableCoinBob}}{\varFundValue}{\varTime} \\
        \< \< (\funStarAlt{\varPreTx}, \funStar{\varSpendableCoinBob}, \varSigBob)  \pcskipln \\
        \< \< \opFunResult \procAptRecvCoins{\varPreTx}{\varFundValue}{\varWit}  \\
        \< \sendmessageleft*[2cm]{\funStarAlt{\varPreTx}, \funStarAlt{\varStatement}} \\
        \pcif \varStatement \opNotEq \cnstFalsum \opAnd \varStatement \opNotEq \funStarAlt{\varStatement} \\
        \t \pcreturn \cnstFalsum \\
        \varSigAptBob \opFunResult \funStarAlt{\varPreTx}.\varSignature \\
        \varSigAliceBob \< \< \varTx \pcskipln \\
        \opFunResult \procDAptFinTxL{\funStarAlt{\varPreTx}}{\varSecKeyAlice}{\varNonceAlice}{\varStatement} \< \< \opFunResult \procDAptFinTxR{\funStarAlt{\varPreTx}}{\varSecKeyBob}{\varNonceBob}{\varSigBob} \\
        \< \sendmessageleft*[2cm]{\varTx} \\
        \varWit \opFunResult \procExtWit{\varTx.\varSignature}{\varSigAliceBob}{\varSigAptBob} \\
        \pcreturn (\varTx, \funStar{\varSpendableCoinAlice}, \varWit) \< \< \pcreturn (\varTx, \funStar{\varSpendableCoinBob})
        }
    \end{varwidth}
    }
    \end{center}
    \caption{$\procDScriptMwTxId$ two-party protocol to build a primitive contract transaction} \label{fig:d-script-tx}
\end{figure}

\paragraph{A note on rogue-key attacks:} In~\cref{sec:sig:definitions}, we mentioned that we need to take special care in the key generation phase in a Two-Party Signature Scheme.
Otherwise the protocol might be vulnerable against rogue-key attacks in which one of the party's public keys is computed as a function of the other.
We see that we do not take this into account in all of the protocols laid out in this section.
As for the receiving party, it will always be possible to generate his keypair as a function of the sender's public key.
We now show how attempting a rogue-key attack in Mimblewimble would play out and why it would not threaten the security of our scheme:\\
Imagine we have an attacker $\cnstAdversary$ who knows the value $\varValue$ of some coin $\varCoin \opEqNoQ \funGen{\varBlindingFactor} \opAddPoint \funGenH{\varValue}$ present in the unspent output list of the blockchain.
He could then compute $\varPubKeyAlice \opEqNoQ \varCoin \opAddPoint {(\funGenH{\varValue}})^{-1}$.
For the rogue-key attack to succeed, $\cnstAdversary$ would now create a transaction spending $\varCoin$ and choose his output coin pubkey as $\varPubKeyBob \opEqNoQ \varPubKeyAlice^{-1}$ with the attempt of canceling out Alice's key.
However, recalling the structure of Mimblewimble transactions the participants sign the excess value $\varExcess \opEqNoQ \varInputs \opSub \varOutputs$, where $\varInputs$ and $\varOutputs$ is the input and output coins list.
Therefore, making the public keys cancel out $\cnstAdversary$ would instead have to choose his key as $\varPubKeyBob \opAssign \varPubKeyAlice$.
Given this setup (a transaction which spends the coin $\varCoin \opEqNoQ \varPubKeyAlice \opAddPoint \funGenH{\varValue}$ to $\funStar{\varCoin} \opEqNoQ \varPubKeyBob \opAddPoint\funGenH{\varValue}$), the excess value $\varExcess$ would calculate like $\varPubKeyAlice \opAddPoint {\varPubKeyBob}^{-1}$.
$\varPubKeyBob$ definition is $\varPubKeyAlice \opAddPoint {\varPubKeyAlice}^{-1}$, which would cancel out and allow the adversary to forge a signature.
However, since we chose $\varPubKeyBob$ as simply $\varPubKeyAlice$ and $\varPubKeyAlice \opEqNoQ \funGen{\varBlindingFactor}$ (from the original Pedersen Commitment) the new output coin $\funStar{\varCoin}$ would be identical to the input coin $\varCoin$, and the transaction spend a coin to itself.
Recalling the instantiation of the transaction verification algorithm $\procVerfTxId$ defined by Fuchsbauer et al.~\cite{fuchsbauer2019aggregate}, which we laid out in~\cref{fig:inst-mw-tx-2}, we see that the union between input and output coin list must be empty.
Otherwise, the transaction will not verify.
Therefore, even though the attacker could create a forged signature for this transaction, it would still be invalid as by definition of the transaction verification algorithm.
We further consider the case in which the attacker would try to add a fee $\varFee$ to the transaction to steal value from a coin.
In this case, the newly created output coin would be $\varPubKeyBob \opAddPoint \funGenH{\varValue - \varFee}$.
Now the output coin is no longer identical to the input coin, yet the input and output values still cancel out due to the fee, and by the definition of $\varPubKeyBob$ the two public keys must as well still cancel out, allowing for a forged signature.
However, in this scenario, $\cnstAdversary$ is faced with the problem that he does not have a valid range proof for this new output coin.
To compute such a proof, he would need to know the original $\varBlindingFactor$ of $\varPubKeyAlice \opEqNoQ \funGen{\varBlindingFactor}$, which he doesn't.
Therefore it is again impossible for him the create a valid transaction, even though he would be able to forge the transaction signature.
We conclude that all possible rogue-key attacks on Mimblewimble are prevented through transaction verification, and we, therefore, do not have to take other special care to avoid them.





\section{Security \& Correctness} \label{sec:atom:security}
In this section we will prove the correctness and security of the instantiation described in~\ref{sec:atom:inst}.
We start by proving \emph{Transaction Scheme Correctness}, \emph{Extended Transaction Scheme Correctness} and \emph{Adapted Transaction Scheme Correctness} for the three outlined transaction schemes $\varMWScheme, \varextMWScheme$ and $\varaptMWScheme$.
We then continue by showing that all protocols described in~\ref{sec:atom:protocols} are secure in the malicious models as defined in~\ref{subsec:pre:security}.

\subsection{Correctness} \label{subsec:atom:correctness}

We will argue \emph{Transaction Scheme Correctness} follows from the correctness of the commitment scheme $\varCommitScheme$, two-party signature scheme $\varSigScheme$ as well as the correctness of the range proof system $\varProofSystem$ used in the transaction protocol.
If the transaction was constructed correctly (that is by calling the procedures $\procSendCoinsId, \procRecvCoinsId, \procFinTxId$, the distributed variants $\procDSendCoinsId, \procDRecvCoinsId, \procDFinTxId$ or the adapted ones $\procAptRecvCoinsId, \procDAptFinTxId$ with valid inputs) it must follow that the final transaction has correct commitments, rangeproofs and a valid signature and $\procVerfTxId$ will therefore return 1.
We construct the following theorem:

\begin{theorem}
    \label{lem:atom:correctness}
    \emph{Transaction Scheme Correctness}, \emph{Extended Transaction Scheme Correctness} or \emph{Adapted Transaction Scheme Correctness} for a transaction system $\varMWSchemeParams{\varCommitScheme}{\varSigScheme}{\varProofSystem}$, $\varextMWSchemeParams{\varCommitScheme}{\varSigScheme}{\varProofSystem}$ or $\varaptMWSchemeParams{\varCommitScheme}{\varSigScheme}{\varProofSystem}$ holds if and only if the underlying Commitment Scheme $\varCommitScheme,$ Two-Party Signature Scheme $\varSigSchemeMP$ and Rangeproof system $\varProofSystem$ are correct.
\end{theorem}

\todo[inline]{I think this theorem states more than what you actually prove. If and only if means that you prove both directions but I think you prove only one direction, that is, assume that COM, two-party signature and rangeproof are correct then your transactions schemes are correct. For the if and only if, you should additionally prove the following: Assume that your transactions schemes are correct then show that COM, two-party signature and range proof are correct. }

\begin{proof}
    We assume there are two honest participants Alice and Bob, there exists a list of input coins $\funArray{\varCoinInp}$ with blinding factors $\funArray{\varBlindingFactor_i}$ and values $\funArray{\varValue_i}$ wrapped inside a list $\funArray{\varSpendableCoin}$ known to Alice, and some amount $\varFundValue$ which Alice wants to transfer to Bob.
    For \emph{Transaction Scheme Correctness} to hold $\procVerfTx{\varTx}$ must return 1 with overwhealming probability for the two parties creating the transaction $\varTx$ in the following three steps:
    \begin{enumerate}
        \item $(\varPreTx, (\varSecKeyAlice, \varNonceAlice)) \opFunResult \procSendCoins{\funArray{\varSpendableCoin}}{\varFundValue}{\cnstFalsum}$
        \item $\funStar{\varPreTx} \opFunResult \procRecvCoins{\varPreTx}{\varFundValue}$
        \item $\varTx \opFunResult \procFinTx{\funStar{\varPreTx}}{\varSecKeyAlice}{\varNonceAlice}$
    \end{enumerate}
    We recall the conditions for $\procVerfTx{\varTx}$ to return 1 found in~\ref{def:atom:mw-tx-scheme} and show that each of them must hold:

    Condition 1 and 2 both must hold if the participants are honest.
    In the case that the sending party provides duplicate inputs the check at the beginning of the $\procSendCoinsId$ procedure will fail and consequently $\procVerfTx{\varTx}$ will return 0.
    The blinding factors to the ouput coins created in $\procSendCoinsId$ and $\procRecvCoinsId$ are generated randomly, which means a duplication can only appear with negligible probability.

    Condition 3 follows from the implementation of the $\procCreateCoinId$ function called in $\procSendCoinsId$ as well as $\procRecvCoinsId$.
    In the function a rangeproof is computed for the new coin $\varCoin$ with value $\varValue$ and blinding factor $\varBlindingFactor$ as $\varProof \opFunResult \procProof{\varCoin}{\varValue}{\varBlindingFactor}$.
    Given that our Rangeproof system $\varProof$ system has to be correct $\procVerfProof{\varProof}{\varCoin} \opEqNoQ 1$ must hold for all coins created with the $\procCreateCoinId$ routine.
    Therefore Condition 2\todo{typo?} must hold if the transaction is computed honestly.

\todo[inline]{I think you have mixed condition 3 and condition 4? In your description of verfTX (Definition 5.1), Condition 4 checks the validity of range proofs and condition 3 checks the validity of the transaction signature.}

    For condition 4 we must look at how the secret keys $\varSecKeyAlice$ and $\varSecKeyBob$ are constructed.
    From the instantiation of $\procSendCoinsId$ we can see that Alice's share will be $\varSecKeyAlice \opAssign \funStar{\varBlindingFactorAlice} \opSub \cnstSumZeroToN \funArray{\varBlindingFactorAlice}$, where $\funStar{\varBlindingFactorAlice}$ is the blinding factor to her output and $\funArray{\varBlindingFactorAlice}$ are the blinding factors to her input coins.
    Bobs secret key is constructed like $\varSecKeyBob \opAssign \funStar{\varBlindingFactorBob}$, so it corresponds to the blinding factor of his output.
    From the construction of the two-party signature scheme in~\ref{def:sig:two-party-sig} we know that therefore the final signature will be valid under the following public key:
    \[ \funStarAlt{\varExcess} \opAssign \funGen{\varSecKeyAlice} \opAddPoint \funGen{\varSecKeyBob} \]
    Given how the secret keys are constructed we arrive at:
    \[ \funStarAlt{\varExcess} \opAssign \funGen{\funStar{\varBlindingFactorAlice}} \opAddPoint \cnstSumZeroToN \funArray{\funGen{- \varBlindingFactorAlice}} \opAddPoint \funGen{\varBlindingFactorBob} \]
    If we can show that the excess value $\varExcess$ computed in $\procVerfTxId$ is the same as above, $\procVerf{\varMsg}{\varSignature}{\varPubKey} \opEqNoQ 1$ must hold and therefore condition 3 would be proven.
    We show this by a stepwise conversion of the initial equation computing $\varExcess$ until we arrive at the equation for $\funStarAlt{\varExcess}$:
    \begin{gather}
        \varExcess \opEqNoQ \funStarAlt{\varExcess} \\
        \cnstSumZeroToN \varOutputs \opSub \cnstSumZeroToN \varInputs \opEqNoQ \funGen{\funStar{\varBlindingFactorAlice}} \opAddPoint \cnstSumZeroToN \funArray{\funGen{- \varBlindingFactorAlice}} \opAddPoint \funGen{\varBlindingFactorBob} \\
        \varCoinOutAlice \opAddPoint \varCoinOutBob \opAddPoint \cnstSumZeroToN \funArray{(\varCoinInp)^{-1}}  \opEqNoQ\\
        (\funGen{\funStar{\varBlindingFactorAlice}} \opAddPoint \funGenH{\varValue \opSub \varFundValue}) \opAddPoint
        (\funGen{\funStar{\varBlindingFactorBob}} \opAddPoint \funGenH{\varFundValue}) \opAddPoint
        \cnstSumZeroToN \funArray{(\funGen{- \varBlindingFactorAlice}, \funGenH{- \varValue_i})} \opEqNoQ \\
        \funGen{\funStar{\varBlindingFactorAlice}} \opAddPoint \funGen{\funStar{\varBlindingFactorBob}} \opAddPoint \cnstSumZeroToN \funGen{- \varBlindingFactorAlice} \opEqNoQ \funGen{\funStar{\varBlindingFactorAlice}} \opAddPoint \funGen{\funStar{\varBlindingFactorBob}} \opAddPoint \cnstSumZeroToN \funGen{- \varBlindingFactorAlice} \\
        1 \opEqNoQ 1
    \end{gather}
    From step 5.3 to 5.4 we replace every coin $\varCoin$ by its instantiation for a pedersen commitment $\varCoin \opEqNoQ \funGen{\varValue} \opAddPoint \funGenH{\varValue}$.\todo{typo? should be $g^r$ instead?}

    From step 5.4 to 5.4 \todo{typo?} we rely on the fact that if Alice is honest $\varValue \opEqNoQ \sum \varValue_i$, therefore also $(\varValue \opSub \varFundValue) \opAddScalar\varFundValue \opEqNoQ \sum \varValue_i$ must hold.
    From that we can infer that $\funGenH{\varValue \opSub \varFundValue} \opAddPoint \funGenH{\varFundValue} \opAddPoint \sum \funGenH{- \varValue_i}$ must cancel out, otherwise the transaction would either create or burn value, which is not allowed and in which case $\procVerfTxId$ should again return 0.

    We have managed to show that condition 1-4 must hold for a valid transaction and can conclude that \emph{Transaction Scheme Correctness} holds for $\varMWSchemeParams{\varCommitScheme}{\varProofSystem}{\varSigSchemeMP}$.

    We will now argue that the same deriviation holds for \emph{Extended Transaction Scheme Correctness} and \emph{Adapted Transaction Scheme Correctness}.

    Condition 1-2 again follow trivially from the construction of $\procDSendCoinsId$ and $\procDRecvCoinsId$ for the same reasons we have already layed out in the previous proof.

    $\procDSendCoinsId$, $\procDRecvCoinsId$, $\procAptRecvCoinsId$ all rely on the same $\procCreateCoinId$ routine to create output coins, thereby condition 3 \todo{same comment as before with mixed conditions} also holds for valid transactions with the same argument as for the previous proof.

    In the case of \emph{Extended Transaction Scheme Correctness} the blinding factors for the input coins $\funArray{\varCoinInp}$ are shared.
    However, we can easily reduce this case to the proof for the regular case:
    In $\procDSendCoinsId$ Alice and Carol construct their secret keys as follows:
    \begin{gather}
        \varSecKeyAlice \opAssign \funStar{\varBlindingFactorAlice} \opSub \cnstSumZeroToN \varBlindingFactorAlice \\
        \varSecKeyCarol \opAssign \funStar{\varBlindingFactorCarol} \opSub \cnstSumZeroToN \varBlindingFactorCarol
    \end{gather}
    $\varSecKeyAlice$ and $\varSecKeyCarol$ are then inputs to $\procDFinTxId$ in which a partial signature $\varSigAliceCarol$ is calculated, by both Alice and Carol signing with their secret key.
    Assume the general key from before\todo{what is exactly before? not clear here for the reader}, in which we have a single secret key $\varSecKeyAlice$.
    We can split $\varSecKeyAlice$ into arbitrarily chosen shares $(\varSecKeyAlice)_1, (\varSecKeyAlice)_2$ with $\varSecKeyAlice \opEqNoQ (\varSecKeyAlice)_1 + (\varSecKeyAlice)_2$.
    By the definition of Two-Party Signatures~\ref{def:sig:two-party-sig} the combined signature from $(\varSecKeyAlice)_1, (\varSecKeyAlice)_2$ will be valid under $\funGen{\varSecKeyAlice}$.
    Thereby we can treat $\varSecKeyAlice$ and $\varSecKeyCarol$ from $\procSendCoinsId$ as arbitrary shares of a combined $\varSecKeyAliceCarol$.
    It follows from the additive homomorphic property of the elliptic curve that a signature valid under $\funGen{\varSecKeyAliceCarol}$ must also be valid under $\funGen{\varSecKeyAlice} \opAddPoint \funGen{\varSecKeyCarol}$.
    The case of two receivers calling $\procDRecvCoinsId$ is symmetric.
    From this we can conclude that condition 4 must also hold for the \emph{Extended Transaction Scheme}. \\
    Now for the \emph{Adapted Extended Transaction Scheme} the same argument holds.
    The only difference in this scheme is that in $\procDAptFinTxId$ Bob (instead of Alice) will call $\procFinSigId$, as only he knows his unadapted partial signature $\varSigBob$.
    However, the construction of the signature remains unchanged, therefore the reduction we provided before must hold for the same reasons.

    We have thereby proven that if $\varCommitScheme, \varProofSystem, \varSigSchemeMP$ are correct and the participants behave honestly (that is by providing valid inputs and calling the respective routines in the given order) $\procVerfTx{\varTx}$ will return 1 for the resulting transaction $\varTx$ and therefore theorem~\ref{lem:atom:correctness} holds.
\end{proof}
\todo[inline]{Right. As I mentioned before, you have proven only one direction (what is what you were supposed/expected to do). Just update the theorem.}

\subsection{Security}\label{subsec:atom:security}

We now want to prove security in the malicious setting as defined in~\ref{subsec:pre:security} for the protocols defined in~\ref{sec:atom:protocols}.
Again we show that the distributed protocols are secure in the hybrid $\procZKfId{\cnstRelation}$-model as already explained in section~\ref{subsec:sig:secureaptscheme}.
We start by proving security of the simple transaction protocol $\procDBuildMwTxId$.

\textbf{Hybrid functionalities}: The parties have access to a trusted third party computing the zero-knowledge proof of knowledge functionalities $\procZKfId{\cnstRelation_1}$, $\procZKfId{\cnstRelation_2}$ and $\procZKfId{\cnstRelation_2*}$.
$\cnstRelation_1$ is the relation between a secret key $\varSecKey$ and its public key $\varPubKey \opEqNoQ \funGen{\varSecKey}$ for the elliptic curve generator point $\varG$.
$\cnstRelation_2$ is the relation between two secret inputs $\varBlindingFactor, \varValue$ and its pedersen commitment $\varCommitment \opEqNoQ \funGen{\varBlindingFactor} \opAddPoint \funGenH{\varValue}$ for two adjacent generators $\varG, \varH$ as defined in~\ref{def:pre:pedersen}.
We shorten the call by the prover to just provide $\varSpendableCoin$ because it is a wrapper that contains the coin commitment, as well as its openings.
$\cnstRelation_2*$ is the same as $\cnstRelation_2$ just for a list of secrets inputs $\funArray{(\varBlindingFactor, \varValue)}$ and its list of commitments $\funArray{\varCommitment}$.
Again to shorten the calls by the prover we simplify the call to $\procZkf{\cnstRelation_2*}{\funArray{\varSpendableCoin}}$.

\textbf{Proof Idea}: We extend the protocol $\procDBuildMwTxId$ instantiated in section~\ref{sec:atom:protocols} with the following calls to the zero-knowledge proof of knowledge functionalities as depicted in figure~\ref{fig:atom:hybrid-dbuild}.

\begin{figure}
    \fbox{
    \begin{varwidth}{\textwidth}
        \procedure[linenumbering,skipfirstln]{$\procDBuildMwTx{\funArray{\varSpendableCoin}}{\varFundValue}{\varTime}$}{
        Alice \< \< Bob \\
        \procZkf{\cnstRelation_2*}{\funArray{\varSpendableCoin}} \\
        (\varPreTx, \funStar{\varSpendableCoinAlice}, (\varSecKeyAlice, \varNonceAlice)) \\ \opFunResult \procSendCoins{\funArray{\varSpendableCoin}}{\varFundValue}{\varTime} \\
        \procZkf{\cnstRelation_1}{(\varSecKeyAlice, \funGen{\varSecKeyAlice})} \\
        \procZkf{\cnstRelation_1}{(\varNonceAlice, \funGen{\varNonceAlice})} \\
        \procZkf{\cnstRelation_2}{\funStar{\varSpendableCoinAlice}} \\
        \< \sendmessageright*{\varPreTx} \< \\
        \< \< \pcif \procZkf{\cnstRelation_2*}{\varTx.\varInputs} \opEqNoQ 0 \\
        \< \< \t \pcreturn \cnstFalsum \\
        \< \< \pcif \procZkf{\cnstRelation_1}{\varTx.\varSigContext.\varPubKey} \opEqNoQ 0 \\
        \< \< \t \pcreturn \cnstFalsum \\
        \< \< \pcif \procZkf{\cnstRelation_1}{\varTx.\varSigContext.\varRand} \opEqNoQ 0 \\
        \< \< \t \pcreturn \cnstFalsum \\
        \< \< \pcif \procZkf{\cnstRelation_2}{\varTx.\varOutputs[0]} \opEqNoQ 0 \\
        \< \< \t \pcreturn \cnstFalsum \\
        \< \< (\funStarAlt{\varPreTx}, \funStar{\varSpendableCoinBob}) \opFunResult \procRecvCoins{\varPreTx}{\varFundValue} \\
        \< \< \procZkf{\cnstRelation_2}{\funStar{\varSpendableCoinBob}} \\
        \< \sendmessageleft*{\funStarAlt{\varPreTx}} \\
        \pcif \procZkf{\cnstRelation_2}{\varTx.\varOutputs[1]} \opEqNoQ 0 \\
        \t \pcreturn \cnstFalsum \\
        \varTx \opFunResult \procFinTx{\funStarAlt{\varPreTx}}{\varSecKeyAlice}{\varNonceAlice} \\
        \< \sendmessageright*{\varTx} \\
        \pcreturn (\varTx, \funStar{\varSpendableCoinAlice}) \< \< \pcreturn (\varTx, \funStar{\varSpendableCoinBob})
        }
    \end{varwidth}
    }
    \todo[inline]{We need to decide what version of the protocols you want to put finally in the thesis. So far you are writing 2 versions: (i) the version as it is in Grin; and (ii) the version that you are able to prove secure (i.e., adding the calls to the zkp system). So, one question that you might get is: what is the "good" version, that is, if I create another implementation of Mimblewimble, which one should I take? And another question could be: the fact that you have 2 versions means that the current implementation is broken? Think about the answers and we should discuss them. Regarding the 2 versions, I see 2 possibilities: (a) keep both and clearly say their difference; or (b) keep only the one that you prove secure and say that it is slightly different from that of Mimblewimble. }
    \caption{$\procDBuildMwTxId$ in the hybrid Model} \label{fig:atom:hybrid-dbuild}
\end{figure}

\begin{theorem}
    \label{theo:atom:sec-tx}
    Let $\varCommitScheme$ be a correct and secure pedersen commitment scheme, $\varProofSystem$ \todo[inline]{I would write $\varProofSystem$ as $\varProofSystem_{\textit{RP}}$ instead to clearly show that it is not a generic zk proof but a range proof. } be a correct and secure rangeproof system and $\varSigSchemeMP$ be a secure and correct two-party signature scheme, then $\procDBuildMwTxId$ securely computes a Mimblewimble transaction transferring the value $\varFundValue$ from a sender (denoted as Alice) to a receiver (denoted as Bob) in the hybrid $\procZKfId{\cnstRelation_1}, \procZKfId{\cnstRelation_2}$-model.\todo{one ideal functionality missing here?}
\end{theorem}

\begin{proof}
    We proof the security of $\procDBuildMwTxId$ by constructing a simulator $\cnstSimulator$ with access to a TTP computing the protocol in the ideal setting upon receiving the inputs from the participants.
    For this the simulator has to extract the inputs used by the adversary.
    The TTP returns the outputs $(\varTx, \funStar{\varSpendableCoinAlice})$ (resp. $(\varTx, \funStar{\varSpendableCoinBob})$) from which he has to construct a transcript that is indistinguishable from the protcol transcript in the real world. The simulator uses the calls to $\procZKfId{\cnstRelation_1}, \procZKfId{\cnstRelation_2}, \procZKfId{\cnstRelation_2*}$ to achieve this.
    We proof that the transcript is indistinguishable in the cases that either Alice or Bob is corrupt and controlled by a deterministic polynomial adversary $\cnstAdversary$.

    \textbf{Alice is corrupt}: Simulator $\cnstSimulator$ works as follows:
    \begin{enumerate}
        \item $\cnstSimulator$ invokes $\cnstAdversary$ and once it calls $\procZKfId{\cnstRelation_2*}, \procZKfId{\cnstRelation_1}$ $\procZKfId{\cnstRelation_2}$ saves the values $\funArray{\varSpendableCoin}, \varSecKeyAlice, \varNonceAlice, \funStar{\varSpendableCoinAlice}$ to its memory.
        \item $\cnstSimulator$ calculates the transaction value $\varFundValue$ as follows:
        \begin{gather*}
            \varValue \opEqNoQ \sum_{\varI \opAssign 0}^{\varI \opSm \varN}(\varSpendableCoin_i.\varValue) \\
            \varFundValue \opEqNoQ \varValue \opSub \funStar{\varSpendableCoinAlice}.\varValue
        \end{gather*}
        \item $\cnstSimulator$ receives $\varPreTx$ from $\cnstAdversary$ and checks for every transaction input $\varI$ if $\varPreTx.\varInputs[\varI] \opEqNoQ \varSpendableCoin[\varI].\varCoin$, and that $\varTx.\varOutputs \opEqNoQ \funArray{\funStar{\varSpendableCoinAlice}.\varCoin}$.
        He also compares $\varTx.\varSigContext.\varPubKey \opEqNoQ \funGen{\varSecKeyAlice}$, $\varTx.\varSigContext.\varRand \opEqNoQ \funGen{\varNonceAlice}$, $\varTx.\varProof[0] \opEqNoQ \funStar{\varSpendableCoinAlice}.\varProof$ and $\varTx.\varCommits[0] \opEqNoQ \funGen{\varSecKeyAlice}$.
        If any of the equalities were invalid $\cnstSimulator$ sends $\cnstAbort$ to the TTP computing $\procDBuildMwTxId$ and returns $\cnstFalsum$.
        Otherwise he extracts $\varTime \opEqNoQ \varTx.\varTime$ and sends the inputs $(\funArray{\varSpendableCoin}, \varFundValue, \varTime)$ to the TTP and receives back the outputs $(\varTx, \funStar{\varSpendableCoinAlice})$.
        \item The simulators task is it now to construct $\funStarAlt{\varPreTx}$ which he can achieve in the following steps:
        \begin{enumerate}
            \item He takes the signature context $\varSigContext$ and final signature $\varSigFin$ from the final transaction $\varSigContext \opEqNoQ \varTx.\varSigContext$ and $\varSigFin \opEqNoQ \varTx.\varSignature$.
            \item He computes the adversaries partial signature as $\varSigAlice \opFunResult \procSignPrt{\varMsg}{\varSecKeyAlice}{\varNonceAlice}{\varSigContext}$
            \item He further computes
            \begin{gather*}
                \varPubKey \opFunResult \varSigContext.\varPubKey \\
                \varPubKeyAlice \opEqNoQ \funGen{\varSecKeyAlice} \\
                (\varSAlice, \varRandAlice, \varSigContext) \opFunResult \varSigAlice \\
                (\varS, \varRand) \opFunResult \varSigFin \\
                \varSBob \opEqNoQ \varS \opSub \varSAlice \\
                \varRandBob \opEqNoQ \varRand \opAddPoint \varRandAlice^{-1} \\
                \varPubKeyBob \opEqNoQ \varPubKey \opAddPoint \varPubKeyAlice^{-1} \\
                \varSigBob \opEqNoQ (\varSBob, \varRandBob, \varSigContext)
            \end{gather*}
            \item He takes further values from the final transaction:
            \begin{gather*}
                \varCoinOutBob \opEqNoQ \varTx.\varOutputs[1] \\
                \varProofBob \opEqNoQ \varTx.\varProof[1] \\
                \varCommitment_\varBob \opEqNoQ \varTx.\varCommits[1]
            \end{gather*}
            \item Now he can compute $\funStarAlt{\varPreTx} \opFunResult \procCreatePreTx{\varMsg}{\varInputs}{\varOutputs \opConc \varCoinOutBob}{\varProofs \opConc \varProofBob}{\varSigContext}{\varCommits \opConc \varCommitment_\varBob}{\varSigBob}{\varTime}$
        \end{enumerate}
        Finally $\cnstSimulator$ will send $\funStarAlt{\varPreTx}$ as if coming from Bob and sends $\cnstContinue$ to the TTP.
        \item When $\cnstAdversary$ calls $\procZKfId{\cnstRelation_2}$ he checks equality to $\varCoinOutBob$ and returns either 0 or 1.
        \item Eventually $\cnstAdversary$ will send a $\funStarAlt{\varTx}$ after which the simulator will output whatever $\cnstAdversary$ outputs.
    \end{enumerate}
    Next we need to proof that the transcript produced by $\cnstSimulator$ is indistinguishable from a real one in every phase of the protocol.
    We separate between the following three phases:
    \textbf{Phase 1}: Alice sends her input coins, signing key and nonce as well as her change output coin to $\procZKfId{\cnstRelation_1}$ and $\procZKfId{\cnstRelation_2}$ and sends the pre-transaction $\varPreTx$ to Bob.
    \textbf{Phase 2}: Bob calls $\procZKfId{\cnstRelation_1}$ and $\procZKfId{\cnstRelation_2}$ as the verifier, after which he calls $\procZKfId{\cnstRelation_2}$ as the prover and proceeds by sending the updated pre-transaction $\funStarAlt{\varPreTx}$ to Alice.
    \textbf{Phase 3}: Alice calls $\procZKfId{\cnstRelation_2}$ as the verifier and sends back the final transaction $\varTx$ to Bob which they then both output.

    \begin{itemize}
        \item \textit{Phase 1}: Due to the deterministic nature of $\cnstAdversary$ we can conclude that this phase has to be indistinguishable as there is not yet any simulation required.
        \item \textit{Phase 2}: If any of the values that $\cnstAdversary$ send to the trusted party computing the zero-knowledge proofs of knowledge are different from the value that $\cnstAdversary$ sends in the pre-transaction the equality checks done by $\cnstSimulator$ will fail in which case he will output $\cnstFalsum$ which is identical to what happens in the real execution.
        We further argue that the updated pre-transaction $\funStarAlt{\varPreTx}$ is identical to the one send in the real execution by Bob.
        The signatures $\varSigAlice$ and $\varSigBob$ have to add up to $\varSigFin$ which is the final signature.
        $\cnstSimulator$ can read $\varSigFin$ from the transaction in the output he received from the TTP, he can further calculate the adversaries signature because he knows their signing secrets.
        From those two values he can then compute the value that $\varSigBob$ must have such that it will complete to $\varSigFin$ when added to Alice's part of the signature.
        All further values $\cnstSimulator$ needs to build $\funStarAlt{\varPreTx}$ he can simply read from the final transaction $\varTx$.
        Therefore $\funStarAlt{\varPreTx}$ is identical to the one sent in the real execution.
        \item \textit{Phase 3}: When $\cnstAdversary$ calls $\procZKfId{\cnstRelation_2}$ as the verifier, $\cnstSimulator$ can simply check equality with the correct value and return 0 or 1, which is identical to the real execution. \todo{Same comment about using "identical" as in the security of previous chapter}
    \end{itemize}
    

    We have managed to show that in the case that Alice is corrupted the simulation is perfect, because the transcript is in fact identical to the transcript of the real execution.

    \textbf{Bob is corrupt}: Simulator $\cnstSimulator$ works as follows:
    \begin{enumerate}
        \item $\cnstSimulator$ computes one (or multiple) input coins as follows:
        \begin{gather*}
            \varBlindingFactor, \varValue \sample \cnstIntegersPrimeWithoutZero{\varPrime} \\
            \varSpendableCoin \opFunResult \procCreateCoin{\varBlindingFactor}{\varValue}
        \end{gather*}
        He chooses $\varFundValue$ randomly and sets $\varTime \opEqNoQ \cnstFalsum$.
        Now he can call $\procSendCoinsId$ and get:
        \[ (\varPreTx, \funStar{\varSpendableCoinAlice}, (\varSecKeyAlice, \varNonceAlice)) \opFunResult \procSendCoins{\funArray{\varSpendableCoin}}{\varFundValue}{\varTime} \]
        \item The simulator invokes $\cnstAdversary$ and sends $\varPreTx$ as if coming from Alice.
        \item When $\cnstAdversary$ calls $\procZKfId{\cnstRelation_1},\procZKfId{\cnstRelation_2}$ \todo{one ideal functionality missing?} as the verifier $\cnstSimulator$ simply checks equality we the values he sent and returns either 0 or 1.
        The adversary proceeds by calling $\procZkf{\cnstRelation_2}{\funStar{\varSpendableCoinBob}}$, $\cnstSimulator$ saves $\funStar{\varSpendableCoinBob}$ and extracts $\varFundValue \opEqNoQ \funStar{\varSpendableCoinBob}.\varValue$
        He then calls the TTP computing $\procDBuildMwTxId$ with the input $\varFundValue$ and receives $(\varTx, \funStar{\varSpendableCoinBob})$.
        \item Next $\cnstAdversary$ sends an updated pre-transaction $\funStarAlt{\varPreTx}$.
        $\cnstSimulator$ verifies the output coin added by $\cnstAdversary$ matches with $\funStar{\varSpendableCoinBob}$, if it does not he sends $\cnstAbort$ to the TTP and outputs $\cnstFalsum$.
        Otherwise $\cnstSimulator$ computes the following values from the signature context $\varSigContext$ provided in the final transaction and $\funStarAlt{\varSigContext}$ provided by $\cnstAdversary$:
        \begin{gather*}
            \varPubKeyBob \opEqNoQ \funStarAlt{\varSigContext}.\varPubKey \opAddPoint {\funGen{\varSecKeyAlice}}^{-1} \\
            \varRandBob \opEqNoQ \funStarAlt{\varSigContext}.\varRand \opAddPoint {\funGen{\varNonceAlice}}^{-1} \\
            \varPubKeyAlice \opEqNoQ \varSigContext.\varPubKey \opAddPoint \varPubKeyBob^{-1} \\
            \varRandAlice \opEqNoQ \varSigContext.\varRand \opAddPoint \varRandBob^{-1}
        \end{gather*}
        \item Next the simulator rewinds to the first step of the simulation but instead of choosing the values for the pre-transacion now he uses $\varTx.\varInputs$ as the pre-transaction input values, $\varTx.\varOutputs[0]$ as the single output value, $\varTx.\varProofs[0]$ as the single rangeproof value and $\varTx.\varCommits[0]$ as the single value in the commitment field.
        Furthermore he constructs the initial signature context as:
        \begin{gather*}
            \varSigContext \opAssign \{ \varPubKey \opEqNoQ 1, \varRand \opEqNoQ 1 \} \\
            \varSigContext \opFunResult \procSetupCtx{\varSigContext}{\varPubKeyAlice}{\varRandAlice}
        \end{gather*}
        And again sends the pre-transaction to $\cnstAdversary$ as if coming from Alice.
        \item The simulator repeats the steps until step 5. where he rewinded earlier, now instead of rewinding $\cnstSimulator$ sends $\cnstContinue$ to the TTP and sends $\varTx$ as if coming from Alice, and finally outputs whatever $\cnstAdversary$ outputs.
    \end{enumerate}

    Again we now claim that the simulation is indistinguishable from a real execution in all three phases.
    Note that due to the rewinding step we need to consider both the message sent before and after the rewind.

    \begin{itemize}
        \item \textit{Phase 1:} In the first iteration the simulator constructs the input values $\funArray{\varSpendableCoin}$ from random values and also chooses a random transaction value $\varFundValue$.
        $\cnstSimulator$ constructs the pre-transaction using those chosen value rather than the real ones.
        We claim that the adversary cannot distinguish the chosen from the real coin commitments (Except with neglible probability).
        If we assume that he would be able to do so, that means he could distinguish for two pedersen commitments $\varCommitment_1 \opEqNoQ \funGen{\varBlindingFactor_1} \opAddPoint \funGenH{\varValue}, \varCommitment_2 \opEqNoQ \funGen{\varBlindingFactor_2} \opAddPoint \funGenH{\funStarAlt{\varValue}}$ which one commits to $\varValue$, from which follows that he could break the hiding property of perdersen commitments.
        Not being able to extract the coin values, the adversary has no chance of knowing if they are correct at this point.
        For the same reasons the pre-transaction sent by $\cnstSimulator$ after the rewind will be indistinguishable from a real one.
        However, as this time the pre-transaction is constructed from the real $\varTx$ which $\cnstSimulator$ received from the TTP, the pre-transaction is in fact identical to the pre-transaction sent in the real execution.
        The calls to $\procZKfId{\cnstRelation_1}$ and $\procZKfId{\cnstRelation_2}$ also behave identically to the real execution in which the parties have access to a TTP computing those protocols.
        \item \textit{Phase 2:} This phase will be identical to the real execution due to the fact that the adversary is deterministic.
        \item \textit{Phase 3:} The transaction sent to $\cnstAdversary$ in this phase is the one received from the TTP and is therefore identical to what would have been sent in the real execution, given $\cnstAdversary$ sends correct values. (Otherwise the execution would have halted with $\cnstFalsum$).
        We like to emphasize that in the case that we wouldn't have done the rewind step, $\cnstAdversary$ would be able to distinguish the transcript from the real one because he can identify differences in the inputs, outputs, proofs and commitment, as well as the signature context of the final transaction $\varTx$ and the pre-transaction $\varPreTx$ sent in the first phase.
        For instance inputs which are spend in the final transaction are not present in the pre-transaction.
        However, due to the rewinding step $\cnstSimulator$ manages to construct the correct pre-transaction which will finalize into $\varTx$ such that $\cnstAdversary$ again has no chance of distinguishing the two transcripts.
    \end{itemize}

    We have manged to show that the transcripts produced by $\cnstSimulator$ in the case that Alice and in the case that Bob is corrupt are indistinguishable from the transcript of a real execution and can therefore conclude that the protocol is secure and theorem~\ref{theo:atom:sec-tx} holds.

\end{proof}

Before we can continue to proof the security of the three other protocols $\procDSharedInpMwTxId$, $\procDSharedOutputMwTxId$, $\procDScriptMwTxId$ we first have to proof that all the protocols which are run as part of those executions are secure too.
That is we have to show security for $\procDSendCoinsId$, $\procDRecvCoinsId$, $\procDFinTxId$, $\procDAptFinTxId$.

We start with the proof for $\procDSendCoinsId$ which is called inside $\procDSharedInpMwTxId$ as well as $\procDScriptMwTxId$.

\textbf{Hybrid functionalities}: For this proof we need to extend our hybrid model.
As previously the parties have access to a trusted third party computing the zero-knowledge proof of knowledge functionalities $\procZKfId{\cnstRelation_1}, \procZKfId{\cnstRelation_2}$ and $\procZKfId{\cnstRelation_2*}$.
Additionally we introduce $\procZKfId{\cnstRelation_3}$, whereas $\cnstRelation_3$ is the relation between a value $\varValue$, two secrets $\varBlindingFactorAlice, \varBlindingFactorCarol$ and the commitment $\varCoin \opEqNoQ \funGenH{\varValue} \opAddPoint \funGen{\varBlindingFactorAlice} \opAddPoint \funGen{\varBlindingFactorCarol}$. \todo[inline]{In general, one cannot add an ideal functionality for a zkp of an arbitrary relation. I mention this because I see here you introduce a new relation $\cnstRelation_3$. In theory, when one introduces such ideal functionality, there should be a protocol that is secure for such functionality. However, in this case is fine as one can foresee that R3 can be done securely with a Sigma Protocol (as the one we saw in the lecture) or in the worst case with a more powerful zkp such as SNARKS.}
This means that for $\cnstRelation_3$ we have two provers, one of them having to provide $\varBlindingFactorAlice$, the other $\varBlindingFactorCarol$.
Both will have to provide the commitment $\varCoin$ and the value $\varValue$.
Both parties can then call the protocol again as the verifier providing the commitment $\funStar{\varCoin}$ and receiving 1 if $\funStar{\varCoin} \opEqNoQ \varCoin_\varAlice \opEqNoQ \varCoin_\varCarol$ (whereas $\varCoin_\varAlice$ is the commitment received from Bob as the prover, resp. for Carol) $\varValue_\varAlice \opEqNoQ \varValue_\varCarol$ and $\funStar{\varCoin} \opEqNoQ \funGenH{\varValue_{\cdot}} \opAddPoint \funGen{\varBlindingFactorAlice} \opAddPoint \funGen{\varBlindingFactorCarol}$.
To simplify the call made by the prover we just write $\procZkf{\cnstRelation_3}{\varPtSpendableCoin}$ as $\varPtSpendableCoin$ is like $\varSpendableCoin$ a wrapper around $\varCoin$, $\varBlindingFactor$, $\varValue$.
As for $\cnstRelation_2$ we again allow to call the protocol with an array of inputs by calling $\procZKfId{\cnstRelation_{3*}}$

\textbf{Proof Idea}: We extend the protocol $\procDSendCoinsId$ instantiated in~\ref{sec:atom:inst} with the following calls to the zero-knowledge proof of knowledge functionalities as can be seen in figure~\ref{fig:atom:hybrid-dsend}.

\begin{figure}
    \fbox{
    \begin{varwidth}{\textwidth}
        \procedure[linenumbering,skipfirstln]{$\procDSendCoins{\funArray{\varPtSpendableCoinAlice}}{\funArray{\varPtSpendableCoinCarol}}{\varFundValue}{\varTime}$}{
        Alice \< \< Carol \\
        \procZkf{\cnstRelation_{3*}}{\funArray{\varPtSpendableCoinAlice}} \< \< \procZkf{\cnstRelation_{3*}}{\funArray{\varPtSpendableCoinCarol}} \\
        \cdots \\
        \procZkf{\cnstRelation_2}{\funStar{\varSpendableCoinAlice}} \\
        \procZkf{\cnstRelation_1}{(\varSecKeyAlice, \funGen{\varSecKeyAlice})} \\
        \procZkf{\cnstRelation_1}{(\varNonceAlice, \funGen{\varNonceAlice})} \\
        \< \sendmessageright*{\varPreTx} \< \\
        \< \< \pcif \procZkf{\cnstRelation_{3*}}{\varPreTx.\varInputs} \opEqNoQ 0 \\
        \< \< \t \pcreturn \cnstFalsum \\
        \< \< \procZkf{\cnstRelation_2}{\varPreTx.\varOutputs[0]} \opEqNoQ 0 \\
        \< \< \t \pcreturn \cnstFalsum \\
        \< \< \pcif \procZkf{\cnstRelation_1}{\varSigContext.\varPubKey} \opEqNoQ 0 \\
        \< \< \t \pcreturn \cnstFalsum \\
        \< \< \pcif \procZkf{\cnstRelation_1}{\varSigContext.\varRand} \opEqNoQ 0 \\
        \< \< \t \pcreturn \cnstFalsum \\
        \< \< \cdots \\
        \< \< \procZkf{\cnstRelation_2}{\funStar{\varSpendableCoinCarol}} \\
        \< \< \procZkf{\cnstRelation_1}{(\varSecKeyCarol, \funGen{\varSecKeyCarol})} \\
        \< \< \procZkf{\cnstRelation_1}{(\varNonceCarol, \funGen{\varNonceCarol})} \\
        \< \sendmessageleft*{\funStarAlt{\varPreTx}} \< \\
        \pcif \procZkf{\cnstRelation_{3*}}{\funStarAlt{\varPreTx}.\varInputs} \opEqNoQ 0 \\
        \t \pcreturn \cnstFalsum \\
        \pcif \procZkf{\cnstRelation_2}{\funStarAlt{\varPreTx}.\varOutputs[1]} \opEqNoQ 0 \\
        \t \pcreturn \cnstFalsum \\
        \{ \varPubKey, \varRand \} \opFunResult \funStarAlt{\varPreTx}.\varSigContext \\
        \pcif \procZkf{\cnstRelation_1}{\varPubKey \opAddPoint \varPubKeyAlice^{-1}} \opEqNoQ 0 \\
        \t \pcreturn \cnstFalsum \\
        \pcif \procZkf{\cnstRelation_1}{\varRand \opAddPoint \varRandAlice^{-1}} \opEqNoQ 0 \\
        \t \pcreturn \cnstFalsum \\
        \pcreturn (\funStarAlt{\varPreTx}, \funStar{\varSpendableCoinAlice}, (\varSecKeyAlice, \varNonceAlice)) \< \< \pcreturn (\funStarAlt{\varPreTx}, \funStar{\varSpendableCoinCarol}, (\varSecKeyCarol, \varNonceCarol))
        }
    \end{varwidth}
    }
    \caption{$\procDSendCoinsId$ in the hybrid model} \label{fig:atom:hybrid-dsend}
\end{figure}

\begin{theorem}
    \label{teo:atom:sec-dspend}
    Let $\varCommitScheme$ be a correct and secure pedersen commitment scheme, $\varProofSystem$ be a correct and secure range proof system and $\varSigSchemeMP$ be a secure and correct two-party signature scheme, then $\procDSendCoinsId$ securely computes a Mimblewimble pre-transaction $\funStarAlt{\varPreTx}$ spending a coin $\varCoinShared$ owned by the two parties in the hybrid $\procZKfId{\cnstRelation_1}$,$\procZKfId{\cnstRelation_2}$,$\procZKfId{\cnstRelation_3}$-model.
\end{theorem}

\begin{proof}
    Again we proof security by constructing a simulator $\cnstSimulator$ with access to a trusted third party (TTP) computing $\procDSendCoinsId$ in the ideal setting upon receiving inputs from the two parties.
    The simulators task is to extract the inputs of the adversary $\cnstAdversary$, send the inputs to the TTP and construct a protocol transcript indistinguishable from a real one.
    We separately look at the case in which Alice is corrupted, and the case the Carol is corrupted. \todo[inline]{Editorial: I see that you have mentioned the proof strategy several times throughout the thesis. This is good, just that, if you find it repetitive, you can mention it once the first time you need it and refer to it in the rest.}

    \textbf{Alice is corrupted}: Simulator $\cnstSimulator$ works as follows:
    \begin{enumerate}
        \item $\cnstSimulator$ invokes $\cnstAdversary$ and saves $\funArray{\varPtSpendableCoinAlice}$, $\funStar{\varSpendableCoinAlice}$, $\varSecKeyAlice$, $\varNonceAlice$ when he calls $\procZKfId{\cnstRelation_{1,2,3}}$
        \item The simulator then receives $\varPreTx$ from $\cnstAdversary$ and compares the input coins, output coin and proof, signature context value with what he has stored in the first step.
        If any of those are not equal $\cnstSimulator$ sends $\cnstAbort$ to the TTP and outputs $\cnstFalsum$.
        Otherwise he extracts $\varFundValue \opAssign \sum \funArray{\varPtSpendableCoinAlice.\varValue} \opSub \funStar{\varSpendableCoinAlice}.\varValue$ as well as $\varTime \opAssign \varPreTx.\varTime$ and sends the inputs $(\funArray{\varPtSpendableCoinAlice}, \varFundValue, \varTime)$ to the TTP and receives the outputs $(\funStarAlt{\varPreTx}, \funStar{\varSpendableCoinAlice}, (\varSecKeyAlice, \varNonceAlice))$.
        \item $\cnstSimulator$ sends $\funStarAlt{\varPreTx}$ to $\cnstAdversary$ as if coming from Carol and sends $\cnstContinue$ to the TTP to make $\cnstAdversary$ receive the outputs in the ideal setting.
        \item When $\cnstAdversary$ calls $\procZKfId{\cnstRelation_{1,2,3}}$ as the verifier he compares the values to what he has sent in $\funStarAlt{\varPreTx}$ and returns either 0 or 1.
        \item Finally the simulator outputs whatever $\cnstAdversary$ outputs.
    \end{enumerate}
    We separate between the following three phases:
    \textbf{Phase 1}: Alice sends her partially owned inputs coins, newly created output coins, as well as her signing secrets to $\procZKfId{\cnstRelation_{1,2,3}}$ and sends $\varPreTx$.
    Carol sends her partiall owned input coins to $\procZKfId{\cnstRelation{3}}$
    \textbf{Phase 2}: Carol calls $\cnstRelation_{1,2,3}$ as the verifier constructs her output coin and signing secrets, now calls $\cnstRelation_{1,2}$ as the prover and sends the updated $\funStarAlt{\varPreTx}$ to Alice.
    \textbf{Phase 3}: Alice calls $\cnstRelation_{1,2,3}$ as the verifier

    We now argue why each phase is indistinguishable from a real execution in the case that Alice is corrupted.

    \begin{itemize}
        \item \textit{Phase 1}: No simulation is required in this phase, we therefore conclude that is indistinguishable from a real execution due to the deterministic nature of $\cnstAdversary$.
        \item \textit{Phase 2}: If $\cnstAdversary$ tried to cheat by providing invalid values in $\varPreTx$ the equalities that $\cnstSimulator$ checks will fail and will lead to a $\cnstFalsum$ output which is identically to what would happen in a real execution.
        $\cnstSimulator$ then sends $\funStarAlt{\varPreTx}$ to $\cnstAdversary$ which he received from the TTP and therefore has to be identical to the real execution, as Carol as the honest party must always provide exactly this message.
        \item \textit{Phase 3}: Again if $\cnstAdversary$ tries to cheat by sending an invalid value, he will receive a 0 bit, which would also happen in the real execution.
    \end{itemize}

    As the transcript is identical to a transcript of a real protocol execution we conclude that the simulation is perfect.

    \textbf{Carol is corrupt}: Simulator $\cnstSimulator$ works as follows:
    \begin{enumerate}
        \item $\cnstSimulator$ invokes $\cnstAdversary$ and saves $\funArray{\varPtSpendableCoinCarol}$ when the adversary calls $\procZKfId{\cnstRelation_{3*}}$
        \item The simulator then chooses $\varBlindingFactorAlice, \funStar{\varBlindingFactorAlice}, \varFundValue \sample \cnstIntegersPrimeWithoutZero{\varPrime}$ and sets \\ $\varPtSpendableCoinAlice \opAssign \{ \varCoin \opAssign \varPtSpendableCoinCarol.\varCoin, \varBlindingFactor \opAssign \varBlindingFactorAlice, \varValue \opAssign \varPtSpendableCoinCarol.\varValue \}$.
        He then proceeds by building $\varPreTx$ as given by the protocol definition with the chosen values and $\funArray{\varPtSpendableCoinAlice}$ and sends it to $\cnstAdversary$ as if coming from Alice.
        \item When Carol calls $\procZKfId{\cnstRelation_{1,2,3}}$ as the verifier $\cnstSimulator$ checks the passed values for equality and returns either 0 or 1.
        As soon as Carol calls $\procZkf{\cnstRelation_{2}}{\varSpendableCoinCarol}$ $\cnstSimulator$ will extract $\varFundValue \funStar{\varSpendableCoinCarol}.\varValue$ and finally call the TTP with inputs $(\funArray{\varPtSpendableCoinCarol}, \varFundValue)$ \todo{you also need $t$ here?} to receive $\funStarAlt{\varPreTx}, \funStar{\varSpendableCoinCarol}, (\varSecKeyCarol, \varNonceCarol)$.
        \item Now the simulator rewinds to step 1 and constructs the actual $\varPreTx$ from $\funStarAlt{\varPreTx}$ as follows:
        \begin{gather*}
            \{ \varMsg, \varInputs, \varOutputs, \varProofs, \varSigContext, \varCommits, \cnstEmptySet, \varTime \} \opFunResult \funStarAlt{\varPreTx} \\
            \varPubKeyAlice \opAssign \funStarAlt{\varPreTx}.\varSigContext.\varPubKey \opAddPoint {\funGen{\varSecKeyCarol}}^{-1} \\
            \varRandAlice \opAssign \funStarAlt{\varPreTx}.\varSigContext.\varRand \opAddPoint {\funGen{\varNonceCarol}}^{-1} \\
            \funStar{\varSigContext} \opAssign \{ \varPubKey \opAssign \varPubKeyAlice, \varRand \opAssign \varRandAlice \} \\
            \varPreTx \opAssign \procCreatePreTx{\varMsg}{\varInputs}{\varOutputs[0]}{\varProofs[0]}{\funStar{\varSigContext}}{\varCommits[0]}{\cnstEmptySet}{\varTime}
        \end{gather*}
        he then sends again $\varPreTx$ Carol and continues as before
        \item When $\cnstAdversary$ sends $\funStarAlt{\varPreTx}$ he compares its inputs, outputs, proofs and signature context to $\funStarAlt{\varPreTx}$ received from the trusted third party and outputs $\cnstFalsum$ and sends $\cnstAbort$ to the TTP if any do not match.
        Otherwise he sends $\cnstContinue$ to the TTP and outputs whatever $\cnstAdversary$ outputs.
    \end{enumerate}

    We again show that in each phase the transcript produced by the simulator is computationally indistinguishable from a real transcript.

    \begin{itemize}
        \item \textit{Phase 1}: In the first iteration (before the rewind) the pre-transaction that is send to $\cnstAdversary$ will be constructed from randomly chosen values except for the transaction inputs which are given by the commitments in $\funArray{\varPtSpendableCoinCarol}$.
        Due to the hiding property of the pedersen commitment the adversary cannot determine if the correct value $\varFundValue$ has been used to construct the output coin, even though he in fact knows the correct value for $\varFundValue$, but does not know the blinding factor $\funStar{\varBlindingFactorAlice}$.
        $\cnstAdversary$ does know the correct values for the input coins from $\funArray{\varPtSpendableCoinCarol}$ thereby it is critical that $\cnstSimulator$ uses the commitments extracted from $\funArray{\varPtSpendableCoinCarol}$ to build the transaction.
        Otherwise the simulation could be detected.
        In the second iteration (after the rewind) $\cnstSimulator$ sends the same $\varPreTx$ which would be sent in a real execution which is therefore identical.
        \item \textit{Phase 2}: When $\cnstAdversary$ calls $\procZKfId{\cnstRelation_{1,2,3}}$ he will receive 0 or 1 identically to the real execution.
        \item \textit{Phase 3}: If $\cnstAdversary$ sends invalid input, output, proof or context values is the final pre-transaction $\funStarAlt{\varPreTx}$ the simulator detects this and ouputs $\cnstFalsum$, otherwise the protocol concludes, which is the same that would happen in the real exeuction.
    \end{itemize}

    We have managed to show that the simulator $\cnstSimulator$ can produce an indistinguishable transcript both in the case that Alice and that Carol is corrupted and can thereby conclude that $\procDSendCoinsId$ is secure in the $\procZKfId{\cnstRelation_1}$,$\procZKfId{\cnstRelation_2}$,$\procZKfId{\cnstRelation_3}$-model and theorem~\ref{teo:atom:sec-dspend} holds.
\end{proof}

We continue by proofing security of the $\procDRecvCoinsId$ which is called inside the $\procDSharedOutputMwTxId$ protocol.

\textbf{Hybrid functionalities}: Again the parties have access to a trusted third party computing the zero-knowledge proof of knowledge functionalities $\procZKfId{\cnstRelation_1}$, $\procZKfId{\cnstRelation_2}$ and $\procZKfId{\cnstRelation_2*}$.
For this proof we do not need $\cnstRelation_3$ as defined in the previous proof, however we extend the model with two further protocols which have already been proven secure.
We extend our model by including the $\procDSignId$ protocol for which security has been proven in section~\ref{sec:sig:two-party-apt-security} and the $\procDRProofId$ for which a secure protocol can be found in~\cite{klinec2020privacy}

\textbf{Proof idea}: We extend the protocol $\procDRecvCoinsId$ instantiated in~\ref{sec:atom:inst} with the following calls to the zero-knowledge proof of knowledge functionalities as outlined in figure~\ref{fig:atom:hybrid-drecv}.

\begin{figure}
    \fbox{
    \begin{varwidth}{\textwidth}
        \procedure[linenumbering,skipfirstln]{$\procDRecvCoins{\varPreTx}{\varFundValue}$} {
        Bob \< \< \< \< Carol \\
        \cdots \\
        \procZkf{\cnstRelation_2}{(\varCoinShared, (\varFundValue, \funStar{\varBlindingFactorBob}))} \\
        \< \sendmessageright*{\varPreTx, \varCoinShared} \< \\
        \< \< \< \< \pcif \procZkf{\cnstRelation_2}{\varCoinShared} \opEqNoQ 0 \\
        \< \< \< \< \t \pcreturn \cnstFalsum \\
        \< \< \< \< \cdots \\
        \< \< \< \< \procZkf{\cnstRelation_1}{(\varSecKeyCarol, \funGen{\varSecKeyCarol})} \\
        \< \sendmessageleft*{\funStarAlt{\varPreTx}, \funGen{\varSecKeyCarol}} \< \\
        \pcif \procZkf{\cnstRelation_1}{\funGen{\varSecKeyCarol}} \opEqNoQ 0 \\
        \t \pcreturn \cnstFalsum \\
        \cdots \< \< \< \< \cdots \\
        \varProofBobCarol \opFunResult \procDRProofL{\funStarAlt{\varCoinShared}}{\varFundValue}{\varSecKeyBob} \< \< \< \< \varProofBobCarol \opFunResult \procDRProofL{\funStarAlt{\varCoinShared}}{\varFundValue}{\varSecKeyCarol} \\
        (\varSigBobCarol, \varPubKeyBobCarol) \opFunResult \< \< \< \< (\varSigBobCarol, \varPubKeyBobCarol) \opFunResult \pcskipln \\
        \procDSignL{\varMsg}{\varSecKeyBob}{\varNonceBob} \< \< \< \< \procDSignR{\varMsg}{\varSecKeyCarol}{\varNonceCarol} \\
        \cdots \< \< \< \< \cdots \\
        \pcreturn (\funStar{\varPreTx}, \funStar{\varPtSpendableCoinBob}) \< \< \< \< \pcreturn (\funStar{\varPreTx}, \funStar{\varPtSpendableCoinCarol})
        }
    \end{varwidth}
    }
    \caption{$\procDRecvCoinsId$ in the hybrid model} \label{fig:atom:hybrid-drecv}
\end{figure}

\begin{theorem}
    \label{teo:atom:sec-drecv}
    Let $\varCommitScheme$ be a correct and secure pedersen commitment scheme, $\varMPRProofSystem$ be a correct and secure multiparty range proof system and $\varSigSchemeMP$ be a secure and correct two-party signature scheme, then $\procDRecvCoinsId$ securely updates a mimbewimble pre-transaction by creating a new output coin $\funStarAlt{\varCoinShared}$ for which the key is shared between two parties Bob and Carol in the $\procZKfId{\cnstRelation_1}$,$\procZKfId{\cnstRelation_2}$,$\procDSignId$,$\procDRProofId$-model.
\end{theorem}

\begin{proof}
    As before we proof security by construction of a simulator $\cnstSimulator$ with access to a trusted third party (TTP) computing $\procDRecvCoinsId$ in the ideal model upon receiving inputs from the two parties.
    The simulators task is to extract the inputs of the adversary $\cnstAdversary$, send the inputs to the TTP and construct a protocol transcript indistinguishable from a real one.
    We first look at the case in which Bob is corrupted and then when Carol is corrupted.

    \textbf{Bob is corrupted}: Simulator $\cnstSimulator$ works as follows:
    \begin{enumerate}
        \item $\cnstSimulator$ invokes $\cnstAdversary$ and saves ($\varCoinShared$, ($\varFundValue, \funStar{\varBlindingFactorBob}$)) when the adversary calls $\procZKfId{\cnstRelation_2}$.
        \item $\cnstAdversary$ sends $(\varPreTx,\varCoinShared)$.
        The simulator then compares $\varCoinShared$ with the values saved in its memory and sends $\cnstAbort$ to the TTP and outputs $\cnstFalsum$ if they don't match.
        Otherwise he sends $(\varPreTx, \varFundValue)$ to the TTP computing $\procRecvCoinsId$ and receives the outputs $(\funStar{\varPreTx}, \funStar{\varPtSpendableCoinBob})$.
        \item $\cnstSimulator$ proceeds by taking the last output $\funStarAlt{\varCoinShared}$ from $\funStar{\varPreTx}.\varOutputs$ and computes $\funGen{\varSecKeyCarol} \opAssign \funStarAlt{\varCoinShared} \opAddPoint {\varCoinShared}^{-1}$.
        The simulator computes $\funStarAlt{\varPreTx}$ by adding $\funStarAlt{\varCoinShared}$ to $\varPreTx$ and sends it together with $\funGen{\varSecKeyCarol}$ to $\cnstAdversary$ as if coming from Carol and sends $\cnstContinue$ to the TTP.
        \item When $\cnstAdversary$ calls $\procZKfId{\cnstRelation_1}$ as the verifier $\cnstSimulator$ check equality with the correct value and returns either 0 or 1.
        \item When the adversary calls $\procDRProofId$ the simulator saves $\varSecKeyBob$ to its memory and returns the last element of $\funStar{\varPreTx}.\varProofs$ as received from the TTP.
        \item For the call to $\procDSignId$ the simulator returns the $\varPreTx.\varSignature$ as the signature and $\funGen{\varSecKeyBob} \opAddPoint \funGen{\varSecKeyCarol}$ as the public key.
        \item $\cnstSimulator$ concludes by outputting whatever $\cnstAdversary$ outputs.
    \end{enumerate}

    We find the following phases:
    \textbf{Phase 1}: Bob calls $\procZKfId{\cnstRelation_2}$ and sends $\varPreTx$ to Carol.
    \textbf{Phase 2}: Carol calls $\procZKfId{\cnstRelation_2}$ as the verifier adds her public key to the commitment and sends back an updated pre-transaction and her public key.
    \textbf{Phase 3}: Bob calls $\procZKfId{\cnstRelation_1}$ as the verifier and the parties call the trusted third parties computing $\procDRProofId$ and $\procDSignId$.

    We argue that in this case the simulation is perfect, that is the transcript produced by $\cnstSimulator$ is identical to the transcript of the real execution.
    \begin{itemize}
        \item \textit{Phase 1}: No simulation is done during this phase, and the transcript is therby indistinguishable by the deterministic nature of $\cnstAdversary$.
        \item \textit{Phase 2}: In case $\cnstAdversary$ sends an invalid value for $\varCoinShared$ the execution will stop with output $\cnstFalsum$ which is identical to what would happen in the real execution.
        The simulator can then send the updated pre-transaction as the honest Carol would do and the extracted real value for $\funGen{\varSecKeyCarol}$.
        \item \textit{Phase 3}: $\cnstAdversary$ will receiver 0 or 1 to the call to $\procZKfId{\cnstRelation_1}$ as in the real execution.
        The simulator further manages to reconstruct the real output values for $\procDRProofId$ and $\procDSignId$ again making the transcript identical in this phase.
    \end{itemize}

    \textbf{Carol is corrupted}: The simulator works as follows:
    \begin{enumerate}
        \item Since Carol does not have any inputs in this protocol $\cnstSimulator$ can simply send $\cnstEmptySet$ to the TTP and receives ($\funStar{\varPreTx}, \funStar{\varSpendableCoinCarol}$) from which he extracts Carols bliding factor (and secret key) as $\varSecKeyCarol \opAssign \funStar{\varSpendableCoinCarol}.\varBlindingFactor$.
        He can now create the initial shared coin $\varCoinShared$ by taking the last output of $\funStar{\varPreTx}.\varOutputs$ as $\funStarAlt{\varCoinShared}$ and calculating $\varCoinShared \opAssign \funStarAlt{\varCoinShared} \opAddPoint {\funGen{\varSecKeyCarol}}^{-1}$.
        he can further create the initial pre-transaction by removing the last entry of the output coin list, last entry of the proof list and signature from $\funStar{\varPreTx}$.
        \item $\cnstSimulator$ invokes $\cnstAdversary$ and send $\varPreTx, \varCoinShared$ (as calculated in step 1) as if coming from Bob.
        \item When $\cnstAdversary$ calls $\procZKfId{\cnstRelation_2}$ as the verifier the simulator checks for equality with what he send in the last step and returns either 0 or 1.
        \item The adversary then sends the updated $\funStarAlt{\varPreTx}$ which the simulator validates by checking if the last entry in $\funStarAlt{\varPreTx}.\varOutputs$ equals $\funStarAlt{\varCoinShared}$.
        If they don't $\cnstSimulator$ will output $\cnstFalsum$ and send $\cnstAbort$ to the TTP halting the execution, otherwise he will send $\cnstContinue$.
        \item Upon the adversary calling $\procDRProofId$ the simulator will return the proof at the last position in the proofs array of $\funStar{\varPreTx}.\varProofs$ received from the TTP.
        \item The simulator then extracts $\varFundValue \opAssign \funStar{\varPtSpendableCoinCarol}$ and computes $\varPubKeyBob \opAssign \varCoinShared \opAddPoint {\funGenH{\varValue}}^{-1}$ and returns $\funStar{\varPreTx}.\varSignature$ and $\funStar{\varSecKeyCarol} \opAddPoint \varPubKeyBob$ when $\cnstAdversary$ calls $\procDSignId$.
        \item The simulation completes with $\cnstSimulator$ outputting whatever $\cnstAdversary$ outputs.
    \end{enumerate}

    We now argue why in each of the three phases the transcript produced by $\cnstSimulator$ is indistinguishable from a real transcript.

    \begin{itemize}
        \item \textit{Phase 1}: Because $\cnstSimulator$ as able to call the TTP already in the first step he is able to receive the protocol outputs.
        The simulator can then extracts carols secret key $\varSecKeyCarol$ from Carols $\funStar{\varPtSpendableCoinCarol}$ output, which must also be her blinding factor in $\funStarAlt{\varCoinShared}$.
        He therefore can reconstruct $\varCoinShared$ which must be sent by Bob in this phase, simply by subtracting Carols part from the output which is present in $\funStar{\varPreTx}.\varOutputs$.
        $\cnstSimulator$ is further able to reconstruct the $\varPreTx$ which must be sent by Bob in this phase simply by removing the values from $\funStar{\varPreTx}$ which get added at a later point in the protocol.
        The transcript is therefore identical to a real one in this phase.
        \item \textit{Phase 2}: If $\cnstAdversary$ tries do cheat by sending an invalid value to $\procZKfId{\cnstRelation_2}$ as the verifier he will receive 0 as a response and 1 otherwise, which is identical to the real case.
        Similarity the execution will halt with $\cnstFalsum$ if $\cnstAdversary$ sends invalid values as $\funStarAlt{\varPreTx}$ and $\funGen{\varSecKeyCarol}$, again identical to a real execution.
        \item \textit{Phase 3}: $\cnstSimulator$ is able to read the output values for $\varProofBobCarol$ and $\varSigBobCarol$ from $\funStar{\varPreTx}$, he further is able to calculate $\varPubKeyBobCarol$ as he knows $\funGen{\varSecKeyCarol}$ and is further able to reconstruct $\varPubKeyBob$ from $\varCoinShared$.
        Therefore the simulation again is perfect in this phase.
    \end{itemize}

    Both in the case the Bob and Carol is corrupted $\cnstSimulator$ is able to produce a transcript indistinguishable from a transcript produced on a real execution we can therefore conclude that the protocol is secure in the $\procZKfId{\cnstRelation_1}$,$\procZKfId{\cnstRelation_2}$,$\procDSignId$,$\procDRProofId$-model and theorem~\ref{teo:atom:sec-drecv} holds.
\end{proof}

We claim that the security of the protocols $\procDFinTxId$ and $\procDAptFinTxId$ can be reduced to the security of $\procDSignId$ as all interaction between the two parties happens in the call to $\procDSignId$.
We have already proven the security of $\procDSignId$ in section~\ref{sec:sig:two-party-apt-security} and can reuse the simulator constructed there for the protcols $\procDFinTxId$ and $\procDAptFinTxId$.

We can now continue to proof security of the protocols found in section~\ref{sec:atom:protocols}.
We start with $\procDSharedOutputMwTxId$.

\textbf{Hybrid functionalities}: For this proof we again assume the access to a trusted third party computing the zero-knowledge proof of knowledge functionalities $\procZKfId{\cnstRelation_1}$, $\procZKfId{\cnstRelation_2}$ and $\procZKfId{\cnstRelation_{3*}}$, with the three relations defined as in previous proofs.
We further require a trusted third party computing $\procDRecvCoinsId$, which we have already proven to be secure in the hybrid model.

\textbf{We extend the protocol $\procDSharedOutputMwTxId$} instantiated in section~\ref{sec:atom:protocols} with the following calls to the zero-knowledge proof of knowledge functionalities shown in figure \ref{fig:atom:hybrid-sharedinp}.

\begin{figure}
    \fbox{
    \begin{varwidth}{\textwidth}
        \procedure[linenumbering,skipfirstln]{$\procDSharedOutputMwTx{\funArray{\varSpendableCoin}}{\varFundValue}{\varTime}$}{
        Alice \< \< Bob \\
        \procZkf{\cnstRelation_{3*}}{\funArray{\varSpendableCoin}} \\
        \cdots \\
        \procZkf{\cnstRelation_{2}}{\funStar{\varSpendableCoinAlice}} \\
        \procZkf{\cnstRelation_{1}}{(\varSecKeyAlice, \funGen{\varSecKeyAlice})} \\
        \procZkf{\cnstRelation_{1}}{(\varNonceAlice, \funGen{\varNonceAlice})} \\
        \< \sendmessageright*{\varPreTx} \< \\
        \< \< \pcif \procZkf{\cnstRelation_{3*}}{\varPreTx.\varInputs} \opEqNoQ 0 \\
        \< \< \t \pcreturn \cnstFalsum \\
        \< \< \pcif \procZkf{\cnstRelation_{2}}{\varPreTx.\varOutputs[0]} \opEqNoQ 0 \\
        \< \< \t \pcreturn \cnstFalsum \\
        \< \< \pcif \procZkf{\cnstRelation_{1}}{\varPreTx.\varSigContext.\varPubKey} \opEqNoQ 0 \\
        \< \< \t \pcreturn \cnstFalsum \\
        \< \< \pcif \procZkf{\cnstRelation_{1}}{\varPreTx.\varSigContext.\varRand} \opEqNoQ 0 \\
        \< \< \t \pcreturn \cnstFalsum \\
        (\funStarAlt{\varPreTx}, \funStar{\varPtSpendableCoinAlice}) \< \< (\funStarAlt{\varPreTx}, \funStar{\varPtSpendableCoinBob}) \pcskipln \\
        \opFunResult \procDRecvCoinsL{\varPreTx}{\varFundValue}  \< \< \opFunResult \procDRecvCoinsR \\
        \varTx \opFunResult \procFinTx{\funStarAlt{\varPreTx}}{\varSecKeyAlice}{\varNonceAlice} \\
        \< \sendmessageright*{\varTx} \\
        \pcreturn (\varTx, \funStar{\varSpendableCoinAlice}, \funStar{\varPtSpendableCoinAlice}) \< \< \pcreturn (\varTx, \funStar{\varPtSpendableCoinBob})
        }
    \end{varwidth}
    }
    \caption{$\procDSharedOutputMwTxId$ in the hybrid model}  \label{fig:atom:hybrid-sharedinp}
\end{figure}

\begin{theorem}\label{teo:atom:sec-sharedout-tx}
    Let $\varCommitScheme$ be a correct and secure pedersen commitment scheme, $\varProofSystem$ be a correct and secure range proof system and $\varSigSchemeMP$ be a secure and correct two-party signature scheme, then $\procDSharedOutputMwTxId$ securely computes a Mimblewimble transaction with a output coin $\funStarAlt{\varCoinShared}$ which spending secret is shared between Alice and Bob.
\end{theorem}

\begin{proof}
    We proof security of the protocol in the malicious setting by constructing a simulator $\cnstSimulator$ with access to a trusted third party (TTP) computing $\procDSharedOutputMwTxId$ in the ideal model upon receiving inputs from the two parties.
    The simulators task is to extract the adversaries inputs, send them to the TTP to receive the protocol outputs and construct a transcript indistinguishable from a transcript produced in a real execution.
    We separately look at the case in which Alice and in which Bob is corrupted.

    \textbf{Alice is corrupted}: Simulator $\cnstSimulator$ works as follows:
    \begin{enumerate}
        \item $\cnstSimulator$ invokes $\cnstAdversary$ and saves $\funArray{\varSpendableCoin}$, $\varSecKeyAlice$, $\varNonceAlice$ and $\funStar{\varSpendableCoinAlice}$ to its memory.
        \item $\cnstAdversary$ sends $\varPreTx$ from which $\cnstSimulator$ extracts $\varTime \opAssign \varPreTx.\varTime$.
        He further extracts $\varFundValue \opAssign \sum\varSpendableCoin_i.\varValue \opSub \funStar{\varSpendableCoinAlice}.\varValue$.
        $\cnstSimulator$ verifies that the values $\varPreTx.\varInputs$, $\varPreTx.\varOutputs$, $\varPreTx.\varProof$ and $\varPreTx.\varSigContext$ correspond to what he has saved to its memory.
        In case this verification failes he sends $\cnstAbort$ to the TTP and outputs $\cnstFalsum$.
        \item $\cnstSimulator$ sends ($\funArray{\varSpendableCoin}$, $\varFundValue$, $\varTime$) to the TTP and receives ($\varTx$, $\funStar{\varSpendableCoinAlice}$, $\varPtSpendableCoinAlice$).
        \item When $\cnstAdversary$ calls $\procDRecvCoinsId$ $\cnstSimulator$ verifies that $\varPreTx$ and $\varFundValue$ passed by $\cnstAdversary$ are correct and only then forwards them to the TTP to receive ($\funStarAlt{\varPreTx}, \funStar{\varPtSpendableCoinAlice}$) which he then returns to $\cnstAdversary$.
        Otherwise he returns $\cnstFalsum$ to $\cnstAdversary$ and sends $\cnstAbort$ to the TTP and halts the protocol
        \item $\cnstSimulator$ sends $\cnstContinue$ to TTP.
        Eventually $\cnstAdversary$ sends $\varTx$ after which $\cnstSimulator$ outputs whatever $\cnstAdversary$ outputs.
    \end{enumerate}

    It is easy to see that the simulation is perfect as every simulated message exchanged between the party is identical to what would be exchanged in a real execution.
    Also if the adversary cheats (by sending an invalid $\varPreTx$) this is noticed by the simulator who then outputs $\cnstFalsum$ which is identical as well to the real case.

    \textbf{Bob is corrupted}: Simulator works as follows:
    \begin{enumerate}
        \item $\cnstSimulator$ invokes $\cnstAdversary$ and sends $()$ to the TTP to receive the outputs ($\varTx$, $\funStar{\varPtSpendableCoinBob}$)
        \item $\cnstSimulator$ now has the following challenge: $\cnstAdversary$ first expects the first pre-transaction $\varPreTx$ coming from Alice, which should not have any signature, only one output (Alice change output) and only a partially setup signature context $\varSigContext$.
        To achieve this $\cnstSimulator$ clones $\varTx$ into $\varPreTx$, removes the last output coin (and proof), and sets the signature field to $\cnstEmptySet$.
        The simulator can now construct the partially setup signature context as follows:
        \begin{gather*}
            \varSecKeyAlice, \varNonceAlice \sample \cnstIntegersPrimeWithoutZero{\varPrime} \\
            \funStarAlt{\varSigContext} \opAssign \{ \varPubKey \opAssign \funGen{\varSecKeyAlice}, \funGen{\varNonceAlice} \}
        \end{gather*}
        He then sets $\varPreTx.\varSigContext \opAssign \funStarAlt{\varSigContext}$ and sends $\varPreTx$ to $\cnstAdversary$ as if coming from Alice.
        \item When $\cnstAdversary$ calls $\procZKfId{\cnstRelation_{1,2,3}}$ as the verifier $\cnstSimulator$ compares the values with what he sent in step 1 in $\varPreTx$ and returns either 0 or 1.
        \item $\cnstAdversary$ will call $\procDRecvCoinsId$ upon which $\cnstSimulator$ calls the TTP computing $\procDRecvCoinsId$ to receive $\funStarAlt{\varPreTx}$, $\funStar{\varPtSpendableCoinBob}$ which $\cnstSimulator$ then returns to $\cnstAdversary$.
        \item Finally $\cnstSimulator$ sends $\varTx$ as if coming from Alice and outputs whatever $\cnstAdversary$ outputs.
    \end{enumerate}

    It is easy to see that $\varTx$ sent by $\cnstSimulator$ in the last step must be identical to the value from a real execution as it has been computed by the TTP.
    Also when $\cnstAdversary$ tries to cheat by sending invalid values to $\procZKfId{\cnstRelation_{1,2,3}}$ will he receive 0, as would be the case in a real execution.
    $\funStarAlt{\varPreTx}$ must be identical to a real execution as computed by the trusted third party computing $\procDRecvCoinsId$.
    Therefore the only thing that remains to show is that $\varPreTx$ constructed by the simulator is indistinguishable from a $\varPreTx$ exchanged in a real transcript.
    We note that $\varSecKeyAlice$ and $\varNonceAlice$ in a real execution are uniformly distributed values in $\cnstIntegersPrimeWithoutZero{\varPrime}$.
    Consequently $\funGen{\varSecKeyAlice}$ and $\funGen{\varNonceAlice}$ are uniformly distributed in $\cnstGroup$.
    By construction of $\cnstSimulator$ this must also hold in the simulated case.
    Therefore the signature context constructed in step 2 for $\varPreTx$ must be indistinguishable from a real one, which also means that the $\varPreTx$ is indistinguishable, as the rest of the values are taken from $\varTx$ as computed by the TTP.
    We must also note that even when $\cnstAdversary$ receives $\funStarAlt{\varPreTx}$ and $\varTx$ later in the protocol, he has no way of realizing that $\varTx$ was constructed by $\cnstSimulator$.
    This follows from the fact that the final $\varRand$ and $\varPubKey$ in the final signature context of $\funStarAlt{\varPreTx}$ and $\varTx$ is composed of three values each:
    $\varSigContext \opEqNoQ {\varPubKeyAlice}_1 \opAddPoint {\varPubKeyAlice}_2 \opAddPoint {\varPubKeyBob}$ (similar for $\varRand$).
    $\cnstAdversary$ only learns one of Alice's public keys (from step 2) and knows his own, but does not know anything about Alice second keypair.
    Therefore he has no way of learning that the final $\varPubKey$ is not constructed correctly.
    The same argument holds for $\varRand$.

    We have shown that the simulator $\cnstSimulator$ is able to produce an indistinguishable transcrip both in the case that Alice and that Bob is corrupted and can thereby conclude that $\procDSharedOutputMwTxId$ is secure in the $\procZKfId{\cnstRelation_1}$,$\procZKfId{\cnstRelation_2}$,$\procZKfId{\cnstRelation_3}$,$\procDRecvCoinsId$-model and consequently theorem~\ref{teo:atom:sec-sharedout-tx} holds.
\end{proof}

Next we proof security for $\procDSharedInpMwTxId$.

\textbf{Hybrid functionalities}: For this proof it is enough to give the parties access to a trusted third party computing the $\procDSendCoinsId$ and the $\procDFinTxId$ protocol.
Further calls to a zero-knowledge proof of knowledge functionality are not needed.
This means that we do not have to extend to original protocol instantiation any further.

\begin{theorem} \label{teo:atom:sec-dshared-inp}
    Let $\varCommitScheme$ be a correct and secure pedersen commitment scheme, $\varProofSystem$ a correct and secure range proof system and $\varSigSchemeMP$ be a secure and correct two-party signature scheme, then $\procDRecvCoinsId$ securely computes a mimewimble transaction spending an input coin $\varCoinShared$ shared between Alice and Bob in the hybrid $\procDSendCoinsId,\procDFinTxId$-model
\end{theorem}

\begin{proof}
    We proof security by construction of simulator $\cnstSimulator$ with access to a trusted third party (TTP) computing $\procDSharedInpMwTxId$ in the ideal model upon receiving inputs from the two parties, as well as a trusted third party computing the functionality of the hybrid model.
    The simulators task is to extract the adversaries inputs, send them to the TTP and construct a protocol transcript indistinguishable from a real one.
    We again look separately at the case in in which Alice, and in which Bob is corrupted.

    \textbf{Alice is corrupted}: Simulator $\cnstSimulator$ works as follows:
    \begin{enumerate}
        \item $\cnstSimulator$ invokes $\cnstAdversary$ and saves $\funArray{\varPtSpendableCoinAlice}$, $\varFundValue$ and $\varTime$ when $\cnstAdversary$ calls $\procDSendCoinsId$.
        \item He then forwards those values as the inputs to the TTP computing $\procDSendCoinsId$ and receives ($\varPreTx$, $\funStar{\varSpendableCoinAlice}, (\varSecKeyAlice, \varNonceAlice)$) which he returns to $\cnstAdversary$.
        He proceeds by sending the inputs ($\funArray{\varPtSpendableCoinAlice}, \varFundValue, \varTime$) to the TTP computing $\procDSharedInpMwTxId$ and receives ($\varTx, \funStar{\varSpendableCoinBob}$).
        \item The simulator now has the challenge to construct a $\funStarAlt{\varPreTx}$ which is partially signed.
        The final signature is composed of A + B1 + B2, where B2 is the signature from Bobs output coins and A + B1 are the signatures from the shared input coin.
        $\funStarAlt{\varPreTx}$ has to contain the partial signature B2, such that the partial signature verification algorithm verifies and such that when combined with the signatures A and B1 it will complete into the final signature $\varTx.\varSignature$.
        Therefore the only way for the simulator to create a valid simulation is to calculate the actual value for the B2 signature, which is challenging since he does not know $\varSecKeyBob$, $\varNonceBob$.
        However, he knows the final siganture $\varSigFin \opAssign \varTx.\varSignature$ and he can create the signature A as $\varSigAlice \opFunResult \procSignPrt{\varTx.\varMsg}{\varSecKeyAlice}{\varNonceAlice}{\varTx.\varSigContext}$.
        $\cnstSimulator$ is able to recompute the value for the B2 signature as follows:
        \begin{enumerate}
            \item $\cnstSimulator$ choses $(\funStarAlt{\varSecKeyBob}, \funStarAlt{\varNonceBob}) \sample \cnstIntegersPrimeWithoutZero{\varPrime}$
            \item He then computes a temporary $\funStarAlt{\varSigBob} \opFunResult \procSignPrt{\varTx.\varMsg}{\funStarAlt{\varSecKeyBob}}{\funStarAlt{\varNonceBob}}{\varTx.\varSigContext}$
            \item He clones $\varTx$ into $\funStarAlt{\varPreTx}$ and sets $\funStarAlt{\varPreTx}.\varSignature \opAssign \funStarAlt{\varSigBob}$
            \item Now the simulator calls the TTP computing $\procDFinTxId$ with the inputs $\funStarAlt{\varPreTx}$, $\varSecKeyAlice$, $\varNonceAlice$ to receive $\funStarAlt{\varTx}$
            \item Note that the signature in $\funStarAlt{\varTx}$ now contains a signature composed of A + B1 + B2', where B2' is the partial signature computed in step b.
            Therefore now it is possible to recompute the value of the partial signature for B1 as follows:
            \begin{gather*}
                (\funStarAlt{\varS}, \funStarAlt{\varRand}) \opFunResult \funStarAlt{\varTx} \\
                (\varSAlice, \varRandAlice, \varSigContext) \opFunResult \varSigAlice \\
                (\funStarAlt{\varSBob}, \funStarAlt{\varRandBob}, \varSigContext) \opFunResult \funStarAlt{\varSigBob} \\
                {\varSBob}_1 \opAssign \funStarAlt{\varS} \opSub \varSAlice \opSub \funStarAlt{\varSBob} \\
                {\varRandBob}_1 \opAssign \funStarAlt{\varRand} \opAddPoint {\varRandAlice}^{-1} \opAddPoint {\funStarAlt{\varRandBob}}^{-1} \\
                {\varSigBob}_1 \opAssign \{ {\varSBob}_1, {\varRandBob}_1, \varSigContext \}
            \end{gather*}
            \item $\cnstSimulator$ now has the correct values for the signatures A and B1 and can therefore recompute the correct value for the partial signature B2 from $\varTx.\varSignature$ with the same calculation as shown in the previous step
        \end{enumerate}
        \item $\cnstSimulator$ is now able to construct $\funStarAltDouble{\varPreTx}$ by again cloning $\varTx$ and setting $\funStarAltDouble{\varPreTx}.\varSignature \opAssign {\varSigBob}_2$.
        The simulator will rewind the call to the TTP computing $\procDFinTxId$ and send $\funStarAltDouble{\varPreTx}$ to $\cnstAdversary$ as if coming from Bob.
        \item When $\cnstAdversary$ calls $\procDFinTxId$ $\cnstSimulator$ will forward the inputs to the TTP party computing $\procDFinTxId$, return the TTP outputs back to $\cnstAdversary$ and finally output whatever $\cnstAdversary$ outputs.
    \end{enumerate}

    As only $\funStarAlt{\varPreTx}$ is constructed by $\cnstSimulator$, it is the only value for which we have to prove indistinguishability.
    We have already shown that the final signature in $\varTx$ is composed of three parts A, B1 and B2.
    Through the calculations layed out the simulator is able to recompute the real value of B2, which must make $\funStarAlt{\varPreTx}$ identical to the one send by Bob in the real execution.

    \textbf{Bob is corrupted}: Simulation in this case is trivial, as there is no message sent from Alice to Bob and $\cnstSimulator$ doesn't need to extract any inputs.
    A perfect simulation is therefore achieved simply by forwarding the inputs sent by $\cnstAdversary$ to the TTP computing $\procDSendCoinsId$ and $\procDFinTxId$ and finally outputting whatever $\cnstAdversary$ outputs.

    We have managed to construct a simulator in the case the Alice as well as that Bob is corrupted which produced a protocol transcript indistinguishable from a real one and can therefore conclude that $\procDSharedInpMwTxId$ is secure in the $\procDSendCoinsId$,$\procDFinTxId$-model and theorem \ref{teo:atom:sec-dshared-inp} must hold.
\end{proof}

We now move to the final proof, proving security of $\procDScriptMwTxId$:

\textbf{Hybrid functionalities}: We proof the security of $\procDScriptMwTxId$ in the hybrid model in which the participants have access to a trusted third party computing $\procDSendCoinsId$ and $\procDAptFinTxId$.
We also again require acces to a trusted third party computing the zero-knowledge proof of knowledge functionality $\procZKfId{\cnstRelation_1}$, with $\cnstRelation_1$ being defined equally as in previous proofs.

\textbf{Proof idea}: We extend the original $\procDScriptMwTxId$ with a single call to $\procZKfId{\cnstRelation_1}$ from each Alice and Bob.
On Bobs side we extend the protocol with the following call at the beginning of the protocol: $\procZkf{\cnstRelation_1}{(\varWit, \funGen{\varWit})}$.
On Alice side we add the following verification at line 2 of the protocol: $\textbf{If}\; \procZkf{\cnstRelation_1}{\varStatement} \opEqNoQ 0 \; \textbf{return}\; \cnstFalsum$.

\begin{theorem} \label{teo:atom:sec-dcontract-mw}
    Let $\varCommitScheme$ be a correct and secure pedersen commitment scheme, $\varProofSystem$ be a correct and secure range proof system and $\varSigSchemeMP$ be a secure and correct two-party signature scheme, then $\procDScriptMwTxId$ securly computes a mimlewimble transaction transfering value from a shared input coin $\varCoinShared$ to Bob, while at the same time reavealing a secret witness value $\varWit$ to Alice for which she knows the $\varStatement$ for which $\varStatement \opEqNoQ \funGen{\varWit}$.
\end{theorem}

\begin{proof}
    We proof security by constructing a simulator $\cnstSimulator$ with access to a trusted third party (TTP) computing $\procDScriptMwTxId$ in the ideal setting upon receiving inputs from the two parties.
    The simulators task is to extract the adversaries inputs, send them to the TTP and construct a transcript indistinguishable from a transcript produced during a real protocol execution.
    We separately look at the case in which Alice and in which Bob is corrupted.

    \textbf{Alice is corrupted}: Simulator $\cnstSimulator$ works as follows:
    \begin{enumerate}
        \item $\cnstSimulator$ invokes $\cnstAdversary$ and saves the inputs $\funArray{\varPtSpendableCoinAlice}$, $\varFundValue$ and $\varTime$ when the adversary calls $\procDSendCoinsId$.
        \item He forwards the inputs received in step 1 to the TTP computing $\procDSendCoinsId$ to receive the outputs ($\varPreTx$, $\funStar{\varSpendableCoinAlice}$, $\varKeyPairAlice$), which he then forwards to $\cnstAdversary$ as the protocol results.
        \item When $\cnstAdversary$ calls $\procZKfId{\cnstRelation_1}$ as the verifier $\cnstSimulator$ saves $\varStatement$ to his memory and sends the inputs ($\funArray{\varPtSpendableCoinAlice}$, $\varFundValue$, $\varTime$, $\varStatement$) to the TTP computing $\procDScriptMwTxId$ to receive the outputs ($\varTx$, $\funStar{\varSpendableCoinAlice}$, $\varWit$).
        \item As in the previous proof the simulator now has the task to construct a pre-transaction $\funStarAlt{\varPreTx}$ with a partial signature B2 of A, B1, B2.
        The simulator can compute $\varSigBob$ in the same way as we layed out in the previous proof.
        Note however that in this case $\cnstAdversary$ expects an adapted signature $\varSigAptBob$ which will verify with the adapted partial signature verification routine passing $\varStatement$.
        $\cnstSimulator$ can easily calculate the adapted signature by computing $\varSigAptBob \opFunResult \procAptSig{\varSigBob}{\varWit}$ and constructing $\funStarAlt{\varPreTx}$ by cloning $\varTx$ and setting the signature field to $\varSigAptBob$.
        Finally $\cnstSimulator$ sends $(\funStarAlt{\varPreTx},\varStatement)$ to $\cnstAdversary$ as if coming from Bob.
        \item When $\cnstAdversary$ calls $\procDAptFinTxId$ $\cnstSimulator$ forwards the inputs to the TTP computing $\procDAptFinTxId$ and returns the TTP outputs to $\cnstAdversary$.
        If the output returned to the adversary was not $\cnstFalsum$ the simulator will send $\varTx$ to $\cnstAdversary$ as if coming from Bob, send $\cnstContinue$ to the TTP computing $\procDScriptMwTxId$ and output whatever $\cnstAdversary$ outputs.
    \end{enumerate}

    In this proof only $\funStarAlt{\varPreTx}$ and $\varStatement$ sent in the first message from Bob to Alice is constructed by $\cnstSimulator$.
    All other values are direclty forwarded from a trusted third party and must therefore trivially be indistinguishable from the real execution.
    As $\cnstSimulator$ knows $\varWit$, constructing the real value of $\varStatement$ is simply calculating $\funGen{\varWit}$.
    That $\funStarAlt{\varPreTx}$ is identical to the value sent in a real execution must hold for the same reasons as outlined in the previous proof and by the fact that $\cnstSimulator$ knows $\varWit$ and can therefore call $\procAptSigId$ as it is called in a real execution by Bob.

    \textbf{Bob is corrupted}: Again finding a perfect simulator is trivial in this case as there are no messages send directly from Alice to Bob and $\cnstSimulator$ doesn't need to extract any inputs.
    Whenever $\cnstAdversary$ calls one of the trusted third parties to compute a hybrid functionality $\cnstSimulator$ externally forwards the call to the TTP and returns whatever was the result to $\cnstAdversary$.

    We have managed to construct a simulator producing a transcript indistinguishable from a real one both in the case that Alice and that Bob is corrupted and controlled by an adversary $\cnstAdversary$ and can therefore conclude that $\procDScriptMwTxId$ is secure in the $\procDSendCoinsId,\procDAptFinTxId,\procZKfId{\cnstRelation_1}$-model and theorem \ref{teo:atom:sec-dcontract-mw} must hold.
\end{proof}

\section{Atomic Swap Protocol}\label{sec:atom:atomic-swap}
With the outlined Adapted Mimblewimble Transaction Scheme from~\cref{def:atom:apt-ext-mw-tx-scheme} and protocols from~\cref{sec:atom:protocols} we can now construct an Atomic Swap protocol with another Cryptocurrency.
In this thesis we will explain a swap with Bitcoin, as at present Bitcoin is the most widely adopted Cryptocurrency.
To abstract away from the details of different Bitcoin implementations, we define here the minimal DPT functions that we require for our Atomic Swap.
These functionalities are inherent to the Bitcoin functionality and thus supported in all implementations.
We define the following three DPT functions $(\procLockAddrId, \procVerifyLockId, \procSpendBtcId)$.
\begin{itemize}
    \item $(\varScriptPubKey) \opFunResult \procLockAddr{\varPubKeyAlice}{\varPubKeyBob}{\varStatement}{\varTime}$:
    The locking script function lets Bob construct a Bitcoin script only spendable by Alice if she receives the discrete logarithm $\varWit$ of $\varStatement$ with $\varStatement \opEqNoQ \funGen{\varWit}$.
    Additionally, the function requires Bobs public key $\varPubKeyBob$ and a timelock $\varTime$ (given as a block number) as input which allows Bob to reclaim his funds after some time if the Atomic Swap was not completed successfully.
    The function will create and return a Bitcon script $\varScriptPubKey$ to which Bob can send funds using a P2SH transaction.
    To spend this output Alice will have to provide a multi-signature under her public key $\varPubKeyAlice$ and $\varStatement$, which she is able to provide, once acquired $\varWit$.
    An alternative way of constructing a locking mechanism on Bitcoin was shown to be secure by Malavolta et al. in~\cite{malavolta2019anonymous}.
    In their construction the two parties cooperate to construct an initial signature for the spending transaction which is however, not yet valid as it is missing some witness value $\varWit$, only known to one of the two parties.
    Once the second party gets hold of the witness value he or she can complete the signature and finalize the transaction.
    Comparing their solution to the more primitive multi-signature script, it achieves greater privacy (from the outside the lock output just looks like a regular P2PKH output), and needs only a single signature, therefore less space, for the unlocking transaction.
    However, the construction is slightly more complex in the case of ECDSA signatures, which are at present the only signature scheme available on Bitcoin.
    Even though the construction by Malavolta et al. would also be applicable in our case, because of the additional complexity involved and since the focus of this thesis is the Mimblewimble side of the swap we decided to implement the simpler script-based locking mechanism in our proof of concept implementation.
    \item $\{ 1,0 \} \opFunResult \procVerifyLock{\varPubKeyAlice}{\varPubKeyBob}{\varStatement}{\varValue}{\varTime}{\varUTXO_{lock}}$:
    The lock verification algorithm takes as input Alice and Bob public keys and the statement $\varStatement$ and the UTXO $\varUTXO_{lock}$.
    The function will compute the Bitcoin lock script $\varScriptPubKey$ as created by $\procLockAddrId$ check equality with $\varUTXO_{lock}$ and if the value locked under the UTXO equals $\varValue$.
    Upon successful verification the function returns 1, otherwise 0.
    \item $\varTx \opFunResult \procSpendBtc{\varInputs}{\varOutputs}{\varSecKey}$:
    The spend Bitcoin functionality is a wrapper around the $\procBuildTransactionId, \procSignTransactionId$ defined in~\cref{subsec:pre:bitcointx}.
    It constructs and signs a transaction spending the UTXOs given in $\varInputs$ and creates the fresh UTXOs in $\varOutputs$.
    It returns a signed transaction which then can be broadcast.
\end{itemize}

In the following we describe the phases of an Atomic Swap protocol executed between two parties.
In the setup phase~\cref{subsec:atom:setup}, the two parties agree on the parameters of the swap, that is the exchange rates, the amount being swapped and the timeout for the refunding.
In the locking phase~\cref{subsec:atom:locking}, the goal is to lock up the funds on both chains, such that they can either be redeemable by the other party in case the swap was successful, or be refunded to the original owners in case the swap has failed.
The precondition for running the locking phase is that the parties have first completed the setup phase.
In the execution phase~\cref{subsec:atom:exec}, the two parties cooperate to redeem the funds locked by the other parties.
This phase can only be entered after successfully completing the locking phase.
When the funds are redeemed on both sides the swap is considered successfully completed.
In case the execution phase fails, for instance if one party stops cooperating, the swap is considered failed and we enter the refunding phase.~\cref{subsec:atom:refund}
A special security requirement here is that in case of failure the funds are refunded to their original owners on both sides.
If the swap completes on one side, but then can't be completed on the other side, one party would lose all of their value, therefore we must make sure that this case is an impossibility.

\subsubsection{Setup Phase}\label{subsec:atom:setup}

We assume Alice owns Mimblewimble coins $\funArray{\varSpendableCoin}$ with the total value $\varValueMw$ and Bob owns
Bitcoin locked in some UTXO $\varUTXO$ with a value of $\varValueBtc$ and secret spending key $\varSecKey_{btc}$.
Before the protocol can start the two parties must agree on the value they want to swap, the exchange rate of the currencies and a time after which the swap should be canceled.
After coming to an agreement the following variables are defined and known by both Alice and Bob:
\begin{itemize}
    \item $\varSecParam$ A security parameter.
    \item $\varAmountBtc$ The amount of Bitcoin Bob will swap to Alice.
    \item $\varAmountMW$ The amount of the Mimblewimble coin Alice will swap to Bob.
    \item $\varTimeBTC$ The locktime as a block height for the Bitcoin side.
    \item $\varTimeMW$ The locktime as a block height for the Mimblewimble side.
\end{itemize}
We collect this shared variables in an initial swap state $\varSwpState$:
\[ \varSwpState \opAssign \{ \varSecParam, \varAmountBtc, \varAmountMW, \varTimeBTC, \varTimeMW \} \]

In practice, we need to consider that exchange rates might fluctuate, furthermore timeouts have to be calculated separately for each chain.
The problems with cross chain payments are discussed by Tairi et al. in~\cite{tairi2019a2l}, they propose to use a fixed exchange rate for each day and to use a real world timeout like one day and then calculate the specific block numbers by taking the average block time of the blockchain into account.
Alternatively, if the chains allow it, we could use a real world unix timestamp as a timeout, instead of a block height.
In our setup we can also fix the exchange rate at the beginning of the protocol, which stays unchanged during protocol execution.
If the exchange rate fluctuates and one party is negatively impacted he or she could still decide to stop being cooperative which means the coins would be returned to the original owners after the timeout.

There is furthermore the problem of transaction fees, which we do not consider for this formalization.
Depending on the current network load the participants need to agree on a fee that they are willing to pay for each network.
It needs to be considered that if fees are picked to low, it might take time for transactions to be confirmed, and the swap will take longer, if they are picked high the participants will lose value.

\subsection{Locking phase}\label{subsec:atom:locking}

We formalize the protocol $\procSetupSwapId$ in figure~\ref{fig:setup-swap}.
The protocol takes as input the shared swap state $\varSwpState$ from both parties.
From Alice her Mimblewimble input coins $\funArray{\varSpendableCoin}$ with the summed up value $\varValueMw$ is furthermore required as an input.
From Bob we require a list of UTXO's $\funArray{\varUTXO}$ he wants to spend, he also needs to provide their spending keys $\funArray{\varSecKey_{btc}}$ and their summed of total value $\varValueBtc$, although this could also be read from the blockchain.

The protocol starts by both parties creating and exchanging keys.
Bob now creates two new Bitcoin outputs $\varUTXO_{lock}$ and $\varUTXO_{B}$, of which one is the locked Bitcoins which Alice might retrieve later (or Bob after time $\varTimeBTC$ has passed), and the other Bobs change output. (Difference between what is stored in the input UTXO and what should be sent to Alice).
After Bob has published the transaction sending value to the new outputs, he will provide Alice with the statement $\varStatement$ under which the Bitcoins' are locked together with Alice's public key.
Alice can now verify that the funds on Bitcoin side are indeed correctly locked.
After that she will collaborate with Bob to spend her Mimblewimble coins into an output shared by both parties.
Immediately after, both parties collaborate again to spend this shared coin back to Alice with a timelock of $\varTimeMW$.
It is immanent that Alice does not publish the first transaction (A -> AB) before the timelocked refund transaction (AB -> A) was signed, otherwise her funds are locked in the shared output without the possibility of refund if Bob refuses to cooperate.
The locking protocol concludes with the funds locked up in both chains and ready to be swapped and outputs the updated swap state $\varSwpState$ to both parties.
Additionally, it outputs Alice's part $\funStar{\varPtSpendableCoinAlice}$ of the locked Mimblewimble coin, her change output on the Mimblewimble side $\funStar{\varSpendableCoinAlice}$, her secret key $\varSecKeyAlice$ for the Bitcoin side and $\funStarAlt{\varSpendableCoinAlice}$, which is a refund coin, only valid after $\varTimeMW$.
For Bob it furthermore outputs his part $\funStar{\varPtSpendableCoinBob}$ of the locked Mimblewimble coin, his change output on the Bitcoin side $\varUTXO_{B}$ and the secret witness value $\varWit$, which shall be revealed to Alice in the execution phase.

\newgeometry{margin=2cm}
\begin{landscape}
    \thispagestyle{plain}
    \begin{figure}
        \fbox{
        \procedure[linenumbering,skipfirstln]{$\procSetupSwap{\varSwpState}{\funArray{\varSpendableCoin}}{\varValueMw}{\funArray{\varUTXO}}{\funArray{\varSecKey_{btc}}}{\varValueBtc}$} {
        Alice \< \< \< \< Bob \\
        \{ \varAmountBtc, \varAmountMW, \varTimeBTC, \varTimeMW \} \opFunResult \varSwpState \< \< \< \< \{ \varAmountBtc, \varAmountMW, \varTimeBTC, \varTimeMW \} \opFunResult \varSwpState \\
        \varKeyPairAlice \opFunResult \procSetup{\varSecParam} \< \< \< \< \varKeyPairBob \opFunResult \procSetup{\varSecParam} \\
        \< \< \< \< (\varWit, \varStatement) \opFunResult \procSetup{\varSecParam} \\
        \< \sendmessagerightx{4}{\varPubKeyAlice} \< \\
        \< \sendmessageleftx{4}{\varPubKeyBob} \< \\
        \< \< \< \< \varScriptPubKey \opFunResult \procLockAddr{\varPubKeyAlice}{\varStatement}{\varPubKeyBob}{\varTimeBTC} \\
        \< \< \< \< \varUTXO_{lock} \opFunResult \procCreateUTXO{\varAmountBtc}{\varScriptPubKey} \\
        \< \< \< \< \varUTXO_{B} \opFunResult \procCreateUTXO{\varValueBtc - \varAmountBtc}{\varPubKeyBob} \\
        \< \< \< \< \varBtcTx \opFunResult \procSpendBtc{\funArray{\varUTXO}}{\funArray{\varUTXO_{lock}, \varUTXO_{B}}}{\funArray{\varSecKey_{btc}}} \\
        \< \< \< \< \procPublishBtc{\funArray{\varBtcTx}} \\
        \< \< \< \< \varSwpState \opAssign \varSwpState \opUnion (\varStatement, \varUTXO_{lock}) \\
        \< \sendmessageleftx{4}{\varStatement,\varUTXO_{lock}} \< \\
        \pcif \procVerifyLock{\varPubKeyAlice}{\varPubKeyBob}{\varStatement}{\varAmountBtc}{\varTimeBTC}{\varUTXO_{lock}} \opEqNoQ 0 \\
        \t \pcreturn \cnstFalsum \< \< \< \< \\
        \varSwpState \opAssign \varSwpState \opUnion (\varStatement, \varUTXO_{lock}) \\
        (\varMwFundTx, \funStar{\varSpendableCoinAlice},\funStar{\varPtSpendableCoinAlice}) \< \< \< \< (\funStar{\varPtSpendableCoinBob}) \pcskipln \\
        \opFunResult \procDSharedOutputMwTxL{\funArray{\varSpendableCoin}}{\varAmountMW}{\cnstFalsum} \< \< \< \< \opFunResult \procDSharedOutputMwTxR{\varAmountMW} \\
        (\varMwRefundTx, \funStarAlt{\varSpendableCoinAlice}) \< \< \< \< \varMwRefundTx \pcskipln \\
        \opFunResult \procDSharedInpMwTxL{\funStar{\varPtSpendableCoinAlice}}{\varAmountMW}{\varTimeMW} \< \< \< \< \opFunResult \procDSharedInpMwTxR{\funStar{\varPtSpendableCoinBob}}{\varAmountMW} \\
        \procPublishMW{\funArray{\varMwFundTx,\varMwRefundTx}}  \\
        \pcreturn (\varSwpState, \funStar{\varPtSpendableCoinAlice}, \funStar{\varSpendableCoinAlice}, \varSecKeyAlice, \funStarAlt{\varSpendableCoinAlice}) \< \< \< \< \pcreturn (\varSwpState, \funStar{\varPtSpendableCoinBob}, \varUTXO_{B}, \varWit)
        }
        }
        \caption{Atomic Swap - $\procSetupSwapId$.}\label{fig:setup-swap}
    \end{figure}
\end{landscape}
\restoregeometry

\subsection{Execution Phase}\label{subsec:atom:exec}

First we need to define an additional auxiliary function $\procVerifyTimeId$ with the following interface:
\[ \{0,1\} \opFunResult \procVerifyTime{\varChain}{\varTime} \]
This function will verify that there is sufficient time left to execute the Atomic Swap protocol.
As input it takes a chain parameter $\varChain$ (in our case this could be either BTC or MW) and a block height $\varTime$.
The routine will verify that the current height of the blockchain is marginally below $\varTime$.
If this is the case it will return 1, or 0 otherwise.
How much time exactly should be left for the function to return 1 is implementation specific, and could be set to for instance one day.
We now define a protocol $\procExcSwapId$ to execute the Atomic Swap between some amount $\varAmountBtc$ on the Bitcoin side and some amount on the Mimblewimble side $\varAmountMW$.
An instantiation of the protocol can be found in~\cref{fig:exec-swap}.
We assume the participants have successfully run the $\procSetupSwapId$ protocol and both know the updated swap state $\varSwpState$ as returned by the setup protocol.
Both parties need to provide their part of the locked Mimblewimble coins as input to the protocol.
Additionally, Alice needs to provide her secret key for the Bitcoin side $\varSecKeyAlice$ and Bob the secret witness value $\varWit$.
The protocol starts with both parties checking that there is enough time left to complete the protocol.
After the check they will run the $\procDScriptMwTxId$ protocol in which they spend the locked Mimblewimble output to Bob, while at the same time revealing $\varWit$ to Alice.
Either one of the parties can now publish the transaction to the Mimblewimble network, which concludes the swap on the Mimblewimble side, as Bob is now in full control of the funds.
Alice, knowing $\varWit$, creates now a new UTXO where she then sends the funds from the Bitcoin lock.
After publishing this transaction to the Bitcoin network, Alice is in full possession of the swapped funds on the Bitcoin side and the Atomic Swap is completed.
The protocol outputs their newly created output/coin to each party.
We note here that after completion of the swap on the Mimblewimble side, Alice is possible to redeem her Bitcoin, however she still has to construct the transaction and get it mined on the network.
Otherwise, if she would take too long and the timeout block height is reached, Bob could still try to refund his coins, even though he already received the funds on the Mimblewimble side.
Therefore it is important to pick long enough timeouts, and also check how much time is left again before running the execution protocol.

\newgeometry{margin=2cm}
\begin{landscape}
    \thispagestyle{plain}
    \begin{figure}
        \fbox{
        \procedure[linenumbering,skipfirstln]{$\procExcSwap{\varSwpState}{\funStar{\varPtSpendableCoinAlice}}{\varSecKeyAlice}{\funStar{\varPtSpendableCoinBob}}{\varWit}$} {
        Alice \< \< \< \< Bob \\
        (\varAmountMW, \varAmountBtc, \varTimeMW, \varTimeBTC, \varUTXO_{lock}, \varStatement) \opFunResult \varSwpState \< \< \< \< (\varAmountMW, \varAmountBtc, \varTimeMW, \varTimeBTC) \opFunResult \varSwpState \\
        \pcif \procVerifyTime{BTC}{\varTimeBTC} \opEqNoQ 0 \opOr \procVerifyTime{MW}{\varTimeMW} \opEqNoQ 0 \< \< \< \< \pcif \procVerifyTime{BTC}{\varTimeBTC} \opEqNoQ 0 \opOr \procVerifyTime{MW}{\varTimeMW} \opEqNoQ 0 \\
        \t \pcreturn \cnstFalsum \< \< \< \< \t \pcreturn \cnstFalsum \\
        (\varMwTx, \cnstEmptySet, \varWit) \< \< \< \< (\varMwTx, \funStar{\varSpendableCoinBob}) \pcskipln \\
        \opFunResult \procDScriptMwTxL{\funStar{\varPtSpendableCoinAlice}}{\varAmountMW}{\cnstFalsum}{\varStatement} \< \< \< \< \opFunResult \procDScriptMwTxR{\funStar{\varPtSpendableCoinBob}}{\varAmountMW}{\varWit} \\
        \< \< \< \< \procPublishMW{\varMwTx} \\
        (\funStarAlt{\varSecKeyAlice}, \funStarAlt{\varPubKeyAlice}) \opFunResult \procSetup{\varSecParam} \< \< \< \< \\
        \varUTXO_{A} \opFunResult \procCreateUTXO{\varAmountBtc}{\funStarAlt{\varPubKeyAlice}} \< \< \< \< \\
        \varBtcTx \opFunResult \procSpendBtc{\funArray{\varUTXO_{lock}}}{\funArray{\varUTXO_{A}}}{\funArray{\varSecKeyAlice, \varWit}} \< \< \< \< \\
        \procPublishBtc{\funStar{\varBtcTx}} \< \< \< \< \\
        \pcreturn (\varUTXO_{A}) \< \< \< \< \pcreturn (\funStar{\varSpendableCoinBob})
        }
        }
        \caption{Atomic Swap - $\procExcSwapId$. \label{fig:exec-swap}}
    \end{figure}
\end{landscape}
\restoregeometry

\subsection{Refunding phase}\label{subsec:atom:refund}

If one party refused to cooperate or goes offline the coins can be returned to the original owner.
On the Bitcoin side this is the case as Bob can simply spend the locked output with his private key $\varSecKeyBob$ after the timeout $\varTimeBTC$ has passed.
He then can simply construct and sign a transaction spending the output to a new UTXO which is in his full possession.
He even could prepare this transaction upfront and broadcast it, once the the block number hits $\varTimeBTC$ the transaction will become valid and get mined.
Again we stress the importance of using appropriate timeouts, if a timeout is too short the swap might get cancelled if there are some delays, if the timeout is too long the funds might be locked for an unnecessary amount of time.

On the Mimblewimble side the second transaction spending the shared output back to Alice guarantees that her funds are returned to her after the timeout $\varTimeMW$ hits.
For this reason it is so important that Alice publishes both the fund and refund transaction at the same time.
If she would publish the funding transaction first, Bob could refuse to cooperate for the refund transaction, in which case the funds would stay in the locking output only retrievable if both parties cooperate.
If the swap executes successful the refund transaction would get discarded by miners, as it then is no longer valid even after the timeout $\varTimeMW$.