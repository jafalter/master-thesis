In this section, we will formalize Mimblewimble transactions and define security properties, first found by Fuchsbauer et al. in~\cite{fuchsbauer2019aggregate} which have to hold for a valid transaction protocol.
We then provide four different types of the transaction protocol, we separate between output coins which are controlled by a single party (denominated as $\cnstSoloCoin$) and a coin which is controlled by multiple parties owning a share of the blinding key, denominated as $\cnstMultiCoin$. These multiparty outputs can only spend by the two owners collaborating and are therefore semantically similiar to multisignature outputs in Bitcoin.~\cite{bistarelli2018analysis}
From this we can then derive the different types of transactions $\cnstSoToMuTx$, $\cnstSoToMuTx$, $\cnstMuToSoTx$ and $\cnstMuToMuTx$. Note that in our case we restrict ourselfs to $\cnstMultiCoin$ outputs which are controlled by two parties, as those are the ones needed to construct the Atomic Swap protocol.
Proving the security for coins controlled by n participants would be an interesting topic for future research.
We then prove that all security definitions for Mimblewimble transactions, hold for all four types of transactions.
Finally, we define an Atomic Swap protocol from these building blocks, which allows us to securely swap funds from a Mimblewimble blockchain with those on another Blockchain, such as Bitcoin.

\section{Definitions}\label{sec:atomic-def}

\begin{definition}[Mimblewimble Transaction Scheme]
    \label{def:tx}
    As we have already discussed in section~\ref{sec:Mimblewimble} for the creation of a transaction, it is immanent that both the sender and receiver collaborate and exchange messages via a secure channel.
    To construct the transaction protocol we assume that we have access to a two-party signature scheme as defined in definition~\ref{def:twoPartySig}, a zero-knowledge Rangeproofs system such as Bulletproofs, as
    described in section~\ref{sec:rangeProof} and a homomorphic commitment scheme as defined in definition~\ref{def:homomorphicCom} such as Pedersen Commitments~\ref{def:pedersenCom}.
    A transaction scheme consist of the following procedure:
    \[ \cnstTransaction \opAssign ( \procSpendOutputId ) \]
    \begin{itemize}
        \item $\procSpendOutput{\varFundValue}{\varCoinInp}{\varBlindingFactorAlice}{\varBlindingFactorBob}$ Takes a fund value $\varFundValue$ which should be transferred to a new output, an input coin $\varCoinInp$ and the coins blinding factor $\varBlindingFactor$. If the input coin is $\cnstMultiCoin$ we require the blinding factor of both owners. The protocol is constructed between a sender and a receiver. For simplicity we assume that if we spend a $\cnstMultiCoin$ that one of the owners will also be the transaction receiver, as this will be the case in the Atomic Swaps. However the procedure could easily be generalized to the case where a shared coin is sent to a third party.
    \end{itemize}
\end{definition}

TODO security definitons

\section{Grin instantiation}\label{sec:atomic-inst}

In this section we describe instantiations for the four transaction types defined in~\ref{def:tx} specifically tailored for the Grin cryptocurrency. In section~\ref{sec:atomic-swap} we use this transaction types as building block for the Mimblewimble side of our Atomic Swap protocol.

\subsection{$\cnstSoToSoTx$}

We assume to have access to a homomorphic commitment scheme such as Pedersen Commitment as defined in definition~\ref{def:pedersenCom}. Furthermore we require a Rangeproof system as defined in
~\ref{sec:rangeProof} and a two-party signature scheme as defined in~\ref{def:twoPartySig}. The instantiation of the transaction protocol can be found in figure~\ref{fig:solo2solotx}.
\begin{figure}
    \begin{center}
        \fbox{
        \begin{varwidth}{\textwidth}
            \procedure[linenumbering]{$\procSpendOutput{\varFundValue}{\varCoinInp}{\varBlindingFactor}{\cdot}$} {
            \varMsg \opAssign \cnstBinary{*} \< \< \\
            \funStar{\varBlindingFactor} \sample \cnstIntegersPrimeWithoutZero{\varPrime} \< \< \\
            \varCoinOutAlice \opFunResult \procCommit{\varValue \opSub \varFundValue}{\funStar{\varBlindingFactor}} \< \< \\
            \varProofAlice \opFunResult \procProof{\varCoinOutAlice}{\varValue \opSub \varFundValue}{\funStar{\varBlindingFactor}} \< \< \\
            \varBlindingFactorFor{s} \opAssign \funStar{\varBlindingFactor} \opSub \varBlindingFactor \< \< \\
            \< \sendmessageright*{\varTx \opAssign (\varMsg \opSeperate \varCoinInp \opSeperate \varCoinOutAlice \opSeperate \varProofAlice \opSeperate \funGen{\varBlindingFactorFor{s}})} \< \\
            \< \< \procVerfProof{\varProofAlice} \opEq 1 \\
            \< \< \varBlindingFactorFor{r} \sample \cnstIntegersPrimeWithoutZero{\varPrime} \\
            \< \< \varCoinOutBob \opFunResult \procCommit{\varFundValue}{\varBlindingFactorFor{r}} \\
            \< \< \varSigBob \opFunResult \procSignPtSingle{\varMsg}{\varBlindingFactorFor{r}} \\
            \< \< \varProofBob \opFunResult \procProof{\varCoinOutBob}{\varFundValue}{\varBlindingFactorFor{r}} \\
            \< \sendmessageleft*{\varTx \opAssign \varTx \opUnion (\varCoinOutBob \opSeperate \funGen{\varBlindingFactorFor{r}} \opSeperate \varSigBob \opSeperate \varProofBob)} \< \\
            \procVerfPtSig{\varSigBob}{\varMsg}{\funGen{\varBlindingFactorFor{s}}}{\funGen{\varBlindingFactorFor{r}}} \opEq 1 \< \< \\
            \procVerfProof{\varProofBob} \opEq 1 \< \< \\
            \varSigAlice \opFunResult \procSignPtSingle{\varMsg}{\varBlindingFactorFor{s}} \< \< \\
            \varSigFin \opFunResult \procFinSig{\varSigAlice}{\varSigBob} \< \< \\
            \varExcess \opAssign \funGen{\varBlindingFactorFor{s}} \opAddPoint \funGen{\varBlindingFactorFor{r}} \< \< \\
            \pcreturn \varTx \opAssign \varTx \opUnion (\varExcess \opSeperate \varSigFin)
            }
        \end{varwidth}
        }
    \end{center}
    \caption{Instantiation of $\cnstSoToSoTx$ transaction protocol. \label{fig:solo2solotx}}
\end{figure}

\subsection{$\cnstSoToMuTx$}

Again we assume that we have access to a homomorphic commitment scheme, a rangeproof protocol and two-party signature scheme. The rangeproof protocol needs to support creation
of multiparty rangeproofs for this protocol to be executeable. The concrete instantiation can be found in figure~\ref{fig:Mu2SoTx}.
\begin{figure}
    \begin{center}
        \fbox{
        \begin{varwidth}{\textwidth}
            \procedure[linenumbering]{$\procSpendOutput{\varFundValue}{\varCoinInp}{\varBlindingFactorAlice}{\varBlindingFactorBob}$} {
            \varMsg \opAssign \cnstBinary{*} \< \< \\
            \funStar{\varBlindingFactorAlice} \opSeperate \varBlindingFactorFor{1} \sample \cnstIntegersPrimeWithoutZero{\varPrime} \< \< \\
            \varCoinOutAlice \opFunResult \procCommit{\varValue \opSub \varFundValue}{\funStar{\varBlindingFactorAlice}} \< \< \\
            \scriptstyle \varProofAlice \opFunResult \procProof{\varCoinOutAlice}{\varValue \opSub \varFundValue}{\funStar{\varBlindingFactorAlice}} \< \< \\
            \varBlindingFactorFor{s} \opAssign \funStar{\varBlindingFactorAlice} \opAddScalar \varBlindingFactorFor{1}  \opSub \varBlindingFactorAlice \< \< \\
            \< \sendmessageright*{ \scriptstyle \varTx \opAssign (\varMsg \opSeperate \varCoinInp \opSeperate \varCoinOutAlice \opSeperate \varProofAlice \opSeperate \funGen{\varBlindingFactorFor{s}} \opSeperate \funGen{\varBlindingFactorFor{1}})} \< \\
            \< \< \procVerfProof{\varProofAlice} \opEq 1 \\
            \< \< \varBlindingFactorFor{2} \sample \cnstIntegersPrimeWithoutZero{\varPrime} \\
            \< \< \varCoinShared \opFunResult \procCommit{\varFundValue}{\varBlindingFactorFor{2}} \opAddPoint \funGen{\varBlindingFactorFor{1}} \\
            \< \< \varSigBob \opFunResult \procSignPtSingle{\varMsg}{\varBlindingFactorFor{r}} \\
            \< \sendmessageleft*{ \scriptstyle \varTx \opAssign \varTx \opUnion (\varCoinShared \opSeperate \funGen{\varBlindingFactorFor{2}} \opSeperate \varSigBob)} \< \\
            \< \scriptscriptstyle \varProof \opFunResult \procMuProof{\varCoinShared}{\varFundValue}{\varBlindingFactorFor{1}}{\varBlindingFactorFor{2}} \< \\
            \varSigAlice \opFunResult \procSignPtSingle{\varMsg}{\varBlindingFactorFor{s}} \< \< \\
            \varSigFin \opFunResult \procFinSig{\varSigAlice}{\varSigBob} \< \< \\
            \varExcess \opAssign \funGen{\varBlindingFactorFor{s}} \opAddPoint \funGen{\varBlindingFactorFor{r}} \< \< \\
            \scriptstyle \pcreturn \varTx \opAssign \varTx \opUnion (\varExcess \opSeperate \varSigFin \opSeperate \varProof)
            }
        \end{varwidth}
        }
    \end{center}
    \caption{Instantiation of $\cnstSoToMuTx$ transaction protocol. \label{fig:So2MuTx}}
\end{figure}

\subsection{$\cnstMuToSoTx$}

TODO

\subsection{$\cnstMuToMuTx$}

TODO

\section{Atomic Swap protocol}\label{sec:atomic-swap}