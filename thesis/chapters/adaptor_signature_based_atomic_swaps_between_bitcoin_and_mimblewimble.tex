In this section, we will formalize Mimblewimble transactions and define security properties, first found by Fuchsbauer et al. in~\cite{fuchsbauer2019aggregate} which have to hold for a valid transaction protocol.
We later provide two instantionons of the transaction protocol, $\cnstPayToCoin$ which is the general case transaction in which a sender wants to transfer some value to a receiver
and $\cnstPayToSharedCoin$, which is unique in that it creates coins which are only spendable by two (or more) parties collaborating. $\cnstPayToSharedCoin$ is similar
to a Mulit-signature transaction script in Bitcoin in which an UTXO, only spendable by a consortium of owners, is created.~\cite{bistarelli2018analysis}
We then prove that all security definitions for Mimblewimble transactions, hold for the outlined transaction protocols.
Finally we define a Atomic Swap protocol from these building blocks, which allows us to securely swap funds from a Mimblewimble blockchain with those on another Blockchain, such as Bitcoin.

\section{Definitions}\label{sec:atomic-def}

\begin{definition}[Mimblewimble Transaction Scheme]
    \label{def:tx}
    As we have already discussed in section~\ref{sec:Mimblewimble} for the creation of a transaction, it is immanent that both the sender and receiver collaborate and exchange messages via a secure channel.
    To construct the transaction protocol we assume that we have access to a two-party signature scheme as defined in definition~\ref{def:twoPartySig}, a zero-knowledge Rangeproofs system such as Bulletproofs, as
    described in section~\ref{sec:rangeProof}, a homomoprhic commitment scheme as defined in definition~\ref{def:homomorphicCom} such as Pedersen Commitments~\ref{def:pedersenCom}.
    A transaction scheme consist of the following two procedures:
    \[ \cnstTransaction \opAssign ( \procCreateOutputId \opSeperate \procSpendOutputId ) \]
    \begin{itemize}
        \item $\procCreateOutputId$ defines the creation of a new spendable output. It is defined as a two party protocol between a sender and a receiver. As an input the
        sender has to provide the funds which should be transfered to the receiver as well as the input coins he or she would like to spend, together with its blinding factor. The procedure
        outputs a transaction which can be broadcast to the network and transfers the agreed on value from the sender to the receiver.
        \item $\procSpendOutputId$ defines how we can spend the output created by $\procCreateOutputId$. In the simplest case this can be done by just calling $\procCreateOutputId$ again.
        However for more advanced transaction protocols such as $\cnstPayToSharedCoin$ the way to spend the create coin will be slightly different.
    \end{itemize}
\end{definition}

TODO security definitons

\section{Grin instantiation}\label{sec:atomic-inst}

In this section we describe instantiations for the transaction protocol defined in~\ref{def:tx} specifically tailored for the Grin cryptocurrency. We show that all of them are secure by proving the
outlined security properties for mimblewimble transactions. In section~\ref{sec:atomic-swap} we use this transaction types as building block for the Grin side of our Atomic Swap protocol.

\subsection{$\cnstPayToCoin$}

We assume to have access to a homomorphic commitment scheme such as Pedersen Commitment as defined in definition~\ref{def:pedersenCom}. Furthermore we require a Rangeproof system as defined in
~\ref{sec:rangeProof} and a two-party signature scheme as defined in~\ref{def:twoPartySig}. The instantiation of the transaction protocol can be found in figure~\ref{fig:payToCoin}.
\begin{figure}
    \begin{center}
        \fbox{
        \begin{varwidth}{\textwidth}
            \procedure[linenumbering]{$\procCreateOutput{\varFundValue}{\varCoinInp}{\varBlindingFactorFor{i}}$} {
            \varMsg \opAssign \cnstBinary{*} \< \< \\
            \varBlindingFactorFor{o} \sample \cnstIntegersPrimeWithoutZero{\varPrime} \< \< \\
            \varCoinOutAlice \opFunResult \procCommit{\varValue \opSub \varFundValue}{\varBlindingFactorFor{o}} \< \< \\
            \varProofAlice \opFunResult \procProof{\varCoinOutAlice}{\varValue \opSub \varFundValue}{\varBlindingFactorFor{o}} \< \< \\
            \varBlindingFactorFor{a} \opAssign \varBlindingFactorFor{\varBlindingFactorFor{o} \opSub \varBlindingFactorFor{i}} \< \< \\
            \< \sendmessageright*{\varTx \opAssign (\varMsg \opSeperate \varCoinInp \opSeperate \varCoinOutAlice \opSeperate \varProofAlice \opSeperate \funGen{\varBlindingFactorFor{a}})} \< \\
            \< \< \procVerfProof{\varProofAlice} \opEq 1 \\
            \< \< \varBlindingFactorFor{b} \sample \cnstIntegersPrimeWithoutZero{\varPrime} \\
            \< \< \varCoinOutBob \opFunResult \procCommit{\varFundValue}{\varBlindingFactorFor{b}} \\
            \< \< \varSigBob \opFunResult \procSignPtSingle{\varMsg}{\varBlindingFactorFor{b}} \\
            \< \< \varProofBob \opFunResult \procProof{\varCoinOutBob}{\varFundValue}{\varBlindingFactorFor{b}} \\
            \< \sendmessageleft*{\varTx \opAssign \varTx \opConc (\varCoinOutBob \opSeperate \funGen{\varBlindingFactorFor{b}} \opSeperate \varSigBob \opSeperate \varProofBob)} \< \\
            \procVerfPtSig{\varSigBob}{\varMsg}{\funGen{\varBlindingFactorFor{a}}}{\funGen{\varBlindingFactorFor{b}}} \opEq 1 \< \< \\
            \procVerfProof{\varProofBob} \opEq 1 \< \< \\
            \varSigAlice \opFunResult \procSignPtSingle{\varMsg}{\varBlindingFactorFor{a}} \< \< \\
            \varSigFin \opFunResult \procFinSig{\varSigAlice}{\varSigBob} \< \< \\
            \varExcess \opAssign \funGen{\varBlindingFactorFor{a}} \opAddPoint \funGen{\varBlindingFactorFor{b}} \< \< \\
            \pcreturn \varTx \opAssign \varTx \opConc (\varExcess \opSeperate \varSigFin)
            }
            \procedure[linenumbering]{$\procSpendOutput{\varFundValue}{\varCoinOut}{\varBlindingFactorFor{o}}$}{
            \pcreturn \procCreateOutput{\varFundValue}{\varCoinOut}{\varBlindingFactorFor{o}}
            }
        \end{varwidth}
        }
    \end{center}
    \caption{Instantiation of $\cnstPayToCoin$ transaction protocol. \label{fig:payToCoin}}
\end{figure}

\subsection{$\cnstPayToSharedCoin$}

Again we assume that we have access to a homomorphic commitment scheme, a rangeproof protocol and two-party signature scheme. The rangeproof protocol needs to support creation
of multiparty rangeproofs for this protocol to be executeable. The concrete instantiation can be found in figure~\ref{fig:payToSharedCoin}.
\begin{figure}
    \begin{center}
        \fbox{
        \begin{varwidth}{\textwidth}
            \procedure[linenumbering]{$\procCreateOutput{\varFundValue}{\varCoinInp}{\varBlindingFactorFor{i}}$} {
            \varMsg \opAssign \cnstBinary{*} \< \< \\
            \varBlindingFactorFor{o} \opSeperate \varBlindingFactorFor{sa} \sample \cnstIntegersPrimeWithoutZero{\varPrime} \< \< \\
            \varCoinOutAlice \opFunResult \procCommit{\varValue \opSub \varFundValue}{\varBlindingFactorFor{o}} \< \< \\
            \scriptstyle \varProofAlice \opFunResult \procProof{\varCoinOutAlice}{\varValue \opSub \varFundValue}{\varBlindingFactorFor{o1}} \< \< \\
            \varBlindingFactorFor{a} \opAssign \varBlindingFactorFor{\varBlindingFactorFor{o}} \opAddScalar \varBlindingFactorFor{\varBlindingFactorFor{sa}  \opSub \varBlindingFactorFor{i}} \< \< \\
            \< \sendmessageright*{ \scriptstyle \varTx \opAssign (\varMsg \opSeperate \varCoinInp \opSeperate \varCoinOutAlice \opSeperate \varProofAlice \opSeperate \funGen{\varBlindingFactorFor{a}} \opSeperate \funGen{\varBlindingFactorFor{sa}})} \< \\
            \< \< \procVerfProof{\varProofAlice} \opEq 1 \\
            \< \< \varBlindingFactorFor{sb} \sample \cnstIntegersPrimeWithoutZero{\varPrime} \\
            \< \< \varCoinShared \opFunResult \procCommit{\varFundValue}{\varBlindingFactorFor{sb}} \opAddPoint \funGen{\varBlindingFactorFor{sa}} \\
            \< \< \varSigBob \opFunResult \procSignPtSingle{\varMsg}{\varBlindingFactorFor{sb}} \\
            \< \sendmessageleft*{ \scriptstyle \varTx \opAssign \varTx \opConc (\varCoinShared \opSeperate \funGen{\varBlindingFactorFor{sb}} \opSeperate \varSigBob)} \< \\
            \< \scriptscriptstyle \varProof \opFunResult \procMuProof{\varCoinShared}{\varFundValue}{\varBlindingFactorFor{sa}}{\varBlindingFactorFor{sb}} \< \\
            \varSigAlice \opFunResult \procSignPtSingle{\varMsg}{\varBlindingFactorFor{a}} \< \< \\
            \varSigFin \opFunResult \procFinSig{\varSigAlice}{\varSigBob} \< \< \\
            \varExcess \opAssign \funGen{\varBlindingFactorFor{a}} \opAddPoint \funGen{\varBlindingFactorFor{b}} \< \< \\
            \scriptstyle \pcreturn \varTx \opAssign \varTx \opConc (\varExcess \opSeperate \varSigFin \opSeperate \varProof)
            }
            \procedure[linenumbering]{$\procSpendOutput{\varFundValue}{\varCoinOut}{\varBlindingFactorFor{o}}$}{
            TODO
            }
        \end{varwidth}
        }
    \end{center}
    \caption{Instantiation of $\cnstPayToSharedCoin$ transaction protocol. \label{fig:payToSharedCoin}}
\end{figure}

\section{Atomic Swap protocol}\label{sec:atomic-swap}