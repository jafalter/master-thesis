\urldef{\urlmonero}\url{https://www.getmonero.org/}

\paragraph{Related Work.} Maurice Herlihy~\cite{herlihy2018atomic} describes how hashed timelock contracts can be leveraged to build Atomic Swaps between cryptocurrencies that have at least some basic scripting capabilities.
In a hashed timelock contract, the hash pre-image $\varX$ of $\styleVariable{h} \opEqNoQ \funHash{\varX}$, where $\funHash{\cdot}$ is a hash function, serves as the secret contract value that makes the swap possible.
Fuchsbauer et al. conduct a complete security analysis of the Mimblewimble protocol, which forms the basis for~\cref{ch:fixedwitnesssignatures}.
Betarte et al.~\cite{betarte2019towards} work towards making implementations of the Mimblewimble protocol formally verifiable.
In 2016 Andrew Poelstra published his specification of the Mimblewimble protocol in~\cite{poelstra2016mimblewimble} and has managed to expand on the ideas posted in the original writing~\cite{jedusor2016mimblewimble} by Jedusor.
In the following year, Poelstra further introduced the notion of Scriptless Scripts~\cite{poelstra2017scriptless}, showing how one can build primitive contracts just by using cryptography alone.
Poelstra's ideas were formalized as Adaptor signatures as a standalone cryptographic primitive by Aumayr et al. in~\cite{aumayr2020bitcoinchannels}.
Poelstra's insights on Scriptless Scripts and the formalization by Aumayr et al. together form the basis for our construction of the Two-Party Fixed Witness Adaptor Signature Scheme in~\cref{ch:fixedwitnesssignatures}.\\
Joël Gugger constructs a cross-chain Atomic Swap between Monero and Bitcoin in~\cite{gugger2020bitcoin}.
Just like Mimblewimble-based crypto coins, Monero\footnote{\urlmonero} is a privacy-enhancing cryptocurrency that lacks scripting capabilities and even lacks timelocking capabilities (which do exist in Mimblewimble).
An extensive introduction into the Monero protocol can be found in the Zero to Monero Paper~\cite{alonso2020zero}.
In Gugger's protocol, the Monero funds are locked in an address shared between the two trading parties.
On the Bitcoin side, funds are locked so that spending them will reveal the missing key for the Monero side, therefore unlocking the Monero funds for the second party.
The author was forced to make some special considerations since the two cryptocurrencies operate on two different elliptic curves.
Apoorvaa Deshpande and Maurice Herlihy define the privacy properties of an Atomic Swap protocol and demonstrate the privacy improvements brought upon by using Adaptor Signatures.~\cite{deshpande2020privacy}

\paragraph{Future Research.} Erkan Tairi et al. have demonstrated in~\cite{tairi2019a2l} how to construct Payment channel hubs as a solution to scalability issues of blockchain technologies.
With their approach relying solely on Scriptless Scripts, it would be interesting to explore the feasibility of such a construction on a Mimblewimble-based cryptocurrency by utilizing the contract transactions outlined in our research.
A. Faz-Hernandez et al. examine the possibility of hash functions that output an elliptic curve point for multiple elliptic curves, including secp256k1~\cite{hernandez2020hashing}.
Based on this work, one could research if more elaborate contracts such as hashed timelock contracts are made possible on Mimblewimble with such a hash function.
Such a solution could help solve the difficulties that arise when the two cryptocurrencies between which the exchange happens operate on different curves as seen in the work, by Joël Gugger~\cite{gugger2020bitcoin}.
Finally, another interesting research topic would be the construction of a cross-chain Atomic Swap between two cryptocurrencies both lacking scripting functionality.
For instance by combining our protocol with the construction by Gugger~\cite{gugger2020bitcoin} to build Atomic Swaps between Monero and a Mimblewimble-based cryptocurrency.