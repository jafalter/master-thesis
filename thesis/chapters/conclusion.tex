\urldef{\urluniswp}\url{https://uniswap.org/}
\urldef{\urlmonero}\url{https://www.getmonero.org/}

This thesis aimed to improve interoperability between privacy-enhancing cryptocurrencies and regular cryptocurrencies, allowing for the listing of more privacy-enhancing currencies on decentralized exchanges such as Uniswap.\footnote{\urluniswp}
We have achieved this goal by constructing an Atomic Swap protocol between Bitcoin and a Mimblewimble-based cryptocurrency.
Similar to Joël Gugger, who constructed an Atomic Swap protocol between Bitcoin and the Monero cryptocurrency in~\cite{gugger2020bitcoin}, we leveraged a Scriptless Scripts approach that was derived from Andrew Poelstra's inital work~\cite{poelstra2017scriptless} and the Adaptor Signature Scheme~\cite{aumayr2020bitcoinchannels} by Auymayr et al.
We further managed to prove Correctness and Security in the malicious setting for four different types of Mimblewimble transactions that allow for the distribution of Mimblewimble coins between multiple keys and the execution of simple contracts, based on our own secure and correct two-party version of the Adaptor Signature Scheme.
We pointed out a weakness in the security analysis of the Mimblewimble protocol done by Fuchsbauer et al. in~\cite{fuchsbauer2019aggregate}, which we have fixed by introducing the notion of Two-Party Signature Schemes into their model.
\todo[inline]{Pedro: I would like to go with you again through this. Please remind me in the next call}

While two-party signatures are prone to Rogue-key attacks, we demonstrate the infeasibility of such an attack in the Mimblewimble protocol.

\paragraph{Related work.} Maurice Herlihy~\cite{herlihy2018atomic} describes how hashed timelock contracts can be leveraged to build Atomic Swaps between cryptocurrencies that have at least some basic scripting capabilities.
In a hashed timelock contract, the hash pre-image $\varX$ of $\styleVariable{h} \opEqNoQ \funHash{\varX}$, where $\funHash{\cdot}$ is a hash function, serves as the secret contract value that makes the swap possible.
Fuchsbauer et al. conduct a complete security analysis of the Mimblewimble protocol, which will form the basis for~\cref{ch:fixedwitnesssignatures}.
Betarte et al.~\cite{betarte2019towards} work towards making implementations of the Mimblewimble protocol formally verifiable.
In 2016 Andrew Poelstra published his specification of the Mimblewimble protocol in~\cite{poelstra2016mimblewimble} and has managed to expand on the ideas posted in the original writing~\cite{jedusor2016mimblewimble} by Jedusor.
In the following year, Poelstra further introduced the notion of Scriptless Scripts~\cite{poelstra2017scriptless}, showing how one can build primitive contracts just by using cryptography alone.
Poelstra's ideas were formalized as Adaptor signatures as a standalone cryptographic primitive by Aumayr et al. in~\cite{aumayr2020bitcoinchannels}.
Poelstra's insights on Scriptless Scripts and the formalization by Aumayr et al. together form the basis for our construction of the Two-Party Fixed Witness Adaptor Signature Scheme in~\cref{ch:fixedwitnesssignatures}.
Joël Gugger constructs a cross-chain Atomic Swap between Monero and Bitcoin in~\cite{gugger2020bitcoin}.
Just like Mimblewimble-based crypto coins, Monero\footnote{\urlmonero} is a privacy-enhancing cryptocurrency that lacks scripting capabilities and even lacks timelocking capabilities (which do exist in Mimblewimble).
In Gugger's protocol, the Monero funds are locked in an address shared between the two trading parties.
On the Bitcoin side, funds are locked so that spending them will reveal the missing key for the Monero side, therefore unlocking the Monero funds for the second party.
The author made some special considerations since the two cryptocurrencies operate on two different elliptic curves.

\paragraph{Future Research.} Erkan Tairi et al. have demonstrated in~\cite{tairi2019a2l} how to construct Payment channel hubs as a solution to scalability issues of blockchain technologies.
With their approach relying solely on Scriptless Scripts, it would be interesting to explore the feasibility of such a construction on a Mimblewimble-based cryptocurrency by utilizing the contract transactions outlined in our research.
A. Faz-Hernandez et al. examine the possibility of hash functions that output an elliptic curve point for multiple elliptic curves, including secp256k1.~\cite{hernandez2020hashing} Based on this work on could research if more elaborate contracts such as hashed timelock contracts are made possible on Mimblewimble with such a hash function.
\todo[inline]{Interesting, I would like to talk to you about this.}
Finally, another interesting research topic would be the construction of a cross-chain Atomic Swap between two cryptocurrencies both lacking scripting functionality.
For instance by combining our protocol with the construction by Gugger~\cite{gugger2020bitcoin} to build Atomic Swaps between Monero and a Mimblewimble-based cryptocurrency.


