\urldef\urlharrypotter\url{https://harrypotter.fandom.com/wiki/Tongue-Tying_Curse}
\urldef\urlgrin\url{https://grin.mw/}
\urldef\urlbeam\url{https://beam.mw/}

\todo[inline]{Pedro: We need to discuss a structure for the introduction. Proposal:\\
- Introduce why coin exchanges are interesting\\
- Explain why atomic swaps protocols (e.g., one could use a trusted server for this and problem solved, right?)\\
- Why coin exchanges between Bitcoin and Mimblewimble?\\
- Why what you are proposing in this thesis is challenging?\\
- What are the main contributions of these thesis?\\
- What do you think is an interesting future research direction?}

\paragraph{Mimblewimble} The Mimblewimble protocol was introduced in 2016 by an anonymous entity named Jedusor, Tom Elvis~\cite{jedusor2016mimblewimble}. The author's name, as well as the protocols name,
are references to the Harry Potter franchise. \footnote{\urlharrypotter} In Harry Potter, Mimblewimble is a tongue-typing curse which reflects
the goal of the protocol's design, which is improving the user's privacy.
Later, Andrew Poelstra took up the ideas from the original writing and published his understanding of the protocol in his paper~\cite{poelstra2016mimblewimble}.
The protocol gained increasing interest in the community and was implemented in the Grin \footnote{\urlgrin} and Beam \footnote{\urlbeam} Cryptocurrencies, which both launched in early 2019. In the same year,
two papers were published, which successfully defined and proved security properties for Mimblewimble~\cite{fuchsbauer2019aggregate,betarte2019towards}.

\todo[inline]{Pedro: I would not add a line break at the end of each paragraph. The template should do that}

\todo[inline]{Pedro: If you are going to compare to Bitcoin, you need to introduce Bitcoin before}

Compared to Bitcoin, there are some differences in Mimblewimble:
\begin{itemize}
    \item Use of Pedersen commitments instead of plaintext transaction values \todo[inline]{Pedro: The reader does not know what Pedersen commitments are at this point. Perhaps say transaction values are hidden from a blockchain observer while this is not the case in Bitcoin}
    \item No addresses. Coin ownership is given by the knowledge of the opening of the coins Pedersen commitment. \todo[inline]{Pedro: This is also unclear. Could one see the commitment as the ``address'' in Mimblewimble? Perhaps you want to say that there is no scripting language supported?}
    \item Spend outputs are purged from the ledger such that only unspent transaction outputs remain.
    \item No scripting features.
\end{itemize}


By utilizing Pedersen commitments in the transactions, we \todo{Pedro: Use ``we'' for contributions that you do in the thesis and ``they'' for parts that are borrowed from other works} hide the amounts transferred in a transaction,
improving the systems user privacy, but also requiring additional range proofs, attesting to the fact that actual amounts transferred are in between a valid range. 
Not having any addresses enables transaction merging and transaction cut through, \todo{Pedro: An intuition of these two terms is required here} which we will explain in section~\ref{sec:Mimblewimble}.
However, this comes with the consequence that building transactions require active interaction between the sender and receiver,
which is different than in constructions more similar to Bitcoin, where a sender can transfer funds to any address without requiring active participation by the receiver. \todo{Pedro: another sentence that shows that you need to explain before how Bitcoin works (the basics)}
Through transaction merging and cut-through and some further protocol features, which we will see later in this section, we gain the third mentioned property of being able
to delete transaction outputs from the Blockchain, which have already been spent before. This ongoing purging in the Blockchain makes it particularly space-efficient as the
space required by the ledger only grows in the number of UTXOs, in contrast to Bitcoin, in which space requirement increases with the number of overall mined transactions.
Saving space is especially relevant for Cryptocurrencies employing confidential transactions because the size of the range proofs attached to outputs can be significant.

\todo[inline]{Pedro: What comes next is hard to read. It requires better organization: Advantages of Mimblewimble are: (i) .., (ii)...; Disadvantages are: (i)..., (ii),...).}
Another advantage of this property is that new nodes joining the system do not have to verify the whole history of the Blockchain to validate the current state, making it much easier to join the network. 
Another limitation of Mimblewimble- based Cryptocurrencies is that at least the current construction does not allow scripts, such as they are available in Bitcoin or similar systems.
Transaction validity is given solely by a single valid signature plus the balancedness of inputs and outputs.
This shortcoming makes it challenging to realize concepts such as multi signatures or conditional transactions which are required for Atomic Swap protocols. However,
as we will see in~\ref{sec:scriptlessScripts} there are ways we can still construct the necessary transactions by merely relying on cryptographic primitives ~\cite{fuchsbauer2019aggregate}.