\urldef\urlgithub\url{https://github.com/jafalter/mw-btc-swap}
\urldef\urlgrinwallet\url{https://github.com/mimblewimble/grin-wallet}
\urldef\urlrustbitcoin\url{https://github.com/rust-bitcoin/rust-bitcoin}
\urldef\urlprone\url{https://github.com/mimblewimble/grin-wallet/pull/565}
\urldef\urlprtwo\url{https://github.com/mimblewimble/grin-wallet/pull/557}
\urldef\urljsonrpc\url{https://www.jsonrpc.org/}

We have developed a prototype implementation\footnote{\urlgithub} of the Atomic Swap protocol outlined in~\cref{sec:atom:atomic-swap} in the Rust programming language between Bitcoin and the Grin Cryptocurrency.
On the Grin side, we have been using the official grin-wallet\footnote{\urlgrinwallet} library, on the Bitcoin side, the rust-bitcoin\footnote{\urlrustbitcoin} library for transaction creation, signing, etc.
We choose the programming language Rust because the official libraries available for the Grin Cryptocurrency are also written in Rust.
We discovered two shortcomings in the grin-wallet library during development, so we used our local forked version of this library to address those shortcomings.
We then submitted both changes to the library as pull requests.\footnote{\urlprone}\footnote{\urlprtwo}
Our implementation sends JSON-RPC\footnote{\urljsonrpc} requests to a locally running Bitcoin Core and Grin Core node to query blockchain state, submit, and verify transactions.
The developed program is executable via the command line and accepts the following commands:
\begin{itemize}
    \item \textbf{init}: The init command is run during the setup phase of the Atomic Swap.
    It creates a new swap slate consisting of a public file containing information about the offered currency, exchange rate, connection details, and other general parameters and a private file in which secret keys and nonce values are stored.
    The creator can share the public file with an interested party to execute a swap.
    \item \textbf{import}: The import command is run during the Atomic Swap setup phase and allows importing of funds both on the Bitcoin and the Grin side.
    Note that both parties need to import funds with a value of at least the desired value to be swapped before the protocol can continue.
    \item \textbf{accept}: An interested party executes the accept command during the setup phase.
    It imports the public slate file shared by the offering party and creates the individual private slate file, to which funds can be imported.
    \item \textbf{listen}: The listen command concludes the setup phase on the offerer's side and starts a TCP server to which peers can connect.
    The precondition for this command to execute is that the offered funds have been successfully imported into the swap slate.
    \item \textbf{lock}: The lock command starts the locking phase of the Atomic Swap.
    It is executed by the accepting party, while the offering party must already be listening on a TCP server.
    Again the command will verify that enough funds have been imported and will otherwise exit with an error.
    During this phase, funds will be locked up on both chains as specified in~\cref{subsec:atom:locking}.
    Private and public slate files will be updated during this command to allow swap execution to be initiated.
    \item \textbf{execute}: The execute command runs the execution phase of the Atomic Swap~\cref{subsec:atom:exec}, unlocking the funds locked during the locking phase and transferring the coins to their new owners.
    Again it has to be run by the accepting party while the offering party is still listening on a TCP server.
    Before the execution starts, the program will check on both chains if enough time is left to finish the swap before the timeout expires.
    If not enough time is left in at least one of the two blockchains, the program will exit with an error.
    \item \textbf{cancel}: The cancel command returns funds that have been locked during the locking phase to the original owners.
    The accepting party will execute the command while the offering party is listening on a TCP server.
    Running the cancel protocol is only allowed as soon as the timeout has been hit on both blockchains.
    Trying to run the command earlier will result in an error.
\end{itemize}

\section{Implementation Bitcoin side}\label{sec:ImplementationBtc}

On the Bitcoin side, we have used a P2SH address to implement the locking mechanism.
In a practical setting, we note that it would be recommended to use a more efficient P2WSH\footnote{\url{https://bitcoincore.org/en/segwit_wallet_dev/}} (Pay to Witness Script Hash) address instead.
Because of the implementation just being a prototype, we went with the slightly simpler P2SH version.
However, one can also use the same script with a P2WSH address.
\Cref{fig:bitcoin-script} shows the locking script used on the Bitcoin side.
The script has two execution paths, depending on the number on the stack when the \textit{OP\_IF} command executes.
To make it evaluate to true, one would have to push any number other than 0 to the stack.
To make it evaluate to false, one would have to push the number 0 to the stack.
If \textit{OP\_IF} evaluates to true, we execute the refunding code.
\textit{OP\_CHECKLOCKTIMEVERIFY} will cause the script to fail unless the nLockTime on the transaction is equal or greater than the value passed in \textit{<refund\_time>}.
This condition makes sure that this part of the script will only be executable after a specific time, such that refunding is not possible prematurely (Note here that one has to set the nLockTime of the transaction, which guarantees that it can only be mined after a particular time).
If the $\text{OP\_IF}$ evaluates to false, we execute the script's redemption path in which Alice needs to provide a valid signature under her public key and the statement $\varStatement$.

\begin{figure}
    \begin{center}
        \fbox{
        \pseudocode[linenumbering] {
        OP\_IF \\
        \t <refund\_time> \\
        \t OP\_CHECKLOCKTIMEVERIFY \\
        \t OP\_DROP \\
        \t <refund\_pub\_key> \\
        \t OP\_CHECKSIGVERIFY \\
        OP\_ELSE \\
        \t 2\;<recv\_pub\_key>\;<X>\;2\;OP\_CHECKMULTISIGVERIFY \\
        OP\_ENDIF
        }
        }
    \end{center}
    \caption{Bitcoin locking script.}\label{fig:bitcoin-script}
\end{figure}

\section{Implementation Grin side}\label{sec:ImplementationGrin}

On the Grin side, we have implemented the three transaction protocols specified in~\cref{sec:atom:protocols}.
Communication between the transacting parties happens via a TCP channel.
For implementing two-party bulletproof range proofs, we used the implementation of grinswap\footnote{https://github.com/vault713/grinswap}, which runs a three-round protocol between the two parties using the secp256k1-zkp\footnote{https://github.com/mimblewimble/secp256k1-zkp} library for bulletproof computation.

As already laid out in~\cref{sec:atom:atomic-swap}, to lock the Mimblewimble side's funds, the two parties first engage in a $\procDSharedOutputMwTxId$, and then a time-locked $\procDSharedInpMwTxId$ transaction refunding the coins to the original owner.
The timelock on the Grin side can be achieved by setting the transaction's respective kernel feature.
The program saves the refunding transaction to the slate file for later use.
In the execution phase the $\procDScriptMwTxId$ transaction protocol is run.
The result is broadcast to the network sending the Grin to its new owner while revealing the witness value $\varWit$ to Alice, who is can now build a valid spending script for the Bitcoin side.

\section{Evaluation}

We successfully managed to run the full protocol between Alice owning Grin and Bob owning Bitcoin on the Bitcoin and Grin testnets.
In transaction \\10536404873e6ae133afde600b5630d6a00f3be0b9dde01a248c6f13a00b3a4b\footnote{\url{https://tinyurl.com/pend7sdk}}, which was mined in block 1937148\footnote{\url{https://tinyurl.com/7jmryv47}} of the Bitcoin testnet 0.000016 BTC were locked in the lock address 2NCJDq4YRQ9C83fgvepMqU2D9kE4x7h36Ji\footnote{\url{https://tinyurl.com/2sz8munw}}.
On the Grin side our lock transaction locking 0.1 GRN was mined in block 718594\footnote{\url{https://floonet.grinscan.net/block/718594}} sending funds to the lock Commitment \\08c2e1a98f5fd328cc67b7df5ab9fdee9cf0c1c1f166d5d08a02a578945fdf6076\footnote{\url{https://tinyurl.com/y7ed5za5}}.
In the execution phase the locked funds on the Grin side were sent to Bob \\(09ef66334dc2e4c74732dafda8af3c32494eed5b23beb483d29d7ef32bf5c3ebb8\footnote{\url{https://tinyurl.com/ychtyf8v}}) in a contract transaction mined in block 718596\footnote{\url{https://floonet.grinscan.net/block/718596}}.
Executing the contract allowed Alice to unlock her funds on the Bitcoin side in the transaction \\ aa2ab77482841571b6413c68de681830c61527bc6a90ef1781d6208d151fea10\footnote{\url{https://tinyurl.com/45ysh9e4}}, which was mined in block 1937150\footnote{\url{https://tinyurl.com/fddeahk8}} on the Bitcoin testnet.
We can reduce the protocol's execution time to the time it takes for the individual transactions to be mined on the networks.
Therefore depends on the current network load and the miner's fee included with the transaction, as transactions are prioritized according to their fees~\cite{kasahara2016effect}.
It is unavoidable to estimate fees according to the current network load and the Atomic Swap timeout in a practical setting.
Otherwise, a redeeming transaction may stay unconfirmed until the timeout expires.
Then the second party could try to refund the locked funds with a higher fee, outspending the redeeming transaction.
Another time-consuming process of our implementation is the validation of locked funds on the Bitcoin side.
The reason for that is that for importing a new address into a Bitcoin Core node, one has to rescan the blockchain, which can take up to an hour.
To reduce this time, it is possible to import the address without performing a rescan.
However, if the funding transaction was mined in a block before the import command was executed, the transaction would be missed, and the address would therefore appear empty.
A faster solution would be, for instance, to use the API of a block explorer such as\footnote{\url{https://blockstream.info/}}.
However, this would introduce a trusted third party to the protocol, which is generally to be avoided.
