\urldef\urlgithub\url{https://github.com/jafalter/mw-btc-swap}
\urldef\urlgrinwallet\url{https://github.com/mimblewimble/grin-wallet}
\urldef\urlrustbitcoin\url{https://github.com/rust-bitcoin/rust-bitcoin}
\urldef\urlprone\url{https://github.com/mimblewimble/grin-wallet/pull/565}
\urldef\urlprtwo\url{https://github.com/mimblewimble/grin-wallet/pull/557}
\urldef\urljsonrpc\url{https://www.jsonrpc.org/}

We have developed a prototype implementation\footnote{\urlgithub} of the Atomic Swap protocol outlined in~\cref{sec:atom:atomic-swap} in the Rust programming language between Bitcoin and the Grin Cryptocurrency.
On the Grin side we have been using the official grin-wallet\footnote{\urlgrinwallet} library and on the Bitcoin side the rust-bitcoin\footnote{\urlrustbitcoin} library for transaction creation, signing etc.
The choice of programming language was made on the basis that the official libraries available for the Grin Cryptocurrency are also written in Rust.
During development, we discovered two shortcomings in the grin-wallet library, which is why we used our own local forked version of this library in which we addressed those shortcomings.
We then submitted both changes to the library as pull requests.\footnote{\urlprone}\footnote{\urlprtwo}
To query blockchain state, submit and verify transactions our implementation sends JSON-RPC\footnote{\urljsonrpc} to a locally running Bitcoin Core and Grin Core node.
The developed program is executable via the command line and accepts the following commands:
\begin{itemize}
    \item init: The init command is run during the setup phase of the atomic swap.
    It creates a new swap slate consisting of a public containing to offered currency, exchange rate, connection details and other public parameters and a private file in which secret keys and nonces are stored.
    The public file can be shared with an interested party to execute a swap.
    \item import: The import command is run during the setup phase of the atomic swap and allows for the importing of funds both of the Bitcoin and the Grin side.
    Note that both parties need to import funds with a value of at least what is the desired value to be swapped before the protocol can be run.
    \item accept: The accept command is executed by an interested party during the setup phase.
    It imports the public slate file shared by the offering party and creates the respective private slate file, to which funds can be imported.
    \item listen: The listen command concludes the setup phase on the offerers side and starts a TCP server to which peers can connect to.
    The precondition for this command to execute is that the offered funds have been successfully imported into the swap slate.
    \item lock: The lock command starts the locking phase of the Atomic Swap.
    It is executed by the accepting party, while the offering party must already be listening on a TCP server.
    Again the command will verify that enough funds have been imported and will otherwise exit with an error.
    During this phase funds will be locked up on both chains as specified in~\cref{subsec:atom:locking}.
    Private and public slate files will be updated during this command to allow swap execution to be initiated.
    \item execute: The execute command runs the execution phase of the Atomic Swap~\cref{subsec:atom:exec} unlocking the funds locked during the locking phase and transferring the coins to their new owners.
    Again it has to be run by the requesting party while the offering party is still listening on a TCP server.
    Before the execution starts the program will check on both chains if enough time is left to finish the swap before the timeout would be hit.
    In the case that not enough time is left in at least of the two blockchains the program will exit with an error.
    \item cancel: The cancel command returns funds that have been locked during locking phase to the original owners.
    The command will be executed by the requesting party, while the offering party is listening on a TCP server.
    Running the cancel protocol is only allowed as soon as the timeout has been hit on both blockchains.
    Trying to run the command earlier will result in an error.
\end{itemize}

We successfully managed to run the full protocol between Alice owning Grin and Bob owning Bitcoin on the Bitcoin and Grin testnet.
In transaction \\10536404873e6ae133afde600b5630d6a00f3be0b9dde01a248c6f13a00b3a4b\footnote{\url{https://tinyurl.com/pend7sdk}}, which was mined in block 1937148\footnote{\url{https://tinyurl.com/7jmryv47}} of the Bitcoin Testnet 0.000016 BTC were locked in the lock address 2NCJDq4YRQ9C83fgvepMqU2D9kE4x7h36Ji\footnote{\url{https://tinyurl.com/2sz8munw}}.
On the Grin side our lock transaction locking 0.1 GRN was mined in block 718594\footnote{\url{https://floonet.grinscan.net/block/718594}} sending funds to the lock commitment \\08c2e1a98f5fd328cc67b7df5ab9fdee9cf0c1c1f166d5d08a02a578945fdf6076\footnote{\url{https://tinyurl.com/y7ed5za5}}.
In the execution phase the locked funds on the Grin side where sent to Bob \\(09ef66334dc2e4c74732dafda8af3c32494eed5b23beb483d29d7ef32bf5c3ebb8\footnote{\url{https://tinyurl.com/ychtyf8v}}) in a contract transaction mined in block 718596\footnote{\url{https://floonet.grinscan.net/block/718596}}.
This allowed Alice to unlock her funds on the Bitcoin side in transaction \\ aa2ab77482841571b6413c68de681830c61527bc6a90ef1781d6208d151fea10\footnote{\url{https://tinyurl.com/45ysh9e4}} which was mined in block 1937150\footnote{\url{https://tinyurl.com/fddeahk8}} on the Bitcoin testnet.

\section{Implementation Bitcoin side}\label{sec:ImplementationBtc}
\section{Implementation Grin side}\label{sec:ImplementationGrin}
\section{Performance Evaluation}\label{sec:Performance}