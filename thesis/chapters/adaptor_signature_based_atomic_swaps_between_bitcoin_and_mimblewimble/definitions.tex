As we have already discussed in section~\ref{sec:pre:mimblewimble} for the creation of a transaction in Mimblewimble, it is immanent that both the sender and receiver collaborate and exchange messages via a secure channel.
To construct the transaction protocol we assume that we have access to a two-party signature scheme $\varSigSchemeMP$ as defined in definition~\ref{def:sig:two-party-sig}, a zero-knowledge Rangeproofs system $\varProofSystem$
such as Bulletproofs, as described in section~\ref{sec:pre:rangeproof} and a homomorphic commitment scheme $\varCommitScheme$ as defined in definition~\ref{def:pre:homo-com} such as Pedersen Commitments~\ref{def:pre:pedersen}.

Fuchsbauer et al. have defined three procedures $\styleFunction{Send}$, $\styleFunction{Rcv}$ and $\styleFunction{Ldgr}$ with regards to the creation of a transaction.
$\styleFunction{Send}$ called by the sender will create a pre-transaction, $\styleFunction{Rcv}$ takes the pre-transaction and adds the receivers output and $\styleFunction{Ldgr}$ (again called by the sender) publishes the final transaction to the blockchain ledger.
As we have already pointed out in this thesis we won't discuss the transaction publishing phase therefore we will not cover the functionality of the $\styleFunction{Ldgr}$ procedure, however the transactions created by our protocol must be compatible with the $\procLedgrId$ functionality formalized by Fuchsbauer et at.
We redefine the $\styleFunction{Send}$ and $\styleFunction{Rcv}$ functionality in out paper, making small adjustments to the original definitions to fit with our requirements.

To improve the readability of our formalizations we introduce a wrapper $\varSpendableCoin$ which represents a spendable coin and contains a reference to the coin commitment as defined in~\ref{def:pre:coin} as well as its (secret) spending information which consist of the coins value and blinding factor and its rangeproof $\varProof$.
\[ \varSpendableCoin \opAssign \{ \varCommitment, \varValue, \varBlindingFactor, \varProof \} \]
If we want to indicate that a spendable coin is a output coin of a transaction we write $\funStar{\varSpendableCoin}$.

\begin{definition}[Mimblewimble Transaction Scheme]
    \label{def:atom:mw-tx-scheme}
    A Mimblewimble Transaction Scheme $\varMWSchemeParams{\varCommitScheme}{\varSigSchemeMP}{\varProofSystem}$ with commitment scheme $\varCommitScheme$, two-party signature scheme $\varSigSchemeMP$, and rangeproof system $\varProofSystem$ consists of the following tupel of procedures:
    \[ \varMWSchemeParams{\varCommitScheme}{\varSigSchemeMP}{\varProofSystem} \opAssign ( \procSendCoinsId,\procRecvCoinsId,\procFinTxId,\procLedgrId ) \]
    \begin{itemize}
        \item $(\varPreTx, \funStar{\varSpendableCoinAlice} (\varSecKeyAlice, \varNonceAlice)) \opFunResult \procSendCoins{\funArray{\varSpendableCoin}}{\varFundValue}{\varTime}$: The spendCoins algorithm is a DPT function called by the sending party to initiate the spending of some input coins.
        As input, it takes a list of spendable coins $\funArray{\varSpendableCoin}$ and a value $\varFundValue$ which should be transferred to the receiver.
        Optionally a sender can pass a block height $\varTime$ when this transaction should become valid.
        It outputs a pre-transaction $\varPreTx$ which can be send to a receiver, Alice spendable change output coin $\funStar{\varSpendableCoinAlice}$ as well as the senders signing key and secret nonce $(\varSecKeyAlice, \varNonceAlice)$ later used for transaction signing.
        \item $(\funStar{\varPreTx}, \funStar{\varSpendableCoinBob}) \opFunResult \procRecvCoins{\varPreTx}{\varFundValue}$: The receiveCoins algorithm is a DPT routine called by the receiver and takes as input a pre-transaction $\varPreTx$ and a fund value
        $\varFundValue$.
        It will construct the receivers output coin and output a modified pre-transaction $\funStar{\varPreTx}$ together with Bobs new spendable output coin $\funStar{\varSpendableCoinBob}$.
        \item $\varTx \opFunResult \procFinTx{\varPreTx}{\varSecKeyAlice}{\varNonceAlice}$: The finalize algorithm is a DPT routine again called by the transaction sender that takes as input a pre-transaction $\varPreTx$ and the senders signing key $\varSecKeyAlice$ and nonce $\varNonceAlice$.
        The function will output a finalized transaction $\varTx$, which can be published to the ledger using the
        \item $\{\varLedger, \cnstFalsum\} \opFunResult \procLedgr{\varLedger}{\varTx}$: The ledger algorithm is taken from the Fuchsbauer et al. paper and can be used to publish a valid transaction.
        If an invalid transaction is passed it will output $\cnstFalsum$, otherwise it will output the updated ledger.
        We assume that every protocol participant can access the current ledger $\varLedger$, as it is publicly available by running a node.
    \end{itemize}
\end{definition}

We say a Mimblewimble Transaction Scheme is correct if the ledger algorithm $\procLedgrId$ returns and updated $\varLedger$ if and only if the transaction is well balanced and its signature is valid.
More formally:
\begin{definition}[Transaction Scheme Correctness]
    \label{def:atom:tx-scheme-correctness}
    For any list of input coins $\funArray{\varCoinInp}$ with total value $\varValue$ and blinding factors
    $\funArray{\varBlindingFactor}$ and a transaction fund value $\varFundValue$ with $\varFundValue \opSmEq \varValue$ the following must hold:
    \[
        \Pr\left[
        \begin{array}{c}
            \: \procLedgr{\varLedger}{\varTx} \opNotEq \cnstFalsum
        \end{array}
        \middle\vert
        \begin{array}{l}
            (\varPreTx, \cdot, (\varSecKeyAlice, \varNonceAlice)) \opFunResult \procSendCoins{\funArray{\varSpendableCoin}}{\varValue}{\cnstFalsum} \\
            (\funStar{\varPreTx}, \cdot) \opFunResult \procRecvCoins{\varPreTx}{\varFundValue} \\
            \varTx \opFunResult \procFinTx{\funStar{\varPreTx}}{\varSecKeyAlice}{\varNonceAlice}
        \end{array}
        \right]=1.
    \]
\end{definition}

\begin{definition}[Extended Mimblewimble Transaction Scheme]
    \label{def:atom:ext-mw-tx-scheme}
    An extended Mimblewimble transaction scheme $\varextMWSchemeParams{\varCommitScheme}{\varSigSchemeMP}{\varProofSystem}$ is an extension to $\varMWScheme$ with the following two procedures:
    \[ \varextMWSchemeParams{\varCommitScheme}{\varSigSchemeMP}{\varProofSystem} \opAssign \varMWSchemeParams{\varCommitScheme}{\varSigSchemeMP}{\varProofSystem} \opConc (\procDSendCoinsId, \procDRecvCoinsId, \procDFinTxId) \]
    \begin{itemize}
        \item $\langle (\varPreTx, \funStar{\varSpendableCoinAlice}, (\varSecKeyAlice,\varNonceAlice)), (\varPreTx,(\varSecKeyCarol,\varNonceCarol)) \rangle$ \\
        $\opFunResult \procDSendCoins{\funArray{\varSpendableCoinAlice}}{\funArray{\varSpendableCoinCarol}}{\varFundValue}{\varTime}$:
        The distributed coin spending algorithm takes as input a list of spendable input coins which ownership is shared between Alice and Carol.
        Note that for each provided input coin Alice and Carol have only a share to the blinding factor.
        The coins full blinding factor can then be calculated as:
        $\varBlindingFactor \opAssign \varBlindingFactorAlice \opAddScalar \varBlindingFactorCarol$.
        Again optionally a block height $\varTime$ can be given to time lock the transaction.
        \item $\langle (\funStar{\varPreTx}, \funStar{\varSpendableCoinBob}), (\funStar{\varPreTx}, \funStar{\varSpendableCoinCarol}) \rangle \opFunResult \procDRecvCoins{\varPreTx}{\varFundValue}$: The distributed coin receive procedure takes as input a pre-transaction $\varPreTx$ and a value $\varFundValue$ which should be transferred with the transaction.
        The distributed algorithm will generate a output coin owned by both Alice and Carol. (each owning a share of the key).
        The output will be an updated pre-transaction $\funStar{\varPreTx}$, the shared output coin $\varCoinShared$ with the respective shares of the blinding factor added as $\funStar{\varSpendableCoinBob}$, $\funStar{\varSpendableCoinCarol}$.
        Note that $\varCoinShared$ will only be spendable if both owners cooperate running the $\procDSendCoinsId$ protocol.
        \item $\langle \varTx, \varTx \rangle \opFunResult \procDFinTx{\varPreTx}{\varSecKeyAlice}{\varNonceAlice}{\varSecKeyCarol}{\varNonceCarol}$: The distributed finalized transaction protocol has to be used if we are creating a transaction spending a shared coin (i.e. the transaction was created with the $\procDSendCoinsId$ algorithm).
        In this case we require signing information from both Alice and Carol.
    \end{itemize}
\end{definition}

Correctness is given very similar to the standard scheme:

\begin{definition}[Extended Transaction Scheme Correctness]
    \label{def:atom:ext-tx-scheme-correctness}
    For any list of spendable coins $\funArray{\varSpendableCoin}$ with total value and split blinding factors
    $(\funArray{\varBlindingFactorAlice}, \funArray{\varBlindingFactorCarol})$ and a transaction fund value $\varFundValue$ with the following must hold:
    \[
        \Pr\left[
        \begin{array}{c}
            \: \procLedgr{\varLedger}{\varTx} \opNotEq \cnstFalsum
        \end{array}
        \middle\vert
        \begin{array}{l}
            \langle (\varPreTx, \cdot, (\varSecKeyAlice, \varNonceAlice)), (\varPreTx, (\varSecKeyCarol, \varNonceCarol)) \rangle \opFunResult \\
            \procDSendCoins{\funArray{\varSpendableCoinAlice}}{\funArray{\varSpendableCoinCarol}}{\varFundValue}{\cnstFalsum} \\
            \langle (\funStar{\varPreTx}, \cdot)(\funStar{\varPreTx}, \cdot) \rangle \opFunResult \procDRecvCoins{\varPreTx}{\varFundValue} \\
            \varTx \opFunResult \procDFinTx{\funStar{\varPreTx}}{\varSecKeyAlice}{\varNonceAlice}{\varSecKeyCarol}{\varNonceCarol}
        \end{array}
        \right]=1.
    \]
\end{definition}

\begin{definition}[Script Mimblewimble Transaction Scheme]
    \label{def:atom:apt-ext-mw-tx-scheme}
    The script version of the extended Mimblewimble Transaction Scheme updates the Extended Mimblewimble Transaction Schme by providing a modified version of the single party receive routine and the distributed finalize transaction protocol.
    \[ \varaptMWSchemeParams{\varCommitScheme}{\varSigSchemeMP}{\varProofSystem} \opAssign \varextMWSchemeParams{\varCommitScheme}{\varSigSchemeMP}{\varProofSystem} \opConc \procAptRecvCoinsId, \procDAptFinTxId \]
    \begin{itemize}
        \item $(\funStar{\varPreTx}, \funStar{\varSpendableCoinBob}, \varSigBob) \opFunResult \procAptRecvCoins{\varPreTx}{\varFundValue}{\varWit}$: The script variant of the receive function takes an additional input a secret witness value
        $\varWit$ which will be hidden in the transactions signature and extractable by the other party after the protocols' completion.
        Note that the routine also returns Bob's unadapted partial signature.
        The reason for this is that we later still need the unadapted version to complete the signature und thereby finalize the transaction.
        By not sharing this unadapted signature with Alice, Bob is the one who gets to finalize the transaction which is different from the simpler protocol and is an important feature for our atomic swap protocol.
        \item $\langle \varSigAliceBob, \varTx \rangle \opFunResult \procDAptFinTx{\funStar{\varPreTx}}{\varSecKeyAlice}{\varNonceAlice}{\varStatement}{\varSecKeyBob}{\varNonceBob}{\varSigBob}$: The script variant of the finalize transaction algorithm is a distributed protocol between the sender(s) and receiver.
        Additionally to the pre-transaction $\funStar{\varPreTx}$ the senders need to input their signing information, Bob needs to input the unadapted version of his partial signature as it is needed for transaction completion.
        This protocol could also be implemented as a three party protocol, two senders controlling a shared coin and a third receiver.
        However, as in the case we will describe later in~\ref{sec:atom:atomic-swap} one of the two senders is also the receiver, we allowed ourselves to model this protocol as being between only two parties to simplify the formalization.
        In this version of the protocol only Bob will be able to finalize the transaction, which is different to $\procFinTxId$ and $\procDFinTxId$.
        This has the practical reason that for the atomic swap execution Bob needs to be the one in control of the final transaction.
        If Alice were to build the final transaction before Bob, she might be able to extract the Witness value before the transaction has been published, which in the atomic swap scenario would mean she could steal the funds stored on the other chain.
        This is why the protocol does not return the final transaction $\varTx$ to Alice, instead the protocol will output the senders partial signature, which Alice can later use to extract a witness value.
    \end{itemize}
\end{definition}

Similar as before we define correctness for the adapted scheme:

\begin{definition}[Adapted Transaction Scheme Correctness]
    \label{def:atom:apt-tx-scheme-correctness}
    For any list of input coins $\funArray{\varSpendableCoin}$ and a transaction fund value $\varFundValue$ with and any witness value $\varX \opIn \cnstIntegersPrimeWithoutZero{*}$ the following must hold:
    \[
        \Pr\left[
        \begin{array}{c}
            \: \procLedgr{\varLedger}{\varTx} \opNotEq \cnstFalsum
        \end{array}
        \middle\vert
        \begin{array}{l}
            (\varPreTx, \funStar{\varSpendableCoinAlice}, (\varSecKeyAlice, \varNonceAlice)) \opFunResult \procSendCoins{\funArray{\varSpendableCoin}}{\varFundValue}{\cnstFalsum} \\
            (\funStar{\varPreTx}, \funStar{\varSpendableCoinBob}, \varSigBob) \opFunResult \procAptRecvCoins{\varPreTx}{\varFundValue}{\varWit} \\
            \langle \varSigAliceCarol, \varTx \rangle \opFunResult \procDAptFinTx{\varPreTx}{\varSecKeyAlice}{\varNonceAlice}{\varStatement}{\varSecKeyCarol}{\varNonceCarol}{\varSigBob}
        \end{array}
        \right]=1.
    \]
\end{definition}