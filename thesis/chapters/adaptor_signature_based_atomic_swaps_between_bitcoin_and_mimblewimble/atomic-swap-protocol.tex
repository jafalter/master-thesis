With the outlined Adapted Mimblewimble Transaction Scheme from~\cref{def:atom:apt-ext-mw-tx-scheme} and protocols from~\cref{sec:atom:protocols}, we can now construct an Atomic Swap protocol with another Cryptocurrency.
This thesis will explain a swap with Bitcoin, as at present, Bitcoin is the most widely adopted Cryptocurrency.
To abstract away from the details of different Bitcoin implementations, we define here the minimal DPT functions that we require for our Atomic Swap.
These functionalities are inherent to the Bitcoin functionality and thus supported in all implementations.
We define the following three DPT functions $(\procLockAddrId, \procVerifyLockId, \procSpendBtcId)$.
\begin{itemize}
    \item $(\varScriptPubKey) \opFunResult \procLockAddr{\varPubKeyAlice}{\varPubKeyBob}{\varStatement}{\varTime}$:
    The locking script function lets Bob construct a Bitcoin script only spendable by Alice if she receives the discrete logarithm $\varWit$ of $\varStatement$ with $\varStatement \opEqNoQ \funGen{\varWit}$.
    Additionally, the procedure requires Bob's public key $\varPubKeyBob$ and a timelock $\varTime$ (given as a block number) as input, allowing Bob to reclaim his funds after some time if the Atomic Swap was not completed successfully.
    The function will create and return a Bitcoin script $\varScriptPubKey$ to which Bob can send funds using a P2SH transaction.
    To spend this output, Alice will have to provide a multi-signature under her public key $\varPubKeyAlice$ and $\varStatement$, which she can provide, once acquired $\varWit$.
    Malavolta et al. in~\cite{malavolta2019anonymous} described an alternative secure way of constructing a locking mechanism on Bitcoin.
    In their construction the two parties cooperate to build an initial signature for the spending transaction which is not yet valid as it is missing some witness value $\varWit$, only known to one of the two parties.
    Once the second party gets hold of the witness value they can complete the signature and finalize the transaction.
    Comparing their solution to the more primitive multi-signature script, it achieves greater privacy (from the outside, the lock output looks like a standard P2PKH output), and needs only a single signature, therefore less space for the unlocking transaction.
    However, the construction is slightly more complex in ECDSA signatures, which are at present the only Signature Scheme available on Bitcoin.
    Even though the structure by Malavolta et al. would also be applicable in our case, because of the additional complexity involved and since the focus of this thesis is the Mimblewimble side of the swap, we decided to implement the more straightforward script-based locking mechanism in our proof of concept implementation.
    \item $\{ 1,0 \} \opFunResult \procVerifyLock{\varPubKeyAlice}{\varPubKeyBob}{\varStatement}{\varValue}{\varTime}{\varUTXO_{lock}}$:
    The lock verification algorithm takes as input Alice and Bob public keys and the statement $\varStatement$ and the UTXO $\varUTXO_{lock}$.
    The function will compute the Bitcoin lock script $\varScriptPubKey$ as created by $\procLockAddrId$ check equality with $\varUTXO_{lock}$, and if the value locked under the UTXO equals $\varValue$.
    Upon successful verification, the function returns 1, otherwise 0.
    \item $\varTx \opFunResult \procSpendBtc{\varInputs}{\varOutputs}{\varSecKey}$:
    The spend Bitcoin functionality is a wrapper around the $\procBuildTransactionId, \procSignTransactionId$ defined in~\cref{subsec:pre:bitcointx}.
    It constructs and signs a transaction spending the UTXOs given in $\varInputs$ and creates the fresh UTXOs in $\varOutputs$.
    It returns a signed transaction which then can be broadcast.
\end{itemize}


In the following, we describe the phases of an Atomic Swap protocol executed between two parties, one owning funds on a Bitcoin-like Cryptocurrency and the other on a Mimblewimble based one.
As both of the currently most prominent implementations of the Mimblewimble protocol operate on the secp256k1 elliptic curve (which is also the curve that Bitcoin uses), we, therefore, assume a secp256k1-based implementation of the Mimblewimble protocol.
For a Mimblewimble based implementation that operates on a different curve, additional considerations would have to be made to guarantee the protocol's security.
In the setup phase~\cref{subsec:atom:setup}, the two parties agree on the swap parameters: the exchange rates, the amount being swapped and the timeout for the refunding.
In the locking phase~\cref{subsec:atom:locking}, the goal is to lock up the funds on both chains, such that they can either be redeemable by the other party in case the swap was successful or be refunded to the original owners in case the trade has failed.
The precondition for running the locking phase is that the parties have first completed the setup phase.
In the execution phase~\cref{subsec:atom:exec}, the two parties cooperate to redeem the funds locked by the other parties.
The peers can only enter this phase after completing the locking step.
When the funds are redeemed on both sides, the swap is considered successfully completed.
In case the execution phase fails, for instance, if one party stops cooperating, the exchange is considered failed, and we enter the refunding phase~\cref{subsec:atom:refund}.
A unique security requirement here is that the funds are refunded to their original owners on both sides only in case of failure.
If the swap completes on one side but then can't be completed on the other side, one party would lose all of their value, therefore we must make sure that this case is an impossibility.

\subsubsection{Setup Phase}\label{subsec:atom:setup}

We assume Alice owns Mimblewimble coins $\funArray{\varSpendableCoin}$ with the total value $\varValueMw$, and Bob owns
Bitcoin locked in some UTXO $\varUTXO$ with a value of $\varValueBtc$ and secret spending key $\varSecKey_{btc}$.
Before the protocol can start, the two parties must agree on the value they want to swap, the exchange rate of the currencies, and a time after which we should cancel the swap.
After agreeing, the following variables are defined and known by both Alice and Bob:
\begin{itemize}
    \item $\varSecParam$ A security parameter.
    \item $\varAmountBtc$ The amount of Bitcoin Bob will swap to Alice.
    \item $\varAmountMW$ The amount of the Mimblewimble coin Alice will swap to Bob.
    \item $\varTimeBTC$ The locktime as a block height for the Bitcoin side.
    \item $\varTimeMW$ The locktime as a block height for the Mimblewimble side.
\end{itemize}
We collect this shared variables in an initial swap state $\varSwpState$:
\[ \varSwpState \opAssign \{ \varSecParam, \varAmountBtc, \varAmountMW, \varTimeBTC, \varTimeMW \} \]

In practice, we need to consider that exchange rates might fluctuate.
Furthermore timeouts have to be calculated separately for each chain.
The problems with cross-chain payments are discussed by Tairi et al. in~\cite{tairi2019a2l}.
They propose using a fixed exchange rate for each day and using real-world timeout like one day and then calculating the specific block numbers by taking blockchain's average block time into account.
Alternatively, if the chains allow it, we could use a real-world Unix timestamp as a timeout instead of a block height.
In our setup, we can also fix the exchange rate at the beginning of the protocol, which stays unchanged during protocol execution.
Suppose the exchange rate fluctuates and one party is negatively impacted.
In that case, they could still decide stop cooperating, which means the coins would be returned to the original owners after the timeout.

There is furthermore the problem of transaction fees, which we do not consider for this formalization.
Depending on the current network load, the participants need to agree on a fee that they are willing to pay for each network.
It needs to be considered that if fees are picked too low, it might take time for transactions to be confirmed, and the swap will take longer.
If they are picked high, the participants will lose value.

\subsection{Locking phase}\label{subsec:atom:locking}

We formalize the protocol $\procSetupSwapId$ in~\cref{fig:setup-swap}.
The protocol takes as input the shared swap state $\varSwpState$ from both parties.
From Alice, her Mimblewimble input coins $\funArray{\varSpendableCoin}$ with the summed up value $\varValueMw$ is furthermore required as an input.
From Bob, we require a list of UTXO's $\funArray{\varUTXO}$ he wants to spend.
He also needs to provide their spending keys $\funArray{\varSecKey_{btc}}$ and their sum of total value $\varValueBtc$, although one could also read this from the blockchain.

The protocol starts with both parties creating and exchanging keys.
Bob now makes two new Bitcoin outputs $\varUTXO_{lock}$ and $\varUTXO_{B}$, one of these is the locked Bitcoins that Alice might retrieve later (or Bob after time $\varTimeBTC$ has passed), and the other Bobs change output. (Difference between what is stored in the input UTXO and what should be sent to Alice).
After Bob has published the transaction sending value to the new outputs, he will provide Alice with the statement $\varStatement$ under which the Bitcoins are locked together with Alice's public key.
Alice can now verify that the funds on the Bitcoin side are indeed correctly locked.
After that, she will collaborate with Bob to spend her Mimblewimble coins into an output shared by both parties.
Both parties immediately collaborate again to spend this shared coin back to Alice with a timelock of $\varTimeMW$.
It is immanent that Alice does not publish the first transaction (A -> AB) before the time-locked refund transaction (AB -> A) was signed.
Otherwise, funds are locked in the shared output without the possibility of a refund if Bob refuses to cooperate.
The locking protocol concludes with the funds locked up in both chains and ready to be swapped and outputs the updated swap state $\varSwpState$ to both parties.
Additionally, it outputs Alice's part $\funStar{\varPtSpendableCoinAlice}$ of the locked Mimblewimble coin, her change output on the Mimblewimble side $\funStar{\varSpendableCoinAlice}$, her secret key $\varSecKeyAlice$ for the Bitcoin side, and $\funStarAlt{\varSpendableCoinAlice}$, which is a refund coin, only valid after $\varTimeMW$.
Bob also outputs his part $\funStar{\varPtSpendableCoinBob}$ of the locked Mimblewimble coin, his change output on the Bitcoin side $\varUTXO_{B}$ and the secret witness value $\varWit$, which shall be revealed to Alice in the execution phase.

\newgeometry{margin=2cm}
\begin{landscape}
    \thispagestyle{plain}
    \begin{figure}
        \begin{center}
        \fbox{
        \procedure[linenumbering,skipfirstln]{$\procSetupSwap{\varSwpState}{\funArray{\varSpendableCoin}}{\varValueMw}{\funArray{\varUTXO}}{\funArray{\varSecKey_{btc}}}{\varValueBtc}$} {
        Alice \< \< \< \< Bob \\
        \{ \varAmountBtc, \varAmountMW, \varTimeBTC, \varTimeMW \} \opFunResult \varSwpState \< \< \< \< \{ \varAmountBtc, \varAmountMW, \varTimeBTC, \varTimeMW \} \opFunResult \varSwpState \\
        \varKeyPairAlice \opFunResult \procSetup{\varSecParam} \< \< \< \< \varKeyPairBob \opFunResult \procSetup{\varSecParam} \\
        \< \< \< \< (\varWit, \varStatement) \opFunResult \procSetup{\varSecParam} \\
        \< \sendmessagerightx{4}{\varPubKeyAlice} \< \\
        \< \sendmessageleftx{4}{\varPubKeyBob} \< \\
        \< \< \< \< \varScriptPubKey \opFunResult \procLockAddr{\varPubKeyAlice}{\varStatement}{\varPubKeyBob}{\varTimeBTC} \\
        \< \< \< \< \varUTXO_{lock} \opFunResult \procCreateUTXO{\varAmountBtc}{\varScriptPubKey} \\
        \< \< \< \< \varUTXO_{B} \opFunResult \procCreateUTXO{\varValueBtc - \varAmountBtc}{\varPubKeyBob} \\
        \< \< \< \< \varBtcTx \opFunResult \procSpendBtc{\funArray{\varUTXO}}{\funArray{\varUTXO_{lock}, \varUTXO_{B}}}{\funArray{\varSecKey_{btc}}} \\
        \< \< \< \< \procPublishBtc{\funArray{\varBtcTx}} \\
        \< \< \< \< \varSwpState \opAssign \varSwpState \opUnion (\varStatement, \varUTXO_{lock}) \\
        \< \sendmessageleftx{4}{\varStatement,\varUTXO_{lock}} \< \\
        \pcif \procVerifyLock{\varPubKeyAlice}{\varPubKeyBob}{\varStatement}{\varAmountBtc}{\varTimeBTC}{\varUTXO_{lock}} \opEqNoQ 0 \\
        \t \pcreturn \cnstFalsum \< \< \< \< \\
        \varSwpState \opAssign \varSwpState \opUnion (\varStatement, \varUTXO_{lock}) \\
        (\varMwFundTx, \funStar{\varSpendableCoinAlice},\funStar{\varPtSpendableCoinAlice}) \< \< \< \< (\funStar{\varPtSpendableCoinBob}) \pcskipln \\
        \opFunResult \procDSharedOutputMwTxL{\funArray{\varSpendableCoin}}{\varAmountMW}{\cnstFalsum} \< \< \< \< \opFunResult \procDSharedOutputMwTxR{\varAmountMW} \\
        (\varMwRefundTx, \funStarAlt{\varSpendableCoinAlice}) \< \< \< \< \varMwRefundTx \pcskipln \\
        \opFunResult \procDSharedInpMwTxL{\funStar{\varPtSpendableCoinAlice}}{\varAmountMW}{\varTimeMW} \< \< \< \< \opFunResult \procDSharedInpMwTxR{\funStar{\varPtSpendableCoinBob}}{\varAmountMW} \\
        \procPublishMW{\funArray{\varMwFundTx,\varMwRefundTx}}  \\
        \pcreturn (\varSwpState, \funStar{\varPtSpendableCoinAlice}, \funStar{\varSpendableCoinAlice}, \varSecKeyAlice, \funStarAlt{\varSpendableCoinAlice}) \< \< \< \< \pcreturn (\varSwpState, \funStar{\varPtSpendableCoinBob}, \varUTXO_{B}, \varWit)
        }
        }
        \end{center}
        \caption{Atomic Swap - $\procSetupSwapId$.}\label{fig:setup-swap}
    \end{figure}
\end{landscape}
\restoregeometry

\subsection{Execution Phase}\label{subsec:atom:exec}

First, we need to define an additional auxiliary function $\procVerifyTimeId$ with the following interface:
\[ \{0,1\} \opFunResult \procVerifyTime{\varChain}{\varTime} \]
This function will verify that there is sufficient time left to execute the Atomic Swap protocol.
As input, it takes a chain parameter $\varChain$ (in our case, this could be either BTC or MW) and a block height $\varTime$.
The routine will verify that the current height of the blockchain is marginally below $\varTime$.
If this is the case, it will return 1 or 0 otherwise.
How much time exactly should be left for the function to return 1 is implementation-specific and could be set to, for instance, one day.
We now define a protocol $\procExcSwapId$ to execute the Atomic Swap between some amount $\varAmountBtc$ on the Bitcoin side, and some amount on the Mimblewimble side $\varAmountMW$.
The reader can find an instantiation of the protocol in~\cref{fig:exec-swap}.
We assume the participants have successfully run the $\procSetupSwapId$ protocol, and both know the updated swap state $\varSwpState$ as returned by the setup protocol.
Both parties need to provide their part of the locked Mimblewimble coins as input to the protocol.
Additionally, Alice needs to provide her secret key for the Bitcoin side $\varSecKeyAlice$ and Bob the private witness value $\varWit$.
The protocol starts with both parties checking that there is enough time left to complete the protocol.
After the check, they will run the $\procDScriptMwTxId$ protocol in which they spend the locked Mimblewimble output to Bob while at the same time revealing $\varWit$ to Alice.
One of the parties can now publish the transaction to the Mimblewimble network, concluding the swap on the Mimblewimble side, as Bob is now in complete control of the funds.
Knowing $\varWit$, Alice now creates a new UTXO where she then sends the funds from the Bitcoin lock.
After publishing this transaction to the Bitcoin network, Alice is in full possession of the Bitcoin side's swapped funds, and the Atomic Swap is completed.
The protocol outputs their newly created output/coin to each party.
We note here that after completion of the swap on the Mimblewimble side, Alice is possible to redeem her Bitcoin.
However, she still has to construct the transaction and get it mined on the network.
Otherwise, if she would take too long and the timeout block height is reached, Bob could still try to refund his coins, even though he already received the funds on the Mimblewimble side.
Therefore, it is crucial to pick long enough timeouts and check how much time is left again before running the execution protocol.

\newgeometry{margin=2cm}
\begin{landscape}
    \thispagestyle{plain}
    \begin{figure}
        \begin{center}
        \fbox{
        \procedure[linenumbering,skipfirstln]{$\procExcSwap{\varSwpState}{\funStar{\varPtSpendableCoinAlice}}{\varSecKeyAlice}{\funStar{\varPtSpendableCoinBob}}{\varWit}$} {
        Alice \< \< \< \< Bob \\
        (\varAmountMW, \varAmountBtc, \varTimeMW, \varTimeBTC, \varUTXO_{lock}, \varStatement) \opFunResult \varSwpState \< \< \< \< (\varAmountMW, \varAmountBtc, \varTimeMW, \varTimeBTC) \opFunResult \varSwpState \\
        \pcif \procVerifyTime{BTC}{\varTimeBTC} \opEqNoQ 0 \opOr \procVerifyTime{MW}{\varTimeMW} \opEqNoQ 0 \< \< \< \< \pcif \procVerifyTime{BTC}{\varTimeBTC} \opEqNoQ 0 \opOr \procVerifyTime{MW}{\varTimeMW} \opEqNoQ 0 \\
        \t \pcreturn \cnstFalsum \< \< \< \< \t \pcreturn \cnstFalsum \\
        (\varMwTx, \cnstEmptySet, \varWit) \< \< \< \< (\varMwTx, \funStar{\varSpendableCoinBob}) \pcskipln \\
        \opFunResult \procDScriptMwTxL{\funStar{\varPtSpendableCoinAlice}}{\varAmountMW}{\cnstFalsum}{\varStatement} \< \< \< \< \opFunResult \procDScriptMwTxR{\funStar{\varPtSpendableCoinBob}}{\varAmountMW}{\varWit} \\
        \< \< \< \< \procPublishMW{\varMwTx} \\
        (\funStarAlt{\varSecKeyAlice}, \funStarAlt{\varPubKeyAlice}) \opFunResult \procSetup{\varSecParam} \< \< \< \< \\
        \varUTXO_{A} \opFunResult \procCreateUTXO{\varAmountBtc}{\funStarAlt{\varPubKeyAlice}} \< \< \< \< \\
        \varBtcTx \opFunResult \procSpendBtc{\funArray{\varUTXO_{lock}}}{\funArray{\varUTXO_{A}}}{\funArray{\varSecKeyAlice, \varWit}} \< \< \< \< \\
        \procPublishBtc{\funStar{\varBtcTx}} \< \< \< \< \\
        \pcreturn (\varUTXO_{A}) \< \< \< \< \pcreturn (\funStar{\varSpendableCoinBob})
        }
        }
        \end{center}
        \caption{Atomic Swap - $\procExcSwapId$. \label{fig:exec-swap}}
    \end{figure}
\end{landscape}
\restoregeometry

\subsection{Refunding phase}\label{subsec:atom:refund}

If one party refused to cooperate, or goes offline the coins can be returned to the original owner.
Bob can spend the locked output with his private key $\varSecKeyBob$ on the Bitcoin side after the timeout $\varTimeBTC$ has passed.
He can then construct and sign a transaction spending the output to a new UTXO in his full possession.
He even could prepare this transaction upfront and broadcast it.
Once the block number hits $\varTimeBTC$, the transaction will become valid and get mined.
Again we stress the importance of using appropriate timeouts.
If a timeout is too short, the swap might get canceled if there are some delays.
If the timeout is too long, the funds might be locked for an unnecessary amount of time.

On the Mimblewimble side, the second transaction spending the shared output back to Alice guarantees that her funds are returned to her after the timeout $\varTimeMW$ hits.
For this reason, it is so vital that Alice publishes both the fund and refund transaction at the same time.
If she would post the funding transaction first, Bob could refuse to cooperate for the refund transaction, in which case the funds would stay in the locking output only retrievable if both parties cooperate.
If the swap executes successfully, the refund transaction would get discarded by miners, as it is no longer valid even after the timeout $\varTimeMW$.