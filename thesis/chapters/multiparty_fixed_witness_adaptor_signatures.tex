In this chapter, we will define a variant of the adaptor signature scheme as defined in ~\cref{def:pre:script:apt}.
This new variant is tailored specifically to meet the requirements of being applicable in the scenario of two-party signature protocols, constructable for signature schemes such as Schnorr~\cite{maxwell2019simple}.
In a two-party signature protocol each party holds only a share of a private key (to a composite public key) for which they want to cooperatively create a signature.
The advantage of our adapted signature scheme in comparison to the original definition is that in the two-party scenario mentioned above we do not need to introduce an additional pre-signature step.
Instead one of the partial signatures created and exchanged by the two parties will serve as what is defined as the pre-signature by Aumayr et al., allowing for a simpler protocol.
In particular our protocol will allow one of the two parties, to mask his signature share with a witness value $\varWit$ of $(\varWit, \varStatement) \opIn \cnstRelation$ (where $\cnstRelation$ is a hard relation as of \cref{def:pre:hard-relation}).
The second party (knowing $\varStatement$, but not $\varWit$) can verify that $\varWit$ is indeed contained in the partial signature received by the peer.
To complete the final signature the party knowing $\varWit$ has to first replace his partial signature (masked with $\varWit$) with the original unmasked version, which corresponds to the adapting step of the original adaptor signature scheme~\cref{def:pre:script:apt}.
The second party, having previously received the masked partial signature is now able to extract $\varWit$ from the final signature, his own partial signature and the other parties masked partial signature.
This feature can then be leveraged to build an Atomic Swap protocol as we will show in~\cref{ch:atomicswap}.

The rest of the chapter is organized as follows:
First we will define the general two-party Schnorr signature protocol, as it is currently implemented in Mimblewimble-based cryptocurrencies.
We will then show that the final signatures output by the protocol fulfill the same properties as regular Schnorr signatures seen in~\cite{schnorr1989efficient} and prove correctness of the protocol.
From this two-party protocol, we then derive the adapted variant already mentioned before.
We start by defining our extended signature scheme in~\cref{sec:sig:definitions}, proceed by providing a Schnorr-based instantiation of the protocol in ~\cref{sec:sig:schnorr-inst} and finally prove its correctness and security in~\cref{sec:sig:two-party-apt-security}.

\section{Definitions} \label{sec:sig:definitions}
\urldef\urlgrinexplained\url{https://tinyurl.com/y63hc4ua}

As we have already discussed in~\cref{sec:pre:mimblewimble} for creating a transaction in Mimblewimble, it is immanent that both the sender and receiver collaborate and exchange messages via a secure channel.
To construct the transaction protocol, we assume that we have access to a Two-Party Signature Scheme $\varSigSchemeMP$ as described in~\cref{def:sig:two-party-sig}, a Range Proof System as shown in~\cref{def:pre:rangeproof} such as Bulletproofs, as described in~\cref{sec:pre:rangeproof} and a homomorphic Commitment Scheme $\varCommitScheme$ as defined in~\cref{def:pre:homo-com} such as Pedersen Commitments seen in~\cref{def:pre:pedersen}.

Fuchsbauer et al. have defined three procedures, $\procFuchsSend$, $\procFuchsRcv$, and $\procFuchsLdgr$, regarding creating a transaction.
$\procFuchsSend$ called by the sender will create a pre-transaction, $\procFuchsRcv$ takes the pre-transaction and adds the receiver's output and $\procFuchsLdgr$ (again called by the sender) verifies and publishes the final transaction to the blockchain ledger.
As we have already pointed out in this thesis, we won't discuss the transaction publishing phase.
Therefore we will not cover the publishing functionality of the $\procFuchsLdgr$ procedure.
However, we will use the verification capabilities of the algorithm.
That means the transactions created by our protocol must be compatible with the $\procFuchsVer$ functionality formalized by Fuchsbauer et at. and internally used by $\procFuchsLdgr$.
We can, however, assume that a transaction $\varTx$ for which $\procFuchsVer \opEqNoQ 1$ holds, could be published to the ledger using the $\procFuchsLdgr$ algorithm. (Given the inputs used in the transaction are present and unspent on the ledger)

Originally Fuchsbauer et al. have defined the creation of a Mimblewimble transaction as a two-step, two-party protocol.
A sender owning a set of input coins calls $\procFuchsSend$ to create an initial pre-transaction signed already by the sender and then forwarded to the fund receiver.
The receiver then calls $\procFuchsRcv$ to add his output coins with the correct value.
His signature is then aggregated with the sender's signature and thereby finalizing the transaction $\varTx$.
Any party (knowing the final $\varTx$) can now call $\procFuchsLdgr$ to verify and publish the transaction to the ledger.

We now want to motivate why in the following, we found it necessary to redefine some of the algorithms already laid out by Fuchsbauer et al.
The main reason is that we are using the notion of two-party signatures as of~\cref{def:sig:two-party-sig} instead of aggregatable signatures, which are employed in their paper.
While aggregatable signatures are a similar concept to the two-party signatures, we can find some essential differences.
Ultimately, the two-party signatures is, as we shall see, the more appropriate and secure choice for the formalization.
First of all, we need to define the notion of an Aggregatable Signature Scheme:
\begin{definition}[Aggregatable Signature Scheme] \label{def:atom:aggsig}
    A Signature Scheme $\varSigScheme$ can be called aggregatable if for two signatures $\varSignature_1$ and $\varSignature_2$, valid for a message $\varMsg$ under the public keys $\varPubKey_1$ and $\varPubKey_2$ we can construct an aggregated signature $\varSignature_a$ valid for the same message $\varMsg$ under the composite public key $\varPubKey_a \opEqNoQ \varPubKey_1 \opAddPoint \varPubKey_2$
\end{definition}
In the Schnorr Signature Scheme, we can only aggregate signatures by primitively concatenating the individual signatures like $\varSignature_1 \opConc \varSignature_2$.
The verifier would then check the validity of $\varSignature_1$ and $\varSignature_2$ independently under the public keys $\varPubKey_1, \varPubKey_2$ and finally check if $\varPubKey_a \opEqNoQ \varPubKey_1 \opAddPoint \varPubKey_2$~\cite{fuchsbauer2019aggregate}. \\
The reason why we can not simply add up the signatures is the following:
Recall the structure of a Schnorr signature $(\varS, \varRand)$, imagine we would try to create an aggregated signature like $\varSignature_a \opEqNoQ (\varS_1 \opAddScalar \varS_2, \varRand_1 \opAddPoint \varRand_2)$, then this would not be a valid signature anymore.
$\varS$ is calculated as $\varS \opEqNoQ \varNonce \opAddScalar \varSchnorrChallenge \opTimesScalar \varSecKey$ where $\varSchnorrChallenge \opEqNoQ \funHash{\varMsg \opConc \varRand \opConc \varPubKey}$.
As we have changed the nonce Commitment $\varRand$ and the public key $\varPubKey$ in our aggregated signature the Schnorr challenge $\varSchnorrChallenge$ will be different from the one used by the individual signers, thereby making the verification algorithm return 0.
We can fix this issue by having the individual signers use the final composite $\varRand$ and $\varPubKey$ for their Schnorr challenge calculation, which is exactly what we are doing in the Schnorr-based instantiation of the Two-Party Signature Scheme in~\cref{fig:twoparty-schnorr}.
This detail, however, introduces the necessity for an initial setup phase in which the parties exchange messages to compute $\varRand$ and $\varPubKey$ from their shares.
Using the two-party Schnorr model instead of the aggregated Schnorr, we save space, as we only need to store one single signature instead of multiple.
Further, we also only need to store the final public key $\varPubKey$ and disregard the public key shares.
We also note that the two-party version is currently implemented in Grin and Beam in practice.\footnote{\urlgrinexplained}
Finally, there is another critical advantage that comes with the two-party Schnorr approach.
For the peers to start the signing process, the final composite $\varPubKey$ and nonce Commitment $\varRand$ need to be known.
That also entails that the flow pointed out in~\cite{fuchsbauer2019aggregate}, in which the transaction sender starts the signing process, and the receiver completes it is no longer possible.
Instead, the signing process can only start with the receiver's turn.
We need to introduce a third round.
The sender receives the partially signed pre-transaction from the receiver, adds his partial signature and only now can finalize the signature and thereby the transaction.
While having to add an additional round would seem like an inconvenience at first, we discover that we avoid being vulnerable to a \emph{Transaction Sniff Attack} by doing so.

For the following attack to be possible, we need to assume that the channel between the sender (Alice) and receiver (Bob) has been compromised therefore can no longer be considered secure.
We show that under this assumption, the formalization laid out by Fuchsbauer et al. would be vulnerable to the \emph{Transaction Sniff Attack}.
In contrast our formalization would still be secure.
\paragraph{Transaction Sniff Attack}
Imagine a sender Alice and receiver Bob.
Alice owns three Mimblewimble coins and wants to send one of them to Bob to pay for Bob's service.
They start the transaction-building process and communicate via a channel that they assume to be secure.
However, in reality, the channel they are using is insecure, and attacker $\cnstAdversary$ has managed to compromise it and is secretly listening to every message exchanged between the two.
With the notions defined by Fuchsbauer et al., Alice starts the protocol by running $\varPreTx \opFunResult \procFuchsSend(\cdot)$ and sending $\varPreTx$ to Bob via the channel.
Bob has received $\varPreTx$ from Alice but decides to wait with the protocol continuation because of some urgent task.
In the meantime, the malicious attacker managed to sniff $\varPreTx$ sent by Alice.
Already containing Alice's signature, all the attacker has to do is guess the value Alice might want to send, create an output coin with that value, add his signature, aggregate it with Alice and broadcast the final transaction to the network.
Since the range of possible amounts that Alice might want to transfer is limited, it is trivial for the attacker to guess it in polynomial time.
When now Bob comes back to finalize the transaction, he will discover that he is unable to continue with the protocol, as the transactions input coins are already spent and are now in possession of the attacker.

Starting the signing process only at the receiver's turn and introducing a third-round solves this issue because Alice adds the signature for her input coins only at the last step.
Using the Two-Party Signature Scheme instead of an Aggregatable Signature Scheme forces us to make this change because of the additional setup phase required.
Even if the attacker were able to sniff one of the pre-transactions exchanged between the parties, because Alice will only ever add the signature for her input coins at the end of the protocol, the attacker would not be able to compute a valid transaction.

We now define the standard \emph{Mimblewimble Transaction Scheme} that intuitively allows a sender to transfer value stored in a Mimblewimble coin to a receiver.
To improve the readability of our following formalizations, we introduce a wrapper $\varSpendableCoin$ that represents a spendable coin and contains a reference to the coin Commitment $\varCommitment$, range proof $\varProof$, and its (secret) spending information of the coins value $\varValue$ and blinding factor $\varBlindingFactor$.
\[ \varSpendableCoin \opAssign \{ \varCommitment, \varValue, \varBlindingFactor, \varProof \} \]
If we want to indicate that a spendable coin is used as an output coin in a transaction, we write $\funStar{\varSpendableCoin}$.

\begin{definition}[Mimblewimble Transaction Scheme]
    \label{def:atom:mw-tx-scheme}
    A Mimblewimble Transaction Scheme $\varMWSchemeParams{\varCommitScheme}{\varSigSchemeMP}{\varRProofSystem}$ with Commitment Scheme $\varCommitScheme$, Two-Party Signature Scheme $\varSigSchemeMP$, and Range Proof System $\varRProofSystem$ consists of the following tuple of procedures:
    \[ \varMWSchemeParams{\varCommitScheme}{\varSigSchemeMP}{\varRProofSystem} \opAssign ( \procSendCoinsId,\procRecvCoinsId,\procFinTxId,\procVerfTxId ) \]
    \begin{itemize}
        \item $(\varPreTx, \funStar{\varSpendableCoinAlice}, (\varSecKeyAlice, \varNonceAlice)) \opFunResult \procSendCoins{\funArray{\varSpendableCoin}}{\varFundValue}{\varTime}$: The spendCoins algorithm is a DPT function called by the sending party to initiate the spending of some input coins.
        As input, it takes a list of spendable coins $\funArray{\varSpendableCoin}$ and a value $\varFundValue$ which should be transferred to the receiver.
        Optionally a sender can pass a block height $\varTime$ to make this transaction only valid after a specific time.
        It outputs a pre-transaction $\varPreTx$ which can be sent to a receiver, Alice's spendable change output coin $\funStar{\varSpendableCoinAlice}$ as well as the senders signing key and secret nonce $(\varSecKeyAlice, \varNonceAlice)$ later used in the transaction signing process.
        \item $(\funStar{\varPreTx}, \funStar{\varSpendableCoinBob}) \opFunResult \procRecvCoins{\varPreTx}{\varFundValue}$: The receiveCoins algorithm is a DPT routine called by the receiver and takes as input a pre-transaction $\varPreTx$ and a fund value
        $\varFundValue$.
        It will output a modified pre-transaction $\funStar{\varPreTx}$ and Bob's new spendable output coin $\funStar{\varSpendableCoinBob}$, added to the transaction.
        At this stage, the transaction already has to be partially signed by the receiver.
        \item $\varTx \opFunResult \procFinTx{\varPreTx}{\varSecKeyAlice}{\varNonceAlice}$: The finalize algorithm is a DPT routine called by the transaction sender that takes as input a pre-transaction $\varPreTx$ and the senders signing key $\varSecKeyAlice$ and nonce $\varNonceAlice$.
        The function will output a finalized signed transaction $\varTx$.
        \item $\{1,0\} \opFunResult \procVerfTx{\varTx}$: The verification algorithm is the same as defined in the paper by Fuchsbauer et al.~\cite{fuchsbauer2019aggregate}, we still add it here for completeness.
        Note that in their work, one can find it under the name $\styleFunction{MW.Ver}$.
        We rename it here to $\procVerfId$ to fit with our naming scheme.
        If an invalid transaction is passed to the routine, it will output 0, 1 otherwise.
        Informally the algorithm verifies four conditions:
        \begin{enumerate}
            \item Condition 1: Every input and output coin only appears once in the transaction.
            \item Condition 2: The union of input and output coins is the empty set.
            \item Condition 3: For every output coin, the range proof verifies.
            \item Condition 4: The transaction signature verifies with the excess value of the transaction as the public key, which is calculated by summing up the output coins and subtracting the input coins. (See~\cref{sec:pre:mimblewimble})
        \end{enumerate}
    \end{itemize}
\end{definition}

We say a Mimblewimble Transaction Scheme is correct if the verification algorithm $\procVerfTxId$ returns 1 upon providing a transaction that is well balanced and contains a valid signature.
More formally:
\begin{definition}[Transaction Scheme Correctness]
    \label{def:atom:tx-scheme-correctness}
    For any transaction fund value $\varFundValue$ and list of spendable input coins $\funArray{\varSpendableCoin}$ with combined value $\varValue \opGreaterEq \varFundValue$ the following must hold:
    \[
        \Pr\left[
        \begin{array}{c}
            \: \procVerfTx{\varTx} \opEqNoQ 1
        \end{array}
        \middle\vert
        \begin{array}{l}
            \varFundValue \opSmEq \sum_{\varI \opAssign 0}^{\varI \opSm \varN}(\varSpendableCoin_{i}.\varValue) \\
            (\varPreTx, \cdot, (\varSecKeyAlice, \varNonceAlice)) \opFunResult \procSendCoins{\funArray{\varSpendableCoin}}{\varValue}{\cnstFalsum} \\
            (\funStar{\varPreTx}, \cdot) \opFunResult \procRecvCoins{\varPreTx}{\varFundValue} \\
            \varTx \opFunResult \procFinTx{\funStar{\varPreTx}}{\varSecKeyAlice}{\varNonceAlice}
        \end{array}
        \right]=1.
    \]
\end{definition}

In the following, we define the \emph{Extended Mimblewimble Transaction Scheme}, which intuitively extends the previous scheme with shared coin ownership functionalities, similar to multisignature addresses available in Bitcoin.

\begin{definition}[Extended Mimblewimble Transaction Scheme]
    \label{def:atom:ext-mw-tx-scheme}
    An extended Mimblewimble Transaction Scheme $\varextMWSchemeParams{\varCommitScheme}{\varSigSchemeMP}{\varMPRProofSystem}$ is an extension to $\varMWScheme$ with the following three distributed protocols:
    \begin{gather*}
        \varextMWSchemeParams{\varCommitScheme}{\varSigSchemeMP}{\varMPRProofSystem} \opAssign \\ \varMWSchemeParams{\varCommitScheme}{\varSigSchemeMP}{\varMPRProofSystem} \opConc (\procDSendCoinsId,\procDRecvCoinsId, \procDFinTxId)
    \end{gather*}
    Note that for this scheme, we require a Two-Party Range Proof System $\varMPRProofSystem$ as shown in~\cref{def:pre:mp-rangeproof}.
    Specifically, we need the system to provide a distributed proof computation protocol $\procDRProofId$.
    \begin{itemize}
        \item $\langle (\varPreTx, \funStar{\varSpendableCoinAlice}, (\varSecKeyAlice,\varNonceAlice)), (\varPreTx, \funStar{\varSpendableCoinCarol}, (\varSecKeyCarol,\varNonceCarol)) \rangle$ \\
        $\opFunResult \procDSendCoins{\funArray{\varSpendableCoinAlice}}{\funArray{\varSpendableCoinCarol}}{\varFundValue}{\varTime}$:
        The distributed coin spending algorithm takes as input a list of spendable input coins for which ownership is shared between Alice and Carol.
        Assume that both Alice and Carol own a coin $\varCoin$, then we have two blinding factors $\varBlindingFactorAlice, \varBlindingFactorCarol$, where $\varBlindingFactorAlice$ is known only to Alice and $\varBlindingFactorCarol$ only to Carol.
        Both blinding factors are required to spend the coin.
        Again optionally a block height $\varTime$ can be given to time lock the transaction.
        Similar to the single party version of the function its outputs are a pre-transaction $\varPreTx$ and change coin for each party $\funStar{\varSpendableCoinAlice}$ (resp. $\funStar{\varSpendableCoinCarol}$), and their signing information.
        \item $\langle (\funStar{\varPreTx}, \funStar{\varPtSpendableCoinBob}), (\funStar{\varPreTx}, \funStar{\varPtSpendableCoinCarol}) \rangle \opFunResult \procDRecvCoins{\varPreTx}{\varFundValue}$: The distributed coin receive procedure takes as input a pre-transaction $\varPreTx$ and a value $\varFundValue$ which should be transferred with the transaction.
        The distributed algorithm will generate an output coin with value $\varValue$, owned by both Bob and Carol (each knowing only a share of the coin Commitment's blinding factor).
        The output will be an updated pre-transaction $\funStar{\varPreTx}$, and the spendable shared output coins for each party $\funStar{\varPtSpendableCoinBob}$ (resp. $\funStar{\varPtSpendableCoinCarol}$).
        Note that the newly generated output coin can only be spent by both parties cooperating, as each share of the blinding factor is strictly required.
        We note here that creating more complex schemes in which a coin is spendable by knowing N out M keys would be possible by implementing Shamir's Secret Sharing algorithm, which can be found in~\cite{shamir1979share}.
        \item $\langle \varTx, \varTx \rangle \opFunResult \procDFinTx{\varPreTx}{\varSecKeyAlice}{\varNonceAlice}{\varSecKeyCarol}{\varNonceCarol}$: The distributed finalized transaction protocol has to be used to create a transaction spending a shared coin (i.e., the transaction was created with the $\procDSendCoinsId$ algorithm).
        In this case, we require signing information from both Alice and Carol.
    \end{itemize}
\end{definition}

Correctness is given very similarly to the standard scheme:

\begin{definition}[Extended Transaction Scheme Correctness]
    \label{def:atom:ext-tx-scheme-correctness}
    For any list of spendable coins $\funArray{\varSpendableCoin}$ with total value $\varValue$ greater than the transaction fund value $\varFundValue$ and split blinding factors $(\funArray{\varBlindingFactorAlice}, \funArray{\varBlindingFactorCarol})$, the following must hold:
    \[
        \Pr\left[
        \begin{array}{c}
            \: \procVerfTx{\varTx} \opEqNoQ 1
        \end{array}
        \middle\vert
        \begin{array}{l}
            \varFundValue \opSmEq \sum_{\varI \opAssign 0}^{\varI \opSm \varN}(\varSpendableCoin_{i}.\varValue) \\
            \langle (\varPreTx, \cdot, (\varSecKeyAlice, \varNonceAlice)), (\varPreTx, (\varSecKeyCarol, \varNonceCarol)) \rangle \opFunResult \\
            \procDSendCoins{\funArray{\varSpendableCoinAlice}}{\funArray{\varSpendableCoinCarol}}{\varFundValue}{\cnstFalsum} \\
            \langle (\funStar{\varPreTx}, \cdot)(\funStar{\varPreTx}, \cdot) \rangle \opFunResult \procDRecvCoins{\varPreTx}{\varFundValue} \\
            \varTx \opFunResult \procDFinTx{\funStar{\varPreTx}}{\varSecKeyAlice}{\varNonceAlice}{\varSecKeyCarol}{\varNonceCarol}
        \end{array}
        \right]=1.
    \]
\end{definition}

We define the \emph{Contract Mimblewimble Transaction Scheme}, which will extend the scheme with additional algorithms to create primitive contracts between the sending and receiving party.

\begin{definition}[Contract Mimblewimble Transaction Scheme]
    \label{def:atom:apt-ext-mw-tx-scheme}
    The contract version of the Mimblewimble Transaction Scheme updates the Extended Mimblewimble Transaction Scheme by providing a modified version of the single party receive routine and the distributed finalize transaction protocol.
    \begin{gather*}
        \varaptMWSchemeParams{\varCommitScheme}{\varSigSchemeMP}{\varMPRProofSystem} \opAssign \\ \varextMWSchemeParams{\varCommitScheme}{\varSigSchemeMP}{\varMPRProofSystem} \opConc (\procAptRecvCoinsId, \procDAptFinTxId)
    \end{gather*}
    \begin{itemize}
        \item $(\funStar{\varPreTx}, \funStar{\varSpendableCoinBob}, \varSigBob) \opFunResult \procAptRecvCoins{\varPreTx}{\varFundValue}{\varWit}$: The contract variant of the receive function takes an additional input, a secret witness value
        $\varWit$, hidden in the transaction signature and extracted by the other party after the completion of the protocol.
        Note that the routine also returns Bob's unmasked partial signature.
        The reason for this is that we later need the unmasked version to complete the signature and finalze the transaction.
        By not sharing this unmasked signature with Alice, Bob is the one who gets to finalize the transaction, which is different from the simpler protocol and is a crucial feature necessary for our Atomic Swap protocol.
        We want to stress here that $\procAptRecvCoinsId$ is only a single-party algorithm.
        We can only use it if we're going to create an output coin owned by a single receiver.
        It would, of course, be conceivable also to define a distributed version similar to $\procDRecvCoinsId$ of this functionality, allowing two receivers (or one of the two) to hide secret witness values, extractable later by the sender(s).
        However, as for the following protocols, such functionality is not needed, we omit it here.
        \item $\langle \varSigAliceBob, \varTx \rangle \opFunResult \procDAptFinTx{\funStar{\varPreTx}}{\varSecKeyAlice}{\varNonceAlice}{\varStatement}{\varSecKeyBob}{\varNonceBob}{\varSigBob}$: The finalize transaction algorithm's contract variant is a distributed protocol between the sender(s) and receiver.
        Additionally to the pre-transaction $\funStar{\varPreTx}$, the senders need to input their signing information.
        Bob needs to input the unmasked version of his partial signature as it is required for transaction completion.
        This protocol could also be implemented as a three-party protocol with two senders controlling a shared coin and a third receiver.
        However, in our case, which we will describe later in~\cref{sec:atom:protocols}, one of the two senders is also the receiver.
        We allowed ourselves to model this protocol as being between only two parties to simplify the formalization.
        In this version of the protocol, only Bob can finalize the transaction, which is different from $\procFinTxId$ and $\procDFinTxId$.
        The reason for that is that for the Atomic Swap execution, Bob needs to be the one in control of building the final transaction.
        If Alice were to build the final transaction before Bob, she will extract the witness value before the transaction has been published, which in the Atomic Swap scenario would mean she could steal the funds stored on the other chain.
        That is why the protocol does not return the final transaction $\varTx$ to Alice.
        Instead, the protocol will output the sender's partial signature, which Alice can later use to extract the final transaction's witness value.
    \end{itemize}
\end{definition}

Similar to before, we define Correctness for the adapted scheme:

\begin{definition}[Contract Transaction Scheme Correctness]
    \label{def:atom:apt-tx-scheme-correctness}
    For any transaction fund value $\varFundValue$ and list of input coins $\funArray{\varSpendableCoin}$ with combined value $\varValue \opGreaterEq \varFundValue$ and any witness value $\varX \opIn \cnstIntegersPrimeWithoutZero{\varPrime}$, the following must hold:
    \[
        \Pr\left[
        \begin{array}{c}
            \: \procVerfTx{\varTx} \opEqNoQ 1
        \end{array}
        \middle\vert
        \begin{array}{l}
            \varFundValue \opSmEq \sum_{\varI \opAssign 0}^{\varI \opSm \varN}(\varSpendableCoin_{i}.\varValue) \\
            (\varPreTx, \funStar{\varSpendableCoinAlice}, (\varSecKeyAlice, \varNonceAlice)) \opFunResult \procSendCoins{\funArray{\varSpendableCoin}}{\varFundValue}{\cnstFalsum} \\
            (\funStar{\varPreTx}, \funStar{\varSpendableCoinBob}, \varSigBob) \opFunResult \procAptRecvCoins{\varPreTx}{\varFundValue}{\varWit} \\
            \langle \varSigAliceCarol, \varTx \rangle \opFunResult \procDAptFinTx{\funStar{\varPreTx}}{\varSecKeyAlice}{\varNonceAlice}{\varStatement}{\varSecKeyCarol}{\varNonceCarol}{\varSigBob}
        \end{array}
        \right]=1.
    \]
\end{definition}


\section{Schnorr-based Instantiation} \label{sec:sig:schnorr-inst}
\urldef\urlbiptaproot\url{https://en.bitcoin.it/wiki/BIP_0341}

We start by providing a general instantiation of a signature scheme (see~\cref{def:pre:signature-scheme}):
We assume we have a group $\cnstGroup$ with prime $\varPrime$ and generator $\varG$, $\cnstHash$ is a secure hash function as defined in~\cref{def:pre:hash-function} and $\varMsg \opIn \cnstBinary{*}$ is a message.


A concrete implementation can be seen in~\cref{fig:schnorr}.
The signature scheme is called Schnorr signature scheme, first defined in~\cite{schnorr1989efficient} and valued for its simplicity and extensively analyized security.
Due to being originally patented its practical use was limited, however since the patent experied in 2008 it sees increasing interest.
Cryptocurrencies such as Grin and Beam now use Schnorr as its primary signature scheme, also Bitcoin is planning to add Schnorr signatures as an alternative to the currently used ECDSA signatures. \footnote{\urlbiptaproot}
\begin{figure}
    \begin{center}
        \fbox{
        \begin{varwidth}{\textwidth}
            \procedure[linenumbering]{$\procSetup{\varSecParam}$} {
            \varKey \sample \cnstIntegersPrimeWithoutZero{\varPrime} \\
            \pcreturn (\varSecKey \opAssign \varKey \opSeperate \varPubKey \opAssign \funGen{\varKey})
            }
            \procedure[linenumbering]{$\procSign{\varMsg}{\varSecKey}$}{
            \varNonce \sample \cnstIntegersPrimeWithoutZero{\varPrime} \\
            \varRand \opAssign \funGen{\varNonce} \\
            \varSchnorrChallenge \opAssign \funHash{\varMsg \opConc \varRand \opConc \varPubKey} \\
            \varS \opAssign \varNonce \opAddScalar \varSchnorrChallenge \opTimesScalar \varSecKey \\
            \pcreturn \varSignature \opAssign (\varS, \varRand)
            }
            \procedure[linenumbering]{$\procVerf{\varMsg}{\varSignature}{\varPubKey}$} {
            (\varS \opSeperate \varRand) \opFunResult \varSignature \\
            \varSchnorrChallenge \opAssign \funHash{\varMsg \opConc \varRand \opConc \varPubKey} \\
            \pcreturn \funGen{\varS} \opEq \varRand \opAddPoint \opPointScalar{\varPubKey}{\varSchnorrChallenge}
            }
        \end{varwidth}
        }
    \end{center}
    \caption{Schnorr Signature Scheme as first defined in~\cite{schnorr1989efficient}}
    \label{fig:schnorr}
\end{figure}
Correctness of the scheme is easy to derive. 
\todo[inline]{Introduce here what you are about to show. For instance, as shown in Figure X, verf, step 3, we need to show that....}
As $\varS$ is calculated as $\varNonce \opAddScalar \varSchnorrChallenge \opTimesScalar \varSecKey$, when generator $\varG$ is raised to $\varS$, we get
$\funGen{\varNonce \opAddScalar \varSchnorrChallenge \opTimesScalar \varSecKey}$ which we can transform into $\funGen{\varNonce} \opAddPoint \funGen{\varSecKey \opTimesScalar \varSchnorrChallenge}$, and finally
into $\varRand \opAddPoint \opPointScalar{\varPubKey}{\varSchnorrChallenge}$ which is the same as the right side of the equation.

From the regular Schnorr signature we now provide an instantiation for the two-party case defined in~\cref{def:sig:two-party-sig}. Note that this two-party variant of the scheme is what is
currently implemented in the mimblewimble-based cryptocurrencies and will provide a basis from which we will build our adapted scheme.\todo[inline]{What is adapted scheme here? you mean the scheme for 2-party adaptor signatures?}

First we define a auxiliary function $\procSetupCtxId$ to use for the instantion:

\begin{center}
    \fbox{
    \begin{varwidth}{\textwidth}
        \procedure[linenumbering]{$\procSetupCtx{\varSigContext}{\varPubKeyAlice}{\varRandAlice}$} {
        \opAccess{\varSigContext}{\varPubKey} \opAssign \opAccess{\varSigContext}{\varPubKey} \opAddPoint \varPubKeyAlice \\
        \opAccess{\varSigContext}{\varRand} \opAssign \opAccess{\varSigContext}{\varRand} \opAddPoint \varRandAlice \\
        \pcreturn \varSigContext
        } \\
    \end{varwidth}
    }
\end{center}

This function helps the participants to setup and update the signature context shared between them. In~\cref{fig:twoparty-schnorr} we show a concrete instantiation of the protocol and functions. In $\procKeyGenPtId$ Alice
and Bob will each randomly chose their secret key and nonce. They further require to create a zero-knowledge proof attesting to the fact that they have generated their key before any message was exchanged. This
is essential for the scheme to achieve EUF-CMA as described by Lindell et al. ~\cite{lindell2017fast}.

In $\procKeyGenPtId$ Alice will initially setup the signature context and send it to Bob, together with her public and zk-proof.
Bob verifies the proof (and exits if it is invalid).
He will proceed by adding his parameters to the signature context and send it back to Alice, together with his public key and zk-proof, which Alice will verify.

\todo[inline]{In your $\procKeyGenPtId$, you generate both the combined public key and the combined randomness. Imagine a scenario where Alice and Bob want to have a shared public key and re-use it to sign multiple messages (and for each, they would need to generate new fresh shared randomness). How would that be handled? Would you need a modified version of $\procKeyGenPtId$ where only randomness is created? I am mentioning this because this could be a question that you might get. You might want to add a note here about that so that you don't forget}

$\procSignPrtId$ and $\procVerfPtSigId$ are generally similiar to the instantiation of the normal Schnorr signature scheme. Note however that for computing the Schnorr challenge $\varSchnorrChallenge$ the input into
the hash function will be the combined public key $\varPubKey$ and combined nonce commitment $\varRand$, which the participants can read from the context object $\varSigContext$. This has the effect that the partial
signature itself are not yet a valid signature (neither under $\varPubKey$ nor under $\varPubKeyAlice$ or $\varPubKeyBob$). This is because to be valid under $\varPubKey$ the partial signatures are missing the $\varS$ values
from the other participants. They are also not valid under the partial public keys $\varPubKeyAlice$ or $\varPubKeyBob$ because the Schnorr challenge is computed already with the combined values. There we have
to introduce the slightly adjusted $\procVerfPtSigId$ to be able to verify specifically the partial signatures.

\begin{figure}
    \begin{center}
        \fbox{
        \begin{varwidth}{\textwidth}
            \procedure[linenumbering,skipfirstln]{$\procKeyGenPt{\varSecParam}{\varSecParam}$} {
            \textbf{Alice} \< \< \textbf{Bob} \\
            \varSecKeyAlice \sample \cnstIntegersPrimeWithoutZero{\varPrime} \< \< \varSecKeyBob \sample \cnstIntegersPrimeWithoutZero{\varPrime} \\
            \varNonceAlice \sample \cnstIntegersPrimeWithoutZero{\varPrime} \< \< \varNonceBob \sample \cnstIntegersPrimeWithoutZero{\varPrime} \\
            \varPubKeyAlice \opAssign \funGen{\varSecKeyAlice} \< \< \varPubKeyBob \opAssign \funGen{\varSecKeyBob} \\
            \varRandAlice \opAssign \funGen{\varNonceAlice} \< \< \varRandBob \opAssign \funGen{\varNonceBob} \\
            \varZkpStatementAlice \opAssign \exists \varSecKeyAlice \textit{ s.t. } \funGen{\varSecKeyAlice} \opEqNoQ \varPubKeyAlice \< \< \varZkpStatementBob \opAssign \exists \varSecKeyBob \textit{ s.t. } \funGen{\varSecKeyBob} \opEqNoQ \varPubKeyBob \\
            \varProofAlice \opFunResult \procZkpProve{\varSecKeyAlice}{\varZkpStatementAlice} \< \< \varProofBob \opFunResult \procZkpProve{\varSecKeyBob}{\varZkpStatementBob} \\
            \varSigContext \opAssign \langle \varPubKey \opAssign \cnstIdentityElement \opSeperate \varRand \opAssign \cnstIdentityElement \rangle \< \< \\
            \varSigContext \opFunResult \procSetupCtx{\varSigContext}{\varPubKeyAlice}{\varRandAlice} \< \< \\
            \< \sendmessageright*{\varSigContext, \varPubKeyAlice, \varProofAlice} \< \\
            \< \< \pcif \procZkpVerify{\varProofAlice} \opEqNoQ 0 \\
            \< \< \t \pcreturn \cnstFalsum \\
            \< \< \varSigContext \opFunResult \procSetupCtx{\varSigContext}{\varPubKeyBob}{\varRandBob} \\
            \< \sendmessageleft*{\varSigContext, \varPubKeyBob, \varProofBob} \< \\
            \pcif \procZkpVerify{\varProofBob} \opEqNoQ 0 \< \< \\
            \t \pcreturn \cnstFalsum \< \< \\
            \pcreturn (\varSecKeyAlice,\varPubKeyAlice,\varNonceAlice,\varSigContext) \< \< \pcreturn (\varSecKeyBob,\varPubKeyBob,\varNonceBob,\varSigContext)
            } \\
            \procedure[linenumbering]{$\procSignPrt{\varMsg}{\varSecKeyAlice}{\varNonceAlice}{\varSigContext}$} {
            (\varRand \opSeperate \varPubKey) \opFunResult \varSigContext \\
            \varRandAlice \opAssign \funGen{\varNonceAlice} \\
            \varSchnorrChallenge \opAssign \funHash{\varMsg \opConc \varRand \opConc \varPubKey} \\
            \varSAlice \opAssign \varNonceAlice \opAddScalar \varSecKeyAlice \opTimesScalar \varSchnorrChallenge \\
            \pcreturn \varSigAlice \opAssign (\varSAlice, \varRandAlice, \varSigContext)
            }
            \procedure[linenumbering]{$\procVerfPtSig{\varSigAlice}{\varMsg}{\varPubKeyAlice}$} {
            (\varSAlice \opSeperate \varRandAlice \opSeperate \varSigContext) \opFunResult \varSigAlice \\
            (\varPubKey \opSeperate \varRand) \opFunResult \varSigContext \\
            \varSchnorrChallenge \opAssign \funHash{\varMsg \opConc \varRand \opConc \varPubKey} \\
            \pcreturn \funGen{\varSAlice} \opEq \varRandAlice \opAddPoint \opPointScalar{\varPubKeyAlice}{\varSchnorrChallenge}
            }
            \procedure[linenumbering]{$\procFinSig{\varSigAlice}{\varSigBob}$} {
            (\varSAlice \opSeperate \varRandAlice \opSeperate \varSigContext) \opFunResult \varSigAlice \\
            (\varSBob \opSeperate \varRandBob \opSeperate \varSigContext) \opFunResult \varSigBob \\
            (\varPubKey \opSeperate \varRand) \opFunResult \varSigContext \\
            \varS \opAssign \varSAlice \opAddScalar \varSBob \\
            \varSigFin \opAssign (\varS, \varRand) \\
            \pcreturn \varSigFin
            }
        \end{varwidth}
        }
    \end{center}
    \caption{Two Party Schnorr Signature Scheme}
    \label{fig:twoparty-schnorr}
\end{figure}

We further formalize a protocol $\procDSignId$ which is a protocol between two parties running the partial signature creation outlined before.
Note that we assume that the secret keys as well as nonces used in the signatures have already been generated, for example by running the $\procKeyGenPtId$ protocol.
Both parties input the shared message $\varM$ as well as their secret keys and secret nonces.
The protocol outputs a signature $\varSigFin$ to the message $\varSigFin$\todo{typo?}, valid under the composite public key $\varPubKey \opEqNoQ \varPubKeyAlice \opAddPoint \varPubKeyBob$.
Additionally to the final signature the protocol also outputs the composite public key $\varPubKey$.

\todo[inline]{If they run $\procKeyGenPtId$, don't they have already the nonces in the context? And if so, are you adding them again here in $\procDSignId$?}

\begin{center}
    \fbox{
    \begin{varwidth}{\textwidth}
        \procedure[linenumbering,skipfirstln]{$\procDSign{\varMsg}{\varSecKeyAlice}{\varNonceAlice}{\varSecKeyBob}{\varNonceBob}$} {
        Alice \< \< Bob \\
        \varSigContext \opAssign \{ \varPubKey \opAssign \cnstIdentityElement, \varRand \opAssign \cnstIdentityElement \} \< \< \\
        \varSigContext \opFunResult \procSetupCtx{\varSigContext}{\funGen{\varSecKeyAlice}}{\funGen{\varNonceAlice}} \< \< \\
        \< \sendmessageright*{\varSigContext, \varPubKeyAlice \opAssign \funGen{\varSecKeyAlice}} \< \\
        \< \< \varSigContext \opFunResult \procSetupCtx{\varSigContext}{\funGen{\varSecKeyBob}}{\funGen{\varNonceBob}} \< \< \\
        \< \< \varSigBob \opFunResult \procSignPrt{\varMsg}{\varSecKeyBob}{\varNonceBob}{\varSigContext} \\
        \< \sendmessageleft*{\varSigBob, \varSigContext, \varPubKeyBob \opAssign \funGen{\varSecKeyBob}} \< \\
        \pcif \procVerfPtSig{\varSigBob}{\varMsg}{\varPubKeyBob} \opEqNoQ 0 \< \< \\
        \t \pcreturn \cnstFalsum \< \< \\
        \varSigAlice \opFunResult \procSignPrt{\varMsg}{\varSecKeyAlice}{\varNonceAlice}{\varSigContext} \< \< \\
        \< \sendmessageright*{\varSigAlice} \< \\
        \< \< \pcif \procVerfPtSig{\varSigAlice}{\varMsg}{\varPubKeyAlice} \opEqNoQ 0 \\
        \< \< \t \pcreturn \cnstFalsum \\
        \varSigFin \opFunResult \procFinSig{\varSigAlice}{\varSigBob} \< \< \varSigFin \opFunResult \procFinSig{\varSigAlice}{\varSigBob} \\
        \varPubKey \opFunResult \varSigContext.\varPubKey \< \< \varPubKey \varSigContext.\varPubKey \\
        \pcreturn (\varSigFin, \varPubKey) \< \< \pcreturn (\varSigFin, \varPubKey)
        }
    \end{varwidth}
    }
\end{center}

The final signature is a valid signature to the message $\varMsg$ with the composite public key $\varPubKey \opAssign \varPubKeyAlice \opAddPoint \varPubKeyBob$.
A verifier knowing the signed message $\varMsg$, the final signature $\varSigFin$ and the composite public key $\varPubKey$ can now verify the signature using the regular $\procVerfId$ procedure.

Note that this way of computing Schnorr signatures is not new.
For a proof of its correctness and a more extensive explanation we refer the reader to a paper by Maxwell et al.~\cite{maxwell2019simple}.

In~\cref{fig:aptSchnorr} we further provide a Schnorr-based instantiation for the fixed witness adapted signature scheme as defined in~\cref{def:sig:two-party-fixed-wit-apt-sig}.

$\procAptSigId$ will add the secret witness $\varWit$ to the $\varS$ value of the signature, this means we will not be able to verify the adapted signature using $\procVerfPtSigId$ anymore. Therefore we
introduce $\procVerifyAptSigId$ which takes as additional parameter the statement $\varStatement$ which will be included in the verifiers equation. Now the function verifies not only validity of the partial
signature, but also that it indeed has been adapted with the witness value $\varWit$, being the discrete logarithm of $\varStatement$. After obtaining $\varSigFin$, we can then cleverly unpack the secret $\varWit$,
which is shown in the $\procExpExtId$\todo{typo? The name of the function here is not the same as in the figure} function.

\begin{figure}
    \begin{center}
        \fbox{
        \begin{varwidth}{\textwidth}
            \procedure[linenumbering]{$\procAptSig{\varSigPt}{\varWit}$}{
            (\varS \opSeperate \varRandAlice \opSeperate \varSigContext) \opFunResult \varSigPt \\
            \varSStar \opAssign \varS \opAddScalar \varWit \\
            \pcreturn \varSigApt \opAssign (\varSStar \opSeperate \varRandAlice \opSeperate \varSigContext)
            } \\
            \procedure[linenumbering]{$\procVerifyAptSig{\varSigAptAlice}{\varMsg}{\varPubKeyAlice}{\varStatement}$} {
            (\varSAlice \opSeperate \varRandAlice \opSeperate \varSigContext) \opFunResult \varSigAptAlice \\
            (\varPubKey \opSeperate \varRand) \opFunResult \varSigContext \\
            \varSchnorrChallenge \opAssign \funHash{\varMsg \opConc \varRand \opConc \varPubKey} \\
            \pcreturn \funGen{\varSAlice} \opEq \varRandAlice \opAddPoint \opPointScalar{\varPubKeyAlice}{\varSchnorrChallenge} \opAddPoint \varStatement
            }
            \procedure[linenumbering]{$\procExtWit{\varSigFin}{\varSigAlice}{\varSigAptBob}$}{
            (\varS \opSeperate \varRand) \opFunResult \varSigFin \\
            (\varSAlice \opSeperate \varRandAlice \opSeperate \varSigContext) \opFunResult \varSigAlice \\
            (\varSAptBob \opSeperate \varRandBob \opSeperate \varSigContext) \opFunResult \varSigAptBob \\
            \varSBob \opAssign \varS \opSub \varSAlice \\
            \varWit \opAssign \varSAptBob \opSub \varSBob \\
            \pcreturn (\varWit)
            }
        \end{varwidth}
        }
    \end{center}
    \caption{Fixed Witness Adaptor Schnorr Signature Scheme}
    \label{fig:aptSchnorr}
\end{figure}

We now define a protocol $\procDAptSignId$ between Alice and Bob creating a signature $\varSigFin$ for the composite public key $\varPubKey \opAssign \varPubKeyAlice \opAddPoint \varPubKeyBob$
Now Bob will hide his secret $\varWit$ which Alice can extract after the signing process has completed.
One thing to note is that in this protocol only Bob is able to call $\procFinSigId$ to create the final signature, which is different to the previous protocol.
This is because the function requires Bob's unadapted partial signature $\varSigBob$ as input, which Alice does not know. (She only knows Bobs adapted variant).
Therefore, one further interaction is needed to send the final signature to Alice.
The protocol outputs $(\varWit, (\varSigFin, \varPubKey))$ for Alice as she manages to learn $\varWit$ and $(\varSigFin, \varPubKey)$ for Bob.

\begin{center}
    \fbox{
    \begin{varwidth}{\textwidth}
        \procedure[linenumbering,skipfirstln]{$\procDAptSign{\varMsg}{\varSecKeyAlice}{\varNonceAlice}{\varSecKeyBob}{\varNonceBob}{\varWit}$} {
        Alice \< \< Bob \\
        \varSigContext \opAssign \{ \varPubKey \opAssign \cnstIdentityElement, \varRand \opAssign \cnstIdentityElement \} \< \< \\
        \varSigContext \opFunResult \procSetupCtx{\varSigContext}{\funGen{\varSecKeyAlice}}{\funGen{\varNonceAlice}} \< \< \\
        \< \sendmessageright*{\varSigContext, \varPubKeyAlice \opAssign \funGen{\varSecKeyAlice}} \< \\
        \< \< \varSigContext \opFunResult \procSetupCtx{\varSigContext}{\funGen{\varSecKeyBob}}{\funGen{\varNonceBob}} \< \< \\
        \< \< \varSigBob \opFunResult \procSignPrt{\varMsg}{\varSecKeyBob}{\varNonceBob}{\varSigContext} \\
        \< \< \varSigAptBob \opFunResult \procAptSig{\varSigBob}{\varWit} \\
        \< \< \varPubKeyBob \opAssign \funGen{\varSecKeyBob} \\
        \< \< \varStatement \opAssign \funGen{\varWit} \\
        \< \sendmessageleft*{\varSigAptBob, \varSigContext, \varPubKeyBob, \varStatement} \< \\
        \pcif \procVerifyAptSig{\varSigBob}{\varMsg}{\varPubKeyBob}{\varStatement} \opEqNoQ 0 \< \< \\
        \t \pcreturn \cnstFalsum \< \< \\
        \varSigAlice \opFunResult \procSignPrt{\varMsg}{\varSecKeyAlice}{\varNonceAlice}{\varSigContext} \< \< \\
        \< \sendmessageright*{\varSigAlice} \< \\
        \< \< \pcif \procVerfPtSig{\varSigAlice}{\varMsg}{\varPubKeyAlice} \opEqNoQ 0 \\
        \< \< \t \pcreturn \cnstFalsum \\
        \< \< \varSigFin \opFunResult \procFinSig{\varSigAlice}{\varSigBob} \\
        \< \sendmessageleft*{\varSigFin} \< \\
        \varPubKey \opFunResult \varSigContext.\varPubKey \< \< \varPubKey \opFunResult \varSigContext.\varPubKey \\
        \pcif \procVerf{\varMsg}{\varSigFin}{\varPubKey} \opEqNoQ 0 \\
        \t \pcreturn \cnstFalsum \\
        \varWit \opFunResult \procFinSig{\varSigAlice}{\varSigAptBob} \\
        \pcreturn (\varWit, (\varSigFin, \varPubKey)) \< \< \pcreturn (\varSigFin, \varPubKey)
        }
    \end{varwidth}
    }
\end{center}

\section{Protocols} \label{sec:sig:protocols}
This section specifies three protocols to build Mimblewimble transactions from the definitions found in~\cref{sec:atom:definitions}.
Later in~\cref{sec:atom:security}, we will prove the security of these protocols, and finally, in ~\cref{sec:atom:atomic-swap}, we will utilize them to build our Atomic Swap.

\subsection{Simple Mimblewimble Transaction - $\procDBuildMwTxId$} \label{subsec:atom:simple-mw-tx}

$\procDBuildMwTxId$ is a protocol between a sender and receiver which builds a Mimblewimble transaction transferring a value $\varFundValue$ from the sender to a receiver for a Mimblewimble Transaction Scheme as defined in~\cref{def:atom:mw-tx-scheme}.
It takes as input a list of spendable coins $\funArray{\varSpendableCoin}$, a transaction value $\varFundValue$, and an optional timelock $\varTime$ from the sender, the same transaction value $\varFundValue$ from the receiver, and uses the functions defined earlier to output a valid transaction $\varTx$ as well as the newly spendable coins to both parties.
\[ \langle (\varTx, \funStar{\varSpendableCoinAlice}), (\varTx, \funStar{\varSpendableCoinBob}) \rangle \opFunResult \procDBuildMwTx{\funStar{\varSpendableCoin}}{\varFundValue}{\varTime} \]
\Cref{fig:d-build-mw-tx} shows the implementation of the $\procDBuildMwTxId$.

\begin{figure}
    \begin{center}
    \fbox{
    \begin{varwidth}{\textwidth}
        \procedure[linenumbering,skipfirstln]{$\procDBuildMwTx{\funArray{\varSpendableCoin}}{\varFundValue}{\varTime}$}{
        Alice \< \< Bob \\
        (\varPreTx, \funStar{\varSpendableCoinAlice}, (\varSecKeyAlice, \varNonceAlice)) \pcskipln \\
        \opFunResult \procSendCoins{\funArray{\varSpendableCoin}}{\varFundValue}{\varTime} \\
        \< \sendmessageright*{\varPreTx} \< \\
        \< \< (\funStarAlt{\varPreTx}, \funStar{\varSpendableCoinBob}) \opFunResult \procRecvCoins{\varPreTx}{\varFundValue} \\
        \< \sendmessageleft*{\funStarAlt{\varPreTx}} \\
        \varTx \opFunResult \procFinTx{\funStarAlt{\varPreTx}}{\varSecKeyAlice}{\varNonceAlice} \\
        \< \sendmessageright*{\varTx} \\
        \pcreturn (\varTx, \funStar{\varSpendableCoinAlice}) \< \< \pcreturn (\varTx, \funStar{\varSpendableCoinBob})
        }
    \end{varwidth}
    }
    \end{center}
    \caption{$\procDBuildMwTxId$ two-party protocol to build a new transaction} \label{fig:d-build-mw-tx}
\end{figure}

\subsection{Shared Output Mimblewimble Transaction - $\procDSharedOutputMwTxId$} \label{subsec:atom:shared-out-mw-tx}

$\procDSharedOutputMwTxId$ is a protocol between a sender and a receiver.
It builds a Mimblewimble transaction transferring value from a sender for the Extended Mimblewimble Transaction Scheme in ~\cref{def:atom:ext-mw-tx-scheme}.
However, instead of simply sending value to a receiver, it sends it to a shared coin, for which both the sender and receiver know one part of the opening.
As input, it again takes a list of spendable coins $\funArray{\varSpendableCoin}$, a transaction value $\varFundValue$ and an optional timelock $\varTime$ from the sender, and the same transaction value $\varFundValue$ from the receiver.
It outputs the final transaction $\varTx$ to both parties, Alice will receive her spendable change output $\funStar{\varSpendableCoinAlice}$ and both parties will receive their part of the shared spendable coin $\funStar{\varPtSpendableCoinAlice}$, $\funStar{\varPtSpendableCoinBob}$.

\[ \langle (\varTx, \funStar{\varSpendableCoinAlice}, \funStar{\varPtSpendableCoinAlice}), (\varTx, \funStar{\varPtSpendableCoinBob}) \rangle \opFunResult \procDSharedOutputMwTx{\funArray{\varSpendableCoin}}{\varFundValue}{\varTime} \]

One use case of this transaction protocol is to lock funds between two users, which can then be redeemed by both parties cooperating.

\Cref{fig:d-shared-out-mw-tx} shows the implementation of the protocol.

\begin{figure}
    \begin{center}
    \fbox{
    \begin{varwidth}{\textwidth}
        \procedure[linenumbering,skipfirstln]{$\procDSharedOutputMwTx{\funArray{\varSpendableCoin}}{\varFundValue}{\varTime}$}{
        Alice \< \< Bob \\
        (\varPreTx, \funStar{\varSpendableCoinAlice}, (\varSecKeyAlice, \varNonceAlice)) \pcskipln \\
        \opFunResult \procSendCoins{\funArray{\varSpendableCoin}}{\varFundValue}{\varTime} \\
        \< \sendmessageright*{\varPreTx} \< \\
        (\funStarAlt{\varPreTx}, \funStar{\varPtSpendableCoinAlice}) \< \< (\funStarAlt{\varPreTx}, \funStar{\varPtSpendableCoinBob}) \pcskipln \\
        \opFunResult \procDRecvCoinsL{\varPreTx}{\varFundValue}  \< \< \opFunResult \procDRecvCoinsR \\
        \varTx \opFunResult \procFinTx{\funStarAlt{\varPreTx}}{\varSecKeyAlice}{\varNonceAlice} \\
        \< \sendmessageright*{\varTx} \\
        \pcreturn (\varTx, \funStar{\varSpendableCoinAlice}, \funStar{\varPtSpendableCoinAlice}) \< \< \pcreturn (\varTx, \funStar{\varPtSpendableCoinBob})
        }
    \end{varwidth}
    }
    \end{center}
    \caption{$\procDSharedOutputMwTxId$ two-party protocol to build a new transaction with a shared output} \label{fig:d-shared-out-mw-tx}
\end{figure}

\subsection{Shared Input Mimblewimble Transaction $\procDSharedInpMwTxId$} \label{subsec:atom:shared-inp-mw-tx}

$\procDSharedInpMwTxId$ is a protocol between a sender and a receiver.
It builds a Mimblewimble transaction transferring value from a coin shared between the sender and receiver to a receiver again for the Extended Mimblewimble Transaction Scheme outlined in~\cref{def:atom:ext-mw-tx-scheme}
As input, it takes a list of partial spendable coins $\funArray{\varPtSpendableCoinAlice}$, a transaction value $\varFundValue$, an optional timelock $\varTime$ from the sender, and the other part of the shared spendable coins $\varPtSpendableCoinBob$ and the same transaction value $\varFundValue$ from the receiver.
It outputs a final transaction $\varTx$ to both parties and the new outputs $\funStar{\varSpendableCoinAlice}, \funStar{\varSpendableCoinBob}$ to the respective owner.

\[ \langle (\varTx, \funStar{\varSpendableCoinAlice}), (\varTx, \funStar{\varSpendableCoinBob}) \rangle \opFunResult \procDSharedInpMwTx{\funArray{\varPtSpendableCoinAlice}}{\varFundValue}{\varTime}{\funArray{\varPtSpendableCoinBob}} \]

The protocol can be used to redeem funds that are locked created with the \\ $\procDSharedInpMwTxId$ protocol.

\Cref{fig:d-shared-inp-mw-tx} shows the implementation of the protocol.

\begin{figure}
    \begin{center}
    \fbox{
    \begin{varwidth}{\textwidth}
        \procedure[linenumbering,skipfirstln]{$\procDSharedInpMwTx{\funArray{\varPtSpendableCoinAlice}}{\varFundValue}{\varTime}{\funArray{\varPtSpendableCoinBob}}$}{
        Alice \< \< Bob \\
        (\varPreTx, \funStar{\varSpendableCoinAlice}, (\varSecKeyAlice, \varNonceAlice)) \< \< (\varPreTx, (\varSecKeyBob, \varNonceBob)) \pcskipln \\
        \opFunResult \procDSendCoinsL{\funArray{\varPtSpendableCoinAlice}}{\varFundValue}{\varTime} \< \< \opFunResult \procDSendCoinsL{\funArray{\varPtSpendableCoinBob}}{\varFundValue}{\varTime} \\
        \< \< (\funStarAlt{\varPreTx}, \funStar{\varSpendableCoinBob}) \opFunResult \procRecvCoins{\varPreTx}{\varFundValue} \\
        \< \sendmessageleft*[2cm]{\funStarAlt{\varPreTx}} \\
        \varTx \opFunResult \procDFinTxL{\funStarAlt{\varPreTx}}{\varSecKeyAlice}{\varNonceAlice} \< \< \varTx \opFunResult \procDFinTxL{\funStarAlt{\varPreTx}}{\varSecKeyBob}{\varNonceBob} \\
        \pcreturn (\varTx, \funStar{\varSpendableCoinAlice}) \< \< \pcreturn (\varTx, \funStar{\varSpendableCoinBob})
        }
    \end{varwidth}
    }
    \end{center}
    \caption{$\procDSharedOutputMwTxId$ two-party protocol to build a new transaction from a shared output} \label{fig:d-shared-inp-mw-tx}
\end{figure}

\subsection{Contract Mimblewimble Transaction - $\procDScriptMwTxId$} \label{subsec:atom:script-mw-tx}

$\procDScriptMwTxId$ is a protocol between a sender and a receiver for the Contract Mimblewimble Transaction Scheme defined in~\cref{def:atom:apt-ext-mw-tx-scheme}.
Similar to the $\procDSharedInpMwTxId$ it spends an input coin which is shared between the sender and receiver.
Additionally, we utilize the adapted signature protocol from~\cref{def:sig:two-party-fixed-wit-apt-sig} to let the receiver hide a secret witness value $\varWit$ in the transaction signature, which the sender can extract from the final transaction, thereby allowing primitive contracts.

\[ \langle (\varTx, \funStar{\varSpendableCoinAlice}, \varWit), (\varTx, \funStar{\varSpendableCoinBob}) \rangle \opFunResult \procDScriptMwTx{\funArray{\varPtSpendableCoinAlice}}{\varFundValue}{\varTime}{\varStatement}{\funArray{\varPtSpendableCoinBob}}{\varWit} \]

\Cref{fig:d-script-tx} shows the implementation of the protocol.

\begin{figure}
    \begin{center}
    \fbox{
    \begin{varwidth}{\textwidth}
        \procedure[linenumbering,skipfirstln]{$\procDScriptMwTx{\funArray{\varPtSpendableCoinAlice}}{\varFundValue}{\varTime}{\varStatement}{\funArray{\varPtSpendableCoinBob}}{\varWit}$}{
        Alice \< \< Bob \\
        (\varPreTx, \funStar{\varSpendableCoinAlice}, (\varSecKeyAlice, \varNonceAlice)) \< \< (\varPreTx, (\varSecKeyBob, \varNonceBob)) \pcskipln \\
        \opFunResult \procDSendCoinsL{\funArray{\varPtSpendableCoinAlice}}{\varFundValue}{\varTime} \< \< \opFunResult \procDSendCoinsL{\funArray{\varPtSpendableCoinBob}}{\varFundValue}{\varTime} \\
        \< \< (\funStarAlt{\varPreTx}, \funStar{\varSpendableCoinBob}, \varSigBob)  \pcskipln \\
        \< \< \opFunResult \procAptRecvCoins{\varPreTx}{\varFundValue}{\varWit}  \\
        \< \sendmessageleft*[2cm]{\funStarAlt{\varPreTx}, \funStarAlt{\varStatement}} \\
        \pcif \varStatement \opNotEq \cnstFalsum \opAnd \varStatement \opNotEq \funStarAlt{\varStatement} \\
        \t \pcreturn \cnstFalsum \\
        \varSigAptBob \opFunResult \funStarAlt{\varPreTx}.\varSignature \\
        \varSigAliceBob \< \< \varTx \pcskipln \\
        \opFunResult \procDAptFinTxL{\funStarAlt{\varPreTx}}{\varSecKeyAlice}{\varNonceAlice}{\varStatement} \< \< \opFunResult \procDAptFinTxR{\funStarAlt{\varPreTx}}{\varSecKeyBob}{\varNonceBob}{\varSigBob} \\
        \< \sendmessageleft*[2cm]{\varTx} \\
        \varWit \opFunResult \procExtWit{\varTx.\varSignature}{\varSigAliceBob}{\varSigAptBob} \\
        \pcreturn (\varTx, \funStar{\varSpendableCoinAlice}, \varWit) \< \< \pcreturn (\varTx, \funStar{\varSpendableCoinBob})
        }
    \end{varwidth}
    }
    \end{center}
    \caption{$\procDScriptMwTxId$ two-party protocol to build a primitive contract transaction} \label{fig:d-script-tx}
\end{figure}

\paragraph{A note on rogue-key attacks:} In~\cref{sec:sig:definitions}, we mentioned that we need to take special care in the key generation phase in a Two-Party Signature Scheme.
Otherwise the protocol might be vulnerable against rogue-key attacks in which one of the party's public keys is computed as a function of the other.
We see that we do not take this into account in all of the protocols laid out in this section.
As for the receiving party, it will always be possible to generate his keypair as a function of the sender's public key.
We now show how attempting a rogue-key attack in Mimblewimble would play out and why it would not threaten the security of our scheme:\\
Imagine we have an attacker $\cnstAdversary$ who knows the value $\varValue$ of some coin $\varCoin \opEqNoQ \funGen{\varBlindingFactor} \opAddPoint \funGenH{\varValue}$ present in the unspent output list of the blockchain.
He could then compute $\varPubKeyAlice \opEqNoQ \varCoin \opAddPoint {(\funGenH{\varValue}})^{-1}$.
For the rogue-key attack to succeed, $\cnstAdversary$ would now create a transaction spending $\varCoin$ and choose his output coin pubkey as $\varPubKeyBob \opEqNoQ \varPubKeyAlice^{-1}$ with the attempt of canceling out Alice's key.
However, recalling the structure of Mimblewimble transactions the participants sign the excess value $\varExcess \opEqNoQ \varInputs \opSub \varOutputs$, where $\varInputs$ and $\varOutputs$ is the input and output coins list.
Therefore, making the public keys cancel out $\cnstAdversary$ would instead have to choose his key as $\varPubKeyBob \opAssign \varPubKeyAlice$.
Given this setup (a transaction which spends the coin $\varCoin \opEqNoQ \varPubKeyAlice \opAddPoint \funGenH{\varValue}$ to $\funStar{\varCoin} \opEqNoQ \varPubKeyBob \opAddPoint\funGenH{\varValue}$), the excess value $\varExcess$ would calculate like $\varPubKeyAlice \opAddPoint {\varPubKeyBob}^{-1}$.
$\varPubKeyBob$ definition is $\varPubKeyAlice \opAddPoint {\varPubKeyAlice}^{-1}$, which would cancel out and allow the adversary to forge a signature.
However, since we chose $\varPubKeyBob$ as simply $\varPubKeyAlice$ and $\varPubKeyAlice \opEqNoQ \funGen{\varBlindingFactor}$ (from the original Pedersen Commitment) the new output coin $\funStar{\varCoin}$ would be identical to the input coin $\varCoin$, and the transaction spend a coin to itself.
Recalling the instantiation of the transaction verification algorithm $\procVerfTxId$ defined by Fuchsbauer et al.~\cite{fuchsbauer2019aggregate}, which we laid out in~\cref{fig:inst-mw-tx-2}, we see that the union between input and output coin list must be empty.
Otherwise, the transaction will not verify.
Therefore, even though the attacker could create a forged signature for this transaction, it would still be invalid as by definition of the transaction verification algorithm.
We further consider the case in which the attacker would try to add a fee $\varFee$ to the transaction to steal value from a coin.
In this case, the newly created output coin would be $\varPubKeyBob \opAddPoint \funGenH{\varValue - \varFee}$.
Now the output coin is no longer identical to the input coin, yet the input and output values still cancel out due to the fee, and by the definition of $\varPubKeyBob$ the two public keys must as well still cancel out, allowing for a forged signature.
However, in this scenario, $\cnstAdversary$ is faced with the problem that he does not have a valid range proof for this new output coin.
To compute such a proof, he would need to know the original $\varBlindingFactor$ of $\varPubKeyAlice \opEqNoQ \funGen{\varBlindingFactor}$, which he doesn't.
Therefore it is again impossible for him the create a valid transaction, even though he would be able to forge the transaction signature.
We conclude that all possible rogue-key attacks on Mimblewimble are prevented through transaction verification, and we, therefore, do not have to take other special care to avoid them.





\section{Correctness \& Security}\label{sec:sig:two-party-apt-security}
We now prove that the outlined Schnorr-based instantiation is correct, i.e. Adaptor Signature Correctness holds, and is secure with regards to the~\cref{subsec:pre:security} .

\subsection{Adaptor Signature Correctness}\label{subsec:sig:aptsig-correctness}

To prove that Adaptor Signature Correctness holds we have 3 statements to prove as given by \cref{def:sig:apt-sig-correctness}, first we prove that $\procVerf{\varMsg}{\varSigFin}{\varSigContext.\varPubKey} \opEqNoQ 1$ holds in our Schnorr-based instantiation of the signature scheme, where $\varSigContext$ is setup such that $\varPubKey \opEqNoQ \varPubKeyAlice \opAddPoint \varPubKeyBob$.

\begin{proof}
    \label{prf:apt-schnorr-pre-sig-corr}
    For this proof we assume the setup already specified in~\cref{def:sig:apt-sig-correctness} .
    The proof is by showing equality of the equation checked by the verifier of the final signature by continuous substitutions in the left side of equation:
    \begin{align}
        \funGen{\varS} &\opEqNoQ \varRand \opAddPoint \opPointScalar{\varPubKey}{\varSchnorrChallenge} \\
        \funGen{\varSAlice} \opAddPoint \funGen{\varSBob} & \\
        \funGen{\varNonceAlice \opAddScalar \varSchnorrChallenge \opTimesScalar \varSecKeyAlice} \opAddPoint \funGen{\varNonceBob \opAddScalar \varSchnorrChallenge \opTimesScalar \varSecKeyBob} & \\
        \funGen{\varNonceAlice} \opAddPoint \opPointScalar{\varPubKeyAlice}{\varSchnorrChallenge} \opAddPoint \funGen{\varNonceBob} \opAddPoint \opPointScalar{\varPubKeyBob}{\varSchnorrChallenge} & \\
        \varRandAlice \opAddPoint \opPointScalar{\varPubKeyAlice}{\varSchnorrChallenge} \opAddPoint \varRandBob \opAddPoint \opPointScalar{\varPubKeyBob}{\varSchnorrChallenge} & \\
        \varRand \opAddPoint \opPointScalar{\varPubKey}{\varSchnorrChallenge} & \opEqNoQ \varRand \opAddPoint \opPointScalar{\varPubKey}{\varSchnorrChallenge} \\
        1 & \opEqNoQ 1
    \end{align}

    It remains to prove that with the same setup $\procVerifyAptSig{\varSigAptBob}{\varMsg}{\varPubKeyBob}{\varStatement} \opEqNoQ 1$ and
    $(\varStatement \opSeperate \varWit) \opIn \cnstRelation$ for $\varWit \opFunResult \procExtWit{\varSigFin}{\varSigAlice}{\varSigAptBob}$:

    \[
        \procVerifyAptSig{\varSigAptBob}{\varMsg}{\varPubKeyBob}{\varStatement} \opEqNoQ 1
    \]
    The proof is by continuous substitutions in the equation checked by the verifier:
    \begin{align}
        \funGen{\varSigAptBob} &\opEqNoQ \varRandBob \opAddPoint \opPointScalar{\varPubKeyBob}{\varSchnorrChallenge} \opAddPoint \varStatement \\
        \funGen{\varSigBob \opAddScalar \varWit} & \\
        \funGen{\varNonceBob \opAddScalar \varSecKeyBob \opTimesScalar \varSchnorrChallenge \opAddScalar \varWit} & \\
        \funGen{\varNonceBob} \opAddPoint \funGen{\varSecKeyBob \opTimesScalar \varSchnorrChallenge} \opAddScalar \funGen{\varWit} & \\
        \varRandBob \opAddPoint \opPointScalar{\varPubKeyBob}{\varSchnorrChallenge} \opAddPoint \varStatement &\opEqNoQ \varRandBob \opAddPoint \opPointScalar{\varPubKeyBob}{\varSchnorrChallenge} \opAddPoint \varStatement \\
        1 &\opEqNoQ 1
    \end{align}
    We now continue to prove the last equation required:
    \[
        (\varStatement \opSeperate \varWit) \opIn \cnstRelation
    \]
    We do this by showing that $\varWit$ is calculated correctly in $\procExtWitId$:
    $\varSAptBob$ is the $\varS$ value in Bob's adapted partial signature
    \begin{align}
        \varWit \opEqNoQ & \varSAptBob \opSub (\varS \opSub \varSAlice) \\
        & \varSAptBob \opSub ((\varSAlice \opAddScalar \varSBob ) \opSub \varSAlice ) \\
        & \varSBob \opAddScalar \varWit \opSub (\varSBob) \\
        \varWit \opEqNoQ & \varWit \\
        1 \opEqNoQ & 1
    \end{align}
\end{proof}

\subsection{Security}\label{subsec:sig:secureaptscheme}

We have shown that the outlined signature scheme is correct, next we have to prove its security.
Our goal is to proof security in the malicious setting (as defined in~\cref{subsec:pre:security}) that means the adversary might or might not behave as specified by the protocol.
For achieving this we will prove security for both the $\procDSignId$ and $\procDAptSignId$ protocols in the hybrid model which was layed out by Yehuda Lindell in~\cite{lindell2017simulate}.
In particular, we will use the $\procZKfId{\cnstRelation}$-model in which we assume that we have access to a constant-round protocol $\procZKfId{\cnstRelation}$ that computes the zero-knowledge proof of knowledge functionality for any $\cnstNP$ relation $\cnstRelation$.
The function is parameterized with a relation $\cnstRelation$ between a witness value $\varWit$ (or potentially multiple)  and a statement $\varStatement$.
One party provides the witness statment pair $(\varWit, \varStatement)$, the second the statement $\funStar{\varStatement}$.
If $\varStatement \opEqNoQ \funStar{\varStatement}$ and $(\varWit, \varStatement) \opIn \cnstRelation$ the functionality returns 1, otherwise 0.
More formally:
\[
    \procZkf{\cnstRelation}{((\varWit, \varStatement), \funStar{\varStatement})} \opEqNoQ
    \begin{cases}
        (\lambda, \cnstRelation(\varStatement, \varWit)) & \text{if } \varStatement \opEqNoQ \funStar{\varStatement} \\
        (\lambda, 0) &\text{otherwise}
    \end{cases}
\]
That a constant-round zero-knowledge proof of knowledge exists was proven in~\cite{lindell2013note}.
A secure zero-knowledge proof must fulfill Completeness, Soundness and Zero-Knowledge properties which are defined for instance in~\cite{groth2010short}.

\paragraph{Hybrid functionalities:} The parties have access to a trusted third party that computes the zero-knowledge proof of knowledge functionality $\procZKfId{\cnstRelation}$. $\cnstRelation$ is the relation between a secret key $\varSecKey$ and its public key $\varPubKey \opEqNoQ \funGen{\varSecKey}$, for the elliptic curve generator point $\varG$.
The participants have to call the functionality in the same order.
That means if the prover first sends the pair $(\varWit_1, \varStatement_1)$ and then $(\varWit_2, \varStatement_2)$ the verifier needs to first send $\varStatement_1$ and then $\varStatement_2$.

\paragraph{Proof idea:} In order to construct our simulation proof in the hybrid-model we make some adjustments to the $\procDSignId$ protocol utilizing the capabilities of the $\procZKfId{\cnstRelation}$ functionality: (The added statements are marked in blue)
\begin{center}
    \fbox{
    \begin{varwidth}{\textwidth}
        \procedure[linenumbering,skipfirstln]{$\procDSign{\varMsg}{\varSecKeyAlice}{\varNonceAlice}{\varSecKeyBob}{\varNonceBob}$} {
        Alice \< \< Bob \\
        \varSigContext \opAssign \{ \varPubKey \opAssign \cnstIdentityElement, \varRand \opAssign \cnstIdentityElement \} \< \< \\
        \varSigContext \opFunResult \procSetupCtx{\varSigContext}{\funGen{\varSecKeyAlice}}{\funGen{\varNonceAlice}} \< \< \\
        \color{blue} \procZkf{\cnstRelation}{(\varSecKeyAlice, \varPubKeyAlice)} \\
        \color{blue} \procZkf{\cnstRelation}{(\varNonceAlice, \varRandAlice)} \\
        \< \sendmessageright*{\varSigContext, \varPubKeyAlice, \varRandAlice} \< \\
        \< \< \varSigContext \opFunResult \procSetupCtx{\varSigContext}{\funGen{\varSecKeyBob}}{\funGen{\varNonceBob}} \< \< \\
        \< \< \varSigBob \opFunResult \procSignPrt{\varMsg}{\varSecKeyBob}{\varNonceBob}{\varSigContext} \\
        \< \< \color{blue} \pcif \procZkf{\cnstRelation}{\varPubKeyAlice} \opEqNoQ 0 \opOr \procZkf{\cnstRelation}{\varSigContext.\varRand} \opEqNoQ 0 \\
        \< \< \color{blue} \t \pcreturn \cnstFalsum \\
        \< \< \color{blue} \procZkf{\cnstRelation}{(\varSecKeyBob, \varPubKeyBob)} \\
        \< \< \color{blue} \procZkf{\cnstRelation}{(\varNonceBob, \varRandBob)} \\
        \< \sendmessageleft*{\varSigBob, \varSigContext, \varPubKeyBob} \< \\
        \color{blue} \pcif \procZkf{\cnstRelation}{\varPubKeyBob} \opEqNoQ 0 \opOr \pcskipln \\
        \t \color{blue} \procZkf{\cnstRelation}{\varSigContext.\varRand \opAddPoint \varRandAlice^{-1}} \opEqNoQ 0 \\
        \t \color{blue} \pcreturn \cnstFalsum \\
        \pcif \procVerfPtSig{\varSigBob}{\varMsg}{\varPubKeyBob} \opEqNoQ 0 \< \< \\
        \t \pcreturn \cnstFalsum \< \< \\
        \varSigAlice \opFunResult \procSignPrt{\varMsg}{\varSecKeyAlice}{\varNonceAlice}{\varSigContext} \< \< \\
        \< \sendmessageright*{\varSigAlice} \< \\
        \< \< \pcif \procVerfPtSig{\varSigAlice}{\varMsg}{\varPubKeyAlice} \opEqNoQ 0 \\
        \< \< \t \pcreturn \cnstFalsum \\
        \varSigFin \opFunResult \procFinSig{\varSigAlice}{\varSigBob} \< \< \varSigFin \opFunResult \procFinSig{\varSigAlice}{\varSigBob} \\
        \varPubKey \opFunResult \varSigContext.\varPubKey \< \< \varPubKey \opFunResult \varSigContext.\varPubKey \\
        \pcreturn (\varSigFin, \varPubKey) \< \< \pcreturn (\varSigFin, \varPubKey)
        }
    \end{varwidth}
    }
\end{center}

That means both Alice and Bob will verify the validity of the public key and nonce commitments of the other party and will stop protocol execution in case an invalid value has been sent.
We assume parties have access to a trusted third party computing $\procZKfId{\cnstRelation}$ which will return 1 if $\varPubKeyAlice \opEqNoQ \funStar{\varPubKeyAlice}$ (where $\funStar{\varPubKeyAlice}$ is the public key that Bob received from Alice) and $\varPubKeyAlice \opEqNoQ \funGen{\varSecKeyAlice}$. (The same holds for the reversed case)

\begin{theorem}\label{lem:sig:security}
Assume we have two key pairs $\varKeyPairAlice$ and $\varKeyPairBob$ which were setup securely as for instance with the distributed keygen protocol $\procKeyGenPtId$ and a hash function $\funHash{\cdot}$ modeled in the random oracle model.
    Then $\procDSignId$ securely computes a signature $\varSigFin$ under the composite public key $\varPubKey \opAssign \varPubKeyAlice \opAddPoint \varPubKeyBob$ in the $\procZKfId{\cnstRelation}$-model.
\end{theorem}

\begin{proof}
    We proof security of the protocol by constructing a simulator $\cnstSimulator$ who is given output $(\varSigFin, \varPubKey)$ from a TTP (trusted third party) that securely computes the protocol in the ideal world upon receiving the inputs from Alice and Bob.
    The task of the simulator will be to extract the inputs used by $\cnstAdversary$ such that he is able to call the TTP and receive the outputs.
    From this output the simulator $\cnstSimulator$ will have to construct a transcript which is indistinguishable from the protocol transcript in the real world in which the corrupted party is controlled by a deterministic polynomial adversary $\cnstAdversary$.
    The simulator uses the calls to $\procZKfId{\cnstRelation}$ in order to do this.
    Furthermore, we assume that the message $\varMsg$ is known to both Alice and Bob.
    All other inputs (including public keys) are only known to the respective party at the start of the protocol.
    We have to prove two cases, one in which Alice is the corrupted party and one in which Bob is the corrupted party.
    
    \textbf{Alice is corrupted: } Simulator $\cnstSimulator$ works as follows:
    \begin{enumerate}
        \item $\cnstSimulator$ invokes $\cnstAdversary$ receives and saves $(\varSecKeyAlice, \varPubKeyAlice)$, as well as $(\varNonceAlice, \varRandAlice)$ that $\cnstAdversary$ sends to $\procZKfId{\cnstRelation}$.
        \item Next $\cnstSimulator$ receives the message $(\varSigContext, \funStar{\varPubKeyAlice}, \funStar{\varRandAlice})$ sent to Bob by $\cnstAdversary$.
        If $\funStar{\varPubKeyAlice} \opNotEq \varPubKeyAlice$ or $\funStar{\varRandAlice} \opNotEq \varRandAlice$ $\cnstSimulator$ externally sends $\cnstAbort$ to the TTP computing $\procDSignId$ and outputs $\cnstFalsum$, otherwise he will send the inputs $(\varMsg, \varSecKeyAlice, \varNonceAlice)$ and receive back $(\varSigFin, \varPubKey)$.
        \item $\cnstSimulator$ now calculates $\varPubKeyBob, \varRandBob$ and $\varSigBob$ as follows:
        \begin{gather*}
            (\varS, \varRand) \opFunResult \varSigFin \\
            \varPubKeyBob \opAssign \varPubKey \opAddPoint \varPubKeyAlice^{-1} \\
            \varRandBob \opAssign \varRand \opAddPoint \varRandAlice^{-1} \\
            \varSigContext \opFunResult \procSetupCtx{\varSigContext}{\varPubKeyBob}{\varRandBob} \\
            \varSigAlice \opFunResult \procSignPrt{\varMsg}{\varSecKeyAlice}{\varNonceAlice}{\varSigContext} \\
            (\varSAlice, \varRandAlice, \varSigContext) \opFunResult \varSigAlice \\
            \varSBob \opAssign \varS \opSub \varSAlice \\
            \varSigBob \opAssign (\varSBob, \varRandBob, \varSigContext)
        \end{gather*}
        \item After having done the calculations $\cnstSimulator$ is able to send $\varSigContext, \varSigBob, \varPubKeyBob$ to $\cnstAdversary$ as if coming from Bob.
        \item When $\cnstAdversary$ calls $\procZKfId{\cnstRelation}$ and $\procZKfId{\cnstRelation}$ (as the verifier) $\cnstSimulator$ checks equality with $\varPubKeyBob$ (respective $\varRandBob$) and thereafter sends back either 0 or 1.
        \item Eventually $\cnstSimulator$ will receive $\funStar{\varSigAlice}$ from $\cnstAdversary$ and checks if $\varSigAlice \opEqNoQ \funStar{\varSigAlice}$ (as calculated in step 3).
        If they are indeed the same the simulator will send $\cnstContinue$ to the TTP and output whatever $\cnstAdversary$ outputs, otherwise he will send $\cnstAbort$ and output $\cnstFalsum$.
    \end{enumerate}

    We now show that the joint output distribution in the ideal model with $\cnstSimulator$ is identically distributed to the joint distribution in a real execution in the $\procZKfId{\cnstRelation}$-hybrid model with $\cnstAdversary$.
    We consider three phases :
    \textbf{(1)} Alice sends $(\varSecKeyAlice, \varPubKeyAlice)$ as well as $(\varNonceAlice, \varRandAlice)$ to $\procZKfId{\cnstRelation}$ and $(\varSigContext, \varPubKeyAlice, \varRandAlice)$ to Bob
    \textbf{(2)} Bob sends $\varPubKeyAlice$ and $\varSigContext.\varRand$ to $\procZKfId{\cnstRelation}$ as the verifier, and  $(\varSecKeyBob, \varPubKeyBob)$, $(\varNonceBob, \varRandBob)$ to $\procZKfId{\cnstRelation}$ as the prover.
    Afterward he sends $(\varSigBob, \varSigContext, \varPubKeyBob)$ to Alice.
    \textbf{(3)} Alice sends $\varPubKeyBob$ and $\varRandBob$ to $\procZKfId{\cnstRelation}$ as the verifier and finally $\varSigAlice$ to Bob.

    \begin{itemize}
        \item \textit{Phase 1} Since $\cnstAdversary$ is required to be deterministic, the distribution is identical to a real execution.
        Also in the case the Alice does not send a message, or sends invalid values which will lead Bob to output $\cnstFalsum$ we also output $\cnstFalsum$ in the simulation, which again is indistinguishable.
        \item \textit{Phase 2} As $\cnstSimulator$ managed to calculate Bobs $\varSigBob, \varPubKeyBob, \varRandBob$ from the final $(\varSigFin, \varPubKey)$ and none of the values depend on any random tape we can say that the values sent in the ideal model are identical to those in the real model.
        As Bob in this case is the honest party, we don't have to consider any deviation from the protocol specification.
        \item \textit{Phase 3} The messages sent by the deterministic $\cnstAdversary$ again have to be identical to the real execution, therefore the transcript will be indistinguishable.
    \end{itemize}

    We have shown that the distributions in each phase are indeed identical, which proves the indistinguishability of the two transcripts in the case Alice is corrupted.
    
    \todo[inline]{Minor points in the formality:\\
    1. You are missing the point that the output of the simulator is also indistinguishable from the output of the attacker. In fact, your simulator is setting as output the same as the adversary.
    2. You often write ``have to be identical to the real execution''. This is not formally correct: By real execution, the reader would understand an execution where the attacker contacts the real Bob and not your simulator. Therefore, the values that you create in the simulator are not identical to that real execution. What you are doing is creating an execution that is computationally indistinguishable to that of a real execution (i.e., imagine that A is given the transcript of two executions, one transcript is the messages exchanged with the real Bob and a second transcript is the messages that you create with the simulator. Then, A cannot tell what transcript is the real one and what transcript is the simulated one). An statement that is correct is that your simulator is creating messages that are identical to that expected by the adversary in the current execution. 
    3. In this case, it is trivial to see that the two transcripts (the one that you create with the simulator and the one that would be created with the real Bob) are indistinguishable, so what you have done so far is fine. However, to be fully formal you would need to do the following: (i) assume that there is an adversary that is able to differentiate between the two transcripts, then you can build an adversary that breaks a certain assumption (e.g., the zero-knowledge of the zkps or the discrete logarithm assumption).  }

    \textbf{Bob is corrupted: } Simulator $\cnstSimulator$ works as follows:
    \begin{enumerate}
        \item $\cnstSimulator$ starts by sampling $\varSecKeyAlice, \varNonceAlice \sample \cnstIntegersPrimeWithoutZero{*}$ and proceeds by setting up the initial signature context as defined in the protocol:
        \begin{gather*}
            \varSigContext \opAssign \{ \varPubKey \opAssign 1, \varRand \opAssign 1 \} \\
            \varSigContext \opFunResult \procSetupCtx{\varSigContext}{\funGen{\varSecKeyAlice}}{\funGen{\varNonceAlice}} \\
        \end{gather*}
        \item $\cnstSimulator$ now invokes $\cnstAdversary$ and sends $(\varSigContext, \varPubKeyAlice, \varRandAlice)$ as if coming from Alice.
        \item When $\cnstAdversary$ calls $\procZKfId{\cnstRelation}$ (as verifier) $\cnstSimulator$ checks equality to the parameters he sent in step 1 and returns either 1 or 0.
        When $\cnstAdversary$ calls $\procZkf{\cnstRelation}{(\varSecKeyBob, \varPubKeyBob)}$ and $\procZkf{\cnstRelation}{(\varNonceBob, \varRandBob)}$ the simulator saves those values to its memory.
        \item Now $\cnstSimulator$ externally sends the inputs $(\varMsg, \varSecKeyBob, \varNonceBob)$ to the TTP and receives back $(\varSigFin, \varPubKey)$
        \item When $\cnstAdversary$ queries $\funHash{\varMsg \opConc \varRandAlice \opAddPoint \varRandBob \opConc \varPubKeyAlice \opAddPoint \varPubKeyBob}$ $\cnstSimulator$ sends back $\funStar{\varSchnorrChallenge}$ such that:
        \begin{gather*}
            \varSigFin \opEqNoQ \varNonceAlice \opAddScalar \varSecKeyAlice \opTimesScalar \funStar{\varSchnorrChallenge} \opAddScalar \varNonceBob \opAddScalar \varSecKeyBob \opTimesScalar \funStar{\varSchnorrChallenge} \\
            \funStar{\varSchnorrChallenge} \opEqNoQ \frac{\varSigFin \opSub \varNonceAlice \opSub \varNonceBob}{\varSecKeyAlice \opAddScalar \varSecKeyBob}
        \end{gather*}
        
        \todo[inline]{Following-up from a previous message, in the description of the protocol in previous page, it does not appear when Bob does this hash computation. This is the reason why I think you should show the complete protocol in the figure above.}
        
        \item $\cnstSimulator$ receives $(\varSigBob, \varSigContext, \varPubKeyBob)$ from $\cnstAdversary$.
        (In case he does not $\cnstSimulator$ sends $\cnstAbort$ to the TTP and outputs $\cnstFalsum$).
        He verifies the values sent to him by comparing them with $\varPubKeyBob$ and $\varRandBob$ from its memory, if they are found to be invalid $\cnstSimulator$ sends $\cnstAbort$ to the TTP, otherwise it sends $\cnstContinue$.
        \item $\cnstSimulator$ calculates as defined in the protocol as $\varSigAlice \opFunResult \procSignPrt{\varMsg}{\varSecKeyAlice}{\varNonceAlice}{\varSigContext}$ and then sends it to $\cnstAdversary$ as if coming from Alice and finally outputs whatever $\cnstAdversary$ outputs.
    \end{enumerate}
    Again we argue why the transcript is indistinguishable from the real one for each of the three phases layed out before:
    \begin{itemize}
        \item \textit{Phase 1: } The values $(\varPubKeyAlice, \varRandAlice)$ sent by $\cnstSimulator$ to $\cnstAdversary$ only depend on Alice's input parameters (and to some extend on the public elliptic curve parameters).
        As $\cnstAdversary$ does not know $\varPubKeyAlice$ or $\varRandAlice$ yet, he has no way of determining for two public keys $\varPubKeyAlice, \funStar{\varPubKeyAlice}$ which of the two is the correct one (other than guessing).
        \item \textit{Phase 2: } When $\cnstAdversary$ calls $\procZKfId{\cnstRelation}$ with the parameters sent to him he will still receive 1 back, and 0 otherwise, which is again exactly the same as in the real execution.
        The hash function $\funHash{\cdot}$ is expected to output a random value for the Schnorr challenge as defined by the hiding property of the hash function. \todo{This is not because of the hiding property, but the random oracle model. If you assume that the the hash function is hiding (but not in the random oracle), it might be that you cannot guess the input from the output, but the output is not random.}
        In the simulated case $\cnstSimulator$ calculates the output value from the final signature and depends on the input values of Alice and Bob of which at least Alice input is chosen randomly by $\cnstSimulator$.
        As dependent on randomly chosen inputs the calculation output will as well be distributed uniformly across the possible values and is therefore indistinguishable from a real hash function output.
        The remaining messages sent by $\cnstAdversary$ are identical to those of the real execution due to the deterministic nature of $\cnstAdversary$.
        \item \textit{Phase 3: } The simulator will now verify the values sent to him by $\cnstAdversary$ and will halt and output $\cnstFalsum$ in the case that he sends something invalid which is identical to the real execution.
        In this case $\cnstAdversary$ must not receive ($\varSigFin, \varPubKey$) in the ideal setting which is modelled by $\cnstSimulator$ sending $\cnstAbort$ to the TTP.
        Otherwise $\cnstSimulator$ will calculate his part of the partial signature as defined by the protocol.
        It will therefore found to be valid by $\cnstAdversary$ and will complete to $\varSigFin$ with $\procFinSigId$, because of the fixed, calculated Schnorr challenge $\cnstSimulator$ calculated in Phase 2.
    \end{itemize}

    We have managed to show that in the case that Bob is corrupted the transcript is indistinguishable to a real transcript and even identical for the most part.
    We can therefore conclude that the transcript output will be indistinguishable from a real one in all cases and have thereby proven that the protocol $\procDSignId$ is secure.
\todo[inline]{As before, you are missing the output of the simulator.}
\end{proof}

We now do the same for $\procDAptSignId$:
Again we adjust the protocol with calls to $\procZKfId{\cnstRelation}$, note that we now have one additional call $\procZKfId{\cnstRelation}$, for the pair $(\varWit, \varStatement)$.
The relation $\cnstRelation$ is equally defined as in the previous proof.
\begin{center}
    \fbox{
    \begin{varwidth}{\textwidth}
        \procedure[linenumbering,skipfirstln]{$\procDAptSign{\varMsg}{\varSecKeyAlice}{\varNonceAlice}{\varSecKeyBob}{\varNonceBob}{\varWit}$} {
        Alice \< \< Bob \\
        \cdots \< \< \\
        \procZkf{\cnstRelation}{(\varSecKeyAlice, \varPubKeyAlice)} \\
        \procZkf{\cnstRelation}{(\varNonceAlice, \varRandAlice)} \\
        \< \sendmessageright*{\varSigContext, \varPubKeyAlice, \varRandAlice} \< \\
        \< \< \cdots \\
        \< \< \pcif \procZkf{\cnstRelation}{\varPubKeyAlice} \opEqNoQ 0 \opOr \procZkf{\cnstRelation}{\varSigContext.\varRand} \opEqNoQ 0 \\
        \< \< \t \pcreturn \cnstFalsum \\
        \< \< \procZkf{\cnstRelation}{(\varSecKeyBob, \varPubKeyBob)} \\
        \< \< \procZkf{\cnstRelation}{(\varNonceBob, \varRandBob)} \\
        \< \< \procZkf{\cnstRelation}{(\varWit, \varStatement)} \\
        \< \sendmessageleft*{\varSigBob, \varSigContext, \varPubKeyBob, \varStatement} \< \\
        \cdots \< \< \\
        \pcif \procZkf{\cnstRelation}{\varPubKeyBob} \opEqNoQ 0 \opOr \pcskipln \\
        \t \procZkf{\cnstRelation}{\varSigContext.\varRand \opAddPoint \varRandAlice^{-1}} \opEqNoQ 0 \opOr \pcskipln \\
        \t \procZkf{\cnstRelation
        }{\varStatement} \opEqNoQ 0 \opOr \\
        \t \pcreturn \cnstFalsum \\
        \< \cdots \< \\
        \pcreturn (\varWit, (\varSigFin, \varPubKey)) \< \< \pcreturn (\varSigFin, \varPubKey)
        }
    \end{varwidth}
    }
    \todo[inline]{Same as before, I would show the whole protocol, highlighting the changes.}
\end{center}

\begin{theorem}
    Assume we have two key pairs $\varKeyPairAlice$ and $\varKeyPairBob$ which were setup securely as for instance with the distributed keygen protocol $\procKeyGenPtId$.
    Additionally we have a pair $(\varWit, \varStatement)$ in the relation $\varStatement \opEqNoQ \funGen{\varWit}$ for which $\varWit$ was chosen randomly.
    Then $\procDAptSignId$ securely computes a signature $\varSigFin$ under the composite public key $\varPubKey \opAssign \varPubKeyAlice \opAddPoint \varPubKeyBob$ after which $\varWit$ is revealed to Alice, in the $\procZKfId{\cnstRelation}$-model.
\end{theorem}

\todo[inline]{Do you use the random oracle model as well?}

\begin{proof}
    We proof the security of $\procDAptSignId$ by constructing a simulator $\cnstSimulator$ who is given the output $(\varSigFin, \varPubKey)$ (resp. ($\varWit, (\varSigFin, \varPubKey)$)) from a TTP that securly computes the protocol in the ideal world after receiving the inputs from Alice and Bob.
    The simulators task again is to extract the adversaries inputs and send them to the trusted third party to receive the protocol outputs.
    From this output the simulator $\cnstSimulator$ will construct a transcript that is indistinguishable from the protocol transcript in the real world.
    The simulator uses the calls to $\procZKfId{\cnstRelation}$ in order to do this.
    As in the proof before we assume the message $\varMsg$ is known to both participants.
    All other inputs (including public keys) are only known to the respective party at the start of the protocol.
    We proof that the transcript is indistinguishable in case Alice is corrupted as well as in the case that Bob is corrupted.

    \textbf{Alice is corrupted: } Simulator $\cnstSimulator$ works as follows:
    \begin{enumerate}
        \item $\cnstSimulator$ invokes $\cnstAdversary$.
        When $\cnstAdversary$ internally calls $\procZKfId{\cnstRelation}$ and $\procZKfId{\cnstRelation}$ $\cnstSimulator$ saves $(\varSecKeyAlice, \varPubKeyAlice)$ and $(\varNonceAlice, \varRandBob)$ to its memory.
        \item $\cnstSimulator$ receives $(\varSigContext, \funStar{\varPubKeyAlice}, \funStar{\varPubKeyBob})$ from $\cnstAdversary$.
        $\cnstSimulator$ checks the equalities $\funStar{\varPubKeyAlice} \opEqNoQ \varPubKeyAlice$ and $\funStar{\varRandAlice} \opEqNoQ \varRandAlice$ as well as checking $\varPubKeyAlice \opEqNoQ \funGen{\varSecKeyAlice}$ and $\varRandAlice \opEqNoQ \funGen{\varNonceAlice}$.
        If any of those checks fail $\cnstSimulator$ sends $\cnstAbort$ to the TTP and outputs $\cnstFalsum$.
        Otherwise he sends $(\varMsg, \varSecKeyAlice, \varNonceAlice)$ to the TTP and receives $(\varWit, (\varSigFin, \varPubKey))$
        \item Again $\cnstSimulator$ calculates $\varSigBob, \varPubKeyBob, \varRandBob$ and finalizes the context $\varSigContext$ as layed out in the proof beforehand in step 3.
        
        \todo[inline]{I would copy the steps here to make the proof self-complete.}
        
        \item $\cnstSimulator$ calculates $\funStar{\varSBob} \opAssign \varSBob \opAddScalar \varWit$ (extracted from the TTP output) from which he sets $\varSigAptBob \opAssign (\funStar{\varSBob}, \varRandBob, \varSigContext)$.
        \item $\cnstSimulator$ sends $(\varSigAptBob, \varSigContext, \varPubKeyBob, \varStatement \opAssign \funGen{\varWit})$ as if coming from Bob.
        \item When $\cnstAdversary$ calls $\procZKfId{\cnstRelation}$ we compare the parameters send by $\cnstAdversary$ to the real one, in case he sent a invalid value $\cnstSimulator$ returns 0, otherwise 1.
        \item $\cnstSimulator$ receives $\funStar{\varSigAlice}$ from $\cnstAdversary$ and checks $\varSigAlice \opEqNoQ \funStar{\varSigAlice}$.
        If the equality holds $\cnstAdversary$ sends $\cnstContinue$ to the TTP and finally sends $\varSigFin$ to $\cnstAdversary$ as if coming from Bob and outputs whatever $\cnstAdversary$ outputs.
    \end{enumerate}

    We reuse the phases defined in the previous proof with two adjustments:
    \begin{itemize}
        \item In \textit{Phase 2} Bob additionally sends $\varStatement$ to Alice
        \item We introduce \textit{Phase 4} in which Bob sends $\varSigFin$ to Alice
    \end{itemize}
    
    \todo[inline]{Here, I would write what you have done: ``The next steps are similar to those in proof x, with the differences a, b''. Then, I would say, ``yet for completeness, we write the full proof in the following'' and write the whole proof. We can see if we want to put in the appendix later if we find it repetitive.}
    

    We now again argue why each phase in the simulation is indistinguishable from a real execution
    \begin{itemize}
        \item \textit{Phase 1:} This phase is identical to phase 1 the previous proof, thereby the argument is the same. \todo{Same. Say that is the same but write it for completeness.}
        \item \textit{Phase 2:} In this phase $\cnstSimulator$ sends $\varStatement \opAssign \funGen{\varWit}$ to $\cnstAdversary$ for which $\varWit$ was received from the TTP, therefore it will be the same $\varWit$ sent in the real execution by the honest party which makes the simulation perfect in this phase.
        \item \textit{Phase 3:} Again the messages send in this phase are produced by the deterministic $\cnstAdversary$ which will be indistinguishable to the real execution.
        In contrast to the $\procDSignId$ protocol now the adversary does not yet finish the protocol.
        \item \textit{Phase 4:} Now the $\cnstAdversary$ expects to receive $\varSigFin$, from which he is able to extract the witness $\varWit$.
        Indeed he will receive a $\varSigFin$ which is identical to the one sent in a real execution by honest Bob, furthermore he will be able to extract $\varWit$ such that $\varStatement \opEqNoQ \funGen{\varWit}$, which again makes this phase identical to the real execution.
    \end{itemize}

\todo[inline]{There are 4 phases here, while in the previous proof you had only three phases. I would recall the phases here as well (and maybe even mark them in the protocol so that it is clearer what you mean)}

    We have shown that in the case Alice is corrupt the simulated transcript produced by $\cnstSimulator$ is indeed distributed equally to a real execution and is thereby computationally indistinguishable.

    \textbf{Bob is corrupted: } Simulator $\cnstSimulator$ works as follows:
    \begin{enumerate}
        \item $\cnstSimulator$ starts by sampling $\varSecKeyAlice, \varNonceAlice \sample \cnstIntegersPrimeWithoutZero{*}$ and proceeds by setting up the initial signature context as defined in the protocol:
        \begin{gather*}
            \varSigContext \opAssign \{ \varPubKey \opAssign 1, \varRand \opAssign 1 \} \\
            \varSigContext \opFunResult \procSetupCtx{\varSigContext}{\funGen{\varSecKeyAlice}}{\funGen{\varNonceAlice}} \\
        \end{gather*}
        \item $\cnstSimulator$ now invokes $\cnstAdversary$ and sends $(\varSigContext, \varPubKeyAlice, \varRandAlice)$ as if coming from Alice.
        \item When $\cnstAdversary$ calls $\procZKfId{\cnstRelation}$ (as the verifier) $\cnstSimulator$ checks for equality with the values sent by him and returns either 0 or 1.
        Once $\cnstAdversary$ sends $(\varSecKeyBob, \varPubKeyBob)$, $(\varNonceBob, \varRandBob)$, $(\varWit, \varStatement)$ internally to $\procZKfId{\cnstRelation}$ as the prover $\cnstSimulator$ saves them to his memory.
        \item $\cnstSimulator$ sends $(\varMsg, \varSecKeyAlice, \varNonceAlice, \varWit)$ to the TTP and receives $(\varSigFin, \varPubKey)$.
        \item When $\cnstAdversary$ queries $\funHash{\cdot}$ the simulator again sets the output to $\funStar{\varSchnorrChallenge}$ calculated with the same steps as layed out in the previous proof in step 5.
        \item $\cnstSimulator$ receives $(\funStar{\varSigAptBob}, \funStar{\varPubKeyBob}, \varSigContext, \funStar{\varStatement})$ from $\cnstAdversary$ and verifies those values checking equality with the ones stored in its memory.
        If the equality checks succeed $\cnstSimulator$ sends $\cnstContinue$ to the TTP, otherwise sends $\cnstAbort$ and outputs $\cnstFalsum$.
        \item The simulator now calculates $\varSigAlice$ as defined by the protocol using the $\procSignPrtId$ procedure and sends the result to $\cnstAdversary$ as if coming from Alice.
        \item Finally $\cnstSimulator$ will receive $\funStar{\varSigFin}$ from $\cnstAdversary$ (if not he outputs $\cnstFalsum$) and will verify that $\funStar{\varSigFin} \opEqNoQ \varSigFin$.
        If the equality holds he will output whatever $\cnstAdversary$ outputs, otherwise $\cnstFalsum$.
    \end{enumerate}
    
    Again we argue why the transcript is indistinguishable in phases 1--4.
    \begin{itemize}
        \item \textit{Phase 1:} This phase is identical to phase 1 in the previous proof, thereby the same argumentation holds.
        \item \textit{Phase 2:} Again this phase is similar to phase 2 in the $\procDSignId$ proof with the only difference that $\cnstAdversary$ will make the additional call to $\procZkf{\cnstRelation}{(\varWit, \varStatement)}$ and send the value $\varStatement$ to $\cnstSimulator$.
        Both these changes do not require any further interaction from $\cnstSimulator$ thereby the arguments from the previous proof in phase 2 still hold.
        \item \textit{Phase 3:} In this section $\cnstSimulator$ will verify equality of the values sent by $\cnstAdversary$ with the variables saved prior to its memory and halts with output $\cnstFalsum$ if any of the values are unequal.
        In this case $\cnstAdversary$ should not receive the final outputs $(\varSigFin, \varPubKey)$ which is modelled by sending $\cnstAbort$ to the TTP.
        The same behaviour is expected in a real execution when Alice calls $\procZKfId{\cnstRelation}$ and receives a 0 bit.
        We have already argued in the prior proof why $\varSigAlice$ is indistinguishable from the one calculated by Alice in a real execution and only refer to the argumentation here.
        \item \textit{Phase 4:} In this phase $\cnstSimulator$ is expected to receive $\funStar{\varSigFin}$ from $\cnstAdversary$ which needs to be equal to $\varSigFin$ received earlier by the TTP.
        $\cnstSimulator$ will do this simple equality check and if successful output whatever $\cnstAdversary$ outputs.
        In the other case we would simply output $\cnstFalsum$ which is identical to the case in which a Bob sends a $\varSigFin$ that does not verify.
    \end{itemize}

    We have shown that the transcript produced by $\cnstSimulator$ in an ideal world with access to a TTP computing $\procDAptSignId$ is indistinguishable from a transcript produced during a real execution both in the case that Alice and that Bob is corrupted.
    By managing to show this we have proven that the protocol is secure.
\end{proof}
