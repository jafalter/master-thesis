\section{General Notation}\label{secGenNot}
\section{Cryptographic Primitives}\label{secCrypPrim}
\section{Mulitparty Fixed Witness Adaptor Signature Scheme} \label{secScheme}

We define a Generalized Multiparty Adaptor Signature Scheme from the standard construction of multiparty Schnorr signatures which
are defined as follows:\\

\fbox{
    \parbox{\textwidth}{
        \procedure[linenumbering, syntaxhighlight=auto]{\procGen} {
            \varK \sample \cnstZq \\
            \varR \sample \cnstZq \\
            \pcreturn (\varK, \varR)
        }
        \procedure[linenumbering, syntaxhighlight=auto]{\procGenPartSig} {
            \varE = \funH{(\varM \opConc \varGk + \varGkSt \opConc \varGr + \varGrSt)} \\
            \varSigPart = \varK + \varE + \varR \\
            \pcreturn (\varSigPart, \varGk, \varGr)
        }\\
        \procedure[linenumbering, syntaxhighlight=auto]{\procVerfPartSig} {
            \varE = \funH{((\varM \opConc \varGk + \varGkSt \opConc \varGr + \varGrSt)} \\
            \pcreturn \varGSigPart = \varGkSt + \funGen{\varE * \varR}
        }\\
        \procedure[linenumbering, syntaxhighlight=auto]{\procFinSig} {
            \pcreturn (\varSigPart + \varSigPartSt, \varGk + \varGkSt, \varGr + \varGrSt)
        }
    }
}
In order to have adaptable partial signature we add the following procedures\\

\fbox{
    \parbox{\textwidth} {
        \procedure[linenumbering, syntaxhighlight=auto]{\procAptSig} {
            \varSigPartApt = \varSigPart + \varX \\
            \pcreturn (\varSigPartApt, \varGx)
        }
        \procedure[linenumbering, syntaxhighlight=auto]{\procExtWit}{
            \varSigPartSt = \varSigFin - \varSigPart \\
            \varX = \varSigPartAptSt - \varSigPartSt \\
            \pcreturn (\varX)
        }\\
        \procedure[linenumbering, syntaxhighlight=auto]{\procAptSig} {
            \varE = \funH{(\varM \opConc \varGk + \varGkSt \opConc \varGr + \varGrSt)} \\
            \pcreturn \varGSigPartAptSt = \varGkSt + \funGen{\varE * \varR} + \varGx
        }
    }
}