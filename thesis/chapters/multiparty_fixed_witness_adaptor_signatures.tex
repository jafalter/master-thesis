\section{General Notation}\label{secGenNot}
\section{Cryptographic Primitives}\label{secCrypPrim}
\section{Mulitparty Fixed Witness Adaptor Signature Scheme} \label{sec:secscheme}

In this chapter, we will define a variant of the Adaptor Signature scheme as seen in~\ref{sec:apt},
which is specifically tailored for the use in an Atomic Swap scenario in which (at least one side of the swap) uses a multi-party protocol to generate transaction signatures.
We will start by recalling some basic notation and definitions, we then explain the general multi-party signature creation protocol as it currently implemented in the Grin Cryptocurrency.
We reduce the generated signatures to the general case~\cite{schnorr1991efficient} and thereby prove its security. From this protocol, we then derive the adapted variant, which allows hiding a fixed witness
value in the signature, which can only be revealed by the other parties after attaining the final signature.
We then proof it's security by showing that all security definitions defined in~\cite{sanches2020bitcoinchannels} hold.
In chapter~\ref{chapAtomic} we will then utilize this scheme to build the Atomic Swap protocol.

\section{General Notation and Definitions}

\paragraph{Signature Scheme}
A valid Signature Sccheme must provide three procedures:
\[ \varSigScheme = (\procGenDefault \opSeperate \procSignDefault \opSeperate \procVerfDefault) \]
$\procGenDefault$ takes as input a security parameter $\varSecParam$ and outputs a keypair $\varKeyPair$, consisting of a secret key $\varSecKey$ and a public key $\varPubKey$, whereas
the secret key has to be kept private and the public key is shared with other parties.
$\varSecKey$ can be used together with a message $\varMsg$ to call the $\procSignDefault$ procedure to create a signature $\varSignature$ over the message $\varMsg$.
Parties knowing $\varPubKey$ can then test the validity of the signature by calling $\procVerfDefault$ with the same message $\varMsg$. The procedure will only output $1$ if the message was
indeed signed with the correct secret key $\varSecKey$ of $\varPubKey$ and therefore proves the possesion of $\varSecKey$ by the signer.
A valid signature scheme have to fullfill two security properties
\begin{itemize}
    \item Correctness: For all messages $\varMsg$ and valid keypairs $\varKeyPair$ the following must hold $\procVerf{\varPubKey}{\procSignDefault}{\varMsg} \opEq 1$
    \item Unforgability: Note that there are different levels of Unforgability:~\cite{goldwasser1988digital}
    \begin{itemize}
        \item Universal Forgery: The ability to forge signatures for any message.
        \item Selective Forgery: The ability to fogre signatures for messages of the adversary's choice.
        \item Existential Forgery: The ability to forge a valid signature / message pair not previously known to the adversary.
    \end{itemize}
\end{itemize}

\paragraph{Hash function}

\paragraph{Zero-Knowledge}

\paragraph{Hard relation}


\fbox{
    \parbox{\textwidth}{
        \procedure[linenumbering, syntaxhighlight=auto]{$\procGenDefault$} {
            \varKey \sample \cnstZq \\
            \varRand \sample \cnstZq \\
            \pcreturn (\varKey \opSeperate \varRand)
        }
        \procedure[linenumbering, syntaxhighlight=auto]{$\procGenPartSigDefault$} {
            \varSchnorrChallenge \opAssign \funH{(\varMsg \opConc \varGk \opAdd \varGkSt \opConc \varGr \opAdd \varGrSt)} \\
            \varSigPart \opAssign \varKey \opAdd \varSchnorrChallenge \opAdd \varRand \\
            \pcreturn (\varSigPart \opSeperate \varGk \opSeperate \varGr)
        }\\
        \procedure[linenumbering, syntaxhighlight=auto]{$\procVerfPartSigDefault$} {
            \varSchnorrChallenge \opAssign \funH{((\varMsg \opConc \varGk \opAdd \varGkSt \opConc \varGr \opAdd \varGrSt)} \\
            \pcreturn \varGSigPart \opEq \varGkSt \opAdd \funGen{\varSchnorrChallenge \opMult \varRand}
        }\\
        \procedure[linenumbering, syntaxhighlight=auto]{$\procFinSigDefault$} {
            \pcreturn (\varSigPart \opAdd \varSigPartSt \opSeperate \varGk \opAdd \varGkSt \opSeperate \varGr \opAdd \varGrSt)
        }
    }
}
In order to have adaptable partial signature we add the following procedures\\

\fbox{
    \parbox{\textwidth} {
        \procedure[linenumbering, syntaxhighlight=auto]{$\procAptSigDefault$} {
            \varSigPartApt \opAssign \varSigPart \opAdd \varWit \\
            \pcreturn (\varSigPartApt \opSeperate \varGWit)
        }
        \procedure[linenumbering, syntaxhighlight=auto]{$\procExtWitDefault$}{
            \varSigPartSt \opAssign \varSigFin \opSub \varSigPart \\
            \varWit \opAssign \varSigPartAptSt \opSub \varSigPartSt \\
            \pcreturn (\varWit)
        }\\
        \procedure[linenumbering, syntaxhighlight=auto]{$\procVrfAptDefault$} {
            \varSchnorrChallenge \opAssign \funH{\varMsg \opConc \varGk \opAdd \varGkSt \opConc \varGr \opAdd \varGrSt} \\
            \pcreturn \varGSigPartAptSt \opEq \varGkSt \opAdd \funGen{\varSchnorrChallenge \opMult \varRand} \opAdd \varGWit
        }
    }
}