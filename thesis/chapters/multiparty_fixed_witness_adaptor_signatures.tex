This chapter will define a variant of the Adaptor Signature Scheme as shown in~\cref{def:pre:script:apt}.
This new variant is explicitly tailored to meet the requirements of being applicable in the scenario of two-party signature protocols that one can construct for Signature Schemes such as Schnorr~\cite{maxwell2019simple}.
In a two-party signature protocol, each party holds only a share of a private key (to a composite public key) for which they want to create a signature cooperatively, without revealing their key share to the other party.
The advantage of our Adaptor Signature Scheme in comparison to the original definition is that we do not need to introduce an additional pre-signature step in the two-party scenario mentioned above.
Instead, one of the partial signatures created and exchanged by the two parties will serve as what is defined as the pre-signature by Aumayr et al., allowing for a more straightforward protocol.
In particular, our protocol will allow one of the two parties to mask his signature share with a witness value $\varWit$ of $(\varWit, \varStatement) \opIn \cnstRelation$ (where $\cnstRelation$ is a hard relation as of \cref{def:pre:hard-relation}).
The second party (knowing $\varStatement$, but not $\varWit$) can verify that $\varWit$ is indeed contained in the peer's partial signature.
To complete the final signature, the party knowing $\varWit$ has to first replace his partial signature (masked with $\varWit$) with the original unmasked version, which corresponds to the adapting step of the original Adaptor Signature Scheme from~\cref{def:pre:script:apt}.
Having previously received the masked partial signature, the second party can now extract $\varWit$ from the final signature, his partial signature, and the other party's masked partial signature.
One can then leverage this feature to build an Atomic Swap protocol, as shown in~\cref{ch:atomicswap}.

The rest of the chapter is organized as follows:
First, we will define the general two-party Schnorr signature protocol, as it is currently implemented in Mimblewimble-based cryptocurrencies.
We will then show that the protocol's final signatures fulfill the same properties as regular Schnorr signatures seen in~\cite{schnorr1989efficient} and prove its correctness.
From this two-party protocol, we then derive the adapted variant already mentioned before.
We start by defining our extended Signature Scheme in~\cref{sec:sig:definitions}, proceed by providing a Schnorr-based instantiation of the protocol in~\cref{sec:sig:schnorr-inst} and finally prove its correctness and security in~\cref{sec:sig:two-party-apt-security}.

\section{Definitions} \label{sec:sig:definitions}
A Two-Party Signature Scheme is an extension of a Signature Scheme shown in~\cref{def:pre:signature-scheme}, which allows us to distribute signature generation for a composite public key shared between two parties Alice and Bob.
Alice and Bob want to collaborate to generate a signature valid under the composite public key $\varPubKey \opAssign \varPubKeyAlice \opAddPoint \varPubKeyBob$ without revealing their secret keys to each other.
The definition below was constructed with the goal in mind of formalizing exactly what is currently implemented and used in Mimblewimble-based cryptocurrencies.

\begin{definition}[Two-Party Signature Scheme]
    \label{def:sig:two-party-sig}

    A \emph{Two-Party Signature Scheme $\varSigSchemeMP$} extends a Signature Scheme $\varSigScheme$ with a tuple of protocols and algorithms\\
    ($\procKeyGenPtId, \procSignPrtId , \procVerfPtSigId, \procFinSigId)$ defined as follows:

    \begin{asparaitem}
        \item $((\varSecKeyAlice, \varPubKeyAlice, \varNonceAlice, \varSigContext), (\varSecKeyBob, \varPubKeyBob, \varNonceBob, \varSigContext)) \opFunResult \procKeyGenPt{\varSecParam}{\varSecParam}$: The distributed key generation protocol takes as input the security parameter from both Alice and Bob.
        It returns the tuple $(\varSecKeyAlice, \varPubKeyAlice, \varNonceAlice, \varSigContext)$ to Alice (similar to Bob) where $(\varSecKeyAlice, \varPubKeyAlice)$ is a pair of private and corresponding public keys, $\varNonceAlice$ a secret nonce and $\varSigContext$ is the signature context containing parameters shared between Alice and Bob.
        We introduce $\varSigContext$ for the participants to share and update parameters with each other during the protocol execution.
        Note that this context always has to be consistent between the two parties.
        If Alice were to update $\varSigContext$, she has to send the updated version to Bob to continue the protocol.

        \item $(\varSigAlice) \opFunResult \procSignPrt{\varMsg}{\varSecKeyAlice}{\varNonceAlice}{\varSigContext}$: The partial signing algorithm is a DPT function that takes as input the message $\varMsg$, the share of the secret key $\varSecKeyAlice$ and nonce $\varNonceAlice$ (similar for Bob), and the shared signature context $\varSigContext$. The procedure outputs $(\varSigAlice)$, that is, a share of the signature to a participant.

        \item $\cnstTrueorFalse \opFunResult \procVerfPtSig{\varSigAlice}{\varMsg}{\varPubKeyAlice}$: The share verification algorithm is a DPT function that takes as input a signature share $\varSigAlice$, a message $m$, and the other participant's public key $\varPubKeyAlice$ ($\varPubKeyBob$ for Bob's partial signature).
        The algorithm returns 1 if the verification was successful or 0 otherwise.

        \item $\varSigFin \opFunResult \procFinSig{\varSigAlice}{\varSigBob}$: The finalize signature algorithm is a DPT function that takes as input two shares of the signatures and combines them into a final signature valid under the composite public key $\varPubKey \opEqNoQ \varPubKeyAlice \opAddPoint \varPubKeyBob$.

    \end{asparaitem}

\end{definition}

We require the Two-Party Signature Scheme to be correct as well as secure as of~\cref{subsec:pre:security}.
For the security of the distributed key-generation protocol $\procKeyGenPtId$, special care needs to be taken to protect the scheme against rogue-key attacks.
In such an attack one of the public keys is computed as a function of the other parties public key, allowing the corrupted signer to produce forged signatures under the honest users public key without knowing its secret key~\cite{maxwell2019simple}.


From~\cref{def:sig:two-party-sig}, we now derive a Two-Party Adaptor Signature Scheme $\varSigSchemeApt$, allowing one of the participants to hide a secret witness value inside his partial signature.
\begin{definition}[Two-Party Fixed Witness Adaptor Signature Scheme]
    \label{def:sig:two-party-fixed-wit-apt-sig}
    Given a pair $(\varWit, \varStatement) \opIn \cnstRelation$, (where $\cnstRelation$ is a hard relation as of \cref{def:pre:hard-relation}) a Two-Party Fixed Witness Adaptor Signature Scheme $\varSigSchemeApt$ is an extension to $\varSigSchemeMP$ with the following algorithms.

    \[ \varSigSchemeApt \opAssign (\varSigSchemeMP \opConc \procAptSigId \opConc \procVerifyAptSigId \opConc \procExtWitId) \]

    \begin{asparaitem}
        \item $\varSigAptAlice \opFunResult \procAptSig{\varSigAlice}{\varWit}$: The mask signature algorithm is a DPT function that takes as input a partial signature $\varSigAlice$ and a secret witness value $\varWit$.
        The procedure will output a masked partial signature $\varSigAptAlice$ that can be verified to contain $\varWit$ using the $\procVerifyAptSigId$ function without revealing $\varWit$.

        \item $\cnstTrueorFalse \opFunResult \procVerifyAptSig{\varSigAptAlice}{\varMsg}{\varPubKeyAlice}{\varStatement}$: The masked signature verification algorithm is a DPT function that takes as input a masked partial signature $\varSigAptAlice$, the other participant's public key $\varPubKeyAlice$ and a statement $\varStatement$.
        The function will verify the partial signature's validity and that it was masked with the secret witness $\varWit$.

        \item $\varWit \opFunResult \procExtWit{\varSigFin}{\varSigAlice}{\varSigAptBob}$: The witness extraction algorithm is a DPT function that lets Alice extract the secret witness $\varWit$ after having learned the final composite signature $\varSigFin$.
        As input, it expects the partial signatures $\varSigAlice$ and $\varSigAptBob$ shared between the participants during protocol execution and the final composite signature $\varSigFin$.
        Consequently, only protocol participants knowing the partial signatures exchanged during the protocol can run this algorithm.
    \end{asparaitem}
\end{definition}

Similar to how it is defined in~\cite{aumayr2020bitcoinchannels}, additionally to regular Correctness, as described  in~\cref{def:pre:signature-scheme}, we require our Signature Scheme to satisfy Adaptor Signature Correctness.
This property is given when one can complete every masked partial signature generated by $\procAptSigId$ into a final signature for all pairs $(\varWit \opSeperate \varStatement) \opIn \cnstRelation$.
And it will then be possible to extract the witness computing $\procExtWitId$ with the required parameters.

\begin{definition}[Adaptor Signature Correctness]
    \label{def:sig:apt-sig-correctness}
    More formally, \emph{Adaptor Signature Correctness} is given if for every security parameter $\varN \in \cnstNatural$, message $\varMsg \in \cnstBinary{*}$, \\ keypairs $\langle (\varSecKeyAlice, \varPubKeyAlice, \varNonceAlice, \varSigContext), (\varSecKeyBob, \varPubKeyBob, \varNonceBob, \varSigContext) \rangle \opFunResult \procKeyGenPt{\varSecParam}{\varSecParam}$ with their composite public key $\varSigContext.\varPubKey \opEqNoQ \varPubKeyAlice \opAddPoint \varPubKeyBob$ and every statement/witness pair $(\varStatement \opSeperate \varWit) \opFunResult \procGenR{\varSecParam}$ it must hold that:
    \[
        \Pr\left[
        \begin{array}{c}
            \:\procVerf{\varMsg}{\varSigFin}{\varSigContext.\varPubKey} \opEqNoQ 1                                         \\
            \opAnd                                                                                              \\
            \: \procVerifyAptSig{\varSigAptBob}{\varMsg}{\varPubKeyBob}{\varStatement} \opEqNoQ 1             \\
            \opAnd                                                                                              \\
            \:(\funStar{\varWit}, \varStatement) \opIn \cnstRelation
        \end{array}
        \middle\vert
        \begin{array}{l}
            (\varWit, \varStatement) \opFunResult \procGenR{\varSecParam} \\
            \varSigAlice \opFunResult \procSignPrt{\varMsg}{\varSecKeyAlice}{\varNonceAlice}{\varSigContext}        \\
            \varSigBob \opFunResult \procSignPrt{\varMsg}{\varSecKeyBob}{\varNonceBob}{\varSigContext}              \\
            \varSigAptBob \opFunResult \procAptSig{\varSigBob}{\varWit}                                             \\
            \varSigFin \opFunResult \procFinSig{\varSigAlice}{\varSigBob}                                           \\
            \funStar{\varWit} \opFunResult \procExtWit{\varSigFin}{\varSigAlice}{\varSigAptBob}
        \end{array}
        \right]=1.
    \]
\end{definition}

\section{Schnorr-based Instantiation} \label{sec:sig:schnorr-inst}
We start by providing a general instantiation of a signature scheme (see definition~\ref{def:pre:signature-scheme}):
We assume we have a group $\cnstGroup$ with prime $\varPrime$, $\cnstHash$ is a secure hash function as defined in definition~\ref{def:pre:hash-function} and $\varMsg \opIn \cnstBinary{*}$ is a message.


A concrete implementation can be seen in figure~\ref{fig:schnorr}. The signature scheme is called schnorr signature scheme, first defined in~\cite{schnorr1989efficient} and is widely employed in many cryptography systems.
\begin{figure}
    \begin{center}
        \fbox{
        \begin{varwidth}{\textwidth}
            \procedure[linenumbering]{$\procSetup{\varSecParam}$} {
            \varKey \sample \cnstIntegersPrimeWithoutZero{\varPrime} \\
            \pcreturn (\varSecKey \opAssign \varKey \opSeperate \varPubKey \opAssign \funGen{\varKey})
            }
            \procedure[linenumbering]{$\procSign{\varMsg}{\varSecKey}$}{
            \varNonce \sample \cnstIntegersPrimeWithoutZero{\varPrime} \\
            \varRand \opAssign \funGen{\varNonce} \\
            \varSchnorrChallenge \opAssign \funHash{\varMsg \opConc \varRand \opConc \varPubKey} \\
            \varS \opAssign \varNonce \opAddScalar \varSchnorrChallenge \opTimesScalar \varSecKey \\
            \pcreturn \varSignature \opAssign (\varS, \varRand)
            }
            \procedure[linenumbering]{$\procVerf{\varMsg}{\varSignature}{\varPubKey}$} {
            (\varS \opSeperate \varRand) \opFunResult \varSignature \\
            \varSchnorrChallenge \opAssign \funHash{\varMsg \opConc \varRand \opConc \varPubKey} \\
            \pcreturn \funGen{\varS} \opEq \varRand \opAddPoint \opPointScalar{\varPubKey}{\varSchnorrChallenge}
            }
        \end{varwidth}
        }
    \end{center}
    \caption{Schnorr Signature Scheme as first defined in~\cite{schnorr1989efficient}}
    \label{fig:schnorr}
\end{figure}
Correctness of the scheme is easy to derive. As $\varS$ is calculated as $\varNonce \opAddScalar \varSchnorrChallenge \opTimesScalar \varSecKey$, when generator $\varG$ is raised to $\varS$, we get
$\funGen{\varNonce \opAddScalar \varSchnorrChallenge \opTimesScalar \varSecKey}$ which we can transform into $\funGen{\varNonce} \opAddPoint \funGen{\varSecKey \opTimesScalar \varSchnorrChallenge}$, and finally
into $\varRand \opAddPoint \opPointScalar{\varPubKey}{\varSchnorrChallenge}$ which is the same as the right side of the equation.

From the regular schnorr signature we now provide an instantiation for the two-party case defined in definition~\ref{def:sig:two-party-sig}. Note that this two-party variant of the scheme is what is
currently implemented in the mimblewimble-based cryptocurrencies and will provide a basis from which we will build our adapted scheme.

First we define a auxiliary function $\procSetupCtxId$ to use for the instantion:

\begin{center}
    \fbox{
    \begin{varwidth}{\textwidth}
        \procedure[linenumbering]{$\procSetupCtx{\varSigContext}{\varPubKeyAlice}{\varRandAlice}$} {
        \opAccess{\varSigContext}{\varPubKey} \opAssign \opAccess{\varSigContext}{\varPubKey} \opAddPoint \varPubKeyAlice \\
        \opAccess{\varSigContext}{\varRand} \opAssign \opAccess{\varSigContext}{\varRand} \opAddPoint \varRandAlice \\
        \pcreturn \varSigContext
        } \\
    \end{varwidth}
    }
\end{center}

This function helps the participants to setup and update the signature context shared between them. In figure~\ref{fig:twoparty-schnorr} we show a concrete instantiation of the protocol and functions. In $\procKeyGenPtId$ Alice
and Bob will each randomly chose their secret key and nonce. They further require to create a zero-knowledge proof attesting to the fact that they have generated their key before any message was exchanged. This
is essential for the scheme to achieve EUF-CMA as described by Lindell et al. ~\cite{lindell2017fast}.

In $\procKeyGenPtId$ Alice will initially setup the signature context and send it to Bob, together with her public and zk-proof.
Bob verifies the proof (and exits if it is invalid). He will proceed by adding his parameters to the signature context and send it back to Alice, together with his public key and zk-proof, which Alice will verify.

$\procSignPrtId$ and $\procVerfPtSigId$ are generally similiar to the instantiation of the normal schnorr signature scheme. Note however that for computing the schnorr challenge $\varSchnorrChallenge$ the input into
the hash function will be the combined public key $\varPubKey$ and combined nonce commitment $\varRand$, which the participants can read from the context object $\varSigContext$. This has the effect that the partial
signature itself are not yet a valid signature (neither under $\varPubKey$ nor under $\varPubKeyAlice$ or $\varPubKeyBob$). This is because to be valid under $\varPubKey$ the partial signatures are missing the $\varS$ values
from the other participants. They are also not valid under the partial public keys $\varPubKeyAlice$ or $\varPubKeyBob$ because the schnorr challenge is computed already with the combined values. There we have
to introduce the slightly adjusted $\procVerfPtSigId$ to be able to verify specifically the partial signatures.

\begin{figure}
    \begin{center}
        \fbox{
        \begin{varwidth}{\textwidth}
            \procedure[linenumbering]{$\procKeyGenPt{\varSecParam}{\varSecParam}$} {
            \textbf{Alice} \< \< \textbf{Bob} \\
            \varSecKeyAlice \sample \cnstIntegersPrimeWithoutZero{\varPrime} \< \< \varSecKeyBob \sample \cnstIntegersPrimeWithoutZero{\varPrime} \\
            \varNonceAlice \sample \cnstIntegersPrimeWithoutZero{\varPrime} \< \< \varNonceBob \sample \cnstIntegersPrimeWithoutZero{\varPrime} \\
            \varPubKeyAlice \opAssign \funGen{\varSecKeyAlice} \< \< \varPubKeyBob \opAssign \funGen{\varSecKeyBob} \\
            \varRandAlice \opAssign \funGen{\varNonceAlice} \< \< \varRandBob \opAssign \funGen{\varNonceBob} \\
            \varZkpStatementAlice \opAssign \exists \varSecKeyAlice \textit{ s.t. } \funGen{\varSecKeyAlice} \opEqNoQ \varPubKeyAlice \< \< \varZkpStatementBob \opAssign \exists \varSecKeyBob \textit{ s.t. } \funGen{\varSecKeyBob} \opEqNoQ \varPubKeyBob \\
            \varProofAlice \opFunResult \procZkpProve{\varSecKeyAlice}{\varZkpStatementAlice} \< \< \varProofBob \opFunResult \procZkpProve{\varSecKeyBob}{\varZkpStatementBob} \\
            \varSigContext \opAssign \langle \varPubKey \opAssign \cnstIdentityElement \opSeperate \varRand \opAssign \cnstIdentityElement \rangle \< \< \\
            \varSigContext \opFunResult \procSetupCtx{\varSigContext}{\varPubKeyAlice}{\varRandAlice} \< \< \\
            \< \sendmessageright*{\varSigContext, \varPubKeyAlice, \varProofAlice} \< \\
            \< \< \pcif \procZkpVerify{\varProofAlice} \opEqNoQ 0 \\
            \< \< \t \pcreturn \cnstFalsum \\
            \< \< \varSigContext \opFunResult \procSetupCtx{\varSigContext}{\varPubKeyBob}{\varRandBob} \\
            \< \sendmessageleft*{\varSigContext, \varPubKeyBob, \varProofBob} \< \\
            \pcif \procZkpVerify{\varProofBob} \opEqNoQ 0 \< \< \\
            \t \pcreturn \cnstFalsum \< \< \\
            \pcreturn (\varSecKeyAlice,\varPubKeyAlice,\varNonceAlice,\varSigContext) \< \< \pcreturn (\varSecKeyBob,\varPubKeyBob,\varNonceBob,\varSigContext)
            } \\
            \procedure[linenumbering]{$\procSignPrt{\varMsg}{\varSecKeyAlice}{\varNonceAlice}{\varSigContext}$} {
            (\varRand \opSeperate \varPubKey) \opFunResult \varSigContext \\
            \varRandAlice \opAssign \funGen{\varNonceAlice} \\
            \varSchnorrChallenge \opAssign \funHash{\varMsg \opConc \varRand \opConc \varPubKey} \\
            \varSAlice \opAssign \varNonceAlice \opAddScalar \varSecKeyAlice \opTimesScalar \varSchnorrChallenge \\
            \pcreturn \varSigAlice \opAssign (\varSAlice, \varRandAlice, \varSigContext)
            }
            \procedure[linenumbering]{$\procVerfPtSig{\varSigAlice}{\varMsg}{\varPubKeyAlice}$} {
            (\varSAlice \opSeperate \varRandAlice \opSeperate \varSigContext) \opFunResult \varSigAlice \\
            (\varPubKey \opSeperate \varRand) \opFunResult \varSigContext \\
            \varSchnorrChallenge \opAssign \funHash{\varMsg \opConc \varRand \opConc \varPubKey} \\
            \pcreturn \funGen{\varSAlice} \opEq \varRandAlice \opAddPoint \opPointScalar{\varPubKeyAlice}{\varSchnorrChallenge}
            }
            \procedure[linenumbering]{$\procFinSig{\varSigAlice}{\varSigBob}$} {
            (\varSAlice \opSeperate \varRandAlice \opSeperate \varSigContext) \opFunResult \varSigAlice \\
            (\varSBob \opSeperate \varRandBob \opSeperate \varSigContext) \opFunResult \varSigBob \\
            (\varPubKey \opSeperate \varRand) \opFunResult \varSigContext \\
            \varS \opAssign \varSAlice \opAddScalar \varSBob \\
            \varSigFin \opAssign (\varS, \varRand) \\
            \pcreturn \varSigFin
            }
        \end{varwidth}
        }
    \end{center}
    \caption{Two Party Schnorr Signature Scheme}
    \label{fig:twoparty-schnorr}
\end{figure}

We further formalize a protocol $\procDSignId$ which is a protocol between two parties running the partial signature creation outlined before.
Note that we assume that the secret keys as well as nonces used in the signatures have already been generated, for example by running the $\procKeyGenPtId$ protocol.
Both parties input the shared message $\varM$ as well as their secret keys and secret nonces.
The protocol outputs a signature $\varSigFin$ to the message $\varSigFin$, valid under the composite public key $\varPubKey \opEqNoQ \varPubKeyAlice \opAddPoint \varPubKeyBob$.
Additionally to the final signature the protocol also outputs the composite public key $\varPubKey$.

\begin{center}
    \fbox{
    \begin{varwidth}{\textwidth}
        \procedure[linenumbering]{$\procDSign{\varMsg}{\varSecKeyAlice}{\varNonceAlice}{\varSecKeyBob}{\varNonceBob}$} {
        Alice \< \< Bob \\
        \varSigContext \opAssign \{ \varPubKey \opAssign \cnstIdentityElement, \varRand \opAssign \cnstIdentityElement \} \< \< \\
        \varSigContext \opFunResult \procSetupCtx{\varSigContext}{\funGen{\varSecKeyAlice}}{\funGen{\varNonceAlice}} \< \< \\
        \< \sendmessageright*{\varSigContext, \varPubKeyAlice \opAssign \funGen{\varSecKeyAlice}} \< \\
        \< \< \varSigContext \opFunResult \procSetupCtx{\varSigContext}{\funGen{\varSecKeyBob}}{\funGen{\varNonceBob}} \< \< \\
        \< \< \varSigBob \opFunResult \procSignPrt{\varMsg}{\varSecKeyBob}{\varNonceBob}{\varSigContext} \\
        \< \sendmessageleft*{\varSigBob, \varSigContext, \varPubKeyBob \opAssign \funGen{\varSecKeyBob}} \< \\
        \pcif \procVerfPtSig{\varSigBob}{\varMsg}{\varPubKeyBob} \opEqNoQ 0 \< \< \\
        \t \pcreturn \cnstFalsum \< \< \\
        \varSigAlice \opFunResult \procSignPrt{\varMsg}{\varSecKeyAlice}{\varNonceAlice}{\varSigContext} \< \< \\
        \< \sendmessageright*{\varSigAlice} \< \\
        \< \< \pcif \procVerfPtSig{\varSigAlice}{\varMsg}{\varPubKeyAlice} \opEqNoQ 0 \\
        \< \< \t \pcreturn \cnstFalsum \\
        \varSigFin \opFunResult \procFinSig{\varSigAlice}{\varSigBob} \< \< \varSigFin \opFunResult \procFinSig{\varSigAlice}{\varSigBob} \\
        \varPubKey \opFunResult \varSigContext.\varPubKey \< \< \varPubKey \varSigContext.\varPubKey \\
        \pcreturn (\varSigFin, \varPubKey) \< \< \pcreturn (\varSigFin, \varPubKey)
        }
    \end{varwidth}
    }
\end{center}

The final signature is a valid signature to the message $\varMsg$ with the composite public key $\varPubKey \opAssign \varPubKeyAlice \opAddPoint \varPubKeyBob$.
A verifier knowing the signed message $\varMsg$, the final signature $\varSigFin$ and the composite public key $\varPubKey$ can now verify the signature using the regular $\procVerfId$ procedure.

Note that this way of computing schnorr signatures is not new.
For a proof of its correctness and a more extensive explanation we refer the reader to a paper by Maxwell et al.~\cite{maxwell2019simple}.

In figure~\ref{fig:aptSchnorr} we further provide a schnorr-based instantiation for the fixed witness adapted signature scheme as defined in definition~\ref{def:sig:two-party-fixed-wit-apt-sig}.

$\procAptSigId$ will add the secret witness $\varWit$ to the $\varS$ value of the signature, this means we will not be able to verify the adapted signature using $\procVerfPtSigId$ anymore. Therefore we
introduce $\procVerifyAptSigId$ which takes as additional parameter the statement $\varStatement$ which will be included in the verifiers equation. Now the function verifies not only validity of the portial
signature, but also that it indeed has been adapted with the witness value $\varWit$, being the discrete logarithm of $\varStatement$. After obtaining $\varSigFin$, we can then cleverly unpack the secret $\varWit$,
which is shown in the $\procExpExtId$ function.

\begin{figure}
    \begin{center}
        \fbox{
        \begin{varwidth}{\textwidth}
            \procedure[linenumbering]{$\procAptSig{\varSigPt}{\varWit}$}{
            (\varS \opSeperate \varRandAlice \opSeperate \varSigContext) \opFunResult \varSigPt \\
            \varSStar \opAssign \varS \opAddScalar \varWit \\
            \pcreturn \varSigApt \opAssign (\varSStar \opSeperate \varRandAlice \opSeperate \varSigContext)
            } \\
            \procedure[linenumbering]{$\procVerifyAptSig{\varSigAptAlice}{\varMsg}{\varPubKeyAlice}{\varStatement}$} {
            (\varSAlice \opSeperate \varRandAlice \opSeperate \varSigContext) \opFunResult \varSigAptAlice \\
            (\varPubKey \opSeperate \varRand) \opFunResult \varSigContext \\
            \varSchnorrChallenge \opAssign \funHash{\varMsg \opConc \varRand \opConc \varPubKey} \\
            \pcreturn \funGen{\varSAlice} \opEq \varRandAlice \opAddPoint \opPointScalar{\varPubKeyAlice}{\varSchnorrChallenge} \opAddPoint \varStatement
            }
            \procedure[linenumbering]{$\procExtWit{\varSigFin}{\varSigAlice}{\varSigAptBob}$}{
            (\varS \opSeperate \varRand) \opFunResult \varSigFin \\
            (\varSAlice \opSeperate \varRandAlice \opSeperate \varSigContext) \opFunResult \varSigAlice \\
            (\varSAptBob \opSeperate \varRandBob \opSeperate \varSigContext) \opFunResult \varSigAptBob \\
            \varSBob \opAssign \varS \opSub \varSAlice \\
            \varWit \opAssign \varSAptBob \opSub \varSBob \\
            \pcreturn (\varWit)
            }
        \end{varwidth}
        }
    \end{center}
    \caption{Fixed Witness Adaptor Schnorr Signature Scheme}
    \label{fig:aptSchnorr}
\end{figure}

We now define a protocol $\procDAptSignId$ between Alice and Bob creating a signature $\varSigFin$ for the composite public key $\varPubKey \opAssign \varPubKeyAlice \opAddPoint \varPubKeyBob$
Now Bob will hide his secret $\varWit$ which Alice can extract after the signing process has completed.
One thing to note is that in this protocol only Bob is able to call $\procFinSigId$ to create the final signature, which is different to the previous protocol.
This is because the function requires Bobs unadapted partial signature $\varSigBob$ as input, which Alice does not know. (She only knows Bobs adapted variant).
Therefore, one further interaction is needed to send the final signature to Alice.
The protocol outputs $(\varWit, (\varSigFin, \varPubKey))$ for Alice as she manages to learn $\varWit$ and $(\varSigFin, \varPubKey)$ for Bob.

\begin{center}
    \fbox{
    \begin{varwidth}{\textwidth}
        \procedure[linenumbering]{$\procDAptSign{\varMsg}{\varSecKeyAlice}{\varNonceAlice}{\varSecKeyBob}{\varNonceBob}{\varWit}$} {
        Alice \< \< Bob \\
        \varSigContext \opAssign \{ \varPubKey \opAssign \cnstIdentityElement, \varRand \opAssign \cnstIdentityElement \} \< \< \\
        \varSigContext \opFunResult \procSetupCtx{\varSigContext}{\funGen{\varSecKeyAlice}}{\funGen{\varNonceAlice}} \< \< \\
        \< \sendmessageright*{\varSigContext, \varPubKeyAlice \opAssign \funGen{\varSecKeyAlice}} \< \\
        \< \< \varSigContext \opFunResult \procSetupCtx{\varSigContext}{\funGen{\varSecKeyBob}}{\funGen{\varNonceBob}} \< \< \\
        \< \< \varSigBob \opFunResult \procSignPrt{\varMsg}{\varSecKeyBob}{\varNonceBob}{\varSigContext} \\
        \< \< \varSigAptBob \opFunResult \procAptSig{\varSigBob}{\varWit} \\
        \< \< \varPubKeyBob \opAssign \funGen{\varSecKeyBob} \\
        \< \< \varStatement \opAssign \funGen{\varWit} \\
        \< \sendmessageleft*{\varSigAptBob, \varSigContext, \varPubKeyBob, \varStatement} \< \\
        \pcif \procVerifyAptSig{\varSigBob}{\varMsg}{\varPubKeyBob}{\varStatement} \opEqNoQ 0 \< \< \\
        \t \pcreturn \cnstFalsum \< \< \\
        \varSigAlice \opFunResult \procSignPrt{\varMsg}{\varSecKeyAlice}{\varNonceAlice}{\varSigContext} \< \< \\
        \< \sendmessageright*{\varSigAlice} \< \\
        \< \< \pcif \procVerfPtSig{\varSigAlice}{\varMsg}{\varPubKeyAlice} \opEqNoQ 0 \\
        \< \< \t \pcreturn \cnstFalsum \\
        \< \< \varSigFin \opFunResult \procFinSig{\varSigAlice}{\varSigBob} \\
        \< \sendmessageleft*{\varSigFin} \< \\
        \varPubKey \opFunResult \varSigContext.\varPubKey \< \< \varPubKey \opFunResult \varSigContext.\varPubKey \\
        \pcif \procVerf{\varMsg}{\varSigFin}{\varPubKey} \opEqNoQ 0 \\
        \t \pcreturn \cnstFalsum \\
        \varWit \opFunResult {\varSigFin}{\varSigAlice}{\varSigAptBob} \\
        \pcreturn (\varWit, (\varSigFin, \varPubKey)) \< \< \pcreturn (\varSigFin, \varPubKey)
        }
    \end{varwidth}
    }
\end{center}

\section{Protocols} \label{sec:sig:protocols}
In this section we specify three protocols to build Mimblewimble transactions from the definitions found in~\ref{sec:atom:definitions}.
Later in section~\ref{sec:atom:security} we will prove the security of those protocols and finally in section~\ref{sec:atom:atomic-swap} we will use those protocols to build our Atomic Swap.

\subsection{Simple Mimblewimble Transaction - $\procDBuildMwTxId$} \label{subsec:atom:simple-mw-tx}

$\procDBuildMwTxId$ is a protocol between a sender and receiver which builds a mimblewimble transaction transferring a value $\varFundValue$ from the sender to a receiver for a Mimblewimble Transaction scheme as defined in~\ref{def:atom:mw-tx-scheme}
It takes as input a list of spendable coins $\funArray{\varSpendableCoin}$, a transaction value $\varFundValue$, and an optional timelock $\varTime$ from the sender, the same transaction value $\varFundValue$ from the receiver and uses the functions defined earlier to output a valid transaction $\varTx$ as well as the newly spendable coins to both parties.
\[ \langle (\varTx, \funStar{\varSpendableCoinAlice}), (\varTx, \funStar{\varSpendableCoinBob}) \rangle \opFunResult \procDBuildMwTx{\funStar{\varSpendableCoin}}{\varFundValue}{\varTime} \]
Figure~\ref{fig:d-build-mw-tx} show the implementation of the $\procDBuildMwTxId$.

\begin{figure}
    \fbox{
    \begin{varwidth}{\textwidth}
        \procedure[linenumbering,skipfirstln]{$\procDBuildMwTx{\funArray{\varSpendableCoin}}{\varFundValue}{\varTime}$}{
        Alice \< \< Bob \\
        (\varPreTx, \funStar{\varSpendableCoinAlice}, (\varSecKeyAlice, \varNonceAlice)) \pcskipln \\
        \opFunResult \procSendCoins{\funArray{\varSpendableCoin}}{\varFundValue}{\varTime} \\
        \< \sendmessageright*{\varPreTx} \< \\
        \< \< (\funStarAlt{\varPreTx}, \funStar{\varSpendableCoinBob}) \opFunResult \procRecvCoins{\varPreTx}{\varFundValue} \\
        \< \sendmessageleft*{\funStarAlt{\varPreTx}} \\
        \varTx \opFunResult \procFinTx{\funStarAlt{\varPreTx}}{\varSecKeyAlice}{\varNonceAlice} \\
        \< \sendmessageright*{\varTx} \\
        \pcreturn (\varTx, \funStar{\varSpendableCoinAlice}) \< \< \pcreturn (\varTx, \funStar{\varSpendableCoinBob})
        }
    \end{varwidth}
    }
    \caption{$\procDBuildMwTxId$ two-party protocol to build a new transaction} \label{fig:d-build-mw-tx}
\end{figure}

\subsection{Shared Output Mimblewimble Transaction - $\procDSharedOutputMwTxId$} \label{subsec:atom:shared-out-mw-tx}

$\procDSharedOutputMwTxId$ is a protocol between a sender and a receiver.
It builds a mimblewimble transaction transferring value from a sender for the Extendend Mimblewimble Transaction Scheme in ~\ref{def:atom:ext-mw-tx-scheme}.
However, instead of simply sending value to a receiver it sends it to a shared coin, for which both the sender and receiver know one part of the opening.
As input it again takes a list of spendable coins $\funArray{\varSpendableCoin}$, a transaction value $\varFundValue$ and an optional timelock $\varTime$ from the sender and the same transaction value $\varFundValue$ from the receiver.
It outputs the final transaction $\varTx$ to both parties, Alice will receiver her spendable change output $\funStar{\varSpendableCoinAlice}$ and both parties will receive their part of the shared spendable coin $\funStar{\varPtSpendableCoinAlice}$, $\funStar{\varPtSpendableCoinBob}$.

\[ \langle (\varTx, \funStar{\varSpendableCoinAlice}, \funStar{\varPtSpendableCoinAlice}), (\varTx, \funStar{\varPtSpendableCoinBob}) \rangle \opFunResult \procDSharedOutputMwTx{\funArray{\varSpendableCoin}}{\varFundValue}{\varTime} \]

One use case of this transaction protocol is to lock funds between two users, which can then be redeemed by both parties cooperating.

Figure~\ref{fig:d-shared-out-mw-tx} shows the implementation of the protocol.

\begin{figure}
    \fbox{
    \begin{varwidth}{\textwidth}
        \procedure[linenumbering,skipfirstln]{$\procDSharedOutputMwTx{\funArray{\varSpendableCoin}}{\varFundValue}{\varTime}$}{
        Alice \< \< Bob \\
        (\varPreTx, \funStar{\varSpendableCoinAlice}, (\varSecKeyAlice, \varNonceAlice)) \pcskipln \\
        \opFunResult \procSendCoins{\funArray{\varSpendableCoin}}{\varFundValue}{\varTime} \\
        \< \sendmessageright*{\varPreTx} \< \\
        (\funStarAlt{\varPreTx}, \funStar{\varPtSpendableCoinAlice}) \< \< (\funStarAlt{\varPreTx}, \funStar{\varPtSpendableCoinBob}) \pcskipln \\
        \opFunResult \procDRecvCoinsL{\varPreTx}{\varFundValue}  \< \< \opFunResult \procDRecvCoinsR{\varFundValue} \\
        \varTx \opFunResult \procFinTx{\funStarAlt{\varPreTx}}{\varSecKeyAlice}{\varNonceAlice} \\
        \< \sendmessageright*{\varTx} \\
        \pcreturn (\varTx, \funStar{\varSpendableCoinAlice}, \funStar{\varPtSpendableCoinAlice}) \< \< \pcreturn (\funStar{\varPtSpendableCoinBob})
        }
    \end{varwidth}
    }
    \caption{$\procDSharedOutputMwTxId$ two-party protocol to build a new transaction with a shared output} \label{fig:d-shared-out-mw-tx}
\end{figure}

\subsection{Shared Input Mimblewimble Transaction $\procDSharedInpMwTxId$} \label{subsec:atom:shared-inp-mw-tx}

$\procDSharedInpMwTxId$ is a protocol between a sender and a receiver.
It builds a mimblewimble transaction transferring value from a coin shared between the sender and receiver to a receiver again for the Extended Mimblewimble Transaction Scheme outlined in~\ref{def:atom:ext-mw-tx-scheme}
As input it takes a list of partial spendable coins $\funArray{\varPtSpendableCoinAlice}$, a transaction value $\varFundValue$ and an optional timelock $\varTime$ from the sender the other part of the shared spendable coins $\varPtSpendableCoinBob$ as well as the same transaction value $\varFundValue$ from the receiver.
It outputs a final transaction $\varTx$ to both parties, as well as the new outputs $\funStar{\varSpendableCoinAlice}, \funStar{\varSpendableCoinBob}$ to the respective owner.

\[ \langle (\varTx, \funStar{\varSpendableCoinAlice}), (\varTx, \funStar{\varSpendableCoinBob}) \rangle \opFunResult \procDSharedInpMwTx{\funArray{\varPtSpendableCoinAlice}}{\varFundValue}{\varTime}{\funArray{\varPtSpendableCoinBob}} \]

The protocol can be used to redeem funds which are locked created with the $\procDSharedInpMwTxId$ protocol.

Figure~\ref{fig:d-shared-inp-mw-tx} shows the implementation of the protocol.

\begin{figure}
    \fbox{
    \begin{varwidth}{\textwidth}
        \procedure[linenumbering,skipfirstln]{$\procDSharedInpMwTx{\funArray{\varPtSpendableCoinAlice}}{\varFundValue}{\varTime}{\funArray{\varPtSpendableCoinBob}}$}{
        Alice \< \< Bob \\
        (\varPreTx, \funStar{\varSpendableCoinAlice}, (\varSecKeyAlice, \varNonceAlice)) \< \< (\varPreTx, (\varSecKeyBob, \varNonceBob)) \pcskipln \\
        \opFunResult \procDSendCoinsL{\funArray{\varPtSpendableCoinAlice}}{\varFundValue}{\varTime} \< \< \opFunResult \procDSendCoinsL{\funArray{\varPtSpendableCoinBob}}{\varFundValue}{\varTime} \\
        \< \< (\funStarAlt{\varPreTx}, \funStar{\varSpendableCoinBob}) \opFunResult \procRecvCoins{\varPreTx}{\varFundValue} \\
        \< \sendmessageleft*{\funStarAlt{\varPreTx}} \\
        \varTx \opFunResult \procDFinTxL{\funStarAlt{\varPreTx}}{\varSecKeyAlice}{\varNonceAlice} \< \< \varTx \opFunResult \procDFinTxL{\funStarAlt{\varPreTx}}{\varSecKeyBob}{\varNonceBob} \\
        \pcreturn (\varTx, \funStar{\varSpendableCoinAlice}) \< \< \pcreturn (\varTx, \funStar{\varSpendableCoinBob})
        }
    \end{varwidth}
    }
    \caption{$\procDSharedOutputMwTxId$ two-party protocol to build a new transaction from a shared output} \label{fig:d-shared-inp-mw-tx}
\end{figure}

\subsection{Contract Mimblewimble Transaction - $\procDScriptMwTxId$} \label{subsec:atom:script-mw-tx}

$\procDScriptMwTxId$ is a protocol between a sender and a receiver for the Script Mimblewimble Transaction Scheme defined in~\ref{def:atom:apt-ext-mw-tx-scheme}.
Similar to the $\procDSharedInpMwTxId$ it spends an input coin which is shared between the sender and receiver.
Additionally, we utilize the adapted signature protocol from~\ref{def:sig:two-party-fixed-wit-apt-sig} to let the receiver hide a secret witness value $\varWit$ in the transaction signature which the sender can extract from the final transaction, thereby allowing the construction of primitive contracts.

\[ \langle (\varTx, \funStar{\varSpendableCoinAlice}, \varWit), (\varTx, \funStar{\varSpendableCoinBob}) \rangle \opFunResult \procDScriptMwTx{\funArray{\varPtSpendableCoinAlice}}{\varFundValue}{\varTime}{\varStatement}{\funArray{\varPtSpendableCoinBob}}{\varWit} \]

Figure~\ref{fig:d-script-tx} shows the implementation of the protocol.

\begin{figure}
    \fbox{
    \begin{varwidth}{\textwidth}
        \procedure[linenumbering,skipfirstln]{$\procDScriptMwTx{\funArray{\varPtSpendableCoinAlice}}{\varFundValue}{\varTime}{\varStatement}{\funArray{\varPtSpendableCoinBob}}{\varWit}$}{
        Alice \< \< Bob \\
        (\varPreTx, \funStar{\varSpendableCoinAlice}, (\varSecKeyAlice, \varNonceAlice)) \< \< (\varPreTx, (\varSecKeyBob, \varNonceBob)) \pcskipln \\
        \opFunResult \procDSendCoinsL{\funArray{\varPtSpendableCoinAlice}}{\varFundValue}{\varTime} \< \< \opFunResult \procDSendCoinsL{\funArray{\varPtSpendableCoinBob}}{\varFundValue}{\varTime} \\
        \< \< (\funStarAlt{\varPreTx}, \funStar{\varSpendableCoinBob}, \varSigBob)  \pcskipln \\
        \< \< \opFunResult \procAptRecvCoins{\varPreTx}{\varFundValue}{\varWit}  \\
        \< \sendmessageleft*{\funStarAlt{\varPreTx}, \funStarAlt{\varStatement}} \\
        \pcif \varStatement \opNotEq \cnstFalsum \opAnd \varStatement \opNotEq \funStarAlt{\varStatement} \\
        \t \pcreturn \cnstFalsum \\
        \varSigAptBob \opFunResult \funStarAlt{\varPreTx}.\varSignature \\
        \varSigAliceBob \< \< \varTx \pcskipln \\
        \opFunResult \procDAptFinTxL{\funStarAlt{\varPreTx}}{\varSecKeyAlice}{\varNonceAlice}{\varStatement} \< \< \opFunResult \procDAptFinTxR{\funStarAlt{\varPreTx}}{\varSecKeyBob}{\varNonceBob}{\varSigBob} \\
        \< \sendmessageleft*{\varTx} \\
        \varWit \opFunResult \procExtWit{\varTx.\varSignature}{\varSigAliceBob}{\varSigAptBob} \\
        \pcreturn (\varTx, \funStar{\varSpendableCoinAlice}, \varWit) \< \< \pcreturn (\varTx, \funStar{\varSpendableCoinBob})
        }
    \end{varwidth}
    }
    \caption{$\procDScriptMwTxId$ two-party protocol to build a primitive contract transaction} \label{fig:d-script-tx}
\end{figure}









\section{Correctness \& Security}\label{sec:sig:two-party-apt-security}
We now prove that the outlined schnorr-based instantiation is correct, i.e. Adaptor Signature Correctness holds, and is secure with regards to the definition~\ref{subsec:pre:security}.

\subsection{Adaptor Signature Correctness}\label{subsec:sig:aptsig-correctness}

To prove that Adaptor Signature Correctness holds we have 3 statements to prove, first we prove that $\procVerf{\varMsg}{\varSigFin}{\varSigContext.\varPubKey} \opEqNoQ 1$ holds in our
schnorr-based instantiation of the signature scheme, whereas $\varSigContext$ is setup such that $\varPubKey \opEqNoQ \varPubKeyAlice \opAddPoint \varPubKeyBob$.

\begin{proof}
    \label{prf:apt-schnorr-pre-sig-corr}
    For this prove we assume the setup already specified in definition~\ref{def:sig:apt-sig-correctness}.
    The proof is by showing equality of the equation checked by the verifier of the final signature by continuous substitutions in the left side of equation:
    \begin{align}
        \funGen{\varS} &\opEqNoQ \varRand \opAddPoint \opPointScalar{\varPubKey}{\varSchnorrChallenge} \\
        \funGen{\varSAlice} \opAddPoint \funGen{\varSBob} & \\
        \funGen{\varNonceAlice \opAddScalar \varSchnorrChallenge \opTimesScalar \varSecKeyAlice} \opAddPoint \funGen{\varNonceBob \opAddScalar \varSchnorrChallenge \opTimesScalar \varSecKeyBob} & \\
        \funGen{\varNonceAlice} \opAddPoint \opPointScalar{\varPubKeyAlice}{\varSchnorrChallenge} \opAddPoint \funGen{\varNonceBob} \opAddPoint \opPointScalar{\varPubKeyBob}{\varSchnorrChallenge} & \\
        \varRandAlice \opAddPoint \opPointScalar{\varPubKeyAlice}{\varSchnorrChallenge} \opAddPoint \varRandBob \opAddPoint \opPointScalar{\varPubKeyBob}{\varSchnorrChallenge} & \\
        \varRand \opAddPoint \opPointScalar{\varPubKey}{\varSchnorrChallenge} & \opEqNoQ \varRand \opAddPoint \opPointScalar{\varPubKey}{\varSchnorrChallenge} \\
        1 & \opEqNoQ 1
    \end{align}

    It remains to prove that with the same setup $\procVerifyAptSig{\varSigAptBob}{\varMsg}{\varPubKeyBob}{\varStatement} \opEqNoQ 1$ and
    $(\varStatement \opSeperate \varWit) \opIn \cnstRelation$ (whereas $\varWit$ is the output for the $\procExtWitId$ function) hold.

    \[
        \procVerifyAptSig{\varSigAptBob}{\varMsg}{\varPubKeyBob}{\varStatement} \opEqNoQ 1
    \]
    The proof is by continuous substitutions in the equation checked by the verifier:
    \begin{align}
        \funGen{\varSigAptBob} &\opEqNoQ \varRandBob \opAddPoint \opPointScalar{\varPubKeyBob}{\varSchnorrChallenge} \opAddPoint \varStatement \\
        \funGen{\varSigBob \opAddScalar \varWit} & \\
        \funGen{\varNonceBob \opAddScalar \varSecKeyBob \opTimesScalar \varSchnorrChallenge \opAddScalar \varWit} & \\
        \funGen{\varNonceBob} \opAddPoint \funGen{\varSecKeyBob \opTimesScalar \varSchnorrChallenge} \opAddScalar \funGen{\varWit} & \\
        \varRandBob \opAddPoint \opPointScalar{\varPubKeyBob}{\varSchnorrChallenge} \opAddPoint \varStatement &\opEqNoQ \varRandBob \opAddPoint \opPointScalar{\varPubKeyBob}{\varSchnorrChallenge} \opAddPoint \varStatement \\
        1 &\opEqNoQ 1
    \end{align}
    We now continue to prove the last equation required:
    \[
        (\varStatement \opSeperate \varWit) \opIn \cnstRelation
    \]
    We do this by showing that $\varWit$ is calculated correctly in $\procExtWitId$:
    $\varSAptBob$ is the $\varS$ value in Bobs adapted partial signature
    \begin{align}
        \varWit \opEqNoQ & \varSApt \opSub (\varS \opSub \varSAlice) \\
        & \varSAptBob \opSub ((\varSAlice \opAddScalar \varSBob ) \opSub \varSAlice ) \\
        & \varSBob \opAddScalar \varWit \opSub (\varSBob) \\
        \varWit \opEqNoQ & \varWit \\
        1 \opEqNoQ & 1
    \end{align}
\end{proof}

\subsection{Security}\label{subsec:sig:secureaptscheme}

We have shown that the outlined signature scheme is correct, next we have to prove its security.
Our goal is to proof security in the malicious setting (as defined in~\ref{subsec:pre:security}) that means the adversary might or might not behave as specified by the protocol.
For achieving this we will prove security for both the $\procDSignId$ and $\procDAptSignId$ protocols in the hybrid model which was layed out by Yehuda Lindell in~\cite{lindell2017simulate}.
In particular, we will use the $\procZKfId{\cnstRelation}$-model in which we assume that we have access to a constant-round protocol $\procZKfId{\cnstRelation}$ that computes the zero-knowledge proof of knowledge functionality for any $\cnstNP$ relation $\cnstRelation$.
The function is parameterized with a relation $\cnstRelation$ between a witness value $\varWit$ (or potentially multiple)  and a statement $\varStatement$.
One party provides the witness statment pair $(\varWit, \varStatement)$, the second the statement $\funStar{\varStatement}$.
If $\varStatement \opEqNoQ \funStar{\varStatement}$ and $\cnstRelation (\varWit, \varStatement)$ the functionality returns 1, otherwise 0.
More formally:
\[
    \procZkf{\cnstRelation}{((\varWit, \varStatement), \funStar{\varStatement})} \opEqNoQ
    \begin{cases}
        (\lambda, \cnstRelation(\varStatement, \varWit)) & \text{if } \varStatement \opEqNoQ \funStar{\varStatement} \\
        (\lambda, 0) &\text{otherwise}
    \end{cases}
\]
That a constant-round zero-knowledge proof of knowledge exists was proven in~\cite{lindell2013note}.
We recall from~\ref{sec:pre:privacy:zeroknowlegde} that a secure zero-knowledge proof must fulfill Completeness, Soundness and Zero-Knowledge.

\paragraph{Hybrid functionalities:} The parties have access to a trusted third party that computes the zero-knowledge proof of knowledge functionality $\procZKfId{\cnstRelation}$. $\cnstRelation$ is the relation between a secret key $\varSecKey$ and its public key $\varPubKey \opEqNoQ \funGen{\varSecKey}$, for the elliptic curve generator point $\varG$.
The participants have to call the functionality in the same order.
That means if the prover first sends the pair $(\varWit_1, \varStatement_1)$ and then $(\varWit_2, \varStatement_2)$ the verifier needs to first send $\varStatement_1$ and then $\varStatement_2$.

\paragraph{Proof idea:} In order to construct our simulation proof in the hybrid-model we make some adjustments to the $\procDSignId$ protocol utilizing the capabilities of the $\procZKfId{\cnstRelation}$ functionality:
\begin{center}
    \fbox{
    \begin{varwidth}{\textwidth}
        \procedure[linenumbering,skipfirstln]{$\procDSign{\varMsg}{\varSecKeyAlice}{\varNonceAlice}{\varSecKeyBob}{\varNonceBob}$} {
        Alice \< \< Bob \\
        \cdots \< \< \\
        \procZkf{\cnstRelation}{(\varSecKeyAlice, \varPubKeyAlice)} \\
        \procZkf{\cnstRelation}{(\varNonceAlice, \varRandAlice)} \\
        \< \sendmessageright*{\varSigContext, \varPubKeyAlice, \varRandAlice} \< \\
        \< \< \cdots \\
        \< \< \pcif \procZkf{\cnstRelation}{\varPubKeyAlice} \opEqNoQ 0 \opOr \procZkf{\cnstRelation}{\varSigContext.\varRand} \opEqNoQ 0 \\
        \< \< \t \pcreturn \cnstFalsum \\
        \< \< \procZkf{\cnstRelation}{(\varSecKeyBob, \varPubKeyBob)} \\
        \< \< \procZkf{\cnstRelation}{(\varNonceBob, \varRandBob)} \\
        \< \sendmessageleft*{\varSigBob, \varSigContext, \varPubKeyBob} \< \\
        \cdots \< \< \\
        \pcif \procZkf{\cnstRelation}{\varPubKeyBob} \opEqNoQ 0 \opOr \pcskipln \\
        \t \procZkf{\cnstRelation}{\varSigContext.\varRand \opAddPoint \varRandAlice^{-1}} \opEqNoQ 0 \\
        \t \pcreturn \cnstFalsum \\
        \< \cdots \< \\
        \pcreturn (\varSigFin, \varPubKey) \< \< \pcreturn (\varSigFin, \varPubKey)
        }
    \end{varwidth}
    }
\end{center}

That means both Alice and Bob will verify the validity of the public key and nonce commitments of the other party and will stop protocol execution in case an invalid value has been sent.
We assume parties have access to a trusted third party computing $\procZKfId{\cnstRelation}$ which will return 1 iff $\varPubKeyAlice \opEqNoQ \funStar{\varPubKeyAlice}$ (where $\funStar{\varPubKeyAlice}$ is the public key that Bob received from Alice) and $\varPubKeyAlice \opEqNoQ \funGen{\varSecKeyAlice}$. (The same holds for the reversed case)

\begin{theorem}\label{lem:sig:security}
Assume we have two key pairs $\varKeyPairAlice$ and $\varKeyPairBob$ which were setup securly as for instance with the distributed keygen protocol $\procKeyGenPtId$.
    Then $\procDSignId$ securely computes a signature $\varSigFin$ under the composite public key $\varPubKey \opAssign \varPubKeyAlice \opAddPoint \varPubKeyBob$ in the $\procZKfId{\cnstRelation}$-model.
\end{theorem}

\begin{proof}
    We proof security of the protocol by constructing a simulator $\cnstSimulator$ who is given output $(\varSigFin, \varPubKey)$ from a TTP (trusted third party) that securely computes the protocol in the ideal world upon receiving the inputs from Alice and Bob.
    The task of the simulator will be to extract the inputs used by $\cnstAdversary$ such that he is able to call the TTP and receive the outputs.
    From this output the simulator $\cnstSimulator$ will have to construct a transcript which is indistinguishable from the protocol transcript in the real world in which the corrupted party is controlled by a deterministic polynomial adversary $\cnstAdversary$.
    The simulator uses the calls to $\procZKfId{\cnstRelation}$ in order to do this.
    Furthermore we assume that the message $\varMsg$ is known to both Alice and Bob.
    All other inputs (including public keys) are only known to the respective party at the start of the protocol.
    We have to proof two cases, one in which Alice and one in which Bob is the corrupted party.
    
    \textbf{Alice is corrupted: } Simulator $\cnstSimulator$ works as follows:
    \begin{enumerate}
        \item $\cnstSimulator$ invokes $\cnstAdversary$ receives and saves $(\varSecKeyAlice, \varPubKeyAlice)$, as well as $(\varNonceAlice, \varRandAlice)$ that $\cnstAdversary$ sends to $\procZKfId{\cnstRelation}$.
        \item Next $\cnstSimulator$ receives the message $(\varSigContext, \funStar{\varPubKeyAlice}, \funStar{\varRandAlice})$ sent to Bob by $\cnstAdversary$.
        If $\funStar{\varPubKeyAlice} \opNotEq \varPubKeyAlice$ or $\funStar{\varRandAlice} \opNotEq \varRandAlice$ $\cnstSimulator$ externally sends $\cnstAbort$ to the TTP computing $\procDSignId$ and outputs $\cnstFalsum$, otherwise he will send the inputs $(\varMsg, \varSecKeyAlice, \varNonceAlice)$ and receive back $(\varSigFin, \varPubKey)$.
        \item $\cnstSimulator$ now calculates $\varPubKeyBob, \varRandBob$ and $\varSigBob$ as follows:
        \begin{gather*}
            (\varS, \varRand) \opFunResult \varSigFin \\
            \varPubKeyBob \opAssign \varPubKey \opAddPoint \varPubKeyAlice^{-1} \\
            \varRandBob \opAssign \varRand \opAddPoint \varRandAlice^{-1} \\
            \varSigContext \opFunResult \procSetupCtx{\varSigContext}{\varPubKeyBob}{\varRandBob} \\
            \varSigAlice \opFunResult \procSignPrt{\varMsg}{\varSecKeyAlice}{\varNonceAlice}{\varSigContext} \\
            (\varSAlice, \varRandAlice, \varSigContext) \opFunResult \varSigAlice \\
            \varSBob \opAssign \varS \opSub \varSAlice \\
            \varSigBob \opAssign (\varSBob, \varRandBob, \varSigContext)
        \end{gather*}
        \item After having done the calculations $\cnstSimulator$ is able to send $\varSigContext, \varSigBob, \varPubKeyBob$ to $\cnstAdversary$ as if coming from Bob.
        \item When $\cnstAdversary$ calls $\procZKfId{\cnstRelation}$ and $\procZKfId{\cnstRelation}$ (as the verifier) $\cnstSimulator$ checks equality with $\varPubKeyBob$ (respective $\varRandBob$) and thereafter send back either 0 or 1.
        \item Eventually $\cnstSimulator$ will receive $\funStar{\varSigAlice}$ from $\cnstAdversary$ and checks if $\varSigAlice \opEqNoQ \funStar{\varSigAlice}$.
        If they are indeed the same the simulator will send $\cnstContinue$ to the TTP and output whatever $\cnstAdversary$ outputs, otherwise he will send $\cnstAbort$ and output $\cnstFalsum$.
    \end{enumerate}

    We now show that the joint output distribution in the ideal model with $\cnstSimulator$ is identically distributed to the joint distribution in a real execution in the $\procZKfId{\cnstRelation}$-hybrid model with $\cnstAdversary$.
    We consider three phases :
    \textbf{(1)} Alice sends $(\varSecKeyAlice, \varPubKeyAlice)$ as well as $(\varNonceAlice, \varRandAlice)$ to $\procZKfId{\cnstRelation}$ and $(\varSigContext, \varPubKeyAlice, \varRandAlice)$ to Bob
    \textbf{(2)} Bob sends $\varPubKeyAlice$ and $\varSigContext.\varRand$ to $\procZKfId{\cnstRelation}$ as the verifier, and  $(\varSecKeyBob, \varPubKeyBob)$, $(\varNonceBob, \varRandBob)$ to $\procZKfId{\cnstRelation}$ as the prover.
    Afterward he sends $(\varSigBob, \varSigContext, \varPubKeyBob)$ to Alice.
    \textbf{(3)} Alice sends $\varPubKeyBob$ and $\varRandBob$ to $\procZKfId{\cnstRelation}$ as the verifier and finally $\varSigAlice$ to Bob.

    \begin{itemize}
        \item \textit{Phase 1} Since $\cnstAdversary$ is required to be deterministic, the distribution is identical to a real execution.
        Also in the case the Alice does not send a message, or sends invalid values which will lead Bob to output $\cnstFalsum$ we also output $\cnstFalsum$ in the simulation, which again is indistinguishable.
        \item \textit{Phase 2} As $\cnstSimulator$ managed to calculate Bobs $\varSigBob, \varPubKeyBob, \varRandBob$ from the final $(\varSigFin, \varPubKey)$ and none of the values depend on any random tape we can say that the values sent in the ideal model are identical to those in the real model.
        As Bob in this case is the honest party, we don't have to consider any deviation from the protocol specification.
        \item \textit{Phase 3} The messages sent by the deterministic $\cnstAdversary$ again have to be identical to the real execution, therefore the transcript will be indistinguishable.
    \end{itemize}

    We have shown that the distributions in each phase are indeed identical, which proves the indistinguishability of the two transcripts in the case Alice is corrupted.

    \textbf{Bob is corrupted: } Simulator $\cnstSimulator$ works as follows:
    \begin{enumerate}
        \item $\cnstSimulator$ starts by sampling $\varSecKeyAlice, \varNonceAlice \sample \cnstIntegersPrimeWithoutZero{*}$ and proceeds by setting up the initial signature context as defined in the protocol:
        \begin{gather*}
            \varSigContext \opAssign \{ \varPubKey \opAssign 1, \varRand \opAssign 1 \} \\
            \varSigContext \opFunResult \procSetupCtx{\varSigContext}{\funGen{\varSecKeyAlice}}{\funGen{\varNonceAlice}} \\
        \end{gather*}
        \item $\cnstSimulator$ now invokes $\cnstAdversary$ and sends $(\varSigContext, \varPubKeyAlice, \varRandAlice)$ as if coming from Alice.
        \item When $\cnstAdversary$ calls $\procZKfId{\cnstRelation}$ (as verifier) $\cnstSimulator$ checks equality to the parameters he sent in step 1 and returns either 1 or 0.
        When $\cnstAdversary$ calls $\procZkf{\cnstRelation}{(\varSecKeyBob, \varPubKeyBob)}$ and $\procZkf{\cnstRelation}{(\varNonceBob, \varRandBob)}$ the simulator saves those values to its memory.
        \item Now $\cnstSimulator$ externally sends the inputs $(\varMsg, \varSecKeyBob, \varNonceBob)$ to the TTP and receives back $(\varSigFin, \varPubKey)$
        \item When $\cnstAdversary$ queries $\funHash{\varMsg \opConc \varRandAlice \opAddPoint \varRandBob \opConc \varPubKeyAlice \opAddPoint \varPubKeyBob}$ $\cnstSimulator$ sends back $\funStar{\varSchnorrChallenge}$ such that:
        \begin{gather*}
            \varSigFin \opEqNoQ \varNonceAlice \opAddScalar \varSecKeyAlice \opTimesScalar \funStar{\varSchnorrChallenge} \opAddScalar \varNonceBob \opAddScalar \varSecKeyBob \opTimesScalar \funStar{\varSchnorrChallenge} \\
            \funStar{\varSchnorrChallenge} \opEqNoQ \frac{\varSigFin \opSub \varNonceAlice \opSub \varNonceBob}{\varSecKeyAlice \opAddScalar \varSecKeyBob}
        \end{gather*}
        \item $\cnstSimulator$ receives $(\varSigBob, \varSigContext, \varPubKeyBob)$ from $\cnstAdversary$.
        (In case he does not $\cnstSimulator$ sends $\cnstAbort$ to the TTP and outputs $\cnstFalsum$).
        He verifies the values sent to him by comparing them with $\varPubKeyBob$ and $\varRandBob$ from its memory, if they are found to be invalid $\cnstSimulator$ sends $\cnstAbort$ to the TTP, otherwise it sends $\cnstContinue$.
        \item $\cnstSimulator$ calculates as defined in the protocol as $\varSigAlice \opFunResult \procSignPrt{\varMsg}{\varSecKeyAlice}{\varNonceAlice}{\varSigContext}$ and then sends it to $\cnstAdversary$ as if coming from Alice and finally outputs whatever $\cnstAdversary$ outputs.
    \end{enumerate}
    Again we argue why the transcript is indistinguishable from the real one for each of the three phases layed out before:
    \begin{itemize}
        \item \textit{Phase 1: } The values $(\varPubKeyAlice, \varRandAlice)$ sent by $\cnstSimulator$ to $\cnstAdversary$ only depend on Alice's input parameters (and to some extend on the public elliptic curve parameters).
        As $\cnstAdversary$ does not know $\varPubKeyAlice$ or $\varRandAlice$ yet, he has no way of determining for two public keys $\varPubKeyAlice, \funStar{\varPubKeyAlice}$ which of the two is the correct one (other than guessing).
        \item \textit{Phase 2: } When $\cnstAdversary$ calls $\procZKfId{\cnstRelation}$ with the parameters sent to him he will still receive 1 back, and 0 otherwise, which is again exactly the same as in the real execution.
        The hash function $\funHash{\cdot}$ is expected to output a random value for the schnorr challenge as defined by the hiding property of the hash function.
        In the simulated case $\cnstSimulator$ calculates the output value from the final signature and depends on the input values of Alice and Bob of which at least Alice input is chosen randomly by $\cnstSimulator$.
        As dependent on randomly chosen inputs the calculation output will as well be distributed uniformly across the possible values and is therefore indistinguishable from a real hash function output.
        The remaining messages sent by $\cnstAdversary$ are identical to those of the real execution due to the deterministic nature of $\cnstAdversary$.
        \item \textit{Phase 3: } The simulator will now verify the values sent to him by $\cnstAdversary$ and will halt and output $\cnstFalsum$ in the case that he sends something invalid which is identical to the real execution.
        In this case $\cnstAdversary$ must not receive ($\varSigFin, \varPubKey$) in the ideal setting which is modelled by $\cnstSimulator$ sending $\cnstAbort$ to the TTP.
        Otherwise $\cnstSimulator$ will calculate his part of the partial signature as defined by the protocol.
        It will therefore found to be valid by $\cnstAdversary$ and will complete to $\varSigFin$ with $\procFinSigId$, because of the fixed, calculated schnorr challenge $\cnstSimulator$ calculated in Phase 2.
    \end{itemize}

    We have managed to show that in the case that Bob is corrupted the transcript is indistinguishable to a real transcript and even identical for the most part.
    We can therefore conclude that the transcript output will be indistinguishable from a real one in all cases and have thereby proven that the protocol $\procDSignId$ is secure.
\end{proof}

We now do the same for $\procDAptSignId$:
Again we adjust the protocol with calls to $\procZKfId{\cnstRelation}$, note that we now have one additional call $\procZKfId{\cnstRelation}$, for the pair $(\varWit, \varStatement)$.
The relation $\cnstRelation$ is equally defined as in the previous proof.
\begin{center}
    \fbox{
    \begin{varwidth}{\textwidth}
        \procedure[linenumbering,skipfirstln]{$\procDAptSign{\varMsg}{\varSecKeyAlice}{\varNonceAlice}{\varSecKeyBob}{\varNonceBob}{\varWit}$} {
        Alice \< \< Bob \\
        \cdots \< \< \\
        \procZkf{\cnstRelation}{(\varSecKeyAlice, \varPubKeyAlice)} \\
        \procZkf{\cnstRelation}{(\varNonceAlice, \varRandAlice)} \\
        \< \sendmessageright*{\varSigContext, \varPubKeyAlice, \varRandAlice} \< \\
        \< \< \cdots \\
        \< \< \pcif \procZkf{\cnstRelation}{\varPubKeyAlice} \opEqNoQ 0 \opOr \procZkf{\cnstRelation}{\varSigContext.\varRand} \opEqNoQ 0 \\
        \< \< \t \pcreturn \cnstFalsum \\
        \< \< \procZkf{\cnstRelation}{(\varSecKeyBob, \varPubKeyBob)} \\
        \< \< \procZkf{\cnstRelation}{(\varNonceBob, \varRandBob)} \\
        \< \< \procZkf{\cnstRelation}{(\varWit, \varStatement)} \\
        \< \sendmessageleft*{\varSigBob, \varSigContext, \varPubKeyBob, \varStatement} \< \\
        \cdots \< \< \\
        \pcif \procZkf{\cnstRelation}{\varPubKeyBob} \opEqNoQ 0 \opOr \pcskipln \\
        \t \procZkf{\cnstRelation}{\varSigContext.\varRand \opAddPoint \varRandAlice^{-1}} \opEqNoQ 0 \opOr \pcskipln \\
        \t \procZkf{\cnstRelation
        }{\varStatement} \opEqNoQ 0 \opOr \\
        \t \pcreturn \cnstFalsum \\
        \< \cdots \< \\
        \pcreturn (\varWit, (\varSigFin, \varPubKey)) \< \< \pcreturn (\varSigFin, \varPubKey)
        }
    \end{varwidth}
    }
\end{center}

\begin{theorem}
    Assume we have two key pairs $\varKeyPairAlice$ and $\varKeyPairBob$ which were setup securly as for instance with the distributed keygen protocol $\procKeyGenPtId$.
    Additionally we have a pair $(\varWit, \varStatement)$ in the relation $\varStatement \opEqNoQ \funGen{\varWit}$ for which $\varWit$ was chosen randomly.
    Then $\procDAptSignId$ securely computes a signature $\varSigFin$ under the composite public key $\varPubKey \opAssign \varPubKeyAlice \opAddPoint \varPubKeyBob$ after which $\varWit$ is revealed to Alice, in the $\procZKfId{\cnstRelation}$-model.
\end{theorem}

\begin{proof}
    We proof the security of $\procDAptSignId$ by constructing a simulator $\cnstSimulator$ who is given the output $(\varSigFin, \varPubKey)$ (resp. ($\varWit, (\varSigFin, \varPubKey)$)) from a TTP that securly computes the protocol in the ideal world after receiving the inputs from Alice and Bob.
    The simulators task again is to extract the adversaries inputs and send them to the trusted third party to receive the protocol outputs.
    From this output the simulator $\cnstSimulator$ will construct a transcript that is indistinguishable from the protocol transcript in the real world.
    The simulator uses the calls to $\procZKfId{\cnstRelation}$ in order to do this.
    As in the proof before we assume the message $\varMsg$ is known to both participants.
    All other inputs (including public keys) are only known to the respective party at the start of the protocol.
    We proof that the transcript is indistinguishable in case Alice is corrupted as well as in the case that Bob is corrupted.

    \textbf{Alice is corrupted: } Simulator $\cnstSimulator$ works as follows:
    \begin{enumerate}
        \item $\cnstSimulator$ invokes $\cnstAdversary$.
        When $\cnstAdversary$ internally calls $\procZKfId{\cnstRelation}$ and $\procZKfId{\cnstRelation}$ $\cnstSimulator$ saves $(\varSecKeyAlice, \varPubKeyAlice)$ and $(\varNonceAlice, \varRandBob)$ to its memory.
        \item $\cnstSimulator$ receives $(\varSigContext, \funStar{\varPubKeyAlice}, \funStar{\varPubKeyBob})$ from $\cnstAdversary$.
        $\cnstSimulator$ checks the equalities $\funStar{\varPubKeyAlice} \opEqNoQ \varPubKeyAlice$ and $\funStar{\varRandAlice} \opEqNoQ \varRandAlice$ as well as checking $\varPubKeyAlice \opEqNoQ \funGen{\varSecKeyAlice}$ and $\varRandAlice \opEqNoQ \funGen{\varNonceAlice}$.
        If any of those checks fail $\cnstSimulator$ sends $\cnstAbort$ to the TTP and outputs $\cnstFalsum$.
        Otherwise he sends $(\varMsg, \varSecKeyAlice, \varNonceAlice)$ to the TTP and receives $(\varWit, (\varSigFin, \varPubKey))$
        \item Again $\cnstSimulator$ calculates $\varSigBob, \varPubKeyBob, \varRandBob$ and finalizes the context $\varSigContext$ as layed out in the proof beforehand in step 3.
        \item $\cnstSimulator$ calculates $\funStar{\varSBob} \opAssign \varSBob \opAddScalar \varWit$ (extracted from the TTP output) from which he sets $\varSigAptBob \opAssign (\funStar{\varSBob}, \varRandBob, \varSigContext)$.
        \item $\cnstSimulator$ sends $(\varSigAptBob, \varSigContext, \varPubKeyBob, \varStatement \opAssign \funGen{\varWit})$ as if coming from Bob.
        \item When $\cnstAdversary$ calls $\procZKfId{\cnstRelation}$ we compare the parameters send by $\cnstAdversary$ to the real one, in case he sent a invalid value $\cnstSimulator$ returns 0, otherwise 1.
        \item $\cnstSimulator$ receives $\funStar{\varSigAlice}$ from $\cnstAdversary$ and checks $\varSigAlice \opEqNoQ \funStar{\varSigAlice}$.
        If the equality holds $\cnstAdversary$ sends $\cnstContinue$ to the TTP and finally sends $\varSigFin$ to $\cnstAdversary$ as if coming from Bob and outputs whatever $\cnstAdversary$ outputs.
    \end{enumerate}

    We reuse the phases defined in the previous proof with two adjustments:
    \begin{itemize}
        \item In \textit{Phase 2} Bob additionally sends $\varStatement$ to Alice
        \item We introduce \textit{Phase 4} in which Bob sends $\varSigFin$ to Alice
    \end{itemize}

    We now again argue why each phase in the simulation is indistinguishable from a real execution
    \begin{itemize}
        \item \textit{Phase 1:} This phase is identical to phase 1 the previous proof, thereby the argument is the same.
        \item \textit{Phase 2:} In this phase $\cnstSimulator$ sends $\varStatement \opAssign \funGen{\varWit}$ to $\cnstAdversary$ for which $\varWit$ was received from the TTP, therefore it will be the same $\varWit$ sent in the real execution by the honest party which makes the simulation perfect in this phase.
        \item \textit{Phase 3:} Again the messages send in this phase are produced by the deterministic $\cnstAdversary$ which will be indistinguishable to the real execution.
        In contrast to the $\procDSignId$ protocol now the adversary does not yet finish the protocol.
        \item \textit{Phase 4:} Now the $\cnstAdversary$ expects to receive $\varSigFin$, from which he is able to extract the witness $\varWit$.
        Indeed he will receive a $\varSigFin$ which is identical to the one sent in a real execution by honest Bob, furthermore he will be able to extract $\varWit$ such that $\varStatement \opEqNoQ \funGen{\varWit}$, which again makes this phase identical to the real execution.
    \end{itemize}

    We have shown that in the case Alice is corrupt the simulated transcript produced by $\cnstSimulator$ is indeed distributed equally to a real execution and is thereby computationally indistinguishable.

    \textbf{Bob is corrupted: } Simulator $\cnstSimulator$ works as follows:
    \begin{enumerate}
        \item $\cnstSimulator$ starts by sampling $\varSecKeyAlice, \varNonceAlice \sample \cnstIntegersPrimeWithoutZero{*}$ and proceeds by setting up the initial signature context as defined in the protocol:
        \begin{gather*}
            \varSigContext \opAssign \{ \varPubKey \opAssign 1, \varRand \opAssign 1 \} \\
            \varSigContext \opFunResult \procSetupCtx{\varSigContext}{\funGen{\varSecKeyAlice}}{\funGen{\varNonceAlice}} \\
        \end{gather*}
        \item $\cnstSimulator$ now invokes $\cnstAdversary$ and sends $(\varSigContext, \varPubKeyAlice, \varRandAlice)$ as if coming from Alice.
        \item When $\cnstAdversary$ calls $\procZKfId{\cnstRelation}$ (as the verifier) $\cnstSimulator$ checks for equality with the values sent by him and returns either 0 or 1.
        Once $\cnstAdversary$ sends $(\varSecKeyBob, \varPubKeyBob)$, $(\varNonceBob, \varRandBob)$, $(\varWit, \varStatement)$ internally to $\procZKfId{\cnstRelation}$ as the prover $\cnstSimulator$ saves them to his memory.
        \item $\cnstSimulator$ sends $(\varMsg, \varSecKeyAlice, \varNonceAlice, \varWit)$ to the TTP and receives $(\varSigFin, \varPubKey)$.
        \item When $\cnstAdversary$ queries $\funHash{\cdot}$ the simulator again sets the output to $\funStar{\varSchnorrChallenge}$ calculated with the same steps as layed out in the previous proof in step 5.
        \item $\cnstSimulator$ receives $(\funStar{\varSigAptBob}, \funStar{\varPubKeyBob}, \varSigContext, \funStar{\varStatement})$ from $\cnstAdversary$ and verifies those values checking equality with the ones stored in its memory.
        If the equality checks succeed $\cnstSimulator$ sends $\cnstContinue$ to the TTP, otherwise sends $\cnstAbort$ and outputs $\cnstFalsum$.
        \item The simulator now calculates $\varSigAlice$ as defined by the protocol using the $\procSignPrtId$ procedure and sends the result to $\cnstAdversary$ as if coming from Alice.
        \item Finally $\cnstSimulator$ will receive $\funStar{\varSigFin}$ from $\cnstAdversary$ (if not he outputs $\cnstFalsum$) and will verify that $\funStar{\varSigFin} \opEqNoQ \varSigFin$.
        If the equality holds he will output whatever $\cnstAdversary$ outputs, otherwise $\cnstFalsum$.
    \end{enumerate}
    
    Again we argue why the transcript is indistinguishable in phases 1--4.
    \begin{itemize}
        \item \textit{Phase 1:} This phase is identical to phase 1 in the previous proof, thereby the same argumentation holds.
        \item \textit{Phase 2:} Again this phase is similar to phase 2 in the $\procDSignId$ proof with the only difference that $\cnstAdversary$ will make the additional call to $\procZkf{\cnstRelation}{(\varWit, \varStatement)}$ and send the value $\varStatement$ to $\cnstSimulator$.
        Both these changes do not require any further interaction from $\cnstSimulator$ thereby the arguments from the previous proof in phase 2 still hold.
        \item \textit{Phase 3:} In this section $\cnstSimulator$ will verify equality of the values sent by $\cnstAdversary$ with the variables saved prior to its memory and halts with output $\cnstFalsum$ if any of the values are unequal.
        In this case $\cnstAdversary$ should not receive the final outputs $(\varSigFin, \varPubKey)$ which is modelled by sending $\cnstAbort$ to the TTP.
        The same behaviour is expected in a real execution when Alice calls $\procZKfId{\cnstRelation}$ and receives a 0 bit.
        We have already argued in the prior proof why $\varSigAlice$ is indistinguishable from the one calculated by Alice in a real execution and only refer to the argumentation here.
        \item \textit{Phase 4:} In this phase $\cnstSimulator$ is expected to receive $\funStar{\varSigFin}$ from $\cnstAdversary$ which needs to be equal to $\varSigFin$ received earlier by the TTP.
        $\cnstSimulator$ will do this simple equality check and if successful output whatever $\cnstAdversary$ outputs.
        In the other case we would simply output $\cnstFalsum$ which is identical to the case in which a Bob sends a $\varSigFin$ that does not verify.
    \end{itemize}

    We have shown that the transcript produced by $\cnstSimulator$ in an ideal world with access to a TTP computing $\procDAptSignId$ is indistinguishable from a transcript produced during a real execution both in the case that Alice and that Bob is corrupted.
    By managing to show this we have proven that the protocol is secure.
\end{proof}
